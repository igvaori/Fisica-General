\PassOptionsToPackage{svgnames}{xcolor} %para los myblock - ha de estar al principio 
\documentclass[a4paper, 12pt, spanish]{book}
\usepackage[utf8]{inputenc}
\usepackage[T1]{fontenc}
\usepackage[spanish]{babel}
\usepackage{float} %posicionar figuras
\usepackage{multicol}
\usepackage{multirow} % esto hará falta para figuras casi seguro
\usepackage{amsmath} %ecuaciones bien
\usepackage{amsthm}
\usepackage{amsfonts}
\usepackage{amssymb}
\usepackage{graphicx} %opciones /includegraphics[...]
\usepackage{indentfirst} % sangria de primera linea
%\setlength{\parindent}{0cm} %elimino sangrado 1a linea
\usepackage{fancyhdr}
\usepackage{caption}
\usepackage{subcaption}
\usepackage{esint} % integrales mejor
\usepackage[top=1.8cm, bottom=1.6cm, left=2cm, right=1.6cm]{geometry} % márgenes (para encuadernado-book: impares más a la izquda, pares más a la dcha.
\usepackage{comment} % comment environment
\usepackage{physics}
\usepackage{enumerate}
%\usepackage{footnote} 
%\makesavenoteenv{tabular} 
%\makesavenoteenv{table}%% write footnotes on tables
%\setlength{\parskip}{10px} %espacio parrafos error con teoremas
%\usepackage{verbatim} % comentarios
\usepackage{comment} % comentarios
\usepackage{accents}
\usepackage{pdfpages} %insertar pdfs
\usepackage{ragged2e} % para  \justifying después de fcolorbox-parbox
\usepackage{mathtools} 

\widowpenalty10000
\clubpenalty10000
\setcounter{tocdepth}{3}

%FLOAT EQNS LEFT
\usepackage{nccmath} 
\makeatletter
\newcommand{\leqnomode}{\tagsleft@true}
\newcommand{\reqnomode}{\tagsleft@false}
\makeatother

%nuevos colores - ahora tengo los "svgnames Colors"
\definecolor{roig}{RGB}{196,49,24}
\definecolor{morat}{RGB}{131,54,147}
\definecolor{verd}{RGB}{85,107,47}
\definecolor{gris}{RGB}{100,100,100}
\definecolor{blau}{RGB}{0,0,100}
\definecolor{fondoblau}{RGB}{232,255,255}
\definecolor{fondoroig}{RGB}{245,194,194}
\definecolor{fondoverd}{RGB}{209,240,192}

\newcommand{\subrayado}[1]{\colorbox{LightYellow}{$\displaystyle #1$}} %fosforito ecuaciones begin{eq... 

\newtheorem{teor}{Teorema}
\newtheorem{coro}{Corolario}
\newtheorem{prop}{Proposición}
\newtheorem{defi}{Definición}
\newtheorem{axio}{Axioma}
\newtheorem{ejem}{Ejemplo}
\newtheorem{ejer}{Ejercicio}
\newtheorem{ejre}{Ejercicio resuelto}
\newtheorem{ayud}{Ayuda.}
\newtheorem{solu}{Solución.}
\newtheorem{prob}{Problema}

\numberwithin{equation}{chapter}
\numberwithin{teor}{chapter}
\numberwithin{coro}{chapter}
\numberwithin{prop}{chapter}
\numberwithin{defi}{chapter}
\numberwithin{axio}{chapter}
\numberwithin{ejem}{chapter}
\numberwithin{ejer}{chapter}
\numberwithin{ejre}{chapter}
\numberwithin{ayud}{chapter}
\numberwithin{solu}{chapter}
\numberwithin{prob}{chapter}

\usepackage{fancyhdr}
\pagestyle{fancy}
\renewcommand{\chaptermark}[1]{%
\markboth{\thechapter.\ #1}{}}
%\newcommand{\TheAuthor}{Ignacio Vallés Oriola} % As given in documentation of **fancyhdr**
%\newcommand{\Author}[1]{\renewcommand{\TheAuthor}{#1}}
\fancyhead{} % clear all fields
%\lhead{\leftmark}
%\rhead{Ignacio Vallés Oriola}
\fancyhead[LO]{\MakeUppercase{\leftmark}}
\fancyhead[RE]{\nouppercase{\rightmark}}
\fancyhead[LE, RO]{\footnotesize{\textcolor{gris}{Ignacio Vallés Oriola}}}

\usepackage{soul} %tachado, usando \textst{lo que se desea tachar} No vale en ecuaciones.
\usepackage[makeroom]{cancel} % tachar en equaciones \cancel{a tachar}; \bcancel tacha al revés ; \xcancel tacha con x; \cancelto {0 (o \infty}{expresion a tachar} flecha con 0 o infty

% blocks
% \PassOptionsToPackage{svgnames}{xcolor}   ---- en primera línea
\usepackage{tcolorbox}
\tcbuselibrary{skins,breakable}
\usetikzlibrary{shadings,shadows}
    
\newenvironment{myexampleblock}[1]{% Verde
    \tcolorbox[beamer,%
    noparskip,breakable,
    colback=Honeydew,colframe=DarkGreen,%
    colbacklower=LimeGreen!75!Honeydew,%
    title={#1}]}%
    {\endtcolorbox}
 
\newenvironment{myalertblock}[1]{% Rojo
    \tcolorbox[beamer,%
    noparskip,breakable,
    colback=LavenderBlush,colframe=DarkRed,%
    colbacklower=Tomato!75!LavenderBlush,%
    title={#1}]}%
    {\endtcolorbox}

\newenvironment{myblock}[1]{% Azul
    \tcolorbox[beamer,%
    noparskip,breakable,
    colback=AliceBlue,colframe=DarkBlue,%
    colbacklower=DarkBlue!75!AliceBlue,%
    title={#1}]}%
    {\endtcolorbox}
    
%párrafo gris claro (sin sangrado) MIPARRAFO
   \usepackage{mdframed}
   \usepackage{xcolor}
   \newenvironment{miparrafo}
      {\begin{mdframed}[
        backgroundcolor=AliceBlue,
        linecolor=gray]}
     % ]\quotation}
   {\endquotation\end{mdframed}}
   
 %párrafo destacado - amarillo claro (sin sangrado) MIPARRAFODESTACADO
   \usepackage{mdframed}
   \usepackage{xcolor}
   \newenvironment{miparrafodestacado}
      {\begin{mdframed}[
        backgroundcolor=LightYellow,
        linecolor=LightYellow]}
     % ]\quotation}
   {\endquotation\end{mdframed}}
   

\renewcommand{\baselinestretch}{1.2} %interlineado
\setlength{\parskip}{2mm} %espacio entre párrafos

\title{Física General.} 
%	\\
% -  \\ 
%\large{(ampliado)}}

\author{Ignacio Vallés Oriola.}
\date{}

\setcounter{tocdepth}{2} %hasta segundo nivel secc


%\rotatebox{180}{\leftline{\textcolor{gris}{texto a escribir }}}
% $\divideontimes$ ejercicio difícil
%\rotatebox{180}{\leftline{\textcolor{gris}{\scriptsize{Sol: $\frac 1 {x^2}$ no está definida en $x0=\in [-3,2]$}}}}. 
%!!!!!!!Texto escrito de dcha a izqda en cualquier parte del texto.
%*************. \boldsymbol{$fórmula en negrita$}
% ********  numero tachado horizontalmente --  $\text{\textst{5}}$

\setlength{\parindent}{0 mm} %elimina sangrado primera línea - recuperarla--> 5 mmm

\spanishdecimal{.}

% Para resúmenes iniciales
\newenvironment{resumen}
{
\begin{center}
%\textbf{Resumen}.  Se puede añadir si se desea que aparezca este título
\end{center}
\begin{quote}\itshape
}
{
\end{quote}
}

%ABSTRACT**************
\usepackage{lipsum}
%*********************************************************
%***********PARRAFO/S SANGRADOS/S*************************
%*********************************************************
%   Reduce the margin of the summary:   
\def\changemargin#1#2{\list{}{\rightmargin#2\leftmargin#1}\item[]}
\let\endchangemargin=\endlist 
% Generate the environment for the abstract:
\newcommand\summaryname{Abstract}
\newenvironment{Abstract}%
    {\small\begin{center}%
    \bfseries{\summaryname} \end{center}}

% BASTA CON PONER:  \begin{changemargin}{1cm}{1cm}
                    % Párrafo a sangrar
                    %\end{changemargin}    
%*********************************************************


% CITAS alineado derecha con pie autor - con \begin{\cita}{Autor} cita-txt \end{cita}
\newenvironment{cita}[1]
{
\def\autorc {#1}
\begin{flushright}
\itshape
}
{
\par
\medskip
\autorc
\end{flushright}
}



%includeonly{TEMA26_chapter-A4, TEMA27_chapter-A4, TEMA28_chapter-A4} %para ver solo esto, o VARIOS separado por comas

%********************************************
%******** CUERPO DEL DOCUMENTO **************
%********************************************

\begin{document}

\begin{titlepage}
	\centering
	\vspace*{\fill}
	{\scshape\LARGE FÍSICA GENERAL\par}
	\vspace{1cm}
	{\Large Ignacio Vallés Oriola \par}
	\vspace{3cm}
	\includegraphics[width=0.75\textwidth]{imagenes/hic-svnt-dracones}
	\vspace{3cm}
\end{titlepage}

\tableofcontents

%\chapter{Intro}
\section{Intoducción}



\centering{
\fcolorbox{black}{fondoblau}{
\parbox{0.95\textwidth}{
	\textit{Este material es un conjunto de apuntes personales que comparto gratuitamente en la red. Se agradecería la comunicación de la detección de cualquier error.}
}}}
\justify

\begin{resumen}{Agradecimientos}

Me he basado, para su confección, en mis viejos apuntes de Física General de la facultad de física de la UV. He consultado y tomado material de los libros clásicos de física general: Alonso y Finn (parece ser que mi profesor seguía este libros, en ocasiones, al pie de la letra), Tipler y Mosca, Sears y Zemanski, Serway y Jewett; así como de mucho material que he encontrado en internet. 

Este texto no tiene más pretensiones que las citadas, una puesta en \LaTeX de mis antiguos apuntes que me sirva para desenpolvar mis conocimientos. Ni la totalidad del contenido ni su organización son del todo de mi gusto. Quizás, en un futuro .. ?`Quién sabe?, de todos modos me decido a compartirlo en la web por si a alguien le resulta de utilidad, aunque dejadme dudarlo. 
\end{resumen}





\vspace{5mm}
\emph{Este documento se comparte bajo licencia `Attribution-NonCommercial 4.0 International (CC BY-NC 4.0)'}
\vspace{5mm}

\begin{multicols}{2}
\begin{figure}[H]
	\centering
	\includegraphics[width=.4
	\textwidth]{imagenes/imagenes00/licencia.png}
\end{figure}
\begin{figure}[H]
	\centering
	\includegraphics[width=.3
	\textwidth]{imagenes/firma.png}
\end{figure}
\end{multicols}

\newpage %****************************************

\begin{myalertblock}{Para aprender física}
La resolución de problemas en física es primordial, aprender a resolver problemas es absolutamente indispensable; es imposible saber física sin poder hacer física.

\vspace{2mm}A la hora de resolver problemas de física no se debe nunca olvidar que las ecuaciones que obtengamos deben ser dimensionalmente correctas. Un análisis de dimensiones siempre son ayudará a decidir si lo que hemos hecho está mal.

\vspace{2mm}La meta de la resolución de problemas en física no es sólo obtener un número o una fórmula; es entender mejor. Ello implica examinar la respuesta para ver qué nos dice. En particular,  deberíamos preguntarnos: ¿Es lógica esta respuesta? ¿Es posible este resultado?

\vspace{2mm}Y lo más importante, al acabar el problema se deben analizar los casos extremos, ?`qué ocurriría si ...?
\end{myalertblock}









\include{TEMA01_chapter-A4}
\part{I Mecánica clásica}


\begin{myalertblock}{Mecánica cásica}

La Mecánica es la parte de la física que tiene por objeto el estudio de los movimientos de los cuerpos materiales.

Sus partes principales son

\begin{itemize}
\item \emph{Cinemática}: Parte de la Mecánica que estudia el movimiento de los cuerpos materiales sin atender al por qué lo hacen.
\item \emph{Dinámica}: Parte de la Mecánica	 que estudia bajo qué condiciones se mueven los cuerpos materiales.
\item \emph{Estática}:  Parte de la Mecánica que estudia las condiciones de equilibrio de los cuerpos materiales (respecto de un determinado observador).
\end{itemize}

\end{myalertblock}
 
\chapter{Cinemática de la partícula}


\setlength{\parindent}{0cm}


\section{Sistema de referencia}

\begin{multicols}{2}
Supongamos dos observadores distintos, uno situado en la Tierra y el otro en el Sol, observando el movimiento de la Luna.

Ambos describen el movimiento de la Luna de dos modos completamente distintos. ¿Cuál de los dos tienen razón?

Ambos, evidentemente, la descripción del movimiento depende de donde de observe éste. Es necesario un \colorbox{LightYellow}{\emph{Sistema de Referencia}}, es decir,  \colorbox{LightYellow}{un origen} (un punto),  \colorbox{LightYellow}{unos ejes} (base de vectores ortonormal)  \colorbox{LightYellow}{y un reloj} (origen de tiempos) sobre el cual referir el movimiento del cuerpo observado.
\begin{figure}[H]
		\centering
		\includegraphics[width=.5\textwidth]{imagenes/imagenes02/T02IM01.png}
	\end{figure}
\end{multicols}

Un sistema de referencia es un conjunto de convenciones usado por un observador para poder medir la posición y otras magnitudes de un sistema físico. Las trayectorias medidas y el valor numérico de muchas magnitudes son relativas al sistema de referencia que se considere, por esa razón, se dice que el movimiento es relativo.

\begin{multicols}{2}
En mecánica clásica un sistema de referencia consta de un punto espacial fijo llamado origen $\mathcal O$, una base de vectores ortonormal orientada $\vec i,\; \vec j,\; \vec k$ \footnote{En el Apéndice \ref{Vectores} se recuerda el concepto de `Base OrtoNormal'.}  y un reloj para el origen de tiempos (dados dos sistemas de coordenadas de ese tipo, existe un giro y una traslación que relacionan las medidas de esos dos sistemas de coordenadas).
\begin{figure}[H]
		\centering
		\includegraphics[width=.35\textwidth]{imagenes/imagenes02/T02IM02.png}
	\end{figure}
\end{multicols}

Cuando $\Delta t \to 0$, $\Delta \vec r$ es tangente a la trayectoria y $\abs{\Delta \vec r} \approx \Delta s$.

Por definición, la velocidad es:

\begin{equation}
\label{velocidad}
\subrayado{
 \vec v= \lim_{\Delta t\rightarrow 0 } \dfrac {\Delta \vec r}{\Delta t}	= \dv{\vec r}{t}
 }
\end{equation}


$\displaystyle \vec v= \lim_{\Delta t\rightarrow 0} \dfrac{\Delta \vec r}{\Delta s}\cdot \dfrac {\Delta s}{\Delta t} =\lim_{\Delta s\rightarrow 0} \dfrac{\Delta \vec r}{\Delta s}\cdot \lim_{\Delta t\rightarrow 0} \dfrac {\Delta s}{\Delta t}= \vec u_T \cdot v$

$\displaystyle \vec u_T=\lim_{t\rightarrow 0} \dfrac{\Delta \vec r}{\Delta s}$: dirección de la recta tangente a la trayectoria y módulo unidad.

$\displaystyle v=\lim_{t\rightarrow 0} \dfrac {\Delta s}{\Delta t}$: un escalar (número real).

\begin{equation}
\vec v=v\cdot \vec u_T
\end{equation}

El \colorbox{LightYellow}{vector velocidad} es, en todo momento, \colorbox{LightYellow}{tangente a la trayectoria} siendo $v$ su módulo (celeridad)\footnote{Celeridad = módulo del vector velocidad.}:
$\quad v=\abs{\vec v}=\sqrt{v_x^2+v_y^2+v_z^2}$

En el SI \emph{(Sistema Internacional de Unidades)}, la velocidad se mide en $\mathrm{m/s} = \mathrm{m\cdot s}^{-1}$

\section{Componentes cartesianas del vector velocidad}
Puesto que el vector de posición de la partícula que se mueve, $\vec r$, se puede expresar en coordenadas cartesinas en nuestro sistema de referencia como: $\;\;\vec r=x\cdot \vec i+y\cdot \vec j+z\cdot \vec k$, para el vector velocidad tendremos:

$\displaystyle \boldsymbol{\vec v}=\dv{\vec r}{t}= \dv{x}{t}\cdot \vec i+\dv{y}{t} \cdot \vec j+\dv{z}{t} \cdot \vec k  \boldsymbol{=v_x\cdot \vec i+v_y\cdot \vec j+v_z\cdot \vec k}$


\section{Movimiento de una partícula en un plano. Componentes polares de la velocidad} \label{velocidad-polares}

\begin{multicols}{2}
$\displaystyle \vec v= \dv{t} \vec 
r = \dv{t} (r\cdot \vec u_r)= $

$\displaystyle=\dv{r}{t} \cdot \vec u_r + r \cdot \dv{\vec u_r}{t}$

$\displaystyle  \dv{\vec u_r}{t}  \; \parallel \;\vec u_\theta$

Con $\quad \vec u_\theta \; \bot \; \vec u_r \quad (*)$ 
\begin{figure}[H]
		\centering
		\includegraphics[width=.3\textwidth]{imagenes/imagenes02/T02IM03.png}
		\end{figure}
\end{multicols}

\begin{footnotesize}
\textcolor{gris}{$\displaystyle (*) \quad \dv{t} (\vec u_r \cdot \vec u_r)=\dv{t} 1=0$}

\textcolor{gris}{Por otra parte: $\displaystyle \dv{t} (\vec u_r \cdot \vec u_r)=\dv{\vec u_r}{t}\cdot \vec u_r+\vec u_r \cdot \dv{\vec u_r}{t}=2\vec u_r \dv{\vec u_r}{t}$}

\textcolor{gris}{Luego $\displaystyle \vec u_r \dv{\vec u_r}{t}=0 \Rightarrow \vec u_r \;\bot\; \dv{\vec u_r}{t}\;\ \parallel ;\vec u_\theta \qquad \Box$}

\textcolor{gris}{Hemos usado que $\vec u \cdot \vec v=\abs{\vec u} \;\abs{\vec v}\; \cos (\varphi_{\vec u, \vec v})$, por lo que $\vec u \cdot \vec v=0 \leftrightarrow \vec u \bot \vec v$ y que $\abs{\vec u_r}=\abs{\vec u_\theta}=1$. También que $\vec r=r\; \vec u_r$}
\end{footnotesize}

Transportando estos vectores al origen: \textcolor{gris}{$\alpha=90^o-\theta$}
\begin{multicols}{2}
$\begin{cases} \;\;\vec u_r=\;\;\cos \theta\; \vec i + \sin \theta\; \vec j \\ \\ \;\;\vec u_\theta = -\sin \theta \; \vec i + \cos \theta \;\vec j \end{cases}$
\begin{figure}[H]
		\centering
		\includegraphics[width=.25\textwidth]{imagenes/imagenes02/T02IM04.png}
		\end{figure}
\end{multicols}

Derivando: $\quad \displaystyle \dv{\vec u_r}{t}=\dv{t} (\cos \theta\; \vec i + \sin \theta\; \vec j)=-\sin \theta \dv{\theta}{t} \; \vec i + \sin \theta \dv{\theta}{t} \; \vec j=(-sin \theta \;\vec i+ \cos \theta \; \vec j)\; \dv{\theta}{t} \quad \Rightarrow \qquad 	\boldsymbol{\dv{\vec u_r}{t}=\vec u_\theta \cdot \dv{\theta}{t}}\quad$ que es la relación matemática que buscábamos.

\begin{equation}
\label{veloc-polares}
\subrayado{
\vec v=\vec u_r \cdot \dv{r}{t} + \vec u_\theta \cdot r\cdot \dv{\theta}{t} 
}
\end{equation}

El primer de término es la `velocidad radial' y el segundo la `velocidad transversal'. Conclusión: \emph{La velocidad radial es perpendicular a la velocidad transversal}.

\section{Movimiento rectilíneo}

Movimiento rectilíneo es el que se realiza a lo largo de una línea recta. Llamamos eje $OX$ a la recta que describe esta trayectoria y fijamos un punto $\mathcal O$ como origen. La posición del objeto al cabo de de un tiempo $T$ viene dada por el escalar, en este caso, $r(t)$ que mide la distancia del móvil al origen $\mathcal O$. En estas condiciones:

$\displaystyle \vec v= \dv{\vec r}{t} \to \int_{r_0}^{r} \dd r=\int_{t_0}^{t}v\; \dd t \quad \Rightarrow \qquad  \quad r-r_0=\int_{t_0}^{t}v\; \dd t $

En el caso particular de que $ \; v=cte	\;$, independiente del tiempo, la expresión anterior queda como: $ \;\; r-r_0=v\; ( t-t_0 ) $

Consideremos ahora que el observador no está situado sobre la recta de la trayectoria o eje de movimiento:


\begin{multicols}{2}
$\displaystyle \vec v=\dv {\vec r}{t}\to \int_{\vec r_0}^{\vec r} \dd r = \int_{t_0}^{t} \vec v \ \dd t$ 

de donde: $\vec r-\vec r_0=\displaystyle \int_{t_0}^{t} \vec v \ \dd t$

\begin{figure}[H]
		\centering
		\includegraphics[width=.25\textwidth]{imagenes/imagenes02/T02IM05.png}
		\end{figure}
\end{multicols}
En el caso particular de que $\boldsymbol{ \;\vec v= \overrightarrow{cte}\; }$, se tiene:

\begin{equation}
\label{MRU}
	 \vec r-\vec r_0=\vec v\; (t-t_0) 
\end{equation}

que es la ecuación del \textbf{MRU} (movimiento rectilíneo uniforme, $\vec v = \overrightarrow{cte}$) en tres dimensiones.

\vspace{-4mm}\section{Movimiento circular. Velocidad angular.}

Llamamos movimiento circular a aquel que describe una circunferencia, por tanto, tiene lugar en un plano. Usaremos coordenadas polares que son más útiles en este caso.

Al ser la trayectoria una circunferencia, $R=\abs{\vec R}=cte$, por lo que:

\begin{multicols}{2}

$\displaystyle \vec v=\dv{\vec R}{t}=\dv{(R\;\vec u_R)}{t}=$

$\displaystyle =\cancelto{0}{\dv{R}{t}}\;\vec u_R+R\; \dv{\vec u_R}{t}$
\begin{figure}[H]
		\centering
		\includegraphics[width=.2\textwidth]{imagenes/imagenes02/T02IM06.png}
		\end{figure}
\end{multicols}

Recordando la ecuación \ref{veloc-polares}:
$\qquad \displaystyle \vec v=R\cdot \dv{\theta}{t}\;\vec u_\theta=R\cdot \omega \; \vec u_\theta$,

Siendo $\displaystyle \omega=\dv{\theta}{t}$ el módulo de la velocidad angular (ángulo barrido en la unidad de tiempo), que en el $SI\;$\footnote{ SI= Sistema Internacional de unidades.} se mide en $\mathrm{s}^{-1}$. Es usual expresarla en $\text{rad}\cdot \mathrm{s}^{-1}$ o $\text{rpm}$.\footnote{ rpm=revoluciones por minuto.}

Supongamos ahora que el observador no se encuentra situado en el centro de la circunferencia que describe la partícula que se mueve sino en una recta perpendicular a ella que pasa por su centro.

\begin{multicols}{2}
Respecto de $\mathcal O'$

$\displaystyle \vec v = \textcolor{gris}{\vec v_{\mathcal O'}}=\dv{\overrightarrow{ r }}{t} = \dv{t}(\overrightarrow{\mathcal O \mathcal O'}+\vec R )=$

$\displaystyle =\dv {\vec R} {t}=R\cdot \omega\; \vec u_\theta\; ; \qquad \textcolor{gris}{\overrightarrow{\mathcal O \mathcal O'}=\overrightarrow{cte}}$

\begin{figure}[H]
		\centering
		\includegraphics[width=.20\textwidth]{imagenes/imagenes02/T02IM07.png}
		\end{figure}
\end{multicols}

\begin{multicols}{2}
Sea $\mathcal O'$ un punto cualquiera de la recta perpendicular a la trayectoria que pasa por el centro, $(\mathcal O)$, de la figura:
$\quad R=r\cdot \sin \alpha$

Luego $\quad \vec v= \omega \cdot r \cdot \sin \alpha \;\; \vec u_\theta$

Recordando la definición de `producto vectorial' de dos vectores (y la regla del sacacorchos para determinar su sentido),

\begin{figure}[H]
		\centering
		\includegraphics[width=.25\textwidth]{imagenes/imagenes02/T02IM08.png}
		\end{figure}
\end{multicols}

\begin{equation}
\label{v-angular}
\subrayado{
\vec v=\vec \omega \times \vec r
}
\end{equation}
donde $\vec \omega$ es el vector velocidad angular.

El vector velocidad angular es perpendicular al plano del movimiento en es sentido de avance de un sacacorchos que gira en el mismo sentido en que lo hace la partícula, en módulo se tiene que $ v=\omega\; r\; \sin \alpha$. 

\begin{multicols}{2}
$\quad$

\emph{Nótese que esto solo es válido para movimientos circulares con el observador situado perpendicularmente al centro de la treyectoria, es decir, siempre que $R=cte \; \wedge \; alpha=cte$.}  
\begin{figure}[H]
		\centering
		\includegraphics[width=.4\textwidth]{imagenes/imagenes02/T02IM09.png}
		\end{figure}
\end{multicols}

Como $\displaystyle \omega=\dv{\theta}{t}\to  \int_{\theta_0}^{\theta}\dd \theta = \int_{t_0}^t \omega \;\dd t \Rightarrow \;\; \;\theta-\theta_0=\int_{t_0}^t \omega \;\dd t\;$

En el caso particular en que $\; \theta = cte$, tendremos: $\;\theta-\theta_0= \omega \;(t-t_0)\;$

Un movimiento es \textbf{periódico} si al pasar un determinado tiempo $\boldsymbol{T}$ llamado \textit{\textbf{periodo}},  todas las variables que describen el movimiento vuelven a tomar el mismo valor. \textit{El movimiento circular uniforme, en que $\omega=cte$, es periódico.}

\textbf{Relación entre $\boldsymbol{T}$ y $\boldsymbol{\omega}$}


\begin{multicols}{2}
Empezamos a contar el tiempo (\emph{origen de tiempos}) cuando la partícula está en el punto A en que el ángulo, medido desde el eje polar, es $\theta_0$. Las variables que describan el movimiento volverán a tomar el mismo valor al cabo de una vuelta:

\begin{figure}[H]
		\centering
		\includegraphics[width=.45\textwidth]{imagenes/imagenes02/T02IM10.png}
		\end{figure}
\end{multicols}

$\theta=\theta_0+\omega \; T \to $, al cabo de una vuelta $\theta=\cancel{\theta_0}+2\pi=\cancel{\theta_0}+\omega \; T \to 2\pi=\omega \; T$

de donde:

\begin{equation}
\label{periodo-vel-angular}
T = \dfrac {2\pi}{\omega} \qquad \vee \qquad \subrayado{\omega=\dfrac{2\pi}{T}}	
\end{equation}

$T$ en $s$. Se llama \textbf{\emph{frecuencia, $\nu$}} a la inversa del periodo y se expresa en $s^{-1}\;(=Hz=rps)$\footnote{rps: revoluciones por segundo}.

\begin{equation}
T=\dfrac 1 \nu	 \qquad \qquad \omega=2\pi\; \nu
\end{equation}

La velocidad angular se usa para movimientos angulares (para movimientos aproximadamente circulares se usa la velocidad angular media).

\section{Aceleración}

La \emph{aceleración} es una magnitud vectorial íntimamente ligada con la \emph{fuerza}. Por definición, la aceleración es: 

\begin{multicols}{2}

\begin{figure}[H]
		\centering
		\includegraphics[width=.6\textwidth]{imagenes/imagenes02/T02IM11.png}
		\end{figure}
\vspace{3mm}
\begin{equation}
\label{aceleracion}
\subrayado{
\vec a= \lim_{\Delta t\rightarrow 0 } \dfrac{\Delta \vec v}{\Delta t}=\dv{\vec v}{t}	
}
\end{equation}
\end{multicols}

\begin{multicols}{2}
Características de $\vec a$:

$\quad$

Mira siempre hacia el interior de la concavidad de la trayectoria.
\begin{figure}[H]
		\centering
		\includegraphics[width=.3\textwidth]{imagenes/imagenes02/T02IM12.png}
		\end{figure}
\end{multicols}

\begin{figure}[H]
		\centering
		\includegraphics[width=1\textwidth]{imagenes/imagenes02/T02IM15.png}
		\end{figure}

Resolvamos el problema de determinar $\vec v$ y $\vec r$ conocido el vector $\vec a$ y las condiciones iniciales $\vec v_0$ y $\vec r_0$:

De la definición de aceleración, ecuación \ref{aceleracion}: $\;\;\dd \vec v= \vec a\; \dd t\;$ . Integrando entre $t_o$ y $t$, intervalo en que las velocidades son $\vec v_0$ y $\vec v$ respectivamente:

$\displaystyle \int_{\vec v_0}^{\vec v}\dd \vec v= \int_{t_0}^{t} \vec a\; \dd t \quad \to \qquad \vec v=\vec v_0+\int_{t_0}^{t} \vec a\; \dd t$

\subsection{Movimiento rectilíneo uniformemente acelerado}

Particularizemos para el caso en que $\boldsymbol{ \vec a=\overrightarrow{cte} }$, \textbf{MRUA} \emph{(Movimiento rectilíneo uniformemente acelerado --$\vec a=\overrightarrow{cte}$--):}

En estas condiciones, $\vec a=\overrightarrow{cte}$, se tiene que:

\begin{equation}
\label{MRUAv}
	\vec v=\vec v_0+\vec a\; (t-t_0)
\end{equation}

$\displaystyle \vec v=\dv{\vec r}{t} \to \dd \vec r=\vec v\dd t : \quad \int_{\vec r_0}^{\vec r}\dd \vec r=\int_{t_0}^t \vec v\;\dd t \to$, teniendo en cuenta \ref{MRUAv}:

$\displaystyle \vec r-\vec r_0=\int_{t_0}^t 
\qty(\; \vec v_0+\vec a\; (t-t_0) \;)
\; \dd t;\; \qquad \vec a=\overrightarrow{cte} \; \wedge \; \vec v_0=\overrightarrow{cte'} \Rightarrow\;:$

\begin{equation}
\label{MRUAr}
	\vec r=\vec r_0+ \vec v_0\; (t-t_0)+\dfrac 1 2 \vec a\; (t-t_0)^2
\end{equation}

Las ecuaciones del MURA en una dimensión (observador situado en la recta de la trayectoria) las ecuaciones son:
$$v=v_0+a\;(t-t_o) \qquad \wedge \qquad r=r_0+v_0\;(t-t_0)+\dfrac 1 2 \; a \; (t-t_0)^2$$

\emph{Cuando la aceleración es constante, la trayectoria es una parábola} ($r \propto t^2$).


\subsection{Componentes cartesianas de la aceleración}

$\displaystyle \vec a=\dv{\vec v}{t}=\textcolor{gris}{
\dv{t}\left( \dv{\vec r}{t}  \right)
}=\dv[2]{\vec r}{t}$

$\displaystyle 
\vec a=\dv{v_x}{t}\;\vec i+\dv{v_y}{t}\;\vec j+\dv{v_z}{t}\;\vec k
\qquad \qquad
\vec a= \dv[2]{x}{t}\;\vec i+\dv[2]{y}{t}\;\vec j+\dv[2]{z}{t}\;\vec k$

$\displaystyle \vec a=a_x\;\vec i+a_y\;\vec j+a_z\;\vec k \quad \qquad \to \qquad \abs{\vec a}=\sqrt{a_x^2+a_y^2+a_z^2}$

 \subsection{Movimiento en un plano.  Aceleración en coordenadas polares}

$\displaystyle \vec a=\dv{\vec v}{t}\;,\qquad$
por \ref{veloc-polares}: $\quad \vec v=\vec u_r \cdot \dv{r}{t} + \vec u_\theta \cdot r\cdot \dv{\theta}{t} $

$\begin{cases} \;\;\vec u_r=\;\;\cos \theta\; \vec i + \sin \theta\; \vec j \\  \;\;\vec u_\theta = -\sin \theta \; \vec i + \cos \theta \;\vec j \end{cases}  \hspace{-3mm} \to \;\;
\displaystyle \dv{\vec u_r}{t}=\vec u_\theta\; \dv{\theta}{t}\;;  \qquad 
\displaystyle \dv{\vec u_\theta}{t}=-\vec u_r \dv{\theta}{t}\; (*)$




$\displaystyle \vec a =\dv{\vec u_r}{t}\; \dv{r}{t} 
+ \vec u_r\; \dv[2]{r}{t} 
+ \dv{\vec u_\theta}{t}\;r\;\dv{\theta}{t}
+\vec u_\theta\;\dv{r}{t}\;\dv{\theta}{t}
+\vec u_\theta\;r\;\dv[2]{\theta}{t}= \quad (\overleftarrow{*}) \; \Rightarrow$

\begin{equation}
\label{aceleracionpolares}	
\displaystyle \vec a\;=\;\vec u_r\;\left[ \dv[2]{r}{t}-r\left(\dv{\theta}{t} \right)^2 \right] + \vec u_\theta \; \left[ 2\;\dv{r}{t}\;\dv{\theta}{t}+r\;\dv[2]{\theta}{t} \right]
\end{equation}

Estas son las componentes polares del vector aceleración y son muy útiles para trabajar en el plano. El primer término es la `componente radial' de la aceleración (responde de la aceleración de la aceleración). El segundo término es la `componente transversal' de la aceleración. Ambos términos son perpendiculares ($\;\vec u_r\;\bot\;\vec u_\theta$). 

\subsection{Componentes intrínsecas de la aceleración}

Supongamos que una partícula $P$ describe la trayectoria que aparece el la siguiente figura. La dirección normal es perpendicular a la tangente y en estas direcciones representamos las componentes normal y tangencial del vector aceleración. A estas componentes, $\vec a_T$ y $\vec a_N$ se les llama \emph{componentes intrínsecas de la aceleración.}

\begin{multicols}{2}
$\displaystyle \vec a=\dv{\vec v}{t}=\dv{t}(v\cdot \vec u_T)=$

$\displaystyle =\vec u_T\; \dv{v}{t}+v\;\dv{\vec u_T}{t}$

$\begin{cases} \;\;\vec u_T=\;\;\cos \theta\; \vec i + \sin \theta\; \vec j \\  \;\;\vec u_N = -\sin \theta \; \vec i + \cos \theta \;\vec j \end{cases} \cdots \to $

$\displaystyle \dv{\vec u_T}{t}=\vec u_N\; \dv{\theta}{t}\;;  \;\; 
\displaystyle \dv{\vec u_N}{t}=-\vec u_T \dv{\theta}{t}$

\begin{figure}[H]
		\centering
		\includegraphics[width=.4\textwidth]{imagenes/imagenes02/T02IM13.png}
		\end{figure}
\end{multicols}


Al pasar de $\;t\;$ a $\;t+dt\;$, el ángulo polar cambia $\;\dd \theta=\theta(t+dt)-\theta(t)$.

Por otro lado, de la relación \emph{arco=ángulo por radio} (cuando el ángulo se expresa en radianes) tenemos:

$\dd s=\dd \theta \cdot \rho\;$ Donde $\rho$ es el llamado \emph{radio de curvatura}.

Como $\; \displaystyle \dv{\theta}{t}=\dv{\theta}{s}\cdot \dv{s}{t}=\dfrac 1 \rho \; v$ 

\begin{figure}[H]
		\centering
		\includegraphics[width=.5\textwidth]{imagenes/imagenes02/T02IM14.png}
		\end{figure}


Se tiene que:

\begin{equation}
\subrayado{
\label{intrinsecas}
\vec a\;=\;\vec u_T\; \dv{v}{t} \;+\; \vec u_N	\; \dfrac {v^2}{\rho}
}
\end{equation}

\begin{miparrafo}

\noindent \textbf{---} La aceleración tangencial, $a_T$, mide el cambio del módulo del vector velocidad (si la partícula viaja más deprisa o más despacio) y es tangente, en todo momento, a la trayectoria.

\vspace{3mm}
\noindent \textbf{---}  La aceleración normal, $a_N$, mide el cambio de la dirección del vector velocidad y es perpendicular (normal) a la trayectoria en todo momento.

\end{miparrafo}

En coordenadas polares, el módulo del vector aceleración se mide por:
$$\displaystyle \abs{\vec a}=\sqrt{\left(\dv{v}{t} \right)^2+\dfrac {v^4}{\rho^2}}$$

-- Para MR, aquellos en los que que no cambia la dirección del vector velocidad, $a_n=0 \to \; \rho = \infty $ (recta).

-- Pra MRU, $\vec v=\overrightarrow{cte}$, tenemos $\abs{\vec v}=cte \to \displaystyle \dv{v}{t}=0 \to a_T=0$

OJO: $\displaystyle \dv{v}{t}= \dv{\abs{\vec v}}{t} \; \neq \; \abs{\dv{\vec v}{t}}$.

\begin{figure}[H]
		\centering
		\includegraphics[width=.8\textwidth]{imagenes/imagenes02/T02IM39.png}
		\end{figure}
\vspace{-10mm} %************
\rule{150pt}{0.4pt}

\begin{multicols}{2}
ATENCIÓN: 

No confundir las componentes polares de la aceleración, en las direcciones $\vec u_r$ y $\vec u_\theta$ con las componentes intrínsecas, en las direcciones tangencial $\vec u_T$ y normal $\vec u_n$.
\begin{figure}[H]
		\centering
		\includegraphics[width=.4\textwidth]{imagenes/imagenes02/T02IM35.png}
		\end{figure}
\end{multicols}
\vspace{-10mm} %**************************************
\rule{150pt}{0.4pt}

\textbf{Los movimientos se pueden clasificar según las componentes intrínsecas de su aceleración. }
\vspace{-2mm}\begin{enumerate}
\vspace{-2mm}\item $a_T=0$ 
	\vspace{-2mm}\begin{enumerate}
	\vspace{-2mm}\item $a_N=0$. Movimiento rectilíneo a velocidad constante. 
	\vspace{-2mm}\item $a_N=cte$. Movimiento circular uniforme. 
	\vspace{-2mm}\item $a_N\neq cte$. Movimiento circular acelerado. 
	\vspace{-2mm}\end{enumerate}
\vspace{-2mm}\item $a_N=0$ 
	\vspace{-2mm}\begin{enumerate}
	\vspace{-2mm}\item $a_T=0$.   Movimiento rectilíneo a velocidad constante. 
	\vspace{-2mm}\item $a_T=cte$. Movimiento rectilíneo uniformemente acelerado. 
	\vspace{-2mm}\item $a_T\neq cte$. Movimiento rectilíneo acelerado. 
	\end{enumerate}
\vspace{-2mm}\item  $a_N=0;\; a_T=0$.  Movimiento curvilíneo.
\end{enumerate}

\vspace{10mm} %***********************************
\begin{figure}[H]
		\centering
		\includegraphics[width=.9\textwidth]{imagenes/imagenes02/T02IM34.png}
		\end{figure}

\newpage %**********************
	\begin{figure}[H]
		\centering
		\includegraphics[width=.95\textwidth]{imagenes/imagenes02/T02IM22.png}
		\end{figure}		
	\begin{figure}[H]
		\centering
		\includegraphics[width=.95\textwidth]{imagenes/imagenes02/T02IM23.png}
		\end{figure}
	\begin{figure}[H]
		\centering
		\includegraphics[width=.6\textwidth]{imagenes/imagenes02/T02IM29.png}
		\end{figure}
	\vspace{-4mm}\begin{figure}[H]
		\centering
		\includegraphics[width=.6\textwidth]{imagenes/imagenes02/T02IM31.png}
		\end{figure}
	\vspace{-5mm}\begin{figure}[H]
		\centering
		\includegraphics[width=.6\textwidth]{imagenes/imagenes02/T02IM32.png}
		\end{figure}


\section{Problemas}

\begin{prob}
Se está usando un vehículo robot para explorar la superficie de Marte. El módulo de descenso es el origen de coordenadas; en tanto que la superficie marciana circundante está en el plano $XY$. El vehículo, que representamos como un punto, tiene coordenadas x y y que varían con el tiempo según: $x=2-0.25t^2$, $y=t+0.025t^3$, estando $t$ en $\mathrm{s}$ y $x$ e $y$ en $\mathrm{m}$ 
\begin{quote}
\begin{enumerate}[a) ]
\item Obtenga las coordenadas del vehículo y su distancia con respecto al módulo en $t=2\;\mathrm{s}$ . 
\item Obtenga los vectores de desplazamiento y velocidad media del vehículo entre $t=2\;\mathrm{s}$ y $t=4\;\mathrm{s}$
\item Deduzca una expresión general para el vector de velocidad instantánea del vehículo. 
\item Exprese la velocidad instantánea en $t=2\; \mathrm{s}$ en forma de componentes y además en términos de magnitud y dirección.  
\item Obtenga el vector aceleración en cualquier instante $t$.
\item Obtenga las componentes de la aceleración en el instantes $t=2\; \mathrm{s}$. 
\item Obtenga las componentes paralela y perpendicular de la aceleración en $t = \;2 \mathrm{s}$. 
\end{enumerate}
\end{quote}	
\end{prob}
 $x=2-0.25t^2\qquad y=t+0.025t^3 \qquad t(\mathrm{s});\;\; x(\mathrm{m}),\; y(\mathrm{m})$
 
 --- a) $t=2\to x=1\;\mathrm{m};\quad y=8.2\; \mathrm{m}\; \quad \vec r=\vec i+8.2\vec j; \quad \abs{\vec r}=8.26\; \mathrm{m}$
 
 --- b) $\displaystyle \overline{\;\vec v\; }=\dfrac{\Delta \vec r}{\Delta t}=\dfrac{\vec r(4)-\vec r(2)}{4-2}$
 
$ \begin{cases}
 \vec r(4)=x(4)\; \vec i+ y(4)\vec j=-2 \vec i +9.6 \vec j
 \\
 \Delta r=(-2,9.6)-(1,8.2)=3\vec i+1.4\vec j 
\end{cases}
\displaystyle \overline{\;\vec v\; }=\dfrac {(3,1,4)}{4-2}=(\;1.5\vec i+0.7\vec j\;)\; \mathrm{ms}^{-1}$

$\abs{\displaystyle \overline{\;\vec v\; }}=\sqrt{1.5^2+0.7^2}=1.66\; \mathrm{ms}^{-1}$

--- c) $\vec v=\displaystyle\dv{\vec r}{t}= \dv{x}{t}\vec i+\dv{y}{t} \vec j= -0.5t\;\vec i + (1+0.075t^2)\;\vec j $

--- d) $t=2s \to \vec v(2)=-\vec i+1.3\; \vec j \to \abs{\vec v(2)}= 1.64\; \mathrm{ms}^{-1};\; \theta_{\vec v(2)}=\arctan \dfrac {v_y} {v_x}=127.6^o$, siendo $\theta_{\vec v(2)}$ el ángulo que forma el vector velocidad con semieje $OX+$ en el instante $t=2\; \mathrm{s}$.

--- e) $\displaystyle \vec a= \dv{\vec v}{t}=-0.5\;\vec i+ 0.15t\; \vec j$

--- f) $\vec a(2)=-0.5\;\vec i+0.3\vec j \quad \to \quad a_x=-0.5 \; ms^{-1}; \; \; a_y(2)=0.3\;\mathrm{ms}^{-2};$

$\displaystyle   \abs{\vec a(2)}= 0.58\; \mathrm{ms}^{-2};\quad \theta_{\vec a_2}=\arctan \dfrac {a_y} {a_x}=149.0^o$


--- g) A partir de los ángulos que forman los vectores $\vec v$ y $\vec a$ con el semieje $OX+$, $\theta_{\vec v}$ y $\theta_{\vec a}$ a los $t=2\; \mathrm{s}$, deducimos que el ángulo que forma el vector aceleración con la recta tangente a la trayectoria (dirección del vector velocidad) es $\theta=\theta_{\vec a(2)}-\theta_{\vec a(1)}=149.0-127.6=21.4^o$, en ese mismo instante, por lo que las componentes tangencial y normal de la aceleración son:




$a_T(2)=\abs{\vec a(2)}\; \cos \theta=0.54\; \mathrm{ms}^{-2}$
$;\quad$
$a_N(2)=\abs{\vec a(2)}\; \sin \theta=0.21\; \mathrm{ms}^{-2}$

\begin{figure}[H]
		\centering
		\includegraphics[width=.35\textwidth]{imagenes/imagenes02/T02IM19.png}
		\end{figure}


\begin{prob}
Una partícula se mueve en línea recta con una aceleración $a=-4v$, siendo $v$ su velocidad. Calcular la distancia que debe recorrer para que su velocidad se reduzca a $8\ \mathrm{ms}^{-1}$ y el tiempo necesario para ello si empezamos a contar cuando la velocidad de la partícula es de $80\ \mathrm{ms}^{-1}$.	
\end{prob}

$\displaystyle a=\dv{v}{t}=-4v \to \int_{v_0}^v \dfrac{\dd v}{v}=-\int_0^t 4 \dd t \to \ln \dfrac {v_0}{v}=4t \Rightarrow $ 

$\displaystyle t=\dfrac 1 4 \ln \dfrac {v_0}{v}=\dfrac 1 4 \ln \dfrac {80}{8}=0.58\ \mathrm{s}$

$\displaystyle a=\dv{v}{t}=\dv{v}{s}\ \dv{s}{t}=\dv{v}{s} \ v=-4v \to \dd v=-4 \ \dd s \to $

$\displaystyle \int_{v_0}^v
 \dd v=-4\int_0^s \dd s \to v-v_0=-4s \Rightarrow s=\dfrac 1 4 (v_0-v)=\dfrac 1 4 (80-8)=18\ \mathrm{m}$

\vspace{10mm} %********************************************
\begin{prob}
La velocidad angular de una rueda viene dada por la ecuación $\omega=2\ \theta^{1/2}$. Calcular el tiempo necesario para dar una vuelta completa así como la aceleración angular en cada instante. ?`Cómo es ese tipo de movimiento?	
\end{prob}
$\omega=\displaystyle \dv{\theta}{t}=2\ \theta^{1/2} \to \dd t=\dfrac{\dd \theta}{2\ \theta^{1/2}}=\dfrac{\dd \theta}{2\sqrt{\theta}}$

El tiempo en dar una vuelta completa, en barrer $2 \pi$ radianes, es el periodo T:

$\displaystyle T=\int_0^T \dd t=\int_0^{2 \pi} \dfrac{\dd \theta}{2\sqrt{\theta}}=\eval{\sqrt{\theta}}_0^{2 \pi}=\sqrt{2 \pi}$

Aceleración angular:

$\displaystyle \alpha= \dv{\omega}{t}=2 \frac 1 2 \theta^{-1/2} \dv{\theta}{t}=\theta^{-1/2} \ \omega =\theta^{-1/2}\ 2 \theta^{1/2}=2=cte.$ 

Como $\alpha=cte$, se trata de un MCUA (movimiento circular uniformemente acelerado).

\vspace{30mm} %********************************************

\begin{prob}
Calcúlese la velocidad angular de la tierra	
\end{prob}

\begin{multicols}{2}
Si utilizamos la relación \ref{periodo-vel-angular} $\;\omega=\dfrac {2\pi}T$ y consideramos $T=24\mathrm{h}=86400 \mathrm{s}$, cometemos un error debido al efecto que tiene la traslación de la tierra, pues al dar una vuelta completa en torno a su eje, el punto $P$ estaría en la posición $P'$ y deseamos que para definir el día vuelva a apuntar al sol, $P''$. Nos hemos pasado un ángulo $\gamma$. Como la traslación es de $365$ días aprox., $\gamma=360/365 =0.986^o\approx 0.017 \text{rad}$
\begin{figure}[H]
		\centering
		\includegraphics[width=.25\textwidth]{imagenes/imagenes02/T02IM18.png}
		\end{figure}
\end{multicols}

El tiempo $t$ aproximado para realizar ese giro por la rotación de la tierra sea corregido a  $t=\theta / \omega= 0.017 / (7.27\times 10^{-5}) \approx 234 \mathrm{s}$, donde hemos tomado como aproximación de $\omega_{rotac.}=2\pi / (24 \text{ días})=7.27\times 10^{-5}\; \mathrm{s}^{-1}$.

Esto hace que el periodo de rotación de la tierra sea $T=24horas - 234 s= 86400-234=86166\; \mathrm{s}$, con lo que:
$$\omega=\displaystyle \dfrac {2\pi}T=7.29\; \mathrm{s}^{-1}$$

\begin{prob}.

	\begin{multicols}{2}
La Tierra rota uniformemente respecto a su eje con una velocidad angular $\;\omega=7.3 \times 10^{-5}\;\mathrm{s}^{-1}\;$ (1 vuelta cada día). Encontrar, en función de la latitud, la velocidad y la aceleración de un punto situado en la superficie terrestre.
\begin{figure}[H]
		\centering
		\includegraphics[width=.25\textwidth]{imagenes/imagenes02/T02IM17.png}
		\end{figure}
\end{multicols}
\end{prob}

Todos los puntos de la tierra se mueven con MRU, movimiento rectilíneo y uniforme ($\vec \omega =\overrightarrow{cte}$), luego $a_T=0$, solo hay $a_N$.

En la figura se observa que un punto A, situado a una latitud $\lambda$, describe una circunferencia paralela al ecuador de radio $r=R\; \cos \lambda$, con $R=\text{radio tierra}=6.4\times 10^6\; \mathrm{m}$.
El vector velocidad es tangente al paralelo $\lambda$, por tanto, paralelo al plano ecuatorial.

De  \ref{v-angular}: $\;\;\;v=\omega \cdot r =\omega\cdot R\cdot \cos \lambda$

De \ref{intrinsecas}: $\;\;a=a_N=\displaystyle \dfrac {v^2}r=\dfrac {\omega^2 \cdot r^2}r=\omega^2\cdot r=\omega\cdot R \cdot \cos \lambda$

Ambos valores máximos se alcanzan en el ecuador, $\lambda=0$:
$v_{max}(\lambda=0)=467 \;\mathrm{ms}^{-1}=1681.2\; \mathrm{kmh}^{-1};$

$ a=3.4\times 10^{-2} \; \mathrm{ms}^{-2} \sim 0.3\%\; g$

\vspace{10mm} %*************************************
\begin{prob}.
	\begin{figure}[H]
		\centering
		\includegraphics[width=.8\textwidth]{imagenes/imagenes02/T02IM25.png}
		\end{figure}
\end{prob}


\begin{prob}
Una esquiadora se mueve sobre una rampa de salto, como se muestra en la figura. La rampa es recta entre A y C, y es curva a partir de C. La rapidez de la esquiadora aumenta al moverse pendiente abajo de A a E, donde su rapidez es máxima, disminuyendo a partir de ahí. Dibuje la dirección del vector de aceleración en los puntos B, D, E y F	
\end{prob}
\begin{figure}[H]
		\centering
		\includegraphics[width=.6\textwidth]{imagenes/imagenes02/T02IM16.png}
		\end{figure}

\vspace{30mm} %*************************************

\begin{prob}.
	\begin{figure}[H]
		\centering
		\includegraphics[width=.9\textwidth]{imagenes/imagenes02/T02IM26.png}
		\end{figure}
\end{prob}

\begin{prob}.
	\begin{figure}[H]
		\centering
		\includegraphics[width=.85\textwidth]{imagenes/imagenes02/T02IM27.png}
		\end{figure}
\end{prob}



\begin{prob} Tiro oblicuo.
\end{prob}	

\vspace{-5mm} \begin{figure}[H]
		\centering
		\includegraphics[width=1.1\textwidth]{imagenes/imagenes02/T02IM20.png}
		\end{figure}
		
\begin{prob} Un cazador dispara a un mono colgado de una rama. Demostrar que si el mono quiere vivir, no debe soltarse de la rama en el momento del disparo.
\end{prob}	
\begin{figure}[H]
		\centering
		\includegraphics[width=1\textwidth]{imagenes/imagenes02/T02IM21.png}
		\end{figure}		





\begin{prob}
Una muchacha que está a $4\;\mathrm{m}$ de una pared vertical lanza contra ella una pelota que sale de su mano que está a $2 \; \mathrm{m}$ del suelo a una velocidad de $10\;\mathrm{ms}^{-1}$ y formando un ángulo de $45^o$ con la horizontal. Cuando la pelota choca contra la pared, se invierte la componente horizontal de la velocidad (rebote) mientras	 que su componente vertical continua sin variar. ¿Dónde caerá la pelota al suelo?

Comprobar que se puede considerar que la pared actúa como un espejo por lo que se puede resolver el problema como si no estuviese, calcular el alcance y, luego, considerar la reflexión especular de este punto respecto de la pared.
\end{prob}
\begin{figure}[H]
		\centering
		\includegraphics[width=.9\textwidth]{imagenes/imagenes02/T02IM24.png}
		\end{figure}
Dividimos el problema en dos: I- Desde el momento del tiro hasta que llega a la pared, y II- desde el rebote en la pared hasta que la pelota llega al suelo.

--- Parte I:  Colocamos el sistema de referencia en los pies de la niña u empezamos a contar el tiempo en el momento del lanzamiento, por tanto: $t_0=0\;\; x_0=0;\; y_0=2;\; v_{0_x}=v_x=v\cos \theta=10\cos 45^o=7.07;\; v_{0_y}=v\sin \theta=10\sin 45^o=7.07;\; a=g=-9.8\vec j\;$ Todo el el SI de unidades.

Cuando la pelota llega a la pared la distancia recorrida en el eje $X$ es de $4\;m$, el tiempo empleado es: $x=x_0+v_x\;t \to 4=0+7.07t \Rightarrow t=0.57\; \mathrm{s}$

Con el paso de este tiempo, calculamos la altura que ha alcanzado la pelota: $y=y_0+v_{0_y}\;t+\frac 1 2 \;a\;t^2 \to y=2+7.07\cdot 0.57-4.9\cdot 0.57^2 \Rightarrow y=4.44\; \mathrm{m}$

Calculemos, finalmente, la velocidad que lleva en su componente $y$ en ese momento: $v_y=v_{0_y}+a\; t \to v_y=7.07-9.8\cdot 0.57=1.48\; \mathrm{ms}^{-1}$

--- Parte II: Dejamos el sistema de referencia espacial a los pies de la niña pero vamos a cambiar, para facilitar el cálculo, el sistema de referencia temporal. Ahora empezamos a contar el tiempo en el momento del rebote en que se invierte el sentido de la componente $x$ del vector velocidad. Tendremos, pues: $t_0=0;\; x_0=4;\; y_0=4.44;\; v_{0_x}=v_x=-7.07;\; v_{0_y}=1.48;\; a=g=-9.8\vec j\;$, como antes, todo en el SI de unidades.

Calculemos el tiempo en que la pelota cae al suelo (desde el rebote). Eso ocurrirá cuando $y=0\to 0=4.44+1.48t-4.9t^2\Rightarrow t=1.11\; \mathrm{s}$.

Calculemos ahora la distancia recorrida durante ese tiempo por la pelota a lo largo del eje $x$: $\;x=4-7.07\cdot 1.11\approx -3.9\;\mathrm{m}$

CONCLUSIÓN: La pelota cae a $3.9\;\mathrm{m}$ a la izquierda de los pies de la niña.

\rule{150pt}{0.4pt} 

Resolvamos el problema suponiendo que la pared es un espejo. Inicialmente actuaremos como si no estuviese y, luego, calcularemos el punto simétrico del de impacto respecto de la pared-espejo.

De nuevo, colocamos el sistema de referencia en los pies de la niña u empezamos a contar el tiempo en el momento del lanzamiento, por tanto: $t_0=0\;\; x_0=0;\; y_0=2;\; v_{0_x}=v_x=v\cos \theta=10\cos 45^o=7.07;\; v_{0_y}=v\sin \theta=10\sin 45^o=7.07;\; a=g=-9.8\vec j\;$ Todo el el SI de unidades.

El tiempo de vuelo del proyectil lo encontraremos exigiendo que $y=0 \to y=y_0+v_{0_y}t+\frac 1 2 a t^2 \to 0=2+7.07t-4.9t^2 \Rightarrow t=1.69\;\mathrm{s}$

Con este tiempo, vamos a calcular el alcance de la pelota, la distancia recorrida en el eje $x$: $\; x=x_0+v_xt\to x=0+7.07\cdot 1.69=11.9\;\mathrm{m}$

Es decir, si no hay pared-espejo, la pelota cea a $11.9\;\mathrm{m}$ de los pies de la niña, es decir, a $11.9-4=7.9\;\mathrm{m}$ de la pared.  El simétrico de $+7.9\;\mathrm{m}$ de la pared es $-7,9\;\mathrm{m}$ de la misma, es decir, a su izquierda. Como la niña está a $4\:\mathrm{m}$ a la izquierda de la pared, la pelota caerá, pues, a $7.9-4=3.9\;\mathrm{m}$ a la izquierda de la niña.

\begin{prob}
Conduces un automóvil por una autopista recta. En el instante $t = 0$, cuando avanzas a $10 \ \mathrm{ms}^{-1}$ en la dirección $+x$, pasas un letrero que está en $x= 50 \ \mathrm{m}$. Tu aceleración es una función del tiempo: $_x = (2.0 - 0.10 t)\ \mathrm{ms}^{-2}$. 

a) Deduce las expresiones para tu velocidad y posición en función del tiempo.  

b) ¿En qué momento es máxima tu velocidad?  

c) ¿Cuál es esa velocidad máxima?  
\end{prob}
$v=10+2t-0.05t^2$ 

$x=50+10t+t^2-0.017t^3$

$a=0$ si $t=20 \to v_{max}=30\ \mathrm{ms}^{-1}$	

\begin{prob}
Sabemos demostrar que, en ausencia de rozamiento con el aire, el alcance máximo de un cuerpo que sigue un movimiento parabólico se consigue para un lanzamiento a $45^o$ sobre la horizontal (siempre que el punto de lanzamiento y el de impacto se encuentren a la misma altura). ¿Es también cierto cuando el suelo está inclinado? 
\end{prob}
\begin{multicols}{2}
Datos: $\vec v_0,\;\; \theta,\;\; \phi$

A encontar $d=d(\vec v_0,\theta,\phi)$

Para $v_0, \ \theta \ { y }\  \phi$ dados $\to d_{max}(\phi)$
\begin{figure}[H]
		\centering
		\includegraphics[width=.5\textwidth]{imagenes/imagenes02/T02IM36.png}
		\end{figure}
\end{multicols}

Sistema de referencia en punto de lanzamiento y empezamos a contar el tiempo en el momento del mismo: $x_0=y_0=t=0$.

Buscamos la ecuación de la trayectoria parabólica que describe el proyectil. Partimos de las ecuaciones de las posiciones:

$\begin{cases} x=v_0 \cos (\theta+\phi) \ t \\ y=v_0 \sin (\theta+\phi) \ t - \frac g 2 t^2 \end{cases}  \to t=\dfrac x{v_0 \cos (\theta+\phi)} $

$\Rightarrow y=\cancel{v_0} \sin (\theta+\phi) \dfrac x{\cancel{v_0} \cos (\theta+\phi)} - \dfrac g 2 \left( \dfrac x{v_0 \cos (\theta+\phi)}  \right)^2$

La ecuación de la trayectoria (parábola) es:

$\textcolor{blue}{y=\tan(\theta+\phi) \ x - \dfrac {g}{2 v_0^2 \cos^2 (\theta+\phi)}\ x^2}$

Por otra parte, la ecuación del plano inclinado es:

$\textcolor{red}{y}=mx\textcolor{red}{=\tan \phi \ x}$

El punto de impacto del \textcolor{blue}{proyectil} contra el \textcolor{red}{suelo} se producirá en la \textbf{intersección} de la \textcolor{blue}{parábola (trayectoria)} con la \textcolor{red}{recta (plano inclinado)}. Igualando ambas expresiones:

$\tan(\theta+\phi) \ x - \dfrac {g}{2 v_0^2 \cos^2 (\theta+\phi)}\ x^2=\tan \phi \ x$

$x\cdot \left[ 
\tan(\theta+\phi) \ x - \dfrac {g}{2 v_0^2 \cos^2 (\theta+\phi)}\ x^2=\tan \phi \ x
 \right]=0 \to  \begin{cases} x=0 \,(inicio) \\ [ \; ]=0 \to \end{cases}$


$[\;]\to x=\left[\tan (\theta+\phi)-\tan \phi  \right] \  \dfrac{2v_0^2 \cos^2(\theta+\phi)}{g} =$

$=\dfrac {\sin(\theta+\phi) \cos \phi-\sin \phi \cos(\theta+\phi)}{\cos(\theta+\phi) \cos \phi} \  \dfrac{2v_0^2 \cos^2(\theta+\phi)}{g} =$

$=\dfrac {\sin [(\theta+\phi)-\phi]}{\cos(\theta+\phi) \cos \phi} \ \dfrac{2v_0^2 \cos^2(\theta+\phi)}{g} $
$=\dfrac {\sin \theta}{\cancel{\cos(\theta+\phi)} \cos \phi} \ \dfrac{2v_0^2 \cos^{\cancel{2}}(\theta+\phi)}{g} =$


Luego, las coordenadas del punto de impacto son:

$x=\dfrac{2v_0^2}{g}\ \dfrac {\sin \theta \ \cos (\theta+\phi)}{\cos \phi}; \qquad y\ \textcolor{gris}{=\tan \phi \ x }= \tan \phi \ \dfrac{2v_0^2}{g}\ \dfrac {\sin \theta \ \cos (\theta+\phi)}{\cos \phi}$

La longitud $d$ del plano inclinado a que el proyectil impacta sobre el plano inclinado, la podemos calcular por trigonometría:

$\cos \phi = \dfrac x d \to \quad \boldsymbol{d= \dfrac{2v_0^2}{g}\ \dfrac {\sin \theta \ \cos (\theta+\phi)}{\cos^2 \phi} }$ 


Para encontrar el ángulo de tiro $\theta$ que para una velocidad inicial dada $v_0=cte$ y una inclinación del plano dada $\phi=cte$, lo obtendremos sin más que derivar $d=d(\theta)$, con $v_0$ y $\phi$ constantes. 

$d'=\dfrac {2 v_0^2}{g  \cos^2 \phi}\ \left[ \cos \theta \cos (\theta+\phi) - \sin \theta \sin(\theta+\phi) \right]=0 \to $

$ \to \quad \tan (\theta + \phi)=\cot \theta \to \dfrac {\tan \theta + \tan \phi}{1-\tan\theta \tan \phi}=\dfrac {1}{\tan \theta} \to $

$\tan^2 \theta +2\tan\phi \tan \theta -1 =0 \Rightarrow \tan \theta= \dfrac {-2\tan \phi \pm \sqrt{4\tan^4 \phi+4}}{2}=-\tan \phi \pm \sqrt{\tan^2 \phi + 1}=-\tan \phi + \dfrac{1}{\cos \phi} \to \quad \boldsymbol { \tan \theta = \dfrac {1-\sin \phi}{\cos \phi} }$


Llevemos estos resultado a la ecuación que da $d$ para obtener el $d_{max}$

como $\cos \theta \textcolor{gris}{= \dfrac 1 {\sqrt{\tan^2}}} =\dfrac {\cos \phi}{\sqrt{2} \sqrt{1-\sin \phi}}; \qquad \sin \theta \textcolor{gris}{=\cos \theta \tan \theta}=\dfrac{\sqrt{1-\sin \phi}}{\sqrt{2}}$

$ d_{max}  =\dfrac {2 v_0^2}{g \cos^2 \phi} \dfrac {\sqrt{1-\sin \phi}}{\sqrt{2}} \left[
\dfrac {\cos \phi}{\sqrt{2}\sqrt{1-\sin \phi}}\cos \phi - \dfrac{\sqrt{1-\sin \phi}}{\sqrt{2}}\sin \phi
\right] $ 

$\boldsymbol{d_{max}} =\dfrac{\cancel{2} v_0^2}{g \cos^2 \phi \ \cancel{2}}[ \ \cos^2 \phi - (1-\sin \phi)\ \sin \phi  \ ]  = \boldsymbol{\dfrac {v_0^2 \ (1-\sin \phi)}{g \ \cos^2 \phi}}$

\vspace{4mm} \textbf{Análisis de dimensiones:} Las razones trigonométricas carecen de dimensiones. $[d]=\dfrac{[(LT^{-1})^2]}{[LT^-{-2}]}=[L]$

\vspace{4mm} \textbf{Análisis de casos extremos:}
\begin{itemize}
\item $\phi=0 \to d=\dfrac {v_0^2} {g}$, que coincide con lo previsto en tiro horizontal para $\theta=45^o$
\item$\phi=90^o \to d= \dfrac {v_0^2}{g}\ \eval{\dfrac {1-\sin \phi}{\cos^2 \phi}}_{\phi\to \pi/2}= [ \widetilde{\phi} =\pi/2 - \phi]=$ 

$= \dfrac {v_0^2}{g}\ \eval{\dfrac {1-\cos \widetilde{\phi}}{\sin^2 \widetilde{\phi}}}_{\widetilde{\phi}\to 0}=[McLauirin]= \dfrac {v_0^2} {g} \eval{\ \dfrac {1-\left( 1-\dfrac {1}{\widetilde{\phi^2}} \right)} {\widetilde{\phi^2}}}_{\widetilde{\phi}\to 0} =\dfrac {v_0^2}{2g}$, que coincide con la altura máxima alcanzada por un proyectil en tiro vertical.
\end{itemize}

\begin{prob}
Un esquiador deja una rampa de salto con una velocidad d $10 \ \mathrm{ms}^{-1}$	formando un ángulo de $15^o$ con la horizontal. La inclinación de la pista de salto es de $50^o$ y la resistencia del aire se considera despreciable. Encuéntrese la distancia $d$ a la que cae el esquiador a lo largo de la pista.
\end{prob}

\begin{multicols}{2}

\begin{figure}[H]
		\centering
		\includegraphics[width=.5\textwidth]{imagenes/imagenes02/T02IM37.png}
		\end{figure}
El problema es similar al anterior pero para $\phi>90^o$. 
Llamando al ángulo del plano inclinado $-\phi$, podremos usar las fórmulas anteriores.

\begin{figure}[H]
		\centering
		\includegraphics[width=.3\textwidth]{imagenes/imagenes02/T02IM38.png}
		\end{figure}

\end{multicols}


\vspace{-3mm}\begin{figure}[H]
		\centering
		\includegraphics[width=1\textwidth]{imagenes/imagenes02/T02IM40.png}
		\end{figure}
		
	
		
\vspace{-3mm}\textbf{Análisis de casos límites:} 

Si en la expresión encontrada, $x=\dfrac{2v_0 \cos^2 \theta}{g}(\tan \theta+\tan \phi)$, hacemos $\phi=0$, se obtiene $x=\dfrac{v_0^2}{g} \sin 2\theta=x_{max}$, alcance máximo de un proyectil en superficie horizontal.

\newpage %**************************************

\begin{myblock}{Cinemática}
	
\begin{small}
El nacimiento de la cinemática moderna tiene lugar con la alocución de Pierre Varignon el 20 de enero de 1700, ante la Academia Real de las Ciencias de París. Fue allí cuando definió la noción de aceleración y mostró cómo es posible deducirla de la velocidad instantánea utilizando un simple procedimiento de cálculo diferencial.

\vspace{2mm} En la segunda mitad del siglo XVIII se produjeron más contribuciones por Jean Le Rond d'Alembert, Leonhard Euler y André-Marie Ampère y continuaron con el enunciado de la ley fundamental del centro instantáneo de rotación en el movimiento plano, de Daniel Bernoulli.

\vspace{2mm} El vocablo cinemática fue creado por André-Marie Ampère, quien delimitó el contenido de esta disciplina y aclaró su posición dentro del campo de la mecánica. Desde entonces, la cinemática ha continuado su desarrollo hasta adquirir una estructura propia.

\vspace{2mm} Con la teoría de la relatividad especial de Albert Einstein en 1905, se inició una nueva etapa, la cinemática relativista, donde el tiempo y el espacio no son absolutos, y sí lo es la velocidad de la luz.

\vspace{2mm} Los elementos básicos de la cinemática son el espacio, el tiempo y un móvil.

\vspace{2mm} La cinemática trata del estudio del movimiento de los cuerpos en general y, en particular, el caso simplificado del movimiento de un punto material, mas no estudia por qué se mueven los cuerpos sino que se limita a describir sus trayectorias y modo de reorientarse en su avance.

\vspace{2mm} El movimiento trazado por una partícula lo mide un observador respecto a un sistema de referencia. Desde el punto de vista matemático, la cinemática expresa cómo varían las coordenadas de posición de la partícula (o partículas) en función del tiempo. La función matemática que describe la trayectoria recorrida por el cuerpo (o partícula) depende de la velocidad (la rapidez con la que cambia de posición un móvil) y de la aceleración (variación de la velocidad respecto del tiempo).

\vspace{2mm} \textbf{Tipos de movimientos}:

--- Si la aceleración es nula, da lugar a un movimiento rectilíneo uniforme y la velocidad permanece constante a lo largo del tiempo.

--- Si la aceleración es constante con igual dirección que la velocidad, da lugar al movimiento rectilíneo uniformemente acelerado y la velocidad variará a lo largo del tiempo.

--- Si la aceleración es constante con dirección perpendicular a la velocidad, da lugar al movimiento circular uniforme, donde el módulo de la velocidad es constante, cambiando su dirección con el tiempo.

--- Cuando la aceleración es constante y está en el mismo plano que la velocidad y la trayectoria, tiene lugar el movimiento parabólico, donde la componente de la velocidad en la dirección de la aceleración se comporta como un movimiento rectilíneo uniformemente acelerado, y la componente perpendicular se comporta como un movimiento rectilíneo uniforme, y se genera una trayectoria parabólica al componer ambas.

--- Cuando la aceleración es constante pero no está en el mismo plano que la velocidad y la trayectoria, se observa el efecto de Coriolis.

--- En el movimiento armónico simple se tiene un movimiento periódico de vaivén, como el del péndulo, en el cual un cuerpo oscila a un lado y a otro desde la posición de equilibrio en una dirección determinada y en intervalos iguales de tiempo. La aceleración y la velocidad son funciones, en este caso, sinusoidales del tiempo.


\vspace{2mm} Un movimiento interesante es el de una peonza, que al girar puede tener un movimiento de precesión y de nutación. Cuando un cuerpo posee varios movimientos simultáneamente, como por ejemplo uno de traslación y otro de rotación, se puede estudiar cada uno por separado en el sistema de referencia que sea apropiado para cada uno, y luego, superponer los movimientos.

\vspace{2mm} \textbf{Movimiento sobre la Tierra}:

\vspace{2mm} Supongamos que un cañón situado en el ecuador lanza un proyectil hacia el norte a lo largo de un meridiano. Un observador situado al norte sobre el meridiano observa que el proyectil cae al este de lo predicho, desviándose a la derecha de la trayectoria. De forma análoga, si el proyectil se hubiera disparado a lo largo del meridiano hacia el sur, el proyectil también se habría desviado hacia el este, en este caso hacia la izquierda de la trayectoria seguida. La explicación de esta «desviación», provocada por el Efecto Coriolis, es debida a la rotación de la Tierra. El proyectil tiene una velocidad con tres componentes: las dos que afectan al tiro parabólico, hacia el norte (o el sur) y hacia arriba, respectivamente, más una tercera componente perpendicular a las anteriores debida a que el proyectil, antes de salir del cañón, tiene una velocidad igual a la velocidad de rotación de la Tierra en el ecuador. Esta última componente de velocidad es la causante de la desviación observada pues si bien la velocidad angular de rotación de la Tierra es constante sobre toda su superficie, no lo es la velocidad lineal de rotación, la cual es máxima en el ecuador y nula en el centro de los polos. Así, el proyectil conforme avanza hacia el norte (o el sur), se mueve más rápido hacia el este que la superficie de la Tierra, por lo que se observa la desviación mencionada. Lógicamente, si la Tierra no estuviese rotando sobre sí misma, no se daría esta desviación.

\vspace{2mm} Otro caso interesante de movimiento sobre la Tierra es el del péndulo de Foucault. El plano de oscilación del péndulo no permanece fijo, sino que lo observamos girar, girando en sentido horario en el hemisferio norte y en sentido antihorario en el hemisferio sur. Si el péndulo se pone a oscilar en el ecuador, el plano de oscilación no cambia. En cambio, en los polos, el giro del plano de oscilación toma un día. Para latitudes intermedias toma valores mayores, dependiendo de la latitud. La explicación de tal giro se basa en los mismos principios hechos anteriormente para el proyectil de artillería \end{small}\normalsize{.}
\end{myblock}




\chapter{Dinámica de la partícula. Campos de fuerza}	
\chaptermark{Dinámica de la partícula}

La dinámica es una parte de la mecánica en que, dadas unas condiciones, se estudia qué movimiento se va a producir.

Consideraciones: En mecánica clásica el \emph{espacio} es como un gran cajón donde está todo el universo físico. El \emph{tiempo} es absoluto: dados dos fenómenos, todos los relojes de todos los posibles observadores medirán lo mismo.

\rule{150pt}{0.4pt} 

Los fenómenos mecánicos se describen mediante sistemas de referencia basados en los conceptos de espacio y tiempo. Por su importancia conviene enunciar los postulados que asume la mecánica clásica para estos conceptos. 

El \emph{espacio}, y por tanto su métrica, tiene las propiedades siguientes. 
\begin{itemize}
\vspace{-2mm}\item Independencia de los objetos en él inmersos. (La métrica del espacio no se ve afectada por los mismos.) 
\vspace{-2mm}\item Constancia a lo largo del tiempo. 
\vspace{-2mm}\item \textbf{Homogeneidad: es igual en todos los puntos}, no existiendo puntos privilegiados. 
\vspace{-2mm}\item \textbf{Isotropía: es igual en todas las direcciones}, no existiendo direcciones privilegiadas. 
\end{itemize}
El espacio se caracteriza por una métrica euclídea:
 
$d(\;(x_1,y_1,z_1)\;(x_2,y_2,z_2)\;)=\sqrt{(x_2-x_1)^2+(y_2-y_1)^2+(z_2-z_1)^2}$ 

El \emph{tiempo} se caracteriza a su vez por las siguientes propiedades. 
\begin{itemize}
\vspace{-2mm}\item \textbf{Homogeneidad}, al no existir instantes privilegiados. 
\vspace{-2mm}\item Fluye constantemente en un sentido, por lo que no se puede retroceder ni volver al pasado. Asimismo, los fenómenos futuros no pueden condicionar los presentes. \textbf{No} se cumple por tanto la \textbf{isotropía}, existiendo un único sentido en el que puede discurrir el tiempo. 
\vspace{-2mm}\item \textbf{Simultaneidad absoluta}: Los fenómenos considerados simultáneos para dos observadores en sendos sistemas de referencia lo son asimismo para cualquier otro observador ligado a cualquier otro sistema de referencia. 
\end{itemize}

\setlength{\parindent}{0pt} %Aparecía sangrado en 1a linea, esto lo evita
En mecánica clásica, el tiempo se considera una variable de naturaleza distinta de las variables espaciales y la métrica euclídea no está influenciada por él. 

Algunos de estos postulados básicos no son aceptados por la mecánica relativista. La teoría de la relatividad restringida establece una referencia en cuatro dimensiones espacio-tiempo. La teoría de la relatividad general establece un espacio curvado, con métrica riemanniana no euclídea, debido a la presencia de masas que condicionan dicha métrica. De esta forma el espacio no sería independiente de los objetos en él inmersos 

\rule{150pt}{0.4pt} 

Una vía para abordar los problemas de la mecánica es averiguar las `causas del movimiento': ``el \emph{movimiento} es debido a la \emph{interacción} que se ejerce entre las diversas partes o cuerpos de universo''.


\section{Leyes del movimiento. Newton (3)}

\begin{cita}
``Si he visto más lejos que los demás es porque estoy sentado sobre los hombros de gigantes'' (Isaac Newton).
\end{cita} 

\vspace{-13mm}\scriptsize{Isaac Newton (1643-1727) escribió esa frase en una carta a Robert Hooke en la que hacía mención a sus predecesores Copérnico, Galileo y Kepler}\normalsize{.}

\vspace{4mm}A las entidades responsables de que los cuerpos varíen su posición en el espacio absoluto newtoniano les llamaremos \emph{interacciones} o \emph{fuerzas.}

En el siglo $XVIII$, Newton postula las tres leyes de la mecánica clásica.

\vspace{3mm}
\begin{miparrafo}

\textbf{1. } \emph{``Todo cuerpo permanece en reposo o en movimiento uniforme a menos que se obre sobre él alguna fuerza''.} Es la llamada \textsf{Ley de Inercia.}

\textbf{2. } \emph{``Todo cuerpo sobre el que actúa una fuerza se mueve de forma tal que la variación de su cantidad de movimiento por unidad de tiempo es igual a la fuerza''.} \textsf{Fuerza igual a masa por aceleración.}

\textbf{3. } \emph{``Cuando dos cuerpos ejercen entre sí fuerzas, éstas son de intensidades iguales y sentidos opuestos y están dirigidas a lo largo de la línea que une a los dos cuerpos considerados como partículas''.} 
\textsf{Ley de Acción y Reacción.}

\end{miparrafo}

En la primera ley se introduce la `fuerza' cualitativamente, en la segunda cuantitativamente.

Hemos introducido una nueva magnitud, la \emph{cantidad de movimiento}, 

\begin{equation}
\label{cant-mov}
\subrayado{\vec p=m\; \vec v} \qquad \qquad  \textcolor{gris}{[\vec p]=[\mathrm{MLT}^{-1}]\to \mathrm{kg\ m\ s}^{-1}}
\end{equation}

\begin{equation}
\label{Newton}
\subrayado{\vec F=\dv{\vec p}{t}} 	\qquad \qquad  \textcolor{gris}{[\vec F]=\dfrac{[\vec p]}{[t]}=[\mathrm{MLT}^{-2}]\to \mathrm{kg\ m\ s}^{-2}=\mathrm{N}}
\end{equation}


\emph{Primera hipótesis de la mecánica clásica:} La \emph{masa} es una característica especial de los cuerpos que interaccionan y no varía respecto al tiempo, es \emph{constante}.

\begin{equation}
\label{fma}
 \boldsymbol{ \vec F}=\dv{\vec p}{t}=\dv{(m\; \vec v)}{t}=m\; \dv{\vec v}{t}=\boldsymbol{ m\; \vec a}
\end{equation}


Supongamos dos cuerpos interactuando: $\vec F_1=m_1\ \vec a_1$ y  $\vec F_2=m_2\ \vec a_2$. Supongamos que forman un sistema aislado en el espacio, es decir, el resto de cuerpos del universo ejercen sobre ellos una interacción despreciable.

Por el tercer postulado (ley) de Newton, tenemos: $\vec F_1=-\vec F_2$, es decir, $m_1\; \vec a_1=-m_2\;\vec a_2\;:$
$$\displaystyle \dfrac {m_2}{m_1}=\dfrac {a_1}{a_2}$$
Tomamos un cuerpo cualquiera del universo de masa $m_1$ que no conocemos y, arbitrariamente, le damos el valor de `masa unidad', con lo que para otro cuerpo, $\displaystyle m_2=\dfrac {a_1}{a_2}\; m_1$ y así introducimos el concepto de masa.

\begin{figure}[]
		\centering
		\includegraphics[width=1\textwidth]{imagenes/imagenes03/T03IM02.png}
		\end{figure}

Por extensión de la ley de Newton, tenemos: $Peso=m_G\; a \;\; \wedge\;\; F=m_I\; a$, donde $g$ es la aceleración de la gravedad, $m_G$ la masa gravitatoria y $m_I$ la masa inercial ($m$ es la constante de proporcionalidad entre $\vec F$ y $\vec a$).

\textbf{Igualdad entre masa inercial y masa gravitatoria: Otro enigma sin resolver.}

Fijémonos despacio en las fórmulas de Newton. Cuando aplicas una fuerza $\vec F$ a un cuerpo con cierta masa $m_I$, éste se acelera con cierta aceleración $\vec a\; / \vec F=m_I\; \vec a$. Hemos llamado $m_I$ a la masa para recordar que estamos hablando de la ``resistencia’’ que opone esa masa a acelerarse (su inercia; de ahí el subíndice `I’).

Pero sabemos que el mismo cuerpo ejerce una atracción gravitatoria sobre todos los demás. Por ejemplo, si pesamos ese cuerpo, lo que estamos haciendo es medir la fuerza gravitatoria entre él y la Tierra, que será $F=G \dfrac{M_T\; m_G}{{R_T}^2}$, donde $M_T$ es la masa de la Tierra, $R_T$ su radio y $m_G$ es la masa del cuerpo en cuestión (que ahora llamamos $m_G$ para remarcar que nos referimos a la masa que produce la atracción gravitatoria). 

Si ahora imaginamos un experimento fácil en el que queremos medir la aceleración que produce la fuerza de la gravedad, no tenemos más que igualar las dos fórmulas:
$$m_I\;a= G \dfrac{M_T\; m_G}{{R_T}^2} \to a= \dfrac{m_G}{m_I}\;G\dfrac{M_R}{{R_T}^2}$$
Midiendo la aceleración $a$ en muchos experimentos diferentes, con objetos de masas diferentes, se ve que es siempre igual (y por eso se le llama $g$, la aceleración de la gravedad en la superficie terrestre) y, por tanto,  $\boldsymbol {m_G/m_I=1}$, o sea encontramos que la masa gravitatoria es igual a la masa inercial en cualquier cuerpo. (Experimentos de EOTVOS-1891 y DICKE-1950)

La pregunta que ya se hizo Newton es: ?`por qué tienen que ser necesariamente iguales $m_I$ y $m_G$? Son conceptualmente diferentes: $m_I$ representa cómo una cantidad de materia se opone a que la aceleren mientras que $m_G$ representa cómo atrae esa cantidad de materia a cualquier otra.

Como hemos dicho, experimentalmente se mide que ambas son iguales. Pero tenemos que recordar que la precisión de los aparatos con los que medimos en los experimentos científicos no es infinita. Por tanto, no podemos estar absolutamente seguros de que sean exactamente iguales, lo más que podemos decir es que son iguales hasta una precisión muy elevada (una parte en diez billones). O sea, en una masa de un kilo, $m_I$ y $m_G$ podrían ser diferentes en una milmillonésima de miligramo, no más. Hecha esta aclaración, podemos asegurar que la masa inercial y la masa gravitatoria son iguales hasta donde alcanza la precisión de nuestras medidas.

Einstein también pensó en este enigma. No lo resolvió (es decir, no encontró una teoría que demostrara que las masas gravitacional e inercial tienen que ser iguales) pero, como estaba convencido de que dicha igualdad no era por casualidad, supuso que era un principio fundamental del universo (como el del límite de la velocidad de la luz) y lo llamo \textbf{\emph{``principio de equivalencia''}}. Este principio es fundamental en su teoría de la gravitación, la teoría general de la relatividad.

		
\section[Fuerzas de rozamiento, estático y dinámico]{Fuerzas de rozamiento, estático y dinámico \sectionmark{Fuerzas de rozamiento}}
\sectionmark{Fuerzas de rozamiento}

Al arrojar un cuerpo sobre una mesa observamos que	éste se desliza por ella y, al cabo de un tiempo	 $t$, se para. \emph{Las fuerzas de rozamiento son aquellas que actúan siempre oponiéndose al sentido del movimiento de los cuerpos.}

En un momento determinado en que hay un determinado número de pesas en la balanza, el cuerpo se pone en movimiento: $\ \boldsymbol{F\ge F_{RE}}$.

\begin{multicols}{2}

La fuerza de rozamiento estático depende de la naturaleza de los cuerpos que rozan.

La relación entre la fuerza máxima de rozamiento estático y la fuerza normal $(N)$ se llama \emph{coeficiente de rozamiento estático:} $\ F_{RE}=\mu_{RE}\ N$

En general,

$\vec F_{tot}=\vec F-\vec F_{RE}=m\ \vec a$

\begin{figure}[H]
		\centering
		\includegraphics[width=.3\textwidth]{imagenes/imagenes03/T03IM48.png}
		\end{figure}		
\end{multicols}		
		
\textbf{Plano inclinado}
		
\begin{multicols}{2}
A medida que aumenta $\alpha$, aumenta la componente tangencial del peso, $\vec F_T$. En el momento en que el cuerpo empiece a deslizar,

$F_T=F_{RE}=\mu_{RE}\vec N \quad \to \quad \mu_E=\dfrac {m g \sin \alpha}{m g \cos \alpha}=\tan \alpha$
\begin{figure}[H]
		\centering
		\includegraphics[width=.3\textwidth]{imagenes/imagenes03/T03IM49.png}
		\end{figure}		
\end{multicols}			
		

El coeficiente de rozamiento estático se puede encontrar como la tangente del ángulo mínimo para el que el cuerpo empieza a deslizar por el plano inclinado.

Experimentalmente se comprueba que, una vez empieza el cuerpo a moverse, la fuerza de mantenimiento puede ser menor que $F_{RE}$. Se suele hablar de \emph{rozamiento dinámico.}

$F_{RE}=\mu_{RE}\ N; \quad F_{RD}=\mu_{RD}\ N \le F_{RE} \Rightarrow \boldsymbol{\mu_{RD} \le \mu_{RE}}$		
		
$F_{RD}$ es la fuerza que hay que aplicar para mantener el cuerpo en movimiento.			


\begin{figure}[H]
		\centering
		\includegraphics[width=.75\textwidth]{imagenes/imagenes03/T03IM06.png}
		\end{figure}
		
\vspace{10mm} %**********************************************
		
\begin{figure}[H]
		\centering
		\includegraphics[width=1\textwidth]{imagenes/imagenes03/T03IM07.png}
		\end{figure}
		
		
\begin{figure}[H]
		\centering
		\includegraphics[width=1\textwidth]{imagenes/imagenes03/T03IM08.png}
		\end{figure}		

\newpage %*****************************************************

\begin{figure}[H]
		\centering
		\includegraphics[width=.95\textwidth]{imagenes/imagenes02/T02IM31.png}
		\end{figure}

\begin{figure}[H]
		\centering
		\includegraphics[width=.95\textwidth]{imagenes/imagenes02/T02IM32.png}
		\end{figure}

			
\section{Trabajo de una fuerza}

\begin{multicols}{2}
Por definición, diremos que el \emph{trabajo elemental} $\boldsymbol{\dd W}$ que realiza un cuerpo $m$  al pasar de una posición $\vec r$ a otra infinitamente próxima $\vec r+ \dd \vec r$ es: 

\begin{equation}
\label{def-trabajo}
\subrayado{\dd W= \vec F \cdot \dd \vec r}
\end{equation}

$ (\ \boldsymbol{\cdot} \ \equiv \text{producto\; escalar})$

\begin{figure}[H]
		\centering
		\includegraphics[width=.45\textwidth]{imagenes/imagenes03/T03IM01.png}
		\end{figure}
\end{multicols} 



El trabajo total para ir de un punto $A$ hasta otro $B$ es:

\begin{equation}
\label{trabajo}
W=\int_A^B\vec F \cdot \dd \vec r	
\end{equation}

En el caso particular en que $\vec F=\overrightarrow{cte}$, 
$W=\vec F \cdot \int_A^B \dd \vec r=\vec F\cdot (\vec r_B-\vec r_A)=\vec F \cdot \Delta \vec r$

\begin{multicols}{2}
Si además el desplazamiento es en línea recta, el trabajo es el producto del desplazamiento por la componente tangencial de la fuerza.º

\vspace{3mm} %**********************************************
$W=\vec F\cdot \Delta \vec r=F\; \Delta r\; 	\cos \theta= (F\; \cos \theta)\; \Delta r= F_T \;\Delta r$

\vspace{3mm} %**********************************************
Dimensionalmente, $[W]=[\mathrm{FL}]=[\mathrm{ML}^2\mathrm{T}^{-2}]$. Su unidad es el Joule, $\mathrm{J}=\mathrm{N} \cdot \mathrm{m}$. El trabajo es un escalar.
\begin{figure}[H]
		\centering
		\includegraphics[width=.4\textwidth]{imagenes/imagenes03/T03IM03.png}
		\end{figure}	
\end{multicols}





\begin{figure}[H]
		\centering
		\includegraphics[width=.8\textwidth]{imagenes/imagenes03/T03IM04.png}
		\end{figure}

\vspace{-4mm}\noindent \small{\textcolor{gris}{$dW=\vec F \cdot \dd \vec r:\begin{cases}  \vec r=x\vec i +y \vec j \\  \vec F=\vec F(x,y) \end{cases} \hspace{-2mm}\to  y=y(x) \Rightarrow  \begin{cases} \dd \vec r=\dd \vec r(x,y(x)) \\ \vec F=\vec F(x) \end{cases} \hspace{-2mm} \dd W= \dd W(x)$}\normalsize{.}}

\begin{figure}[H]
		\centering
		\includegraphics[width=.8\textwidth]{imagenes/imagenes03/T03IM13.png}
		\end{figure}

\section{Concepto de energía}

\normalsize{Un} \emph{fenómeno físico} es una variación en las condiciones que se dan en un sistema físico.

Podemos distinguir entre dos tipos de fueras:

\vspace{-2mm}\begin{itemize}
\vspace{-2mm}\item \emph{fuerzas exteriores} que actúan sobre un sistema físico y pueden modificarlo (al soltar un objeto de la mano cae al suelo por efecto de la fuerza externa de la gravedad que actúa sobre él).
\vspace{-2mm}\item \emph{fuerzas interiores} que actúan sobre un sistema físico y pueden modificarlo 	(un átomo de carbono 14 se convierte en nitrógeno 14 emitiendo un electrón --- desintegración $\beta:\; ^{14}_6C \to ^{14}_7N+^0_{-1}e$---).
\end{itemize}

Cuando se modifica el estado de un sistema físico se atribuye a las fuerzas y producen movimientos. \emph{Las fuerzas en movimiento son trabajo}, luego los sistemas físicos tienen \emph{energía}. 

Llamamos \emph{\colorbox{LightYellow}{energía}} a \emph{la \colorbox{LightYellow}{capacidad} que tienen los sistemas físicos para \colorbox{LightYellow}{realizar un trabajo}.}, es una propiedad que tienen los cuerpos y se va a medir con las mismas unidades y dimensiones que el trabajo. La energía es un escalar.

Podemos distinguir entre tres tipos de energia:

\begin{miparrafo}

\textbf{---} \emph{Energía cinética}: capacidad para realizar trabajo que tiene un cuerpo debido a su estado de movimiento (proyectil estrellado contra una pared).

\textbf{---} \emph{Energía potencial}: capacidad para realizar trabajo que tiene un cuerpo debido a la posición que ocupa en el espacio  (objeto que cae desde una determinada altura)

\textbf{---} \emph{Energía interna}: capacidad para realizar trabajo que tiene un cuerpo debido a su propia estructura interna (petardo de pólvora que explota por la temperatura).

\end{miparrafo}

La cantidad total de energía que tiene un sistema físico no es medible por la mecánica clásica, solo podemos medir variaciones de energía.


\section[Energía cinética. Teorema de las fuerzas vivas]{Energía cinética. Teorema de las fuerzas vivas\sectionmark{Energía cinética}}
\sectionmark{Energía cinética}

Definimos la energía cinética $\;\mathcal E_c\;$ como el trabajo necesario para que un cuerpo de masa $m$ cambie su celeridad (módulo de la velocidad) desde $v_0$ a $v_f$.
$$\subrayado{\Delta \mathcal E_c=W}=\displaystyle \int_1^2 \vec F \cdot \dd \vec r$$
desde el estado $1\to v_o$ hasta el estado $2\to v_f$. Por la segunda de Newton:

$\Delta \mathcal E_c=\displaystyle \int_1^2 \dv{\vec p}{t}\cdot \dd \vec r =\int_1^2 \dd \vec p \cdot \dv{\vec r}{t}
= \int_1^2 \vec v \cdot \dd \vec p=\dfrac 1 m\int_1^2 m\;\vec v \cdot \dd \vec p=
\dfrac 1 m \int_1^2 \vec p\cdot \dd \vec p$

$p^2=\vec p\cdot \vec p \to $ tomando diferenciales: $\cancel{2\;} p\;\dd  p= \cancel{2}\; \vec p \cdot \dd \vec p \to\;\; p\;\dd p=\vec p \cdot \dd \vec p$

$\Delta \mathcal E_c=\displaystyle \int_1^2 \dfrac 1 m \;p\; dp=\int_1^2 v\;\dd p =\int_1^2\; v  \dd (mv)=\int_1^2 m (m\dd v + v\dd m)$

\begin{equation}
	\Delta \mathcal E_c=\displaystyle\int_1^2 m\ v\;\dd v + \int_1^2 v^2 \ \dd m
\end{equation}

Ecuación que da la expresión exacta de la energía cinética entre deos estados $1$ y $2$


En el caso especial de partículas o cuerpos rígido, $ m=cte  \to  \Delta \mathcal E_c= \displaystyle m\ \int_1^2 v \ \dd v $:

\begin{equation}
\label{fuerzas-vivas}
\Delta \mathcal E_c= \frac 1 2 \ m \ v_2^2 - \frac 1 2 \ m \ v_1^2 = \mathcal E_{c_2}-\mathcal E_{c_1}
\end{equation}

Ecuación que recibe el nombre de \emph{\colorbox{LightYellow}{`Teorema de las fuerzas vivas'}.}
(Leibniz, en el $s.\ XVII$ llamó ``fuerza viva'' al producto $m \ v^2$).

\emph{El trabajo realizado por las fuerzas del campo al evolucionar el sistema dede un estado $1$ con $v_1$ a un estado $2$ con $v_2$ viene dado por la diferencia entre energías cinéticas ($\frac 1 2$ fuerza viva) entre el estado final $2$ y el estado inicial $1$.}


En el supuesto que:
\vspace{-2mm}\begin{itemize}
\vspace{-2mm}\item $v_1>v_2 \to \mathcal E_1 < \mathcal E_2 \to \Delta \mathcal E_c<0$ y el trabajo desarrollado se usa en frenar a la partícula.
\vspace{-2mm}\item $v_1<v_2 \to \mathcal E1 < \mathcal E_2 \to \Delta \mathcal E_c>0$ y el trabajo desarrollado se invierte en acelerar a la partícula.
\vspace{-2mm}\item $v_1=v_2 \to \mathcal E_1 = \mathcal E_2 \to \Delta \mathcal E_c=0$ y no se realiza trabajo. 	
\end{itemize}

\begin{figure}[H]
		\centering
		\includegraphics[width=1\textwidth]{imagenes/imagenes03/T03IM05.png}
		\end{figure}


\begin{multicols}{2}
	$W=\Delta \mathcal E_c==\frac 1 2 m v_f^2-\frac 1 2 m v_i^2$
	
	$\vec F=\vec N+\vec T$
	
	$\dd W _M=\vec N \cdot \dd \vec s=$
	
	$=(mg\sin \theta) \dd s \cos 90^o=0\to $ 
	
	$\to \quad W_N=0$
	\begin{figure}[H]
		\centering
		\includegraphics[width=.55\textwidth]{imagenes/imagenes03/T03IM47.png}
		\end{figure}
\end{multicols}
\vspace{-5mm}$\dd W_T= \vec T \cdot \dd \vec s=(mg \sin \theta) \dd s \cos 0^o \to W_T=mg\sin \theta \int_1^2 ds
=ms \sin \theta L=mgh$

$W=0+mgh=mgh=\mathcal E_c=\frac 1 2 m v_f^2$, parten del reposo: $v_i=0$

Luego $v_f=\sqrt{2gh}$. Ambas patinadoras llegan a meta con la misma velocidad y la menos experta gana la apuesta.

\textcolor{gris}{La patinadora que baja por la pista para expertos invierte menos tiempo en la bajada, pero esa no era la apuesta.}

\section{Potencia}

Se llama \emph{potencia} suministrada por una fuerza al trabajo por unidad de tiempo que realiza dicha fuerza. 

Potencia media: $\quad P=\dfrac{\Delta W}{\Delta t}$

Potencia instantánea: $\;\; P=\dfrac {\dd W}{\dd t}=\dfrac {\vec F \cdot \dd \vec r}{\dd t}=\vec F \cdot \vec v$

En general, para cualquier transferencia de energia, la potencia es: $\quad P=\dfrac{\dd E}{\dd t}$

Unidades: En el $SI:\;\; P\to  \ \mathrm{J s}^{-1} = \mathrm{W}$, wat. Una unidad de energía muy usada en ingeniería es el $\mathrm{kWh}=(10^3\ \mathrm{W})\ (3600\ \mathrm{s})=3.6\ 10^6 \ \mathrm{J}$.

\begin{ejem}
Un ascensor de $1000\ \mathrm{kg}$ soporta una carga máxima de $800 \ \mathrm{kg}$. Una fuerza de rozamiento constante de $400 \ \mathrm{N}$ retarda su movimiento hacia arriba haciendo que la subida sea a velocidad constante de $3 \ \mathrm{ms}^{-1}$. ?`Qué potencia suministra el motor?	
\end{ejem}
$F-F_R-(M+m)g=0 \to F=(1000+800)\ 9.8-4000=13640\ \mathrm{N}$

$P=F\ v=13640 \cdot 3 = 40920 \ \mathrm{W}= 40.920 \ \mathrm{KW}$

\textbf{Potencia y energía cinética.}

En una dimensió: $\displaystyle P=Fv=mav = m \dv{v}{t} v =\dv{t}(\dfrac 1 2 mv^2)\Rightarrow P=\dv{\mathcal E_c}{t}$.

Si $P=cte \to \quad P=\dfrac{\Delta \mathcal E_c}{\Delta t} $

\begin{ejem}
\normalsize{Un} móvil acelera desde el reposo hasta $100	\ \mathrm{ms}^{-1}$ en $10 \mathrm{s}$. Su masa es de $2\ \mathrm{Kg}$. Suponiendo que esta aceleración se alcance a potencia constante, ?`cuál es la potencia desarrollada por el motor? ?`Cuánto tiempo necesitará para, en las mismas condiciones, doblar su velocidad?
\end{ejem}

$P=\dfrac{\Delta \mathcal E_c}{\Delta t}=\dfrac{\frac 1 2 \ 2 \ 100^2 - 0}{10}=1000 \mathrm{W}$

$P=\dfrac{\Delta \mathcal E_c}{\Delta t} \to 1000=\dfrac{\frac 1 2 \ 2 \ 200^2 - \frac 1 \ 2 \ 2 100^2}{\Delta t} \to \quad \Delta t=30\ \mathrm{s}$


\section{Campos de fuerza}

En una determinada región del espacio existe un \emph{campo de fuerza} cuando por el hecho de situar en cualquier parte de esta región un cuerpo, \emph{instantáneamente} aparece sometido a una fuerza.

En `teoría clásica de campos' la interacción entre cuerpos del universo es \emph{instantánea}. La `teoría relativista de campos'  no permite la interacción instantánea, ninguna interacción se propaga a velocidad superior a la de la luz en el vacío, $c$. \footnote{Ver en Apéndice \ref{concepto-campo} el artículo ``Los conceptos de campo''.}

Llamamos \emph{magnitud activa de campo, $A$}, a la propiedad que tienen los cuerpos al interaccionar con los campos: $m$ en la interacción gravitatoria, $q$ en la eléctrica, .... Si $\vec E$ es la intensidad del campo en cuestión, $\vec F=A\ \vec E$.

\subsection{Campos conservativos. Campo central}

\begin{multicols}{2}
Propiedad de los campos de fuerza \textbf{conservativos}: \emph{``El \underline{trabajo} realizado por un campo de fuerza (conservativo) para que la magnitud activa del campo se desplace desde un punto $A$ hasta otro $B$ es \underline{independiente del camino recorrido}, depende solo de los instantes inicial y final.''}
\begin{figure}[H]
		\centering
		\includegraphics[width=.2\textwidth]{imagenes/imagenes03/T03IM09.png}
		\end{figure}
\end{multicols}

\begin{multicols}{2}
Un caso particular de campo conservativo muy importante es el \emph{campo central:} cuando se deposita una magnitud activa en el campo aparecen sobre ella las fuerzas del campo que están dirigidas hacia el \emph{centro de fuerzas} y suelen ser \emph{inversamente proporcionales al cuadrado de la distancia} de la posición que ocupa la magnitud activa al el centro de fuerzas.
\begin{figure}[H]
		\centering
		\includegraphics[width=.25\textwidth]{imagenes/imagenes03/T03IM10.png}
		\end{figure}
\end{multicols}

\emph{Un campo de fuerzas central es necesariamente conservativo.} Al ser central solo depende de la posición del centro de fuerzas y varía con la distancia (no necesariamente como $1/r^2$), veamos que es conservativo:

Sea $\mathcal A$, la magnitud activa del campo central $\vec E$,

$\Vec F=\mathcal A \; \vec E= \mathcal A \; \vec u_r \; \varphi(r)$, donde $\vec u_r=\dfrac {\vec r}{r}$ es un vector unitario que apunta al centro de fuerzas desde la posición que ocupe la magnitud activa y $\varphi(r)$ es la forma en que varía $\vec E$ con la distancia al centro de fuerzas.

 Calculemos el trabajo necesario para desplazar la magnitud activa $\mathcal A$ bajo la acción del campo central $\vec E$ desde un punto $A$ hasta otro $B$.
 
 $\displaystyle W=-\int_A^B \mathcal A\; \varphi(r) \;\vec u_r \cdot \dd r= -\int_A^B \mathcal A\; \varphi(r) \; \dfrac {\vec r}{r} \cdot \dd r =(*) -\int_A^B \mathcal A\; \varphi(r) \; \dfrac {\cancel{r}\;\dd r}{\cancel{r}}=-\int_A^B \mathcal A\; \varphi(r)\; \dd r=(**)-\eval{\Phi(r)}_{A}^B=-[\Phi(B)-\Phi(A)]$
 
 \vspace{2mm}
 \textcolor{gris}{
 $(*)\quad \vec r\cdot \dd \vec r=r\;\dd r \cos 0^o=r\; \dd r $}
 
  \textcolor{gris}{
 $(**)\quad \mathcal A\; \varphi(r)$ depende solo de $r$ y su primitiva será de la forma $\Phi(r)$ }

Luego \colorbox{LightYellow}{``todo campo central es conservativo''} ya que el $W$ depende solo de las posiciones inicial $A$ y final $B$ y no del camino seguido.

\section{Energia Potencial. Concepto de gradiente}

 
\begin{miparrafo}
Definición de energía potencial: \emph{El trabajo que realiza un campo de fuerza \underline{conservativo} al desplazar un cuerpo desde un punto $A$ hasta un punto $B$ es igual a la diferencia de energías potenciales existentes en los puntos $A$ y $B$.}
\end{miparrafo}

\begin{equation}
\label{Ener-poten}
\int_A^B \vec F \cdot \dd \vec r = \mathcal E_p(A)- \mathcal E_p(B)\; \qquad \textcolor{gris}{(\vec F\; \text{conservativa})}	
\end{equation}

Esto que llamamos energía potencial $ \mathcal E_p$ es una función escalar característica del campo conservativo.

\rule{150pt}{0.4pt} 

\textbf{Otra definición de `campo conservativo':} 

\begin{multicols}{2}
\emph{En un campo conservativo, el trabajo a lo largo de una trayectoria cerrada es cero.}
\begin{figure}[H]
		\centering
		\includegraphics[width=.3\textwidth]{imagenes/imagenes03/T03IM11.png}
		\end{figure}
\end{multicols}

$W_{A\to A}=W_{A\to B}+W_{B \to A}=\mathcal E_p(A)-\mathcal E_p(B) \;+ \; \mathcal E_p(B)-\mathcal E_p(A)=0$

$$ \text{Campo conservativo:}\qquad \oint \vec F \cdot \dd \vec r =0$$

\textcolor{gris}{$\oint$ representa a la integral curvilínea cerrada.}

\rule{150pt}{0.4pt} 

\begin{equation}
\int_A^B \vec F \cdot \dd \vec r = \mathcal E_p(A)- \mathcal E_p(B)=-\int_A^B\dd \mathcal E_p	
\end{equation}

\begin{multicols}{2}
$\dd r = \dd l$, en la dirección de movimiento

$\vec F\cdot \dd \vec r=F\; \dd l\; \cos \theta =-\dd \mathcal E_p $
\begin{figure}[H]
		\centering
		\includegraphics[width=.4\textwidth]{imagenes/imagenes03/T03IM12.png}
		\end{figure}
\end{multicols}

$F \cos \theta=- \dfrac{\dd \mathcal E_p}{\dd l} \; \to \;\; \vec u_r\;F \cos \theta=-\vec u_r\; \dfrac{\dd \mathcal E_p}{\dd l}$

En matemáticas sabemos que un vector $\vec F$ tal que sus componentes, según una dirección determinada del espacio se obtengan a partir de la derivada direccional de una función escalar $\mathcal E_p$, tal vector es el \textbf{\emph{gradiente}} de la función escalar $\mathcal E_p$. (El signo negativo es por convenio).
$$\vec F=-grad\;( \mathcal E_p)$$


$\displaystyle F=\vec i\; F_x +\vec j\; F_y +\vec k\; F_z; \qquad F_x=- \pdv{\mathcal E_p}{x};\; F_y=- \pdv{\mathcal E_p}{y};\; F_z=- \pdv{\mathcal E_p}{z}$


$\displaystyle \vec F=-\left[\;\vec i\;\;\pdv{\mathcal E_p}{x} + \vec j\;\;\pdv{\mathcal E_p}{y} + \vec k\;\;\pdv{\mathcal E_p}{z}\; \right] =$

\begin{equation}
\displaystyle \vec F=- \left( \;\vec i\;\;\pdv{x} + \vec j\;\;\pdv{y} + \vec k\;\;\pdv{z}\; \right)\; \mathcal E_p
\end{equation}

Si llamamos `gradiente' al operador \emph{nabla}: $\displaystyle \grad= \vec i\;\;\pdv{x} + \vec j\;\;\pdv{y} + \vec k\;\;\pdv{z}$

\begin{equation}
\label{Ep-gradiente}
\subrayado{\vec F=-\overrightarrow{ \grad } \mathcal E_p	}
\end{equation}

\emph{\colorbox{LightYellow}{La energía potencial es característica de los campos conservativos.}} 

\emph{En campos no conservativos no tiene sentido hablar de energía potencial.}

$\displaystyle \pdv{F_x}{y}=-\pdv{\mathcal E_p}{y}{x} \quad = \quad -\pdv{\mathcal E_p}{x}{y}=\pdv{F_y}{x} \quad \to \quad \pdv{F_x}{y}-\pdv{F_y}{x}=0$ 

Esto se cumple, por analogía, para las tres componentes:

$\displaystyle \pdv{F_z}{y}-\pdv{F_y}{z}=0 \; \textcolor{gris}{\to\;  \vec i}\;;\quad \pdv{F_z}{x}-\pdv{F_x}{z}=0 \;\textcolor{gris}{\to\;  \vec j}\;;\quad \pdv{F_y}{x}-\pdv{F_x}{y}=0 \;\textcolor{gris}{\to\;  \vec k}$

Y estas son las 3 condiciones que permiten definir matemáticamente a un campo conservativo: \emph{Un campo $\vec F$ es conservativo sus componentes satisfacen simultáneamente las tres ecuaciones anteriores.}

\normalsize{Si} lo escribimos vectorialmente:

$\displaystyle \left( \pdv{F_z}{y}-\pdv{F_y}{z} \right) \;  \vec i+
\left( \pdv{F_z}{x}-\pdv{F_x}{z} \right) \;  \vec j+
\left( \pdv{F_y}{x}-\pdv{F_x}{y} \right) \;  \vec k\;$,

que podemos escribir como:

$\displaystyle \left| \begin{matrix} \vec i&\vec j&\vec k \\ \pdv{x}&\pdv{y}&\pdv{z} \\ F_x&F_y&F_z  \end{matrix} \right|=0 \to \overrightarrow{\grad } \times \vec F = 0$, si hacemos uso del operador nabla.

Cuando el operador nabla actúa como producto vectorial sobre un vector se le llama \textbf{\emph{rotacional}}.


$$\vec F \text{ es conservativo } \leftrightarrow \overrightarrow{\grad } \times \vec F = 0$$ 

\emph{Un campo de fuerzas es conservativo si su rotacional es cero en todos los puntos.}

\textcolor{gris}{Por ejemplo, el campo $\vec F=3x\vec i+5\vec j+7\vec k \to \overrightarrow{\grad } \times \vec F = \left| \begin{matrix} \vec i&\vec j&\vec k \\ \pdv{x}&\pdv{y}&\pdv{z} \\ 3x&5&7  \end{matrix} \right|=0$ y el campo es conservativo}.

\textbf{Teorema de Stookes}: Un campo es conservativo si el trabajo para desplazar la masa activa de un punto $A$ a otro $B$ es independiente del camino elegido.

\begin{equation}
\label{Th-Stookes}
	\vec F \text{\;conservativo} \quad \leftrightarrow \quad  \overrightarrow{\grad } \times \vec F = 0 \quad \leftrightarrow \quad \oint \vec F \cdot \dd \vec r =0
\end{equation}

-- La integral de curvilínea cerrada de $\vec F$ escalarmente por $\dd \vec r$ es cero.

-- El rotacional del campo de fuerzas $\vec F$ es cero en todos los puntos del campo.

Al ser $\vec F=- \overrightarrow{\grad} \mathcal E_p = - \overrightarrow{\grad} \mathcal (E_p+ Cte)$, pues $(Cte)'=0$, tenemos que \emph{la energía potencial es un valor funcional definida salvo una constante arbitraria (origen de energía potencial).} Esa constante no influye en los cálculos pues, al calcular la fuerza la constante desparecen al derivar y al calcular el trabajo, diferencia de energías potenciales, la constante desaparece al restar. Es usual, para campos eléctricos y gravitatorios, escoger esa constante de modo que se anule para puntos alejados de la partícula que crea en campo ($\mathcal E_p \to 0$ cuando $r\to \infty$).
 
 
 \emph{Tanto la energía cinética como la potencial dependen del sistema de referencia elegido.}

\rule{150pt}{0.4pt} 

En el ejemplo anterior, $\vec F = 3x\vec i + 5 \vec j+7\vec k=\displaystyle - \overrightarrow{\grad} \mathcal E_p$, es fácil obtener que el funcional energía potencial es $\mathcal E_p=3\frac{x^2}2+5y+7j+\mathcal K$, con $\mathcal K$ la constante arbitraria.

$\mathcal E_p(2,-3,1)=6-15+7=-2$, si deseamos que el origen de potencial este en el punto $(2,-3,1)$, definiremos la energía potencial como: $E_p=3\frac{x^2}2+5y+7j-2$.

\rule{150pt}{0.4pt} 



\subsection{Energía potencial gravitacional}

La energía potencia gravitacional es la asociada al peso de un cuerpo y a su altura sobre el suelo. Consideraremos que la altura sobre la superficie de la tierra es tal que puede considerarse la fuerza peso constante [$g=g(r)$].

\vspace{30mm} %****************************************
\begin{multicols}{2}
Calculemos el trabajo que efectúa la fuerza peso (única presente) para bajar un cuerpo de masa $m$ desde la posición $y_1$ hasta la $y_2$ a $v=cte$ (proceso \textit{a)} en la figura adjunta). La fuerza es paralela al desplazamiento $\dd W=\vec F\cdot \dd \vec r$. Como la fuerza y es desplazamiento son paralelos y del mismo sentido ($+\vec j$):

\begin{figure}[H]
		\centering
		\includegraphics[width=.25\textwidth]{imagenes/imagenes03/T03IM54.png}
		\end{figure}
\end{multicols}

 $W=F\cdot \Delta y=mg(y_1-y_2)>0$ (movimiento en eje $Y$).
 
 El trabajo es positivo, lo realizan las fuerzas del campo y el movimiento es espontáneo.
 
 Llamamos \emph{energía potencial (gravitacional)} de una partícula, de masa $m$ a una altura $y$ sobre el origen de energías potenciales que se tome, como:

\begin{equation}
	\label{EP-gravitacional}
	\subrayado{\mathcal E_p=mgy}
\end{equation}

Lueqo: $W_{1\to 2}=mgy_1-mgy_2=\mathcal E_{p_1}-\mathcal E_{p_2}=-(\mathcal E_{p_2}-\mathcal E_{p_1})=-\Delta \mathcal E_p$

En el caso \textit{b)} de la figura, subrir el cuerpo de la posición $2$ a la $1$, tenemos:
$\ W_{2\to 1}=-\Delta \mathcal E_p (2 \to 1)=-(\mathcal E_{p_1}-\mathcal E_{p_2})=-(mgy_1-mgy_2)=mh(y_2-y_1)<0$

Ahora, el $W<0$, debe existir una $\vec F$ externa que realice el trabajo contra las fuerzas del campo (gravitatorio).

\subsection{Energía potencial elástica} \label{Hooke}
Supongamos un muelle en el que llamamos $0$ a su posición de equilibrio y lo desplazamos una pequeña distancia $x \vec i$ de ésta (pequeña para que se verifique la Ley de Hooke). 
\begin{multicols}{2}
Actúa sobre él una fuerza que trata de devolverlo a su posición inicial: $\vec F=-K x \vec i$, con $k$ la constante de elasticidad del muelle y el signo menos denota que tiende a devolver al muelle a su posición inicial.
\begin{figure}[H]
		\centering
		\includegraphics[width=.5\textwidth]{imagenes/imagenes03/T03IM55.png}
		\end{figure}
\end{multicols}
Sabemos que $F=-\dv{\mathcal E_p}{x}=-k x \to \dd \mathcal E_p= kx \dd x$

Tomamos el origen de energía potencial en la posición $0$ de equilibrio del muelle, integrando:

$\displaystyle \int_0^{\mathcal E_p} \dd \mathcal E_p=\mathcal E_p=\int_0^x k x \dd x= \dfrac 1 2 k x^2$

Para un muelle de constante $K$ alejado una distancia $x$ de su posición de equilibrio, \emph{llamamos energía potencial elástica} a la expresión:

\begin{equation}
	\label{EP-gravitacional}
	\subrayado{\mathcal E_p=\dfrac 1 2 k x^2}
\end{equation}

Analicemos los casos posibles:

El muelle vuelve desde su posición $x_2$ más alejada a la posición $x_1$ más próxima a la posición de equilibrio:

$W_{2 \to 1}=-\Delta \mathcal E=-(\mathcal E_{p_1}-\mathcal E_{p_2}=\mathcal E_{p_2}-\mathcal E_{p_1}=\dfrac 1 2 k x_2^2- \dfrac 1 2 k x_1^2=\dfrac 12 k (x_2^2-x_1^2)>0 \quad (x_2>x_1>0)$, por lo que el sistema evoluciona libremente desde $x_2$ hasta $x_1$.

El trabajo necesario para estirar al muelle desde $x_1$ hasta $x_2$ será:

$W_{1 \to 2}=-\Delta \mathcal E_p (1 \to 2)=-(\mathcal E_{p_2}-\mathcal E_{p_1}=\mathcal E_{p_1}-\mathcal E_{p_2}=\dfrac 1 2 k (x_1^2-x_2^2)<0$. Una fuerza externa debe actuar para desarrollar este trabajo contra las fuerzas del campo elástico.

\vspace{10mm} %********************************* 
\begin{figure}[H]
	\centering
	\includegraphics[width=.95\textwidth]{imagenes/imagenes03/T03IM50.png}
	\end{figure}
	
\begin{figure}[H]
	\centering
	\includegraphics[width=.95\textwidth]{imagenes/imagenes03/T03IM51.png}
	\end{figure}


\subsection{Significado físico-matemático del gradiente}
\label{potenciales-decrecientes}
\begin{multicols}{2}
Supongamos una región del campo en que $\mathcal E_p(x,y,z)=cte$, matemáticamente esto es una \emph{superficie}.
\begin{figure}[H]
		\centering
		\includegraphics[width=.5\textwidth]{imagenes/imagenes03/T03IM14.png}
		\end{figure}
\end{multicols}

$\mathcal E_p(x,y,z)=cte \to - \overrightarrow{\grad} \mathcal E_p = 0 = \vec F \cdot \dd \vec r \to \left( \; \vec F \;\parallel \; \overrightarrow{\grad}  \mathcal E_p \; \right) \; \; \bot \; \; \dd \vec r$

El vector $\vec F$ o el vector gradiente es \emph{perpendicular} a la superficie equipotencial.

Sea $\vec u_R$ un vector unitario en la dirección en que actúa la fuerza sobre la magnitud activa. Como
$\vec F = \vec u_R \; F=- \vec u_R \; \dv{\mathcal E_p}{R}\;$
para que $F>0$, $\dv{\mathcal E_p}{R} <0$. La fuerza está orientada em el sentido de potenciales decrecientes: $\mathcal E_{p_2} < \mathcal E_{p_1} \to \dd \mathcal E_p < 0$

El vector fuerza (y el vector gradiente de energía potencial) es perpendicular a las superficies equipotenciales en cada punto y está orientado hacia los potenciales decrecientes.
\vspace{-5mm}\begin{figure}[H]
		\centering
		\includegraphics[width=.4\textwidth]{imagenes/imagenes03/T03IM15.png}
		\end{figure}


\section{Problemas}

\begin{prob}
Una partícula de $100 \ \mathrm{g}$ se lanza con una velocidad inicial de $100 \ \mathrm{ms}^{-1}$ en una región del espacio que le ofrece una resistencia  $F=0.25\ v^2\ \mathrm{N}$, si $v$, velocidad instantánea de la partícula, se expresa en $\mathrm{ms}^{-1}$. Considerando el movimiento unidimensional, averiguar la velocidad que lleva la partícula cuando ha recorrido $1\ \mathrm{m}$, así como el tiempo empleado en recorrerlo.
\end{prob}

$F$ ofrece resistencia al movimiento, se opone a él: 

$F=-0.25 v^2=-Av^2$; $\quad A=0.25\ \dfrac{\mathrm{N}}{\mathrm{m}^2\mathrm{s}^2}=0.25 \dfrac{\mathrm{kg}}{\mathrm{m}}$

$F=ma\ \to \ a=\dfrac F m = \dfrac {-A} m v^2 =\displaystyle \dv{v}{t}$

$\displaystyle \int_{v_0}^v \dfrac{\dd v}{v^2}=-\dfrac A m  \int_0^t \dd t \to \ \dfrac 1 {v_0}-\dfrac 1 v =-\dfrac A m \ t $

$\boldsymbol{v(t)=\left(\dfrac 1 {v_0}+\dfrac A m\ t \right)^{-1}}$

$v(t)=\displaystyle \dv{r}{t}\to \ \int_0^r \dd r = \int_0^t \left(\dfrac 1 {v_0}+\dfrac A m\ t \right)^{-1}\dd t$

$r=\displaystyle \int_0^t \dfrac{mv_0}{m+At}\dd t= \dfrac{mv_0}{\boldsymbol {A} }\int_0^t \dfrac{\boldsymbol{ A}}{m+At} \dd t= \dfrac{mv_0}{A} \eval {\ln(m+At)}_0^t \to $

$\boldsymbol{r= \dfrac {mv_0}{A} \ln \dfrac {m+At}{m}}$


Despejando: $\displaystyle \dfrac {rA} {mv_0} =\ln \dfrac {m+At}{m}; \quad \dfrac {m+At}{m} = e^{ \dfrac {rA} {mv_0} } $

Por último: $\quad \boxed{ \displaystyle t=\dfrac {m}{A}\ \left[ e^{\dfrac{rA}{mv_0}} - 1 \right] }$

De ahí, $\quad \boxed{ v=}\left(
\dfrac 1 {v_0}+\dfrac A m \dfrac {m}{A} 
 \left[ e^{ \dfrac{rA}{mv_0} } - 1  \right] 
\right)$
$=\boxed{ \left(  \dfrac 1 {v_0} + e^{\dfrac{rA}{mv_0}} - 1 \right)^{-1} }$

Solo queda sustituir valores para obtener $\quad t=0.011\ \mathrm{s};\quad v=97.53\ \mathrm{ms}^{-1}$

\textcolor{gris}{$0.25 \mathrm{kgm}{-1}; \quad v_0=100\ \mathrm{ms}^{-1};\quad m=0.1\ \mathrm{kg};\quad r=1\ \mathrm{m}$}

%\rightline{\textsf{\textcolor{DarkBlue}{--- Inacabado ---}}}


\begin{prob}
Una cadena está sobre una mesa sin fricción con la quinta parte de su longitud colgando de un borde. La cadena tiene una masa $M$ y una longitud $L$, ?`cuánto trabajo se requiere para subir la cadena a la mesa?	
\end{prob}

\begin{multicols}{2}
La única fuerza que actúa es la de la gravedad que es una fuerza central y, por ello, conservativa $\to W= \int_A^B \dd \mathcal E_p$.

Suponemos la cadena homogénea: $\lambda=\frac M L$, con $\lambda$ la densidad lineal.
\begin{figure}[H]
	\centering
	\includegraphics[width=.5\textwidth]{imagenes/imagenes03/T03IM33.png}
	\end{figure}
\end{multicols}
Una porción infinitesimal de cadena colgante es $\dd m$ y está a una altura $y$ del borde de la mesa, con $y\in [0,l]$ (en nuestro caso $l=L/5$).

Tenemos que $\dd m=\lambda \; \dd y =\frac M L \; \dd y$. La energía potencial de este elemento de cadena es: $\dd \mathcal E_p=\dd m \; g \; y=\frac M L \ g \ y \ \dd y$.

El trabajo para subir el trozo de cadena a la mesa lo obtendremos sumando (integrando) para todos los elementos de cadena desde la posición $l$ hasta la posición $0$:

$W=\displaystyle \int_l^0 \dd \mathcal E_p=\int_l^0 \frac M L \ g \ y \ \dd y=\frac M L \ g \ \eval{y\ \dd y}_l^0=- \frac M L \ g\ \frac {l^2}{2}$

En nuestro caso, $l=L/5 \to  W=-\displaystyle \dfrac{M\ g\ L }{50}$

El sentido negativo indica que el trabajo se efectúa contra las fuerzas del campo. Hay que realizar este trabajo externo para que la cadena suba a la mesa.

\textcolor{gris}{Comprobación dimensional: $-\dfrac{M\ g\ L }{50} \to 
[\mathrm{M}][\mathrm{LT}^-{2}][\mathrm{L}]=[\mathrm{ML}^2\mathrm{T}^{-2}]\to \mathrm{W}$}


\textcolor{gris}{\textsf{Hubiésemos podido resolver el problema imaginando el trozo $L/5$ de cadena, de peso $M/5$ como una partícula de masa $M/5$ a una distancia $-\frac 1 2 \ \frac L 5=-\frac {L}{10}$ por debajo de la superficie de la mesa (toda la masa del extremo colgante de la cadena concentrada en su centro de gravedad, a la mitad de su longitud). En estas condiciones, $\mathcal E_p=m \ g \ h=\frac M 5 \ g \ \left( -\frac L {10} \right) = -\dfrac{M\ g\ L }{50}$.}}

\textbf{análisis de caso límites.} $W=-\dfrac M L g \dfrac {l^2}2$
\begin{itemize}
\item $l=L \to W=-\dfrac M L g \dfrac {L^2}2=-Mg\dfrac L 2$. Es como si toda la cadena estuviese colgando a distancia $L/2$ del borde de la mesa (lo que concuerda si la usamos la aproximación de considerarla como una partícula centrada en su c.d.g\footnote{centro de gravedad.})
\item $l=0 \to W=0$, toda la cadena está sobre la mesa y no hay que realizar ningún trabajo para subirla.
\end{itemize}


\begin{prob}
Una ametralladora dispara balas de $50\ \mathrm{g}$ con una velocidad de $10^3\ \mathrm{ms}^{-1}$. El soldado que sostiene la ametralladara en sus manos puede ejercer una fuerza meia de $180\ \mathrm{N}$ contra el arma. Determínese el número máximo de balas que puede disparar en $1 \mathrm{min}$.	
\end{prob}

$F=\displaystyle \dv{t} (mv)=\dv{m}{t} v + m \cancelto{0}{\dv{v}{t}}=\dv{m}{t} v$, la velocidad no varía al salir del  arma.

$\overline{F} \displaystyle =\dfrac 1 T \int_0^T F \dd t=\dfrac 1 T \int_0^T \dv{m}{t} \ v\  \dd t=\dfrac 1 T \int_0^T v\  \dd m= \dfrac 1 T\  v \int_0^T \dd t = \dfrac v m T$

$T=60 \ \mathrm{s}; \quad \overline{F}=180\ \mathrm{N}; \quad M=n\ m \to \ \overline{F}=\dfrac v T \ n \ m$

Luego $\ \ n=\dfrac {\overline{F} \ T}{m \ v}= \dfrac{180 \cdot 60}{10^3 \cdot 5\ 10^{-2}}=216\ $balas en un minuto.


\begin{prob}
Una partícula de masa $m$ se mueve	en línea recta sometida a una fuerza constante de módulo $F$ y a una resistencia $kv^2$, donde $v$ es la velocidad. Si en el instante inicial la partícula tiene velocidad $v_0$, ?`qué distancia habrá recorrido cuando lleve velocidad $v$? ?`Cuál es la velocidad límite de la partícula? 
\end{prob}

$2^a$ de Newton: $\displaystyle \ \ F-kv^2=ma=m \dv{v}{t}=m\dv{v}{s} \dv{s}{t}=mv\dv{v}{s}$

$(F-kv^2)\ \dd s= mv\ \dd v \to \ \ \dd s=\dfrac{mv}{F-kv^2}\dd v$, integrando:

$s=\displaystyle \int_0^s \dd s= -\dfrac m{2k} \int_{v_0}^v \dfrac{-2k\ v\ \dd v}{F-kv^2} =-\dfrac m{2k} \eval{\ln (F-kv_2)}_{v_0}^v  \to$

$\displaystyle s=\dfrac m{2k} \ \ln \dfrac{F-kv_0^2}{F-kv^2}$

La velocidad límite se alcanzará cuando: 

$\ a=cte \to F_{total}=0 \to F-kv^2=0 \Rightarrow v_L=\sqrt{\dfrac{F}{K}}$

\textcolor{gris}{De otro modo: $\ v_L \ $ se alcanzará cuando $\ s\to +\infty \ $ es decir, cuando  $\ \dfrac m{2k} \ \ln \dfrac{F-kv_0^2}{F-kv} \to + \infty$ ; lo que ocurrirá cuando el denominador tienda a cero: $F-kv^2 \to 0$, luego: $v_L=\sqrt{\dfrac{F}{K}}$}.

\begin{prob}
Un vagón cargado de arena tiene un agujero en el fondopor el que ésta sale con una rapidez constante $w=-\dd m / \dd t$. Una fuerza constante actúa sobre el vagón en la dirección de su movimiento. Determínese la ecuación del movimiento y la velocidad instantánea del vagón.	
\end{prob}

$\displaystyle w=-\dv{m}{t} \to \dd m = -w \dd t ;\ \ \int_{m_0}^m \dd m=-w \int_0^t \dd t \Rightarrow \quad m=m_0-wt$

$\displaystyle F=cte=\dv{p}{t}=\dv{mv}{t}=\dv{m}{t}v+m\dv{v}{t}=-wv+m\dv{v}{t}$

Multiplicando por $\dd t$ 

$\displaystyle F\dd t=-wv\dd t + m \dd v \to (F+wv)\dd t=(m_o-wt)\dd v$

Separando variables e integrando: 

$\displaystyle - \dfrac 1 w \int_0^t (-w)\ \dfrac{\dd t}{m_0-wt}=\dfrac 1 w \int_{v_0}^v (w) \ \dfrac{\dd v}{F+mv}$

$\displaystyle -\ln \eval{(m_0-wt)}_0^t=\ln \eval{(F+wv)}_{v_0}^v \to
\dfrac{F+wv}{F+wv_0}=\dfrac{m_0}{m_0-wt}$, 

despejando:
$\displaystyle v=\dfrac{m_0v_0+F}{m_o-wt}=\quad \dv{s}{t}$

luego: $\displaystyle \dd s= \dfrac{m_ov_0+F}{m_0-wt} \ \dd t$, integrando:

$\displaystyle s=\int_0^s \dd s= -\dfrac 1 w \int_0^t (-w) \dfrac{m_0v_0+F}{m_0-wt} \dd t \Rightarrow \quad s=\dfrac 1 w \ln \dfrac {m_0}{m_0-wt}$

\vspace{10mm} %*****************************************
\begin{prob}
	\begin{multicols}{2}
$\quad$

$\quad$

Calcúlese las fuerzas normal y tangencial que actúan sobre un proyectil lanzado horizontalmente desde una altura $h$	.
\begin{figure}[H]
		\centering
		\includegraphics[width=.45\textwidth]{imagenes/imagenes03/T03IM57.png}
		\end{figure}
	\end{multicols}
\end{prob}


$MRUA:\quad \left| \begin{matrix}\ a_x=0 \quad \  \\ \ a_y=-g \     \end{matrix} \right|
\left. \begin{matrix}  v_x=v_0 \quad \ \\ v_y=-gt  \   \end{matrix} \right|
\left. \begin{matrix} x=v_0t  \quad \quad \quad \quad \ \\ y=h-gt-\frac 1 2 g t^2     \end{matrix} \right.$

De las ecuaciones de velocidades: $v=\sqrt{v_x^2+v_y^2}=\sqrt{v_0^2+g^2t^2}$

Fuerza tangencial: $\ F_T=\displaystyle m \dv{v}{t}=\dfrac{mg^2t}{\sqrt{v_0^2+g^2t^2}}$

Para encontrar la fuerza normal usando la ecuación $\ F_N=\displaystyle m\dfrac{v^2}{\rho}$, necesitaríamos conocer el radio de curvatura en $\rho$ en cualquier instante; la partícula describe una parábola. De otro modo, podemos hacer:

$F=\sqrt{F_T^2+F_N^2}=mg \to F_N=\sqrt{mg-F_T^2}=\dfrac{mgv_0}{\sqrt{v_o^2+g^2t^2}}$

\begin{prob}
Un astronauta que construye una estación espacial empuja un bloque de masa $m_1$ con una fuerza constante $F$. Este bloque está en contacto directo con un segundo bloque de masa $m_2$. Determina la aceleración en los bloques así como el módulo de la fuerza que ejerce cada bloque sobre el otro.	
\end{prob}

\vspace{30mm} %**********************************************
\begin{multicols}{2}
\begin{figure}[H]
	\centering
	\includegraphics[width=.5\textwidth]{imagenes/imagenes03/T03IM59.png}
	\end{figure}
$\quad$
\begin{figure}[H]
	\centering
	\includegraphics[width=.5\textwidth]{imagenes/imagenes03/T03IM60.png}
	\end{figure}
\end{multicols}	

A la derecha del astronauta hemos representado los diagramas de fuerzas de cuerpo libre de los dos bloques.

Llamamos $R$, de fuerza de reacción, a $R=|\vec F_{1,2}|=|\vec F_{2,1}|$

$m_1: \quad F-R=m_1 a; \qquad \qquad m_2: \quad R=m_2 a$

De donde: $\quad a=\dfrac{F}{m_1+m_2};\qquad R=\dfrac{m_2}{m_1+m_2}F$



\begin{prob}
Analizar el efecto de la rotación de la tierra sobre un cuerpo.
\end{prob}

Llamamos $W_G$ a la fuerza gravitacional ejercida a la atracción de la tierra. Si la tierra no rotara, la aceleración que sentiría un cuerpo cerca de la superficie terrestre sería $a=W_G/m$, pero debido a la rotación, parte de de esta fuerza se ha de dedicar la fuerza normal necesaria para que el punto $A$ describa una trayectoria circular de radio $\overline{CA}=r\cos \lambda$ con velocidad angular $\omega$, $\ F_N=m\omega^2 r$. 
\begin{multicols}{2}
La diferencia $W_G-F_N$ nos da la fuerza total $W$ que actúa hacia abajo sobre la partícula de modo que la \emph{aceleración efectiva} de la gravedad que siente ésta es $g=W/m$.

Si la partícula $A$ estuviese suspendida de un hilo como una plomada, la cuerda tendría la dirección de $W$. Este es el motivo por el que las plomadas no apuntan al centro de la tierra, si ésta parase de rotar veríamos a todos los edificios inclinados. Solo en el ecuador y en los polos $W$ y $W_G$ tienen la misma dirección y solamente allí la dirección de la plomada es radial.
\begin{figure}[H]
	\centering
	\includegraphics[width=.4\textwidth]{imagenes/imagenes03/T03IM58.png}
	\end{figure}
\end{multicols}


\vspace{30mm} %**************************************

\begin{prob}.
\begin{multicols}{2}
Una persona de $80 \ \mathrm{kg}$ está de pie sobre una balanza, sujeta al suelo de un ascensor. La balanza está calibrada en $\mathrm{N}$. ?`Qué peso indicará la balanza cuando a) el ascensor se mueva con aceleración $a$ hacia arriba; b) el ascensor se mueva con aceleración $a'$ hacia abajo; c) el ascensor se mueva hacia arriba a $20\ \mathrm{ms}^{-1}$, mientras su velocidad decrece a razón de $8\ \mathrm{ms}^{-2}$; d) si se rompe el cable del ascensor.

Lo que mide la bascula es la reacción normal a la fuerza que soporta. En el dibujo aparece el peso como $\omega=mg$.
	\begin{figure}[H]
	\centering
	\includegraphics[width=.45\textwidth]{imagenes/imagenes03/T03IM21.png}
	\end{figure}
\end{multicols}
\end{prob}
--- a) $\quad \uparrow a \to \Sigma F= N-mg = ma \to N=mg+ma=m(g+a)$

--- b) $\quad \downarrow a \to \Sigma F= N-mg = -ma \to N=mg-ma=m(g-a)$

--- c) $\quad \uparrow a:\quad \Sigma F=m(g-a)=80(9.8-8)=144\; \mathrm{N}$

--- d) $\quad \downarrow a=g \to \Sigma F= N-mg = -mg \to N=0$

El peso que mide la báscula es la reacción de soporte que ofrece al hombre. En caída libre, tanto hombre como báscula notan la sensación de ingravidez. 


\footnotesize{\textsf{Peso aparente e ingravidez aparente.}}
 
\footnotesize{\textsf{Generalicemos los resultados . Cuando un pasajero de masa $m$  viaja en un elevador con aceleración $a_y$  una báscula da como peso aparente del pasajero $N=m(g+a_y)$.}}

\footnotesize{\textsf{Cuando el elevador está acelerando hacia arriba, $a_y$   es positiva y $N$ es mayor que el peso del pasajero $ \textit{w} = mg$. Si el elevador acelera hacia abajo, $a_y$   es negativa y $N$ es menor que el peso. Si el pasajero no sabe que el elevador está acelerando, sentirá que su peso cambia y, de hecho, la báscula asá lo indica.}} 

\footnotesize{\textsf{El caso extremo sucede cuando el elevador tiene una aceleración hacia abajo  $a_y=-g$, es decir, cuando está en caída libre. En este caso, $N=0$  y el pasajero siente que no tiene peso. Asimismo, un astronauta en órbita alrededor de la Tierra experimenta ingravidez aparente. En ambos casos, la persona aún tiene peso, porque actúa sobre ella una fuerza gravitacional; sin embargo, el efecto de esta condición de caída libre es el mismo que si el cuerpo estuviera en el espacio exterior sin experimentar gravedad. En ambos casos, la persona y su vehículo (elevador o nave) están cayendo juntos con la misma aceleración $g$, así que nada empuja a la persona contra el piso o las paredes del vehículo}}\normalsize{.}

\begin{prob}
Cuando una avión acelera en la pista de una aeropuerto para despegar, un viajero decide determinar la aceleración mediante un yo-yo comprobado que éste se separa de la vertcial un ángulo de $22^o$. ¿Cuál es la aceleración del avión?. Si la masa del yo-yo es de $40\ \mathrm{g}$, ?`cuál es la tensión en el hilo?	
\end{prob}

\vspace{30mm} %**************************************

\begin{multicols}{2}
La fuerza neta sobre el yo-yo es en la dirección de la aceleración y la suministra la componente horizontal de la tensión. La componente vertical equilibra el peso.

$\Sigma F_x: \ \ T \cos \theta-mg=0$

$\Sigma F_y: \ \ T \sin \theta = m a$

Despejando:  
$a=g \tan \theta; \quad T=\dfrac{mg}{\cos \theta}$

\begin{figure}[H]
	\centering
	\includegraphics[width=.5\textwidth]{imagenes/imagenes03/T03IM64.png}
	\end{figure}
\end{multicols}	
Para los datos del problema:  
$\ \ a=3.96\ \mathrm{ms}^{-2}; \quad T=0.423\ \mathrm{N}$
	
\begin{prob}
\begin{multicols}{2}
Un elevador y su carga tienen masa total de $800 \mathrm{kg}$  y originalmente está bajando a $10.0 \ \mathrm{ms}^{-1}$; se le detiene con aceleración constante en una distancia de $25.0 \mathrm{m}$. Calcule la tensión T en el cable de soporte mientras el elevador se está deteniendo.	
\begin{figure}[H]
	\centering
	\includegraphics[width=.4\textwidth]{imagenes/imagenes03/T03IM39.png}
	\end{figure}
\end{multicols}
\end{prob}
$\Sigma F_y=t-mg=ma_y \to T=m(g+a_y)$
$ \quad \begin{cases} v=v_0+at \\ x=x_0+v_0 t + \frac 1 2 a t^2  \end{cases} \to a=2\ \mathrm{m s}^{-2}$ $\qquad T=9440\ \mathrm{N}$


\vspace{10mm} %************************************
\begin{prob}
En la figura, un deslizador de masa $m_1$ se mueve sobre un riel de aire horizontal, sin fricción, en el laboratorio de física. El deslizador está conectado a una pesa de masa $m_2$ mediante un cordón ligero, flexible e inelástico que pasa por una pequeña polea sin fricción. Calcule la aceleración de cada cuerpo y la tensión en el cordón.	
\begin{figure}[H]
	\centering
	\includegraphics[width=.75\textwidth]{imagenes/imagenes03/T03IM40.png}
	\end{figure}
\end{prob}

\small{Los dos cuerpos tienen diferente movimiento, uno horizontal y el otro vertical.}

\small{Si bien las direcciones de las dos aceleraciones son distintas, sus magnitudes son iguales. Ello se debe a que el cordón no se estira; por lo tanto, los dos cuerpos deberán avanzar distancias iguales en tiempos iguales, y así sus rapideces en cualquier instante dado deberán ser iguales. Cuando las rapideces cambian, lo hacen en la misma cantidad en un tiempo dado, de manera que las aceleraciones de los dos cuerpos deben tener la misma magnitud a. Podemos expresar esta relación así $a_{1_x}=a_{2_y}=a$}

\small{Gracias a esta relación, en realidad sólo tenemos dos incógnitas: $a$ y la tensión $T$}\normalsize{.}

Deslizador: $\begin{cases} \Sigma F_x=T=m_1 a \\ \Sigma F_y=N-m_1g=0 \end{cases}$ peas: $\Sigma F_y=m_2 g-T=m_2 a$

$a=\dfrac {m_2}{m_1+m_2}g;\qquad \qquad T=\dfrac {m_1 m_2}{m_1+m_2}g$


\vspace{4mm}\textbf{Análisis de casos extremos:}
\begin{itemize}
\item $m_1=m_2 \to a=\dfrac g 2 \quad (T=mg/2)\ $.
\item $m_1=0 \to a=g \quad (T=0)\ $, $m_2$ cuerpo libre.
\item $m_2=0 \to a=0 \quad (T=0)\ $, cuerpo $m_1$ en reposo.	
\end{itemize}


\begin{prob}.
	\begin{figure}[H]
	\centering
	\includegraphics[width=1\textwidth]{imagenes/imagenes03/T03IM22.png}
	\end{figure}
\end{prob}

\vspace{6mm} %*****************************************
\small{\textsf{Se conoce como fuerza de tensión a la fuerza que, aplicada a un cuerpo elástico, tiende a producirle una tensión.}}

\small{\textsf{Las cuerdas, por ejemplo, permiten transmitir fuerzas de un cuerpo a otro. Cuando en los extremos de una cuerda se aplican dos fuerzas iguales y contrarias, la cuerda se pone tensa. Las fuerzas de tensión son, en definitiva, cada una de estas fuerzas que soporta la cuerda sin romperse}}\normalsize{.}

\begin{multicols}{2}
Observado cada cuerpo por separado (diagrama de cuerpo libre).

Elegimos el sentido de giro hacia la masa mayor $m_1$. En principio es arbitrario, si nos equivocamos saldrá $a<0$ lo que indicará que el giro es en el sentido contrario:

\noindent \small{$y \downarrow >0 \to a \downarrow > 0;\; \; y \uparrow < 0 \to a \uparrow < 0$}\normalsize{.}
\begin{figure}[H]
	\centering
	\includegraphics[width=.4\textwidth]{imagenes/imagenes03/T03IM34.png}
	\end{figure}
\end{multicols}


Aplicamos a ambos cuerpos que $\Sigma F=m \ a$:

$\left.
\begin{matrix} 
 m_1 \ g -T =m_1 \ a \\ T - m_2 \ g =m_2 \ a	
\end{matrix}
\right\} \displaystyle \to a=\dfrac{m_2 - m_1}{m_1+,_2} \ g;  \qquad T=\dfrac {2m_1m_2}{m_1+m_2}\ g $
 \textbf{Caso particular:} $m_1=m_2 \to a=0 \quad (T=mg)$, ambos cuerpos estarían en reposo.


\begin{prob}
	En la figura, el motor de peso \textit{w} cuelga de una cadena unida mediante un anillo \textit{O} a otras dos cadenas, una sujeta al techo y la otra a la pared. Calcule las tensiones en las tres cadenas en términos de \textit{w}. Los pesos de las cadenas y el anillo son despreciables.
\end{prob}
	\begin{figure}[H]
	\centering
	\includegraphics[width=1\textwidth]{imagenes/imagenes03/T03IM38.png}
	\end{figure}

motor: $\Sigma F_y=T_1+\textit{w}=0 \to T_1=\textit{w}$

anillo \textit{O}: $\begin{cases}
\Sigma F_x=T_3 \cos 60^o-T_2=0 \\ \Sigma F_y=T_3 \sin 60^o-T_1=0	
\end{cases} \hspace{-4mm} \to  \begin{cases}
 T_2=0.58 \textit{w} \\ T_3=1.16 \textit{w}	
\end{cases}$

\newpage %**************************************
\begin{prob}.
\begin{multicols}{2}
Un escalador de masa $m_2$ cae por el borde de un glaciar. Afortunadamente está sujeto mediante una cuerda a su compañero, de masa $m_1$ que lleva un piolet. Antes de que éste clave su piolet para detener el movimiento, desliza por la superficie sin rozamiento del glaciar que está inclinada un ángulo $\theta$ respecto de la horizontal. Determinar la aceleración con que cae cada persona y la tensión de la cuerda. El coeficiente de rozamiento entre el escalador del piolet y el bloque de hielo es $\mu$.
	\begin{figure}[H]
	\centering
	\includegraphics[width=.55\textwidth]{imagenes/imagenes03/T03IM61.png}
	\end{figure}
\end{multicols}
\end{prob}

\begin{multicols}{2}
\begin{figure}[H]
	\centering
	\includegraphics[width=.4\textwidth]{imagenes/imagenes03/T03IM62.png}
	\end{figure}
	
\begin{figure}[H]
	\centering
	\includegraphics[width=.4\textwidth]{imagenes/imagenes03/T03IM63.png}
	\end{figure}
La cuerda, de masa despreciable, ni se alarga ni se encoge por lo que ambos escaladores se mueven con la misma aceleración
\end{multicols}	

La dirección de movimiento es en los ejes $y'$ e $y$.

$m_1:\quad \begin{cases} \ (reposo) \quad  \Sigma F_{x'}: \quad N-m_1g\cos \theta =0 \\ \ (movto.) \quad \Sigma F_{y'}: \quad T-N\mu+m_1g\sin \theta=m_1 a \end{cases}$

$m_2: \quad \begin{cases} \ (movto.) \quad \Sigma F_y: \quad m_2g-T=m_2 a \end{cases} $

Despejando, se obtiene:

$a=\dfrac{m_1(sin \theta - \mu \cos \theta)+m_2}{m_1+m_2}\ g; \qquad  T=\dfrac {m_1m_2(1-\sin \theta+\mu \cos \theta)}{m_1+m_2}\ g$

\vspace{15mm} %********************************
\begin{prob}
Una camarera empuja una botella de salsa de tomate con masa de $0.45 \mathrm{kg}$ a la derecha sobre un mostrador horizontal liso. Al soltarla, la botella tiene una rapidez de $2.8 \mathrm{ms}{-1}$, pero se frena por la fuerza de fricción horizontal constante ejercida por el mostrador. La botella se desliza 1.0 m antes de detenerse. ?`Qué magnitud y dirección tiene la fuerza de fricción que actúa sobre la botella?
\end{prob}

\vspace{8mm} %********************************
\begin{figure}[H]
	\centering
	\includegraphics[width=.75\textwidth]{imagenes/imagenes03/T03IM37.png}
	\end{figure}
	
\vspace{-4mm} Del diagrama de fuerzas observamos que la fuerza peso $\mathcal W$ se cancela con la reacción normal del suelo $N$ en el eje y pues no hay movimiento $(a_y=0)$. En el eje $X$, la $a_x$ la encontraremos por cinemática y la única fuerza que actúa es la de rozamiento de la botella con la barra, $F$. Elegimos sentido positivo del eje $X$ el del movimiento de la botella:

$x=x_0+ v_= t + \frac 1 2 a t^2 \; \wedge \; v=v_0+a t$

$x_0=0$, $v_0=2.8\ \mathrm{m s}^{-1}$, $x=1\ \mathrm{m}$ $\quad \to a=-3.9 \mathrm{ms}^{-2}$, negativa, es un movimiento de frenado.

$\Sigma F_x=m a_x \to -F=ma $ $\quad m=0.35\ \mathrm{Kg}$ $\quad F=1.8 \mathrm{N}$

\emph{Las fuerzas de rozamiento siempre se oponen al movimiento.}

\begin{prob}.
\begin{multicols}{2}
Problema del mono y los plátanos. Un mono de $20 \mathrm{kg}$ sujeta firmemente una cuerda ligera que pasa por una polea sin fricción y está atada a un racimo de plátanos de $20 \mathrm{kg}$. El mono ve los plátanos y comienza a trepar por la cuerda para alcanzarlos. a) Al subir el mono, ¿los plátanos suben, bajan o no se mueven? b) Al subir el mono, ¿la distancia entre él y los plátanos disminuye, aumenta o no cambia? c) El mono suelta la cuerda. ¿Qué pasa con la distancia entre él y los plátanos mientras él cae? d) Antes de tocar el suelo, el mono sujeta la cuerda para detener su caída. ¿Qué sucede con los plátanos?
\begin{figure}[H]
	\centering
	\includegraphics[width=.25\textwidth]{imagenes/imagenes03/T03IM36.png}
	\end{figure}	
\end{multicols}	
\end{prob}

El mono y los plátanos tienen la misma masa y la tensión en la cuerda es la misma. Por tanto, el mono y los plátanos tendrán la misma fuerza neta y, por ello, la misma aceleración, tanto en magnitud como en dirección.

(a) Para que el mono suba, $T> mg$. Los plátanos también suben.

(b) Los plátanos y el mono se mueven con la misma aceleración y la distancia entre ellos permanece constante. 

 (c) Tanto el mono como los plátanos están en caída libre. Tienen la misma velocidad inicial y, a medida que caen, la distancia entre ellos no cambia.

(d) Los plátanos se ralentizarán al mismo ritmo que el mono. Si el mono se detiene, también lo harán los plátanos.

Conclusión: ninguna de estas acciones acerca al mono a los plátanos.
 

\begin{prob}.
	\begin{figure}[H]
	\centering
	\includegraphics[width=.8\textwidth]{imagenes/imagenes03/T03IM28.png}
	\end{figure}
\end{prob}

\vspace{-4mm} $v=cte \to MCU: \;\; a_x=a_N= {v^2} / r$

$\Sigma F_y=m a_y \to N \cos \theta - mg =0 \to N=\dfrac {mg}{\cos \theta}$

$\Sigma F_x=m a_x\to N \sin \theta = m \dfrac {v^2} r \downarrow \to \;\tan \theta =\dfrac{v^2}{rg}$

Con los datos del problema: $\theta = 22.8^o$






\begin{prob}.
	\begin{figure}[H]
	\centering
	\includegraphics[width=.75\textwidth]{imagenes/imagenes03/T03IM32.png}
	\end{figure}
\end{prob}

El trabajo que realiza el muelle (de masa despreciable) para recuperar su posición original se invierte en aumentar la velocidad del bloque (que desliza sobre el plano sin rozamiento):

$W=\displaystyle \int_0^{x_0} F \dd x=\int_0^{x_0} K x \dd x= \frac 1 2 k x_0^2$

$\Delta \mathcal E_c=\frac 1 2 m v^2 - 0$

$W=\Delta \mathcal E_c \to v=\displaystyle \sqrt{\dfrac k m }\ x_0$


\rule{150 pt}{0.4 pt}

\begin{multicols}{2}
\textbf{Fuerza centrípeta}
\small{!`CUIDADO!: Evite usar ``fuerza centrífuga’’. La figura muestra tanto un diagrama de cuerpo libre correcto para el movimiento circular uniforme (figura a) como un diagrama común incorrecto (figura b). La figura b es incorrecta porque incluye una fuerza adicional hacia afuera de magnitud $m(v^2/R)$ para ``mantener el cuerpo en equilibrio’’. Hay tres razones para no incluir tal fuerza hacia fuera, que solemos llamar \textit{ fuerza centrífuga} (`centrífuga’ significa `que se aleja del centro’). }


\small{En primer lugar, el cuerpo no está en equilibrio; está en movimiento constante con trayectoria circular. Puesto que su velocidad está cambiando constantemente de dirección, el cuerpo está acelerado. }

\small{En segundo lugar, si hubiera una fuerza adicional hacia afuera para equilibrar la fuerza hacia adentro, no habría fuerza neta y el cuerpo se movería en línea recta, no en un círculo.}

\small{Y, en tercer lugar, la cantidad $m(v^2/R)$ no es una fuerza real, no aparece en la (figura a). }

\begin{figure}[H]
	\centering
	\includegraphics[width=.35\textwidth]{imagenes/imagenes03/T03IM41.png}
	\end{figure}
\end{multicols}
\normalsize{Es} cierto que un pasajero en un automóvil que sigue una curva en un camino horizontal tiende a deslizarse hacia fuera de la curva, como si respondiera a una “fuerza centrífuga” pero, lo que realmente sucede es que el pasajero tiende a seguir moviéndose en línea recta (principio de inercia), y el lado del coche `choca’ contra el pasajero cuando el auto da vuelta. En un marco de referencia inercial no existe ninguna ``fuerza centrífuga’’. No volveremos a mencionar este término, y le recomendamos no usarlo nunca. 

\rule{150 pt}{0.4 pt}

\begin{prob}
Un trineo con masa de $25.0\ \mathrm{kg}$ descansa en una plataforma horizontal de hielo prácticamente sin fricción. Está unido con una cuerda de $5.00\ \mathrm{m}$ a un poste clavado en el hielo. Una vez que se le da un empujón, el trineo da vueltas uniformemente alrededor del poste. Si el trineo efectúa cinco revoluciones completas cada minuto, calcule la fuerza $F$ que la cuerda ejerce sobre él.	
\end{prob}
\begin{multicols}{2}
$MCU: $

$a=a_n=\dfrac {v^2}R=\omega^2 R = $

$=\left(\dfrac {2 \pi}{T}\right)^2 R $

$F=m \ a_N = 34.3 \ \mathrm{N}$
\begin{figure}[H]
	\centering
	\includegraphics[width=.3\textwidth]{imagenes/imagenes03/T03IM42.png}
	\end{figure}
\end{multicols}

\begin{prob}
	Un inventor propone fabricar un reloj de péndulo usando una lenteja de masa $m$ en el extremo de un alambre delgado de longitud $L$. En vez de oscilar, la lenteja se mueve en un círculo horizontal con rapidez constante $v$, con el alambre formando un ángulo constante $\beta$ con la vertical. Este sistema se llama \textit{péndulo cónico} porque el alambre suspendido forma un cono. Calcule la tensión $F_T$ en el alambre y el periodo $T$ (el tiempo de una revolución de la lenteja) en términos de $\beta$.
\end{prob}
\begin{figure}[H]
	\centering
	\includegraphics[width=.8\textwidth]{imagenes/imagenes03/T03IM43.png}
	\end{figure}

$a_{rad}=A_N=\omega^2 R=\left( \dfrac {2\pi}{T} \right)^2 R;\qquad L=R \sin \beta; \quad F_T=T$

$\left. \begin{array}{ll}
\Sigma F_x=F \sin \beta = m a_N \\ \Sigma F_y=F \cos \beta -mg=0
 \end{array}\right\} \quad F=\dfrac{mg}{\cos \beta}; \quad \tan \beta=\dfrac{a_N}{g} $

$ a_N=\dfrac{4 \pi L^2 \sin \beta}{T^2}; \quad \tan \beta =\dfrac{4 \pi L^2 \sin \beta}{gT^2}\; \to \;  \quad T=2\pi \ \sqrt{\dfrac{L \cos \beta}{g}}$

\textbf{caso particular:} $\beta=0^o \to T=2\pi \sqrt{\dfrac L g}$, periodo del péndulo simple. \\ %línea en blanco ****************

\begin{prob}.

\begin{multicols}{2}
$\quad$

Un automóvil deportivo va por una curva sin `peralte' de radio $R$. Si el coeficiente de fricción estática entre los neumáticos y la carretera es $\mu_s$, ¿cuál es la rapidez máxima $v_{max}$ con que el conductor puede tomarse la curva sin derrapar?	
\begin{figure}[H]
	\centering
	\includegraphics[width=.5\textwidth]{imagenes/imagenes03/T03IM44.png}
	\end{figure}
\end{multicols}
\end{prob}

$\Sigma F_x= m a_N = m \dfrac {v^2}{R} \quad \quad  \Sigma F_y=N-mg=0
  $
 
 La primera ecuación muestra que la fuerza de fricción necesaria para mantener el auto en su trayectoria circular aumenta con la rapidez del auto. No obstante, la fuerza máxima de fricción disponible es
 $F_{max}=\mu_s N =\mu_s m g$, lo cual determinará la velocidad máxima del coche:
 
 $\mu_s m g= m \dfrac {v^2}{R} \to v_{max}=\sqrt{\mu_s g R}$
 
 \small{\textsf{Si la velocidad del coche es menor que $msgR$, la fuerza de fricción requerida es menor que el valor máximo $f_{max} = \mu_s m g$ y el coche puede tomar la curva fácilmente. Si tratamos de tomar la curva con una rapidez mayor que la máxima, el auto aún podrá describir un círculo sin derrapar, pero el radio será mayor y se saldrá de la carretera.}}
 
 \small{\textsf{Cabe señalar que la aceleración centrípeta (normal) máxima es $\mu_s g$. Si se reduce el coeficiente de fricción, la aceleración centrípeta máxima y $v_{max}$ también se reducen. Por ello, es mejor tomar las curvas a menor rapidez si el camino está mojado o cubierto de hielo (pues ambas cuestiones reducen el valor de $\mu_s$)}}\normalsize{.}
 
 \begin{prob}
 Para un automóvil que viaja a cierta rapidez, es posible ``peraltar'' una curva con un ángulo tal que los coches que viajan con cierta rapidez no necesiten fricción para mantener el radio con que dan vuelta. El auto podría tomar la curva aun sobre hielo húmedo. (Las carreras de trineos se basan en la misma idea). Un ingeniero propone reconstruir una curva de modo que un automóvil con rapidez $v$ pueda dar la vuelta sin peligro aunque no haya fricción. ?`Qué ángulo de peralte b debería tener la curva?	
 \end{prob}
 \vspace{-5mm}%*******************************************
\begin{figure}[H]
	\centering
	\includegraphics[width=.9\textwidth]{imagenes/imagenes03/T03IM45.png}
	\end{figure}
 
 
 $\left. \begin{array}{ll}
\Sigma F_x= N\sin \beta=m a_N = m \dfrac {v^2}{R} \\ \Sigma F_y=N\cos \beta-mg=0
 \end{array}\right\} \quad N=\dfrac{mg}{\cos \beta};\qquad \tan \beta=\dfrac{a_N}{g}$
 
 $a_N=\dfrac {v^2}{R} \to \quad \tan \beta = \dfrac {v^2}{gR};\qquad \beta=\beta(v,R)$
 

 \begin{prob}.
 \begin{figure}[H]
	\centering
	\includegraphics[width=1\textwidth]{imagenes/imagenes03/T03IM46.png}
	\end{figure}
 \end{prob}
 

\begin{prob}
Un disco de hockey se desliza sobre una mesa de aire, sin fricción; sus coordenadas son $(x,y)$ y sobre él actúa una fuerza conservativa descrita por la función de energía potencial	: $\mathcal E_p(x,y) )= \frac 1 2 k(x^2+y^2)$. Deduzca una expresión para la fuerza que actúa sobre el disco y obtenga una expresión para el módulo de la fuerza en función de la posición.
\end{prob}

En dos dimensione: $\ F_x=-\pdv{x}\mathcal E_p=-Kx; \quad F_y=-\pdv{y}\mathcal E_p=-Ky$

Luego: $\overrightarrow{F}=F_x \vec i + F_y \vec j = -k \ (x\vec i + y \vec j)=-k \ \overrightarrow{r}$

\textcolor{gris}{La fuerza es central, $\overrightarrow{F}=-K \vec r$, luego es conservativa.}

$F=\abs{\vec F}=\sqrt{F_x^2+F_y^2}=k\sqrt{x^2+y^2}=k\ r$, con $r$ la distancia del disco al origen.

\begin{prob}
En cierta región del espacio, la fuerza que actúa sobre un electrón es $\vec F=  Cx\vec j$, donde $C$ es una constante positiva. El electrón se mueve en sentido antihorario en un cuadrado sobre el plano $xy$ . Las esquinas del cuadrado están en $(x, y):\  (0, 0),\ (L, 0),\ (L, L) \text{ y } (0, L)$. Calcule el trabajo de $\vec F$ sobre el electrón durante una vuelta. ¿Esta fuerza es conservativa o no conservativa?	
\end{prob}

\begin{multicols}{2}
$W=\displaystyle \int_{cuadrado}\vec F \cdot \dd \vec l$. 

Calcularemos el trabajo tramo a tramo y sumaremos todos los resultados obtenidos.
De $(0,0)\ a \ (L,0)$, el $W=0$, pues $\vec F \bot \dd \vec l$, lo mismo ocurre para el trabajo desde $(L,L)$ hasta $(=,L)$.

Desde $(=,L)$ hasta $(0,0)$, también es $W=o$, pues $\vec F=0$
	\begin{figure}[H]
	\centering
	\includegraphics[width=.4\textwidth]{imagenes/imagenes03/T03IM53.png}
	\end{figure}
\end{multicols}
Trabajo desde $(L,O)$ hasta $(L,L)$. En este tramo la fuerza $\vec F$ es paralela al desplazamiento $\dd \vec l= \dd y\ \vec j$, luego $\vec F \cdot \dd \vec l = C\ L \ \dd y$

Así, $\ \displaystyle W_{(L,0) \to (L,L)} = \int_C \vec F \cdot \dd \vec l = \int_{y=0}^{y=L} C L \ \dd y= C L \ \left[ \eval{y}_0^L \right.=C L^2$

$W=0+CL^2+0+0=CL^2$

Como el trabajo desarrollado a lo largo de un camino cerrado (espira cuadrada) no ha resultado ser cero, entonces, la fuerza no es conservativa (no se deriva de una función escalar --potencial.) 


\begin{prob}
Curva peraltada con rozamiento.	
\end{prob}

Las fuerzas que aparecen son el peso $mg$, la reacción normal del plano $\vec N$

La aceleración normal $a_n$ no es paralelo al plano, es horizontal. Ver figura adjunta:

\begin{figure}[H]
	\centering
	\includegraphics[width=1\textwidth]{imagenes/imagenes03/T03IM56.png}
	\end{figure}

\noindent $\Sigma F_y:\;\; N\cos \theta-F_R \sin \theta -mg=0;\qquad \Sigma F_x:\;\; N\sin \theta+F_R \cos \theta =m \dfrac {v^2}{R}$

El vehículo deslizará en la dirección radial cuando su velocidad sea tal que $F_R=\mu N$:

$F_R=\mu N \ \to \ \begin{cases} \ N(\cos \theta-\mu \sin \theta)=mg \\ \ N(\sin \theta+\mu \cos \theta)=mv^2/R \end{cases} $


De donde, la velocidad máxima que puede llegar el vehículo es: 

$$ v=\sqrt{Rg \ \dfrac {\sin \theta + \mu \cos \theta} {\cos \theta - \mu \sin \theta}} $$

\textbf{Caso límite:} Si no hay rozamiento, $\mu=0 \to v= \sqrt{RG\tan \theta}$, ecuación que ya habíamos obtenido en un problema anterior al analizar una curva peraltada sin rozamiento.


\begin{prob}
Una partícula de $8 \ \mathrm{g}$  de masa se deja caer desde el reposo bajo la influencia de su propio peso	 en un medio en que existe una fuerza retardadora $\vec F=-9.8\times 10^{-3}\ \vec v\  \mathrm{N}$ , si $v$ se expresa en $\mathrm{ms}^{-1}$. Al cabo de $3\ \mathrm{s}$  toca el suelo. ?`A qué velocidad lo hace y desde qué altura se soltó?
\end{prob}

\begin{multicols}{2}
$v_0=0;\ t=3 \ \mathrm{s};\ m=8\times 10^{-3}\ \textrm{Kg}$

$\overrightarrow{F_R}=9.8\times 10^{-3}\ \vec v=Av\ \vec j$

$ A=9.8\times 10^{-3} \dfrac{\mathrm{N}}{\mathrm{ms}^{-1}}=$

$=9.8\times 10^{-3}\ \mathrm{kgs}^{-1}$
\begin{figure}[H]
	\centering
	\includegraphics[width=.25\textwidth]{imagenes/imagenes09/T09IM11.png}
\end{figure}		
\end{multicols}
$F_T=F_R-mg=Av-mg=ma \to a=\dfrac Am v - g =\displaystyle \dv{v}{t}$

$\boldsymbol{\frac m A} \displaystyle  \ \int_{v_0=0}^{v} {\dfrac{\boldsymbol{\frac A m} \dd v}{\frac A m v - g}}=\int_0^t \dd t \to $ 
$\quad \frac m A  \displaystyle \eval{ \ln(\frac A m v - g) }_0^v=t$

$\dfrac m A \ln \left( 1-\dfrac{Av}{mg} \right)=t \ \to \quad \boxed{\boldsymbol{v=\dfrac{mg}{A} \left( e\ ^{\dfrac {At}{m} }-1 \right)}}$

$v=\displaystyle \dv{y}{t} \to \dd y= v\ \dd t; \qquad t=0 \to y(0)=0;\quad t\to y=0$

$-H=\displaystyle \int_{H}^0 \dd y = \int_0^t \dfrac{mg}{A} \left( e\ ^{\dfrac {At}{m} }-1 \right) \dd t =
   \boldsymbol{\dfrac m A} \dfrac{mg}{A}  \int_0^t  \boldsymbol{\dfrac A m} e\ ^{\dfrac {At}{m} } \dd t  \ -
   \dfrac{mg}{A} \int_0^t  \dd t 
$

$\boxed{ \boldsymbol{ H= \dfrac{mg}{A} \left[ t - \dfrac m A \left( e^{\dfrac{At}{m}}-1 \right) \right] } }$

Sustituyendo los datos del problema, obtenemos: 

$v= \cdots \ \mathrm{ms}^{-1}; \qquad H= \cdots \ \mathrm{m}$

\rightline{\textsf{\textcolor{DarkBlue}{--- Inacabado --- \small{comporbar soluciones}\normalsize{.}}}}



\newpage %********************************

 \begin{myblock}{Fuerzas fundamentales de la naturaleza}
 
 \begin{small}
 Hemos visto fuerzas de varios tipos ---peso, tensión, fricción, resistencia de fluidos y la fuerza normal--- y veremos otras más al seguir estudiando física. Pero, ?`cuántas clases distintas de fuerzas hay? Actualmente, se considera que todas las fuerzas son expresiones de tan solo cuatro clases de fuerzas o interacciones fundamentales entre las partículas. Dos de ellas las conocemos por la experiencia cotidiana; las otras dos implican interacciones entre partículas subatómicas que no podemos observar directamente con nuestros sentidos. 

\vspace{2mm} Las \textbf{interacciones gravitacionales} incluyen la fuerza familiar del peso, que se debe a la acción de la atracción gravitacional terrestre sobre un cuerpo. La mutua atracción gravitacional entre las diferentes partes de la Tierra mantienen a nuestro planeta unido. Newton demostró que la atracción gravitacional del Sol mantiene a la Tierra en su órbita casi circular en torno al Sol. 

\vspace{2mm} La otra clase cotidiana de fuerzas, las \textbf{interacciones electromagnéticas}, incluye las fuerzas eléctricas y magnéticas. Si nos frotamos un peine por el cabello, al final el peine tendrá una carga eléctrica; es posible usar la fuerza eléctrica para atraer trocitos de papel. Todos los átomos contienen carga eléctrica positiva y negativa, así que átomos y moléculas pueden ejercer fuerzas eléctricas unos sobre otros.  Las fuerzas de contacto, incluidas la normal, la de fricción y la de resistencia de fluidos, son la combinación de todas estas fuerzas ejercidas sobre los átomos de un cuerpo por los átomos de su entorno. Las fuerzas magnéticas, como las que se dan entre imanes o entre un imán y un trozo de hierro, son realmente el resultado de cargas eléctricas en movimiento. Por ejemplo, un electroimán causa interacciones magnéticas porque las cargas eléctricas se mueven por sus alambres. 

\vspace{2mm} En el nivel atómico o molecular, las fuerzas gravitacionales no son importantes porque las fuerzas eléctricas son muchísimo más intensas: la repulsión eléctrica entre dos protones a cierta distancia es $10^{35}$ veces más fuerte que su atracción gravitacional. Sin embargo, en cuerpos de tamaño astronómico las cargas positivas y negativas suelen estar presentes en cantidades casi idénticas, y las interacciones eléctricas resultantes casi se anulan. Por ello, las interacciones gravitacionales son la influencia dominante en el movimiento de los planetas y en la estructura interna de las estrellas. 

\vspace{2mm} Las otras dos clases de interacciones son menos conocidas. La \textbf{interacción fuerte} mantiene unido el núcleo de un átomo. Los núcleos contienen neutrones (eléctricamente neutros) y protones (con carga positiva). La fuerza eléctrica entre protones hace que se repelan mutuamente; la enorme fuerza de atracción entre las partículas nucleares contrarresta esta repulsión y mantiene el núcleo estable. En este contexto, la interacción fuerte también se denomina fuerza nuclear fuerte; tiene un alcance mucho menor que las interacciones eléctricas, pero es mucho más fuerte dentro de ese alcance. La interacción fuerte juega un papel fundamental en las reacciones termonucleares que ocurren en el núcleo del Sol, y que generan el calor y su luz. 

\vspace{2mm} Por ultimo, tenemos la \textbf{interacción débil} cuyo alcance es tan pequeño que es relevante sólo a una escala de núcleo o menor. La interacción débil causa una forma común de radioactividad, llamada desintegración beta, en la que un neutrón de un núcleo radioactivo se transforma en protón al tiempo que expulsa un electrón y una partícula casi sin masa llamada antineutrino electrónico. La interacción débil entre un antineutrino y la materia ordinaria es tan tenue que el antineutrino fácilmente podría atravesar una pared de plomo ¡`de un millón de kilómetros de espesor! 

\vspace{2mm} En la década de 1960 los físicos elaboraron una teoría que describe las interacciones electromagnética y débil, como aspectos de una sola interacción electrodébil. Esta teoría ha superado todas las pruebas experimentales a las que se ha sometido, lo cual motivó a los físicos a realizar intentos similares que describan las interacciones fuerte, electromagnética y débil dentro de una sola gran teoría unificada (GUT), y se han dado ciertos pasos hacia una posible unificación de todas las interacciones en una teoría de todo (TOE). Tales teorías aún son especulativas, y hay muchas preguntas sin respuesta en este campo de investigación tan activo. 
	
 \end{small}

\end{myblock}
 
 
 
 
 

\include{TEMA04_chapter-A4}
\chapter{Estática de la partícula libre}	


\begin{miparrafo}
La estática es la parte de la mecánica que estudia el punto material sobre el que actúan fuerzas y momentos totales  cuyas resultantes son nulas, de forma que permanece en reposo o en movimiento rectilíneo y uniforme ($\vec v=\overrightarrow{cte}$), en equilibrio.	
\end{miparrafo}


\section{Estática del punto material. Ligaduras}

Si un \textbf{punto material, masa puntual o partícula} no tiene su movimiento impedido por ninguna restricción, decimos que es un \textbf{punto libre}. Para definir su posición en un espacio tridimensional se requiere el conocimiento de tres coordenadas, diremos que tiene tres grados de libertad, pues tiene la posibilidad de desplazarse en las tres direcciones de los ejes $X$, $Y$ y $Z$. 

Si el punto material tiene alguna limitación en su movilidad, se dice que es un \emph{punto ligado o vinculado}, y a la causa de esa limitación, se la denomina ligadura, vínculo, o \emph{enlace}.

Un punto obligado a permanecer en una línea está ligado a la misma. Su posición puede ser definida ahora por un único parámetro. Ahora el punto sólo tiene un grado de libertad, esto es, únicamente tiene la posibilidad de desplazarse en la dirección tangencial de la línea. 

Otro ejemplo: un punto obligado a permanecer en una superficie. Su posición se puede definir con dos parámetros, y tendrá dos grados de libertad. 


\colorbox{LightYellow}{\textbf{Ligadura}} de un punto material (o de un cuerpo) es cada una de las \colorbox{LightYellow}{condiciones} que se le imponen que \colorbox{LightYellow}{limitan su de movimiento}.

\vspace{4mm} %**************************
Las ligaduras pueden ser:

\vspace{-2mm} %**************************
\begin{itemize}

\item \emph{Ligaduras bilaterales o completas}:	 se pueden expresar por una \emph{igualdad matemática}.

Ejemplo: $x^2+y^2+z^2=r^2 \to $ la partícula se puede mover por una suferficie esférica de radio $r$.

\item \emph{Ligaduras unilaterales o incompleta}s: se pueden expresar por medio de \emph{desigualdades matemáticas}.

Ejemplo: $x^2+y^2+z^2 \le r^2 \to $ la partícula se puede mover por el interior de una esfera de radio $r$.

\end{itemize}

\vspace{-4mm} %**************************
\begin{figure}[H]
	\centering
	\includegraphics[width=.7\textwidth]{imagenes/imagenes05/T05IM01.png}
	\caption*{Tipos de ligaturas.}
\end{figure}

\section{Principio de las fuerzas de ligadura}

Consideremos un cierto punto material, sometido a una cierta ligadura o enlace y sobre el que actúan una serie de fuerzas, cuya resultante es $\overrightarrow{F}$ . 

Si dicho punto está  ligado y por lo tanto tiene su movilidad impedida de algún  modo, no asumirá por efecto de la fuerza  $\overrightarrow{F}$ el mismo movimiento que si dicho punto estuviese libre, por lo que las ligaduras que actúan sobre los puntos materiales, pueden ser equiparables a fuerzas.
 
 De este planteamiento se deriva el principio de las fuerzas de ligadura: 

\emph{En un punto material ligado, la acción de las ligaduras,  se puede sustituir por la de una fuerza que llamaremos fuerza reactiva o de ligadura}. 

\begin{itemize}
\item Estas fuerzas reactivas o de ligadura tienen unas características propias que las diferencian de las fuerzas aplicadas: 
\item Las fuerzas de ligadura dependen de las fuerzas aplicadas. 
\item Si las fuerzas aplicadas se anulan, las fuerzas de ligadura también se anulan, incapaces por sí mismas de producir movimiento. 
\end{itemize}

\vspace{-5mm} %*******************************
$$\text{fuerza de ligadura } + \text{ fuerza aplicada }\to \text{ trayectoria de la partícula}$$



\section[Ecuaciones de equilibrio de la partícula libre]{Ecuaciones de equilibrio de la partícula libre\sectionmark{Ecuaciones de equilibrio de la partícula libre}}
\sectionmark{Equilibrio de la partícula libre}

Desde el punto de vista de la estática, un cuerpo libre es aquel que puede moverse en cualquier dirección y sentido, el que no está sujeto a ligaduras. (Dinámicamente, dos cuerpos son libres cuando la distancia que les separa es infinita, cuando no interaccionan.)

Un cuerpo se encuentra en equilibrio si las fuerzas que actúan sobre él lo mantienen inmóvil o en movimiento rectilíneo y uniforme. Las condiciones de equilibrio se deben dar respecto de un determinado observador.

Condiciones de equilibrio:
\begin{enumerate}
\item $\vec v_{observador}=\vec 0$
\item $\displaystyle \overrightarrow{F}_{total}=\vec 0=\sum_{i=1}^N \vec F_i$	

Lo que implica: $\displaystyle \sum_{i=1}^N \vec F_{x_i}=0;\quad \sum_{i=1}^N \vec F_{y_i}=0;\quad \sum_{i=1}^N \vec F_{z_i}=0$
\end{enumerate}


Como caso particular vamos a estudiar las condiciones de equilibrio de un punto material (partícula) en una región del espacio en que hay un \textbf{campo conservativo.}


Es trivial que $\vec v=\vec 0$. Vamos con la fuerza: $\overrightarrow{F}=0=-\overrightarrow{\grad}{\mathcal E_p} \to $

$$\displaystyle \pdv{\mathcal E_p}{x}=0;\quad \pdv{\mathcal E_p}{x}=0;\quad \pdv{\mathcal E_p}{z}=0;\qquad \mathcal E_p=\mathcal E_p(x,y,z)$$

\begin{miparrafodestacado}
	En un campo conservativo, una partícula está en equilibrio cuando ocupa un máximo o un mínimo de energía potencial. (Si en un punto hay un máximo o mínimo, la derivada en él es cero.)
\end{miparrafodestacado}






\begin{figure}[H]
	\centering
	\includegraphics[width=.75\textwidth]{imagenes/imagenes05/T05IM02.png}
\end{figure}

\section{Teorema de Lejeune Dirichlet}
\label{Lejeune-Dirichlet}

\emph{Siempre que el campo que actúe sobre una partícula derive de un potencial (esto es, que sea un\colorbox{LightYellow}{ campo conservativo}) la condición necesaria y suficiente para que el la partícula esté en \colorbox{LightYellow}{equilibrio} es que en dicho punto exista un \colorbox{LightYellow}{máximo o mínimo para la energía potencial}; siendo el equilibrio \colorbox{LightYellow}{inestable} en el primer caso y \colorbox{LightYellow}{estable} en el segundo.}

Se dice que un cuerpo está en \emph{equilibrio estable} cuando al separarlo de esa posición una distancia infinitesimal, aparece aplicada sobre él una fuerza que tiende a restablecer el equilibrio, a devolver a la partícula a su posición inicial.

Una partícula ocupa una posición de \emph{equlibrio inestable} cuando al separarla una cantidad infinitesimal de esa posición aparece una fuerza que tiende a alejarla de ella infinitamente.

El \emph{equilibrio} es \emph{indiferente} cuando no depende de la posición de la partícula.

Vamos a demostrar que:
\vspace{-3mm}
\begin{quotation} % SANGRADO ************************
	$\begin{array}{lcl}
\text{energía potencial es máxima }&\to&\text{ equilibrio es INESTABLE}	
\\
\text{energía potencial es mínima }&\to&\text{ equilibrio es ESTABLE}		
\end{array}$
\end{quotation}

\begin{figure}[H]
	\centering
	\includegraphics[width=.9\textwidth]{imagenes/imagenes05/T05IM03.png}
\end{figure}

Como vimos en la sección \ref{potenciales-decrecientes}, al ser la fuerza el \emph{menos} gradiente de la energía potencial, estará orientada hacia potenciales decrecientes (y es perpendicular a las superficies equipotenciales), por ello:

--- $\mathcal E_{p_1}>\mathcal E_{p_2}$, aparece una fuerza, orientada hacia potenciales decrecientes, que tiende a alejar a la partícula de la posición que ocupa.

--- $\mathcal E_{p_1}<\mathcal E_{p_2}$, aparece una fuerza, orientada hacia potenciales decrecientes, que tiende a aproxima a la partícula de la posición que ocupaba.

\section[Equilibrio de la partícula libre sometida a ligaduras]{Equilibrio sometido a ligaduras\sectionmark{Equilibrio sometido a ligaduras}}
\sectionmark{Equilibrio sometido a ligaduras}
Las ligaduras se pueden representar por \emph{fuerzas de ligaduras}.

\vspace{-3mm}$$F_{resultante}=F_{\text{aplicada, solicitante o externa}}+F_{ligaduras}$$

\vspace{-3mm}\emph{Para que una partícula ligada se encuentre en equilibrio, basta con que sean nulas las componentes de la fuerza que, de existir, darían lugar a movimientos compatibles con las condiciones de ligadura}.

\vspace{-3mm}$$\displaystyle \sum_{i=1}^N (\overrightarrow{F}_{aplicadas}+\overrightarrow{F}_{ligaduras})_i=0$$

\vspace{-3mm}Veamos dos casos particulares a modo de ejemplos:

\begin{ejem}
\textbf{Equilibrio de un punto ligado a una superficie fija}.	
\end{ejem}
\begin{figure}[H]
	\centering
	\includegraphics[width=.75\textwidth]{imagenes/imagenes05/T05IM04.png}
\end{figure}
Vamos a por las ecuaciones matemáticas que proporcionan el equilibrio:

$\overrightarrow{F}_{Resultante}= \overrightarrow{F}+\overrightarrow{N}=\vec 0 \Rightarrow \begin{cases} F_x+N_x=0\\F_y+N_y=0\\F_z+N_z=0\end{cases}$. Suponemos $\overrightarrow{F}_R=\overrightarrow{cte}$.

$f$ es la ecuación de la superficie, $f=cte \to \dd f=0 \quad \textcolor{gris}{(1*)}$:

$\displaystyle \dd f=\pdv{f}{x} \dd x+\pdv{f}{y} \dd y+\pdv{f}{z} \dd z=0\quad \textcolor{gris}{\leftarrow (1*)}$

Ecuación que vectorialmente se escribe así:

$\displaystyle \left( \vec i \ \pdv{f}{x} + \vec j \ \pdv{f}{y} +\vec k \ \pdv{f}{z} \right) \cdot \left( \vec i \ \dd x+ \vec j \ \dd y + \vec k \ \dd z \right)=0$


El primer vector le llamaremos $\overrightarrow{\lambda}=\overrightarrow{\grad}f$ y, al ser el gradiente de la función escalar que da la superficie, es perpendicular a ésta, paralelo a su vector normal: $\overrightarrow{\lambda}\ \parallel \ \overrightarrow{N}$. 

Un vector unitario en la dirección de la normal lo podemos obtener como $\vec u_N=\dfrac{\overrightarrow{\lambda}}{\lambda}$. Con esto: $\quad \overrightarrow{\lambda}=\lambda \ \vec u_N$

El segundo vector es $\dd \vec r$ y coincide con la dirección tangente del movimiento de la partícula por la superficie $S$ debido a la acción de la fuerza $F_T$.

$\vec u_N=\displaystyle \dfrac { \vec i \ \pdv{f}{x} + \vec j \ \pdv{f}{y} +\vec k \ \pdv{f}{z} }{\sqrt{ \left(\pdv{f}{x} \right)^2 + \left(\pdv{f}{y} \right)^2 + \left(\pdv{f}{z} \right)^2 }} =  \dfrac{\overrightarrow{\lambda}}{\lambda}= \vec i \ \dfrac 1 \lambda \ \pdv{f}{x} + \vec j \ \dfrac 1 \lambda \ \pdv{f}{y} +\vec k \ \dfrac 1 \lambda \ \pdv{f}{z}$


$\displaystyle \overrightarrow{N}=N\ \vec u_N=\vec i \ N_x+\vec j \ N_y+\vec k \ N_z= \dfrac N \lambda \ \pdv{f}{x} \ \vec i +  \dfrac N \lambda \ \pdv{f}{y} \ \vec j + \dfrac N \lambda \ \pdv{f}{z} \ \vec k$

$\ \displaystyle N_x=\dfrac N \lambda \pdv{f}{x} = n\ \pdv{f}{x}; \quad \displaystyle N_y=\dfrac N \lambda \pdv{f}{y} = n\ \pdv{f}{y}; \quad \displaystyle N_z=\dfrac N \lambda \pdv{f}{z} = n\ \pdv{f}{z}$, donde $\dfrac N \lambda = cte = n$

\vspace{30mm} %******************************************

\begin{multicols}{2}
$\quad$

$\quad$

Las ecuaciones de equilibrio son: 

$\quad$

(4 ecuaciones con 4ingógnitas $x,y,x$ del punto de equilibrio y $n$)

$$\displaystyle
\begin{cases}
\displaystyle \ F_x+n\ \pdv{f}{x}=0 \\ \\ \displaystyle \ F_y+n\ \pdv{f}{y}=0 \\  \\ \displaystyle \ F_z+n\ \pdv{f}{z}=0 \\ \\ \ f(x,y,z)=cte
\end{cases}$$
\end{multicols}


\begin{ejem}
\textbf{Equilibrio de un punto ligado a una curva fija}.	
\end{ejem}
\begin{multicols}{2}
	
Una curva es la intersección de dos superficies:

$Curva: \begin{cases} \ f_1(x,y,z)=\mathcal C_1 \\ \ f_2(x,y,z)=\mathcal C_2 \end{cases}$

$Equilibrio \ \leftrightarrow \ \overrightarrow{F}_{resultante}=\vec 0$

$\overrightarrow{F}+\overrightarrow{N}_1 + \overrightarrow{N}_2=\vec 0$

$\begin{cases}\ F_x+N_{1_x}+N_{2_x}=0  \\  \ F_y+N_{1_y}+N_{2_y}=0 \\ \ F_z+N_{1_z}+N_{2_z}=0 \end{cases}$
 
\begin{figure}[H]
	\centering
	\includegraphics[width=.55\textwidth]{imagenes/imagenes05/T05IM05.png}
\end{figure}
\end{multicols}

Por un razonamiento matemático similar al anterior, las ecuaciones de equilibrio en este caso estan formadas por 5 ecuaciones con 5 incóngintas, $x,y,z,n_1,n_2$.

$$\displaystyle
\begin{cases}
\displaystyle \ F_x+n_1 \ \pdv{f_1}{x} + \ n_2\ \pdv{f_2}{x}=0 \\ \\ \displaystyle \ F_y+n_1\ \pdv{f_1}{y} + \ n_2\ \pdv{f_2}{y}=0  \\  \\ \displaystyle \ F_z+n_2\ \pdv{f_1}{z} + \ n_2\ \pdv{f_2}{z}=0  \\ \\ \ f_1(x,y,z)=\mathcal C_1 \\ \\f_2(x,y,z)=\mathcal C_2
\end{cases}$$

\section{Problemas}
\begin{prob}
Expresar matemáticamente el hecho de que una partícula se encuentre limitada a moverse en la región interna de un paralelepípedo de aristas $a$, $b$ y $c$. ?`A qué tipo de ligadura está sometida?	
\end{prob}
\begin{multicols}{2}
Colocamos el origen de nuestro sistema de referencia en uno de los vértices del paralelepípedo.

Ligadura: $\ \ \begin{cases} \ x<a \\ \ y<b \\ \ z < c \end{cases}$

La ligadura es incompleta o bilateral.
\begin{figure}[H]
	\centering
	\includegraphics[width=.4\textwidth]{imagenes/imagenes05/T05IM06.png}
\end{figure}
\end{multicols}

\begin{prob}
Determínense las posiciones de equilibrios de una partícula libre atraída por $n$ centros fijos $\mathcal O_1,\mathcal O_2, \cdots , \mathcal O_n$, en razón inversa de la distancia.
\end{prob}
\begin{multicols}{2}
Tenemos campos centrales $\vec F=\dfrac k r \ \vec u_r$, luego son conservativas, se pueden obtener como el menos gradiente de la energía potencial.

Buscamos las coordenadas $(x,y,z)$ del punto de equilibrio.

\begin{figure}[H]
	\centering
	\includegraphics[width=.4\textwidth]{imagenes/imagenes05/T05IM07.png}
\end{figure}
\end{multicols}

$\vec F_{res}=\vec 0=-\vec {\grad} \mathcal E_p= $

$=\displaystyle \pdv{\mathcal E_p}{x} \ \vec i + \pdv{\mathcal E_p}{y} \ \vec j + \pdv{\mathcal E_p}{z} \ \vec k $

Fuerza que ejerce sobre la partícula $m$ el centro atractivo $\mathcal O_i$

$\vec F_i=\dfrac K r_i \vec u_i = \dfrac k {\sqrt{(x-x_i)^2+(y-t_i)^2+(z-z_i)^2}} \ \vec u_i=$

$\vec u_i= \vec i \ \cos \alpha_i + \vec j \ \cos \beta_i + \vec k \ \cos \gamma_i$

Componentes de $\vec u_i$, cosenos directores de $\vec r_i$:

$\cos \alpha_i=\dfrac {x-x_i}{\sqrt{(x-x_i)^2+(y-t_i)^2+(z-z_i)^2}}; $ análogamente $\cos \beta_i$ y $\cos \gamma_i$.

\begin{comment}
Ecuación extralarga, la corto como más abajo, con 'eqnarray*'
$\vec F_i=k  
\vec i \ \dfrac {x-x_i}{\sqrt{(x-x_i)^2+(y-t_i)^2+(z-z_i)^2}} + 
\vec j \ \dfrac {y-y_i}{\sqrt{(x-x_i)^2+(y-t_i)^2+(z-z_i)^2}} + 
\vec k \ \dfrac {z-z_i}{\sqrt{(x-x_i)^2+(y-t_i)^2+(z-z_i)^2}}  
$	
\end{comment}


\begin{eqnarray*}
\vec F & = & \vec i \ \dfrac {x-x_i}{\sqrt{(x-x_i)^2+(y-t_i)^2+(z-z_i)^2}} \ +  \nonumber \\
    & & \vec j \ \dfrac {y-y_i}{\sqrt{(x-x_i)^2+(y-t_i)^2+(z-z_i)^2}} \ + \nonumber \\
  & & \vec k \ \dfrac {z-z_i}{\sqrt{(x-x_i)^2+(y-t_i)^2+(z-z_i)^2}}  \ = \ \nonumber \\
  -\vec{\grad} \mathcal E_p &= & -\vec i \ \pdv{\mathcal E_p}{x}+ \vec j \ \pdv{\mathcal E_p}{y}+ \vec k \ \pdv{\mathcal E_p}{z}
\end{eqnarray*}


Como $\displaystyle \pdv{\mathcal E_p}{x}=0 \to \sum_{i=1}^N 
\dfrac {x-x_i}{\sqrt{(x-x_i)^2+(y-t_i)^2+(z-z_i)^2}}=0$

$\displaystyle  \to \sum_{i=1}^N x-x_i=0=\sum_{i=1}^N x - \sum_{i=1}^N x_i \to \ \ Nx-\sum_{i=1}^N x_i=0$

$\displaystyle Nx=\sum_{i=1}^N x_i \to \quad x=\dfrac 1 N \sum_{i=1}^N x_i$ y análgogamente para las otras coordenadas del punto de equilibrio.


Luego la posición de equilibrio de la partícula sometida a la acción de estos $N$ centros de atracción es:

$$ x= \dfrac 1 N \ \sum_{i=1}^N x_i; \quad y= \dfrac 1 N \ \sum_{i=1}^N y_i; \quad z= \dfrac 1 N \ \sum_{i=1}^N z_i$$




\newpage %**************************************

\begin{myblock}{Estática en la ingeniería.}

Las aplicaciones prácticas de la estática en la ingeniería son muy numerosas, siendo quizás la parte de la mecánica más empleada. Esto es así especialmente en la ingeniería civil y en el análisis estructural: por lo general las estructuras se diseñan para estar y permanecer en reposo bajo las cargas de servicio estáticas, o para que su movimiento bajo cargas dinámicas sea pequeño y estable (vibraciones).
	
\end{myblock}




\chapter{Equilibrio del sólido rígido. Principio de los trabajos virtuales}	
\chaptermark{Equilibrio del sólido rígido}

\section{Equilibrio de los sistemas de partículas}

Las fuerzas externas que actúan son debidas al lugar que ocupa el sistema de partículas en el universo, son debidas a campos exteriores.

Las fuerzas interiores son las que se ejercen entre sí las partículas del sistema. Se supone que para estas fuerzas se cumple la tercera ley de Newton, la ley de acción y reacción:

$$ \overrightarrow{F}_{ij}=- \overrightarrow{F}_{ji} \ ; \qquad \sum_i \sum_j  \overrightarrow{F}_{ij}=0 $$

La condición de equilibrio es que la  $\overrightarrow{F}_{resultante}$ de todas las partículas del sistema sea cero. (Hemos llamado $\overrightarrow{F}_i' $ a las fuerzas de ligadura.)

$$\sum_j  \overrightarrow{F}_{ji} +  \overrightarrow{F}_i^{\ externas}+  \overrightarrow{F}_i'=\vec 0= \overrightarrow{F}_i+ \overrightarrow{F}_i'$$


Se han de satisfacer $N$ ecuaciones como la siguiente, una para cada una de las $N$ partículas que forman el sistema, con $\overrightarrow{F}_{ji} +  \overrightarrow{F}_i^{\ externas}=\overrightarrow{F}_i\ $. 

\section{Desplazamiento virtual}

Por \emph{desplazamiento virtual} infinitesimal de un sistema se entiende una variación de su configuración como resultado de cualquier cambio infinitesimal arbitrario $\ \boldsymbol{\var \vec r_i} \ $de las coordenadas de las partículas compatible con las fuerzas y ligaduras impuestas al sistema en un instante dado $ t $. Es un cambio infinitesimal del sistema de coordenadas que ocurre mientras el tiempo se mantiene fijo. Se le llama \emph{virtual} en vez de real dado que ningún desplazamiento real puede ocurrir sin que el tiempo avance.



En la mecánica analítica el concepto de desplazamiento virtual solo es significativo cuando se analiza un sistema físico sujeto a ligaduras que restringen su movimiento. El desplazamiento virtual $\var r$  es un caso especial de desplazamiento infinitesimal (normalmente denotado por $\dd r$) que se refiere a un cambio infinitesimal en las coordenadas de posición de un sistema de manera que las ligaduras se satisfacen.

\begin{multicols}{2}
Por ejemplo, una tiza sobre una mesa. En ambas posiciones hay equilibrio.

\begin{figure}[H]
	\centering
	\includegraphics[width=.4\textwidth]{imagenes/imagenes06/T06IM01.png}
\end{figure}
\end{multicols}

En \emph{estática} el tiempo no va a intervenir como una magnitud fundamental para el estudio.

La condición de equilibrio es independiente del tiempo real, es como si congelásemos el tiempo. Introduciremos como parámetro el \emph{tiempo virtual}.

\section{Ligaduras sin rozamiento}

Las ligaduras se llaman sin rozamiento si el trabajo realizado por esas fuerzas de ligadura es cero.  (Una fuerza de rozamiento no es una fuerza de ligadura de rozamiento.)


Si el sistema se encuentra sometido exclusivamente a enlaces ideales (sin rozamiento), las fuerzas de enlace son ortogonales al propio enlace material, y como los desplazamientos $\var r_i$ deben ser compatibles con los enlaces, serán vectores ortogonales a las fuerzas de ligadura, por lo que no realizarán trabajo.	

Fuerzas de ligadura sin rozamiento $\ \bot \ $ a los desplazamientos. (*)

\section{Principio de los trabajos virtuales}

Solo es aplicable a fuerzas aplicadas y fuerzas de ligadura sin rozamiento.

\begin{miparrafo}
\emph{Para que un sistema de puntos materiales sujeto a `enlaces sin rozamiento' esté en equilibrio bajo la acción de un conjunto de fuerzas solicitantes es necesario y suficiente que sea nula la suma de los trabajos virtuales elementales efectuados por las fuerzas solicitantes para cualquier desplazamiento de dicho sistema efectuado a partir de una posición de equilibrio.}
\end{miparrafo}

$\displaystyle \sum_i (\vec F_i+\vec F_i')\cdot \var \vec r_i=
\sum_i \vec F_i\cdot \var \vec r_i +
\cancelto{0,\ (*)}{\sum_i \vec F_i'\cdot \var \vec r_i} \to 
 $
 
 \begin{equation}
 \subrayado{\boxed{\ \boldsymbol{\sum_i \vec F_i\cdot \var \vec r_i}=0 \ }}	
 \end{equation}


El trabajo total efectuado por las fuerzas aplicadas o solicitantes es nulo.

$ \displaystyle \sum_i  (\ F_{x_i}\cdot \var  x_i + F_{y_i}\cdot \var  y_i +F_{z_i}\cdot \var  z_i \ )=0 $

\begin{multicols}{2}
La figura muestra una aplicación sencilla de este principio a un sistema de una sola partícula y una sola fuerza aplicada, el peso.
El punto $B$ no es un punto de equilibrio, pues en el desplazamiento virtual indicado en la figura el peso realiza trabajo. El punto $A$ si es de equilibrio, pues cualquier desplazamiento virtual es perpendicular al peso y, por tanto, el peso no realiza trabajo.
\begin{figure}[H]
	\centering
	\includegraphics[width=.4\textwidth]{imagenes/imagenes06/T06IM02.png}
\end{figure}
\end{multicols}

\section{Sólido rígido}
Un cuerpo \emph{sólido rígido} es aquel cuerpo que está constituido por un sistema de partículas tal que las distancias entre las distintas partículas permanecen inalteradas y constantes en el tiempo.

En la naturaleza no existen sólidos rígidos, los cuerpos están formados por átomos y moléculas que vibran en sus posiciones de equilibrio. El sólido rígido es una idealización (abstracción de la realidad), es un conjunto de puntos del espacio que se mueven de tal manera que no se alteran las distancias entre ellos, sea cual sea la fuerza actuante.

\begin{figure}[H]
	\centering
	\includegraphics[width=1\textwidth]{imagenes/imagenes06/T06IM03.png}
\end{figure}

La mecánica de un cuerpo rígido es aquella que estudia el movimiento y equilibrio de sólidos materiales ignorando sus deformaciones. Se trata, por tanto, de un modelo matemático útil para estudiar una parte de la mecánica de sólidos, ya que todos los sólidos reales son deformables. 

\section{Equilibrio del solido rígido libre}

\vspace{30mm} %**************************************************
\begin{multicols}{2}
Para determinar la posición de un cuerpo rígido debemos dar las coordenadas de un punto, su centro de masas o centro de gravedad por ejemplo y de un eje del cuerpo. La posición del eje puede determinarse dando tres ángulos, los ángulos directores.

\begin{figure}[H]
	\centering
	\includegraphics[width=.4\textwidth]{imagenes/imagenes06/T06IM04.png}
\end{figure}
\end{multicols}

La posición de un cuerpo rígido depende de $6$ parámetros, tiene $6$  grados de libertad: $3$ coordenadas del punto y $3$ ángulos directores del eje. De estos $6$ parámetros o grados de libertad, $3$ corresponden al movimiento y otros $3$ a la rotación.

Utilizaremos el método o principio de los trabajos virtuales: $\displaystyle \sum_i \vec F_i \cdot \var \vec r_i=0$

Non inventamos un tiempo virtual: $\displaystyle \vec v_i=\dv{\vec r_i}{t} \ \to \ \fdv{\vec r_i}{t}=\vec v_i$

Dividienndo por el tiempo virtual (sin existencia real) la ecuación de los trabajos virtuales:

\begin{equation}
 \sum_i\vec F_i\cdot \vec v_i=0	
\end{equation}

Ecuación que recibe el nombre de \emph{principio de las potencias virtuales.}

Particularizando al sólido rígido, se verifica que dadas 2 partículas $i$, $j$ de éste se cumple que (la demostración se verá en capítulos futuros):

$\vec v_i=\vec v+\vec w \times \vec r_i\ , \quad$
con $\vec v$ y $\vec w$, en principio, dos velocidades arbitrarias.
Sustituyendo en la ecuación de las potencias virtuales:

$\displaystyle \sum_i \vec F_i \cdot (\vec v+\vec w \times \vec r_i)=0 \ \to \ \vec v \cdot \sum_i \vec F_i + \sum_i \vec F_i \cdot (\vec \omega \times \vec r_i)=0 \quad (**)$

Como aparece un \emph{producto mixto} en el segundo término de la ecuación anterior y recordando sus propiedades (al cambiar dos filas de un determinante, éste no varía):

$\left| \begin{matrix} F_{i_x}&F_{i_y}&F_{i_z} \\ \omega_x&\omega_y&\omega_z \\ x_i&y_i&z_i \end{matrix} \right| \ = \ \left| \begin{matrix}\omega_x&\omega_y&\omega_z \\ x_i&y_i&z_i \\ F_{i_x}&F_{i_y}&F_{i_z} \end{matrix} \right|$

Por lo que: $\displaystyle \sum \vec F_i \cdot (\vec \omega \times \vec r_i) = \sum_i \vec \omega \cdot (\vec r_i \times \vec F_i) = \vec \omega \cdot \sum_i (\vec r_i \times \vec F_i) $

Por lo que la ecuación $(**)$ queda como:$\quad \vec v \cdot \sum_i \vec F_i +\vec \omega \cdot \sum_i (\vec r_i \times \vec F_i)=0$

Puesto que, como hemos dicho anteriormente, $\vec v$ y $\vec w$ son dos velocidades arbitrarias, necesariamente los coeficientes que las multiplican en la ecuación anterior deben ser cero:

\begin{eqnarray}
\sum_i \vec F_i&=&\vec 0 \\
\sum \vec r_i \times \vec F_i&=&\vec 0
\end{eqnarray} 

La \emph{suma de fuerzas solicitantes igual a cero} y la \emph{suma de los momentos de las fuerzas solicitantes igual a cero} son las dos ecuaciones vectoriales ($3+3$ ecuaciones escalares) del equilibrio del sólido rígido.

Las $6$ ecuaciones escalares del equilibrio del sólido rígido son:

\begin{eqnarray*}
\sum_i F_{i_x}=0 &\quad \quad \quad & \sum_i (y_i F_{z_i}-F_{y_i}z_i)=0 \nonumber \\
\sum_i F_{i_y}=0 &\quad \quad \quad & \sum_i (z_i F_{x_i}-F_{z_i}x_i)=0 \nonumber \\
\sum_i F_{i_z}=0 &\quad \quad \quad & \sum_i (x_i F_{y_i}-F_{x_i}y_i)=0 \nonumber  	
\end{eqnarray*}

Como ejemplo veremos el equilibrio del \emph{sólido plano}, que no existe en la realidad pero con ello queremos decir que el movimiento transcurre en un plano, supongamos el $XY$ 

Harán falta tres ecuaciones: 

\hspace{2cm} $\displaystyle \sum_i F_{x_i}=0;\ \ \sum_i F_{i_y}=0; \ \ \sum_i (x_iF_{y_i}-F_{x_i}y_i)=0$
 

\section{Equlibrio del sólido plano}


Supongamos una varilla rígida de masa $m$ unida a dos puntos $\mathcal O$ y $C$ por dos resortes  con una fuerza atractiva proporcional a la distancia. Tenemos que averiguar la posición de equilibrio, que sabemos quedará determinada por 3 parámetros. La longitud $2L$ de la barra es un dato.

Plano de movimiento $XY \quad \to \quad \begin{cases}
 \sum_i F_{x_i}=0 \\ \sum_i F_{i_y}=0 \\ \sum_i (x_iF_{y_i}-F_{x_i}y_i)=0	 \end{cases}$
 
 La única componente del momento de las fuerzas aplicadas o solicitantes que es perpendicular al plano $XY$ de movimiento es $M_z$, en la dirección $\vec k$.
 
 \begin{figure}[H]
	\centering
	\includegraphics[width=.5\textwidth]{imagenes/imagenes06/T06IM05.png}
\end{figure}
 
 
 $\vec F_A=K\ \overrightarrow{AO}=K\ (-\vec i x_A-\vec j y_A)$
 
 $\vec F_B=K\ \overrightarrow{BC}=k\ (\vec i (x_C-x_B)+\vec j (y_C-y_B))$
 
 $\vec F_G=-\vec j\ mg$
 
 $x_A=\overline x-L\cos \alpha \qquad x_B=\overline x+L\cos \alpha$
 
  $y_A=\overline y-L\cos \alpha \qquad y_B=\overline y+L\cos \alpha$
 
  
$ \sum_i F_{x_i}=0 \ \to$ 
 
$\qquad k(-x_A+x_c-x_B)=0 $ 
    
$\sum_i F_{y_i}=0 \ \to$
 
$\qquad k(-y_A+y_c-y_B)-mg=0$
  
$ \sum_i(x_iF_{y_i}-y_iF_{x_i})=0 \ \to$

$\qquad  -x_AKy_A+y_Akx_A+x_Bk(y_C-y_B)-y_Bk(x_c-x_B)-\overline x mg=0 $
  
 Sustituyendo los valores de $x_A,\ x_B,\ y_A,\ y_B$ en función de $\overline x,\ \overline y,\ \alpha$ se obtiene un sistema de tres ecuaciones con las tres incógnitas, $\overline x,\ \overline y,\ \alpha$ , que determinan la posición de equilibrio del sólido plano.
 
 \section{Equilibrio del sólido rígido sometido a enlaces}
 \subsection{Equilibrio del sólido rígido que tiene un punto fijo}

\vspace{-5mm} %*********************************** 
  \begin{figure}[H]
	\centering
	\includegraphics[width=.4\textwidth]{imagenes/imagenes06/T06IM06.png}
\end{figure}
 
 $$\vec v \cdot \displaystyle \sum_i \vec F_i + \vec w\cdot \sum_i(\vec r_i \times \vec F_i)=0$$
 
 Elegimos como velocidad $\vec v$ la que tiene el punto ligado $\mathcal O \ \to \ \vec v=0;\ \ \vec v_i=\vec \omega \times \vec r_i \to $ condiciones de equilibrio: $\displaystyle \sum_i (\vec r_i \times \vec F_i)=\vec 0$
 
 Luego, $\quad \begin{cases} 
 \ \displaystyle \sum_i (y_iF_{z_i}-z_iF_{y_i}=0\\
  \ \displaystyle \sum_i (z_iF_{x_i}-x_iF_{z_i}=0\\
   \ \displaystyle \sum_i (x_iF_{y_i}-y_iF_{x_i}=0\\	
 \end{cases}$


  \subsection{Equilibrio del sólido rígido que tiene dos puntos fijos}
 
\vspace{-5mm} %***********************************  
   \begin{figure}[H]
	\centering
	\includegraphics[width=.6\textwidth]{imagenes/imagenes06/T06IM07.png}
\end{figure}

Por el mismo razonamiento que en el caso anterior, la condición de equilibrio es $\displaystyle \sum_i \vec r_i \times \vec F_i=\vec 0$. En el plano $X-Y$ queda la tercera componente por ser perpendicular al eje giro  $\mathcal O A$:

$$\displaystyle \sum_i (x_iF_{y_i}-y_iF_{x_i})=0$$
 
 
 \section{Problemas}
 
 \begin{prob}
 	Usando el principio de los trabajos virtuales, determinar la posición de equilibrio de una partícula de masa $m$, móvil sobre la circunferencia $x^2+y^2=r^2$ y repelida por un punto fijo $C$, de coordenadas $(r/2,r/2)$, proporcionalmente a la distancia.
 \end{prob}

\begin{multicols}{2}
Principio trabajos virtuales:

$\displaystyle \sum_i \ \vec F_i \cdot \var \vec r_i = 0 \to$

$\displaystyle F_x \var x + F_y \var y =0$

$F_C=kd=$

$=k[(x-r/2)^2+(y-r_2)^2]^{1/2}$
\begin{figure}[H]
	\centering
	\includegraphics[width=.35\textwidth]{imagenes/imagenes06/T06IM08.png}
\end{figure}	
\end{multicols}

\vspace{-5mm} %*******************************************
$F_{C_x}=k(x-r/2)\ \vec i;\ \ F_{C_y}=k(y-r/2)\ \vec j;$
$\quad m\vec g=mg\ \vec j$ 

$F_x=k(x-r/2);\quad F_y=k(y-r/2)-mg$

Ppio. trabajos virtuales: $k(x-r/2)\ \var v+[k(y-r(2)-mg]\ \var y$

Ligadura: $x^2+y^2=r^2 \to \begin{cases} \ y=(r^2-x^2)^{1/2} \\ 2x\var x+2y \var  y=0 \to \var y=-\dfrac x{(x^2-r^2)^{1/2}} \var x \end{cases}$

Llevando estas relaciones al principio de los trabajos virtuales se obtiene: $f(x)\var x=0$, como $\var x \neq 0$ (el principio se enuncia para desplazamientos virtuales arbitrarios) $\to$ necesariamente, ha de ser $f(x)=0 \to x=x_{eq} \to (x^2+y^2=r^2) \to y=y_{eq}$.

\rightline{\textsf{\textcolor{DarkBlue}{--- Inacabado ---}}}

\begin{prob}
\begin{multicols}{2}
En la figura se observa una barra de peso $P$ y longitud $L$ que está oblgada a deslizar sus extremos entre dos rectas verticales entre sí. Hallar los puntos de equilibrio si es el peso que pende de la polea $\mathcal O$.
\begin{figure}[H]
	\centering
	\includegraphics[width=.4\textwidth]{imagenes/imagenes06/T06IM09.png}
\end{figure}
\end{multicols}	
\end{prob}

$\vec F_A=-Q \vec i; \ \ \vec F_C=-P \vec j; \qquad \vec r_A=x_A \vec i = L\cos \theta \vec i; \ \  \vec r_C=\frac L 2 \cos \theta \vec i + \frac L 2 \sin \theta \vec j$

$\var r_A=-L\sin \theta \ \var \theta \ \vec i; \qquad \var r_C= \frac L 2 \ (-\sin \theta \ \vec i + \cos \theta \ \vec j)\ \var \theta$


Trabajos virtuales: $\vec F_C \cdot \var \vec r_C+\vec F_A \cdot \var \vec r_A=0$

$-P\ \vec j \ \ [\frac L 2\ (\-sin \theta \ \vec i + \cos \theta \ \vec j) \ \var \theta] - Q\ \vec i \ (-L \sin \theta \ \vec i) \ \var \theta =0$

$(-P \frac L 2 \cos \theta + Q L \sin \theta ) \cancelto{\neq 0}{\var \theta}=0 \to \ QL\sin \theta=P\frac L 2 \cos \theta$

Posición de equilibrio: $\quad \tan \theta=\dfrac P{2Q} \quad \to \qquad \theta=\arctan \dfrac{P}{2Q}$


\begin{prob}
Suponiendo que la barra $\overline{AB}$ de longitud $L$ está obligada a deslizar sus extremos por una recta y una circunferencia como muestra la figura, determinar la relación que debe existir entre la fuerza $F$ que actúa en el eje $X$ y la fuerza $T$, tangencial en el punto de contacto con la circunferencia, para que la barra quede en equilibrio para el ángulo $\theta=\pi/6$.	
\end{prob}
\begin{figure}[H]
	\centering
	\includegraphics[width=.75\textwidth]{imagenes/imagenes06/T06IM10.png}
\end{figure}

$\vec F_A=F\ \vec i;\quad \vec F_B=-T\sin \theta \ \vec i + T\cos \theta \vec j$

$\vec r_A=[a\cos \theta +(L^2-a^2\sin^2 \theta)^{1/2}]\ \vec i; \quad \vec r_B=a\cos \theta \ \vec i + a \sin \theta \vec j$

$\var \vec r_A=[-a \sin \theta + \frac 1 2 (L^2-a^2 \sin^2 \theta)^{-1/2}\ 2 a^2 \sin\theta \cos \theta] \ \var \theta \ \vec i$

$\var \vec r_B=(-a \sin \theta \ \vec i+a\cos \theta \ \vec j)\ \var \theta$


Trabajos virtuales:  $\vec F_A \cdot \var \vec r_A+\vec F_B \cdot \var \vec r_B=0$


\small{$[-F a \sin \theta -F a^2 \sin \theta \cos \theta \ (L^2-a^2\sin^2 \theta)^{-1/2}\ + T a \sin^2 \theta + T a \cos^2 \theta]\ \cancelto{\neq 0}{\var \theta}=0 \to$}

\normalsize{$T=F\sin \theta \left( 1 + \dfrac{a \cos \theta}{(L^2-a^2\sin^2 \theta)^{1/2}} \right) \Rightarrow$}

$\dfrac T F = \sin \theta \left( 1+\dfrac {a \cos \theta}{(L^2-a^2\sin^2 \theta)^{1/2}}  \right) \ \to \ \ \ \theta=\pi/6 \ \Rightarrow \ \cdots$

\rightline{\textsf{\textcolor{DarkBlue}{--- Inacabado ---}}}




\begin{prob}
Un anillo está obligado a moverse en una elipse de 	semiejes $a$ y $b$. Sabiendo que el anillo pesa $P$ y que es solicitado por el eje $Y$ con una fuerza proporcional a la distancia a eje eje, determinar la posición de equilibrio.
\end{prob}
\begin{multicols}{2}
$\quad$

$ F_x=-Kx;\quad F_y=0 $

$P_x=0;\quad \quad \ \ P_y=-P$

$\quad$

$f : \ \dfrac{x^2}{a^2}+\dfrac{y^2}{b^2}=1$
\begin{figure}[H]
	\centering
	\includegraphics[width=.5\textwidth]{imagenes/imagenes06/T06IM11.png}
\end{figure}
\end{multicols}

$\left. \begin{matrix}  
 	\displaystyle N_x=\lambda \ \fdv{f}{x} \ \quad \\ \\ \displaystyle N_y=\lambda\  \fdv{f}{y} \ \quad 
 \end{matrix} \right|
 \left. \begin{matrix}  
 	\quad \lambda \ \displaystyle \fdv{f}{x}-Kx=0 \ \quad \\ \\ \quad \displaystyle \lambda \ \fdv{f}{y}-P=0 \quad 
 \end{matrix} \right|
\left. \begin{matrix}  
 	\quad \lambda \dfrac{2x}{a^2}-kx=0\ \quad (1*)\\ \\ \quad \lambda \dfrac{2y}{b^2}-P=0\  \quad (2*)
 \end{matrix} \right.$

$(1*)\ \to \ x\left(\dfrac{2\lambda}{a^2}-k \right)=0 \begin{cases}\ x=0 \to (f\to) \ y=\pm b \\ \ 2\lambda
=ka^2 \quad (3*) \end{cases}$

$(2*) \ \wedge \ (3*)\ \to \ y=\dfrac{Pb^2}{2\lambda}=\dfrac{Pb^2}{ka^2} \ \to f: \ x=\pm \dfrac 1{ka}\sqrt{k^2a^2-P^2b^2}$

Hay cuatro soluciones:

$(0,b);\ (0,-b);\ \left(\dfrac 1{ka}\sqrt{k^2a^2-P^2b^2},\dfrac{p^2b^2}{ka^2} \right);\ \left(-\dfrac 1{ka}\sqrt{k^2a^2-P^2b^2},\dfrac{p^2b^2}{ka^2} \right) $

\begin{prob}
	Una partícula de masa m se puede desplazar sin rozamiento siguiendo una parábola vertical $y=ax^2$, con a>0.
	
	Si en el punto $(0,H)$, con $H>0$, hay un centro de fuerzas que atrae a la partícula con una fuerza directamente proporcional a la distancia, ?`cuáles serán las posiciones de equilibrio de la partícula a lo largo de la parábola?
\end{prob}

\begin{multicols}{2}
$\sum F_x:\quad -Kx\ \vec i$

$\sum F_y:\quad -mg-k(y-H)\ \vec j$

$f:\quad y=ax^2 \quad (\var y= 2ax\ \var x)$

$\var \vec r_x=\var x\ \vec i;\quad \var \vec r_y=2ax\ \var x \ \vec j$

Trabajos virtuales:

$\vec F_x \cdot \var \vec r_x + \vec F_y \cdot \var \vec r_y=0$
\begin{figure}[H]
	\centering
	\includegraphics[width=.4\textwidth]{imagenes/imagenes06/T06IM12.png}
\end{figure}
\end{multicols}
$-kx \var x+ [-mg-k(y-H)]\ 2ax)\ \var v=0 \to \ (\var x \neq o)$

Despejando y teniendo en cuenta la relación entre $x$ e $y$ ($f:\ y=ax^2$), se obtienen las posiciones de equilibrio.

\rightline{\textsf{\textcolor{DarkBlue}{--- Inacabado ---}}}



\begin{comment}

% *******************************************************
% *** Dos problemas exámenes NO SEGURO e inacabados *****
% *******************************************************


\begin{multicols}{2}
\textbf{P.I. 4. } 

\textit{Una varilla de $6 \ \mathrm{kg}$ y longitud $0.8\ \mathrm{m}$ está colocada sobre un ángulo recto liso como muestra la figura adjunta. Determinar la posición de equilibrio y las fuerzas de reacción en función del ángulo $\theta$}.
\begin{figure}[H]
	\centering
	\includegraphics[width=.45\textwidth]{imagenes/imagenes09/T09IM13.png}
\end{figure}
\end{multicols}

$m=6 \ \mathrm{kg};\quad l=0.8\ \mathrm{m}; \qquad \to \qquad \boldsymbol{ N_1 (\theta), \ N_2 (\theta) , \ \varphi (\theta) }$

\begin{figure}[H]
	\centering
	\includegraphics[width=1\textwidth]{imagenes/imagenes09/T09IM12.png}
\end{figure}

$\displaystyle \sum F_x=0: \qquad \boldsymbol{N_2 \cos \theta-N_1 \sin \theta = 0}$


$\displaystyle \sum F_y=0: \qquad  \boldsymbol{N_2 \sin \theta + N_1 \cos \theta - m g=0}$

Tomamos el origen de momentos en el punto $C$, centro de gravedad de la barra.

$\displaystyle \sum M_z=0: \ mg\cdot 0+ N_1 \dfrac L 2 \sin(90^ o-\theta-\varphi)-N_2\dfrac L 2 \sin(90^ o-\theta-\varphi)=0 \to $

$\displaystyle \boldsymbol{\dfrac {N_1L} 2 \cos (\theta-\varphi) - \dfrac {N_2L} 2 \cos (\theta-\varphi)=0}$

Incógnitas $N_1,\ N_2, \ \varphi; \qquad \text{datos: }\ m,\ l,\theta $

\rightline{\textsf{\textcolor{DarkBlue}{--- Inacabado ---}}}


\begin{multicols}{2}
\textbf{P.I. 5. } \textit{Dos partículas de masas $m_1$ y $m_2$ están situadas una sobre un plano inclinado de ángulo $\alpha$ y longitud $L$ y la otra sobre un punto de cuarto de aro de radior $R$ y unidas entre sí por un hilo inextensible de longitul $l$ y masa despreciable.}
\begin{figure}[H]
	\centering
	\includegraphics[width=.5\textwidth]{imagenes/imagenes09/T09IM14.png}
\end{figure}
\end{multicols}

\textit{Si tanto la superficie del aro como la del plano son lisas, determinar cuál es la posición de equilibrio. Como particularización, calcular la posición de equilibrio para $m_1=m_2=1\ \mathrm{kg},\ \alpha=30^o \text{ y } L=l=1\ \mathrm{m}$}


Datos: $\quad m_1=m_2=1\ \mathrm{kg},\ \alpha=30^o \text{ y } L=l=1\ \mathrm{m}$

Incógnitas: $\quad \boldsymbol{N_1, \ N_2,\; x \text{ ó } \varphi}$

\begin{figure}[H]
	\centering
	\includegraphics[width=.75\textwidth]{imagenes/imagenes09/T09IM15.png}
\end{figure}

arco=ángulo (rad) x radio $\ \to x=\varphi R \ \to \boldsymbol{\varphi=\dfrac x R }$

$\displaystyle \sum F_y:\quad (m_1+m_2)g-N_1\cos \theta-N_2 \cos \phi$

$\displaystyle \sum F_x:\quad -N_1\sin \theta + N_2 \sin \varphi=0$

Tomamos como centro de momentos el punto $\ \mathcal O$:

$\displaystyle \sum M:\ \  -N_1 (l-x)+m_1g(l-x)\sin(\frac \pi 2 - \theta)+N_2x-m_2gx \sin (\dfrac \pi 2 - \varphi)=0$

$\boxed{ \  \begin{cases}
\ N_1\sin \theta-N_2\sin \varphi=0 \\
\ N_1\cos \theta+N_2\cos \varphi=(m_1+m_2)g \\
\ (m_1g\cos \theta - N_1)\ (l-x)=(m_2g\cos \varphi-N_2)\ x
\end{cases} \ }$

Incógnitas $N_1, \ N_2, \ \varphi$; datos: $m_1,\ m_2,\ \theta, R, l$

Relación $x,\ l:\quad \varphi=x R;\quad \sin \theta=\dfrac R L$

\rightline{\textsf{\textcolor{DarkBlue}{--- Inacabado ---\textit{\footnotesize{principio trabajos virtuales???}\normalsize{---}}}}}

$N_1,\ N_2\ \bot \ \var r \to W=m_1g\cos (\frac \pi 2 - \theta)-m_2g \cos (\frac \pi 2 - \varphi) \ \var r=0 \to $

$m_1g\sin \theta =m_2g\sin \varphi \to \sin \varphi =\dfrac {m_1}{m_2}\sin \theta $

\rightline{\textsf{\textcolor{DarkBlue}{--- Tan fácil ???? ---}}}


Y si pruebo a calcular la energía potencial y exijo que sea mínima

\rightline{\textsf{\textcolor{DarkBlue}{--- Probarlo!!!!!! ---}}}

\begin{figure}[H]
	\centering
	\includegraphics[width=.75\textwidth]{imagenes/imagenes09/T09IM16.png}
\end{figure}

$E_p=m_1gh_1+m_2gh_2$

$h_1=(l-x) \sin \theta; \qquad x=\varphi L \sin \theta:\quad \varphi=\dfrac{x}{L\sin \theta}$

$h=2=R\cos \varphi=L \sin \theta \cos \varphi = L \sin \theta \cos \left( \dfrac{x}{L\sin \theta} \right)=E_p(x)$

$\displaystyle \dv{E_p}{x}=0 \to -m_1g\sin \theta \textcolor{red}{+} m_2gL\sin \theta \dfrac {1}{L \sin \theta} \sin \left( \dfrac{x}{L\sin \theta} \right)=0$

$\sin \left( \dfrac{x}{L\sin \theta} \right)= \dfrac{m_1}{m_2}\tan \theta$

Datos: $ \dfrac{x}{1/2}=\arcsin \left( \dfrac{\sqrt{3}}{3} \right) \to x=0.308\ \mathrm{m};\quad l-x=0.616\ \mathrm{m};\quad \varphi=0.161 \ \mathrm{rad}=35.29^o$
 
\textcolor{red}{Debería ser un MENOS}

\rightline{\textsf{\textcolor{DarkBlue}{--- Probar dos cuerpos separados, T, ....!!!!!! ---}}}

	
\end{comment}



\begin{prob}
\begin{multicols}{2}.
Una varilla de longitud $2L$ y peso $P$ está en equilibrios en el borde y en la superficie de una cápsula hemiesférica de radio $R<L$, sin rozamiento. 

Determinar la posición de equilibrio.	
\begin{figure}[H]
	\centering
	\includegraphics[width=.55\textwidth]{imagenes/imagenes06/T06IM13.png}
\end{figure}
\end{multicols}	
\end{prob} 

La posición de equilibrio viene determinada por el ángulo $\boldsymbol{\theta}$, como se muestra en la siguiente figura en que se han dibujado las fuerzas que actúan y se han elegido los ejes $x$ e $y$ como allí se indica. Exigiremos que $\sum F_x=0$, $\sum F_y=0$ y $\sum M_z=0$. Escogemos como centro de momentos el punto $A$, con lo anulamos el de la reacción $R_A$.

\begin{figure}[H]
	\centering
	\includegraphics[width=1.1\textwidth]{imagenes/imagenes06/T06IM14.png}
\end{figure}

$\sum F_x=0 \ \to \ -R_A \cos \theta + P \sin \theta =0 \quad \textcolor{gris}{(1*)}$

$\sum F_y=0 \ \to \ R_C-P\cos \theta+R_A \sin \theta=0 \quad \textcolor{gris}{(2*)}$

$\sum M_z=0 \ \to \ -R_C \overline{AC} + PL \cancelto{\cos \theta}{\sin(90^o-\theta)}=0 \quad \textcolor{gris}{(3*)}$

De la figura, $\ \dfrac{\overline{AC}}{2}=R \cos \theta \ \to \ \overline{AC}=2R\cos \theta \quad \textcolor{gris}{(4*)}$

De $\textcolor{gris}{(1*)} \ \to \ -R_a+P\tan \theta = 0;\qquad \boldsymbol{R_A=P \tan \theta}$

De $\textcolor{gris}{(3) \text{ y } (4*)} \ \to \ -R_c 2R \cos \theta + PL \cos \theta = 0;$

$ \cos \theta \cdot \left( -2R_cR+PL \right) =0; \quad \cos \theta \neq 0 \quad  \to \ \boldsymbol{R_C=\dfrac{PL}{2R}}$

Llevándolo a $\textcolor{gris}{(2*)} \ \to \ \dfrac {PL}{2R} - P \cos \theta + P \tan \theta \sin \theta = 0$

$P\cdot \left( \dfrac {PL}{2R} - \cos \theta +  \tan \theta \sin \theta \right) = 0; \quad  P\neq 0 \ \to$

$\dfrac {L}{2R} - \cos \theta +  \tan \theta \sin \theta = 0 $

$ \tan \theta \sin \theta = \dfrac{\sin^2 \theta}{\cos \theta}$, multiplicando por $\cos \theta$ la ecuación anterior,

 $\dfrac {L}{2R} \cos \theta - \cos^2 \theta + 1- \sin^2 \theta =0 \ \to \ -2\cos^2 \theta + \dfrac L{2R} \cos \theta + 1 = 0$
 
 $\boldsymbol{\cos \theta=} \dfrac{-\dfrac{l}{2R} \pm \sqrt{\left(\dfrac{L}{2R}\right)^2 + 8} }{-4}=\cdots =\boldsymbol{
 \dfrac L {8R} + \sqrt{\left( \dfrac{l}{8R} \right)^2 + \dfrac 1 2} }$


\begin{prob}.
	Una barra uniforme de masa $M$ y longitud $L$ se sos	tiene es sus extremos por medio de una cuña como se muestra en la figura. Encontrar las reacciones normales en los puntos de apoyo y el ángulo de equilibrio.
\end{prob}

\begin{figure}[H]
	\centering
	\includegraphics[width=1\textwidth]{imagenes/imagenes06/T06IM17.png}
\end{figure}

$\sum F_x=0:\quad N_A\sin30^o-N_B\sin 60^0=0 \quad \to \quad N_B=N_A\tan 30^o \ \textcolor{gris}{(*)}$

$\sum F_y=0:\quad N_A \cos 30^o+ N_B \cos 60^o - mg = 0 \quad \to $

$N_A \cos 30^0 + N_A \tan 30^0 \sin 30^0 =mg;\quad N_A(\cos^2 30ô + \sin^2 30^o)=mg \cos 30^0$

$ \boldsymbol{N_A=mg\cos 30^0}$

$\textcolor{gris}{(*)}\ \to \quad \boldsymbol{N_B=}mg\cos 30^o \tan 30^o =\boldsymbol{mg \sin 30^o}$

Tomamos el origen de momentos es $A$, así anulamos uno de ellos (el de $N_A$).

$\sum M_z=0:\quad N_a\cdot 0 +mgr_C-NB\sin \theta r_B=0;\quad mg\dfrac L 2 \cos(\theta-30^o)-mg \cancelto{1/2}{\sin 30^o} \sin \theta L = 0$

$mg\dfrac L 2 \cos (\theta-30^o)=mgL \dfrac 1 2 \sin \theta \quad \to \quad \cos(\theta-30^o) = \sin \theta$

Como $\cos \alpha=\sin (90-\alpha) \quad \to \quad \cos(\theta-30^o)=\cos(90^o-\theta)$,

por lo que: $\ \theta-30^o=90^o - \theta \quad \to \quad \boldsymbol{\theta= 60^o}$



\newpage %*******************************

\begin{myblock}{El Principio de los Trabajos Virtuales.}
Principio de los trabajos virtuales.

El Principio de Trabajos Virtuales fue utilizado por Galileo (1564-1642) para el diseño y c cálculo de mecanismos y desarrollado teóricamente con un enunciado más matemático y formal por Lagrange (1736-1813. 
Enúcleo teórico del Pincipio  de los Trabajos Virtuales fue enunciado por Santiago Bernouilli (1654-1705) y por Daniel Bernouilli (1700-1782): “Si una estructura, estando en equilibrio, sufre una deformación virtual debido a la acción de una carga adicional , el trabajo virtual externo de la carga en cuestión, es igual al trabajo virtual interno, desarrollado por las tensiones causadas por la carga”. 
En cuanto a lo que concierne a la mecánica de cuerpos rígidos, dado que por definición estos cuerpos no sufren deformación sino desplazamientos, el Pincipio  de los Trabajos Virtuales debe ser reformulado. El mismo fue enunciado por Johann Bernouilli en el año 1717 de la siguiente manera: “Dado un cuerpo rígido mantenido en equilibrio por un sistema de fuerzas, el trabajo virtual efectuado por este sistema, durante un desplazamiento virtual, es nulo”. 
\end{myblock}



\include{TEMA07_chapter-A4}
\chapter{Fenómenos de superficie}	

\begin{miparrafo}
\begin{multicols}{2}
\small{Una diferencia fundamental entre gases y líquidos es la existencia de una superficie libre en éstos.}
\small{Gran variedad de fenómenos físicos están asociados a la existencia de esta superficie, y que se explican a partir de las propiedades contráctiles de ésta.}
\small{Este estado de tensión de la superficie libre del líquido tiende a reducir el área de ésta a un valor mínimo compatible con los vínculos y con las fuerzas externas. }
\begin{figure}[H]
	\centering
	\includegraphics[width=.5\textwidth]{imagenes/imagenes08/T08IM01.png}
\end{figure}
\end{multicols}
 \small{1. Las gotas de lluvia o las burbujas de aire en el interior de un líquido tienden a tomar la forma que ofrece una superficie mínima, esférica. }
 
 \small{2. Una aguja de coser (a pesar de que su densidad es mayor que la del agua) o un insecto pueden flotar en el agua.}
 
 \small{3. Cuando sumergimos parcialmente en agua un tubo de vidrio, limpio y de pequeño calibre, el agua asciende en su interior, pero si lo sumergimos en mercurio, el líquido desciende}\normalsize{.}

\end{miparrafo}
	
\section{Tensión superficial}
	
\normalsize{Para} realizar un estudio macroscópico del comportamiento de la superficie límite de un líquido vamos a ver el comportamiento microscópico de una molécula	 y después, sumando contribuciones de muchas moléculas, pasaremos al punto de vista macroscópico.

Experimentalmente sabemos que las fuerzas que mantienen unidas a las moléculas o átomos son de naturaleza eléctrica, atractivas, que se establecen entre las cortezas electrónicas de los átomos y los núcleos atómicos (positivos) de los átomos vecinos. Estas fuerzas se conocen con el nombre de \emph{fuerzas de Van der Waals}, su intensidad es muchísimo mayor a la de la interacción gravitatoria y son de \emph{corto alcance}, tienen gran intensidad para distancias cortas (del orden de unos cuantos diámetros atómicos) pero decaen rápidamente al aumentar la distancia.

Podemos distinguir dos tipos de \emph{fuerzas de Van der Waals:}
\begin{itemize}
\vspace{-2mm} \item \emph{Fuerzas de cohesión:} son fuerzas de Van der Waals que se ejercen entre moléculas de una misma sustancia (agua-agua).
\vspace{-2mm} \item \emph{Fuerzas de adhesión:} son fuerzas de Van der Waals que se ejercen entre moléculas de sustancias distintas (agua-vaso).	
\end{itemize}
Las moléculas en el líquido están sometidas a las fuerzas de Van der Waals de corto alcance.
\begin{multicols}{2}
En el interior del fluido, por simetría, estas fuerzas se anulan.

Sobres las moléculas situadas en la superficie (libre o límite) del líquido no ocurre lo mismo y existe una fuerza resultante que trata de meterlas hacia dentro.
\begin{figure}[H]
	\centering
	\includegraphics[width=.4\textwidth]{imagenes/imagenes08/T08IM02.png}
\end{figure}
\end{multicols}
Supongamos que sacamos a una de estas moléculas desde el interior a la superficie libre del líquido, tendremos que realizar un trabajo contra la resultante de la fuerza de la que hemos hablado anteriormente. Como los campos que generan las fuerzas de Van der Waals son conservativos, este trabajo vendrá de una diferencia de energías potenciales. Cuantas más moléculas subamos mayor será la superficie y más energía potencial tendremos: a mayor $\mathcal E_p$, mayor superficie $S$.

Teniendo en cuenta el teorema de Lejeune-Dirichlet, para que la molécula que subimos a la superficie ocupe una posición de equilibrios estable debe tener energía potencial mínima. Como la superficie mínima más estable es la esfera, ello explica por qué \emph{las gotas son esféricas}.

Cada vez que se modifica la superficie libre de un líquido una cantidad $\dd S$ hay que realizar un trabajo $\dd W$ que será proporcional ($\sigma=$constante de proporcionalidad)al cambio en la superficie:$ \subrayado{\ \boldsymbol{\dd W=\sigma \ \dd S\ }}$.

Esta constante de proporcionalidad, $\ \boldsymbol{\sigma} \ $, recibe el nombre de \emph{\textbf{coeficiente de tensión superficial}} y su valor depende de las propiedades físico-químicas del líquido del que se trate.


Unidades y dimensiones del coeficiente de tensión superficial:

$[\sigma]=\dfrac {[\mathrm{W}]}{[\mathrm{S}]}=\dfrac {[\mathrm{FL}]}{[\mathrm{S}]}=\mathrm{M}T^{-2}$; $\quad$ unidad SI: $\mathrm{J\ m}^{-2}$.


\emph{El coeficiente de tensión superficial es debido a fuerzas de cohesión}.
\begin{itemize}
\item No se manifiesta en los sólidos debido a la propia rigidez de éstos, a su incapacidad de movimiento.
\item No se manifiesta en los gases pues la moléculas del gas están a gran distancia unas de otras y las fuerzas de cohesión son de corto alcance.	
\end{itemize}

\subsection{Otra forma de introducir la tensión superficial}

Ésta, basada en la experiencia. 

En la figura, al tirar del alambre se observa cierta resistencia a la separación (fuerza), si se suelta vuelve a su posición inicial. Este fenómeno es debido a la tensión superficial.

Para cuantificar esta fuerza (de cohesión) consideremos una estructura de alambre con un lado deslizante, en la que se coloca una capa de líquido (solución jabonosa).
 
El líquido tratará de minimizar la superficie S ejerciendo una fuerza F sobre el lado deslizante, que S podemos medir. Se observa que: 
\begin{multicols}{2}

--- La tensión superficial $\sigma$ es la fuerza por unidad de longitud que ejerce la superficie de un líquido sobre una línea cualquiera situada sobre ella (borde de sujeción).

--- La fuerza debida a la tensión superficial es perpendicular a la línea y tangente a la superficie. 

\begin{figure}[H]
	\centering
	\includegraphics[width=.5\textwidth]{imagenes/imagenes08/T08IM03.png}
\end{figure}
\end{multicols}

\vspace{-3mm}La explicación física de este fenómeno es la siguiente:

Al estirar, aumentamos la superficie con lo que, según hemos visto, aumenta la energía potencial y se realiza un trabajo: $\ \dd W=F_T \dd x$, puesto que la película de solución jabonosa (fluido) se extiende por arriba y por abajo (un líquido en un vaso solo tiene una), el trabajo en deformar la superficie es: $\ \dd W=\sigma \dd S=\sigma 2L \dd x$, por lo que $F_T=2\sigma L=2F$, una para la superficie de arriba y otra para la de abajo. 

Se deduce: $\subrayado{\ \sigma=\dfrac F L \ }$, fuerza que actúa sobre la unidad de longitud sobre la superficie libre de un líquido. Esta es la causa por la que se llama \emph{tensión} superficial.
\begin{multicols}{2}
\footnotesize{El peso del insecto queda compensado por la resistencia de la superficie del agua a ser deformada. Esta fuerza sólo tiene componente vertical, pues la horizontal se anula.}
\footnotesize{$F_y=2\pi r \sigma \cos \theta \ n$, que compensará al peso del insecto. $\sigma$ es la tensión superficial, $r$ el radio de la depresión que producen las patas y $n$ el número de patas}\normalsize{.}
\begin{figure}[H]
	\centering
	\includegraphics[width=.5\textwidth]{imagenes/imagenes08/T08IM19.png}
\end{figure}
\end{multicols}

\section{Superficie de contacto. Meniscos}
Debido a la interacción que se ejerce entre el líquido y el recipiente que lo contiene (fuerzas de adhesión y cohesión de Van der Waals), la superficie libre del líquido deja de ser plana y adopta la forma que aparece en la figura. Este fenómeno recibe el nombre de \emph{menisco}.

Se llama \emph{ángulo de conjunción} o de contacto, $\theta$ al formado por la tangente a la superficie libre del líquido en el punto de contacto con el recipiente contenedor con la pared de éste último.
\vspace{-3mm} %***************************
\begin{figure}[H]
	\centering
	\includegraphics[width=.9\textwidth]{imagenes/imagenes08/T08IM04.png}
\end{figure}
\vspace{-3mm} %***************************
\begin{itemize}
\item Si el ángulo de conjunción es $\theta<90^0$ el menisco se llama \emph{cóncavo} y se dice que \emph{el líquido moja el vaso} (agua-vidrio).

\item Si el ángulo de conjunción es $\theta>90^0$ el menisco se llama \emph{convexo} y se dice que \emph{el líquido no moja el vaso} (mercurio-vidrio)
\end{itemize}

Veamos a que se debe que el líquido forme un menisco.
\vspace{-3mm} %***************************
\begin{figure}[H]
	\centering
	\includegraphics[width=1\textwidth]{imagenes/imagenes08/T08IM05.png}
\end{figure}

Suponemos la superficie del líquido horizontal y estudiamos las fuerzas que aparecen cuando se forma el menisco.

$(\overrightarrow{\sigma \dd l)})_1=\vec i\ \sigma \dd l \sin \varphi - \vec j \ \sigma \dd l \cos \varphi\, \quad$
$(\overrightarrow{\sigma \dd l)})_2=  - \vec j \ \sigma \dd l;$
$\quad (\overrightarrow{a \dd l)})_1=\vec i\ a \dd l$

La resultante, $\dd \vec R$ será: 
$\quad \dd \vec R=\vec i\ (\sigma \sin \varphi - a)\dd l -\vec j \ (1+\cos \varphi) \sigma \dd l$

Ángulo de conjunción: $\quad \tan \varphi=\dfrac {\dd R_y}{\dd R_x}=\dfrac{-\sigma(1+\cos \varphi)}{\sigma \sin \varphi - a}=\dfrac{\sin \varphi}{\cos \varphi}$

Multiplicando en cruz las dos últimas expresiones:

$\sigma \cos^2 \varphi+\sigma \cos \varphi=-\sigma \sin^2 \varphi + a \sin \varphi$

$\sigma=a\sin \varphi - \sigma \cos \varphi$, despejando \small{($\sin \varphi=\sqrt{1-\cos^2 \varphi}$)}\normalsize{:}

\begin{equation}
\cos \varphi=+ \dfrac{a^2-\sigma^2}{a^2+\sigma^2}	
\end{equation}

Expresión que nos da el ángulo de conjunción en función de las propiedades macroscópicas de los cuerpos que actúan.

$\boldsymbol{\sigma}$ es el coeficiente de tensión superficial, depende solo del del líquido. $\boldsymbol{a}$ es el coeficiente de adhesión y depende de la naturaleza del líquido y del recipiente que lo contiene. Un mismo líquido, en distintos recipientes, puede adoptar distintos tipos de meniscos.

Consecuencias:

\vspace{-2mm} \begin{itemize}
\vspace{-2mm} \item $a>\sigma \to \varphi<90^o$, adhesión mayor que cohesión, `el líquido moja el vaso'.	 \textsf{Menisco cóncavo}.
\vspace{-2mm} \item $a<\sigma \to \varphi<90^o$, adhesión menor que cohesión, `el líquido no moja el vaso'. \textsf{Menisco convexo}.	
\end{itemize}

\begin{figure}[H]
	\centering
	\includegraphics[width=.9\textwidth]{imagenes/imagenes08/T08IM13.png}
\end{figure}

\vspace{10mm} %******************************

\section{Presión de Laplace}

La existencia de los meniscos tiene como consecuencia el que aparezca una sobrepresión o una depresión sobre toda la masa del líquido, es la conocida como \emph{presión de Laplace.} Para superficies grandes (vaso de agua) es casi imperceptible pero no así para superficies pequeñas (capilar).

Supongamos un elemento de superficie, superficie infinitesimal, de un menisco.

\vspace{40mm} %******************************

\begin{multicols}{2}
\begin{figure}[H]
	\centering
	\includegraphics[width=.45\textwidth]{imagenes/imagenes08/T08IM06.png}
\end{figure}
$\quad \quad$ Corte transversal:

\begin{figure}[H]
	\centering
	\includegraphics[width=.5\textwidth]{imagenes/imagenes08/T08IM07.png}
\end{figure}
\end{multicols}
Las fuerzas $df_1$ tendrám una resultante $dF_1$ hacia abajo en la dirección normal y que va a influir en todo el líquido.

$\dd F_1=2 \dd f_1 \cos \alpha = 2 \dd f_1 \sin \dfrac{\dd \beta}{2};\ \ \  \qquad \alpha \text{ y } \beta\ $ complementarios.

De matemáticas sabemos que $\sin \dd \beta \approx \dd \beta \qquad  $ 
\textcolor{gris}{
\small{
$\sin x \text{ y } x \ $ son infinitésimos equivalentes: $\ x<1 \to \sin x \approx x$}
\normalsize{.}
} 

$\dd F_1=2\dd f_1 \dfrac {\dd \beta}{2}=\dd f_1\ \dd \beta=\sigma \ \dd l_1 \cdot \dfrac{\dd l_2}{r_2}=\dfrac{\sigma}{r_2}\cdot \dd S \quad $ 
\textcolor{gris}{\small{arco=ángulo x radio}\normalsize{.}}

Análogamente: $\quad \dd F_2= \dfrac{\sigma}{r_1} \cdot \dd S$


La fuerza total: $\quad \dd F=\dd F_1+\dd F_2= \left( \dfrac{\sigma}{r_1}+\dfrac{\sigma}{r_2} \right) \ \dd S$
Como $P=\dfrac {\dd P}{\dd S}$, finalmente tendremos que:

\begin{equation} 
\subrayado{ \ \boldsymbol{ P=	\left( \dfrac{\sigma}{r_1}+\dfrac{\sigma}{r_2} \right) } \ } \qquad \text{ Presión de Laplace.}
\end{equation}

En el caso en que la presión de Laplace no sea despreciable, la \emph{ecuación fundamental de la hidrostática} se escribirá como: $\subrayado{\boldsymbol{P=P_0+\rho g h+ P_L}}$, siendo $P_L$ la presión de Lapalace. Cuando el \textsf{menisco es cóncavo}, $P_L$ representa una \textsf{sobrepresión}; en \textsf{meniscos convexcos}, cuando la presión del Laplace no sea despreciable, $P_L$ representará una \textsf{depresión} (la resultante $\dd F$ es hacia arriba, hacia fuera de la superficie del líquido).

\vspace{40mm} %******************************
Convenio de signos:	
\begin{multicols}{2}
Tomamos los radios como positivos cuando trazados desde el centro de curvatura apuntan al exterior del fluido (meniscos cóncavos - sobrepresión).

Tomamos los radios como negativos cuando trazados desde el centro de curvatura apuntan al interior del fluido (meniscos convexos - depresión).

\begin{figure}[H]
	\centering
	\includegraphics[width=.5\textwidth]{imagenes/imagenes08/T08IM08.png}
\end{figure}
\end{multicols}

\section{Gotas. Ley de Tate}

La forma que tienden a adoptar las masas de líquidos es esférica y se da solo en masas pequeñas debido a la acción del campo gravitatorio terrestre.

La fuerza gravitatoria es proporcional a la masa, es decir, a todo el volumen del cuerpo; la fuerza debida a la tensión superficial es proporcional a la superficie: $\quad F_G \propto V;\quad F_{\sigma} \propto S \ \ \to \quad$
$\dfrac{F_G} {F_{\sigma}}$
$=\dfrac{\frac 4 3 \pi r^3}{4 \pi r^2}=\dfrac 1 3 r$

Para valores grandes de $r$ \textcolor{gris}{los matemáticos dicen: $r>>1$}, la fuerza gravitatoria es mucho mayor que la de tensión superficial, $F_G>>F_\sigma$.  Sabemos, además, que las posiciones de equilibrio exigen una energía potencial mínima (Lejeune Dirichlet \textcolor{gris}{ válido en campos conservativos como el gravitatorio y el electromagnético --Van der Waals--}).

Imaginemos una gota muy grande, gota en posición 1 de la siguiente figura. La energía potencial de una partícula $\dd m$ situada a una altura $y$ es $\dd \mathcal E_p=gy\dd m$. Mientras no exista un mecanismo que lo impida, la gota pasará a la posición 2.

\begin{figure}[H]
	\centering
	\includegraphics[width=1\textwidth]{imagenes/imagenes08/T08IM09.png}
\end{figure}


Para gotas muy pequeñas, $r \to 0 \Rightarrow \dfrac {F_G}{F_\sigma} \to 0 \Rightarrow F_G<<F_\sigma$ y se adpota la forma esférica de energía potencial mínima.

\textbf{Caída de una gota en un cuentagotas:}


\begin{multicols}{2}
En el momento en que cae la gota, la tensión superficial tiene una resultante hacia arriba que sujeta la gota al cuentagotas hasta que aumentemos la presión en el chupón. Caerá cuando las fuerzas hacia abajo, positivas por convenio, sean mayores que las fuerzas hacia arriba (negativas). En el instante de caer la gota, ambas fuerzas son iguales.

Fuerzas que intervienen:

$+mg;\quad -2\pi r \sigma;\quad -P_0\pi r^2;\quad +P \pi r^2$
\begin{figure}[H]
	\centering
	\includegraphics[width=.2\textwidth]{imagenes/imagenes08/T08IM10.png}
\end{figure}	
\end{multicols}

En el instante que cae la gota, la suma total de fuerzas ha de ser cero:

$mg+P\pi r^2=2\pi r \sigma+ P_0\pi r^2$

Por Pascal (ecuación fundamental de la hidrostática): 

$\ P=P_0+\cancelto{0}{P_{\text{altura}}}+P_{Laplace}=P_0+0+\dfrac \sigma r$

\textcolor{gris}{Laplace: $r_1=r;\ \ r_2\to \infty$}

De donde: $\ mg+\pi r^{\cancel{2}} \dfrac{\sigma}{\cancel{r}}+\bcancel{P_0 \pi r^2}=2\pi r \sigma+\bcancel{P_0\pi r^2}$

Se obtiene:

\begin{equation}
\subrayado{\ \boldsymbol{ mg=\pi r \sigma }\ }	 \qquad \text{Ley de TATE}
\end{equation}

La gota se desprende en el mismo momento en que su peso, $mg$ es igual a $\pi r \sigma$ ($r\approx\ $ radio del capilar con el que trabajamos). \textcolor{gris}{Capilar: que tiene un tamaño similar al de una cabello, se dice del tubo muy delgado}.

La ley de Tate es útil para el cálculo de tensiones superficiales y, conocidas éstas y el tamaño del capilar, para el cálculo de la masa de un fluido sin más que contar gotas.

\vspace{3mm}\textbf{Consideraciones:}
\begin{itemize}
\vspace{-2mm}\item Cuando la gota se desprende adopta forma esférica. Para la ley de Laplace a una gota, $r_1=r_2=r\to P_L=2\sigma /r$. La presión $P$ en el interior de la gota es: $\ P=P_0+\dfrac{2\sigma}{r}.\ $ Si la gota es muy pequeña, $P$ es muy grande. Las moléculas que forman la gota están sometidas a una gran presión desde el interior hacia el exterior por lo que las gotas más pequeñas se evaporan más rápidamente.
\vspace{-2mm}\item La gota grande absorbe a la más pequeña:  en la gota grande la presión interna es menor que en la gota pequeña, ello trae como consecuencia que se establezca un \emph{gradiente de presiones} desde la mayor presión (gota pequeña) a la menor presión (gota grande) y ésta última acaba absorbiendo a la gota pequeña.
\end{itemize}

\begin{figure}[H]
	\centering
	\includegraphics[width=1\textwidth]{imagenes/imagenes08/T08IM18.png}
\end{figure}


\subsection{Pompa de jabón.}

En la superficie $a$, que es convexa respecto a la masa del líquido, la fuerza debida a la presión de Laplace apunta hacia afuera del líquido. La superficie $b$ es cóncava respecto de la masa de fluido y, según Laplace, la fuerza irá hacia el interior de la masa de líquido.
\begin{multicols}{2}
$P=P_0+2\pi \ \dfrac \sigma r+2\pi \ \dfrac \sigma r$

$\quad$

$P=P_0+2\pi \ \dfrac {4\sigma} r$

$\quad$

También explica esta fórmula el efecto de que la gota más grande \emph{chupe} a la más pequeña. A mayor radio, menor presión.
\begin{figure}[H]
	\centering
	\includegraphics[width=.3\textwidth]{imagenes/imagenes08/T08IM11.png}
\end{figure}
\end{multicols}

\section{Capilaridad. Ley de Jurin}

Supongamos que tenemos un fluido en reposo en un recipiente en el que introducimos un tubo estrecho y abierto.


\vspace{40mm} %******************************
\begin{multicols}{2}
$\quad$

Se observa el fenómeno que aparece en la figura. \textcolor{gris}{Imagen tomada de Wikipedia.}\footnote{\footnotesize{De MesserWoland - own work created in Inkscape, based on the graphics by Daniel Stiefelmaier, CC BY-SA 3.0, https://commons.wikimedia.org/w/index.php?curid=1353236}\normalsize{.}}

$\quad$

Cuando el tubo es los suficientemente estrecho, \emph{capilar}, el fluido asciende (desciende) por él respecto del nivel del fluido que hay fuera. 

\begin{figure}[H]
	\centering
	\includegraphics[width=.2\textwidth]{imagenes/imagenes08/T08IM12.png}
\end{figure}

\end{multicols}
En los líquidos que mojan el vaso (meniscos cóncavos), el líquido asciende; en los que no mojan el vaso 8meniscos convexos), el líquido desciende.

\begin{multicols}{2}
Condiciones de equilibrio del fluído en el capilar:

$P_0=P_0`\rho g h-2\dfrac {2\sigma}r$

La presión de la curvatura de Laplace, según el criterio de signos establecido, es negativa.

La presión a nivel del líquido externo en el interior del capilar también es $P_0$ por hidrostática, estamos s al misma altura ($h=0$) y la presión se reparte por igual.


\begin{figure}[H]
	\centering
	\includegraphics[width=.3\textwidth]{imagenes/imagenes08/T08IM14.png}
\end{figure}	
\end{multicols}

\begin{multicols}{2}
Luego: $\ \rho g h = \dfrac {2\sigma}r$

Como: $\ R=r\cos \phi $,

$\rho g h=\dfrac{2\sigma \cos \varphi}R$
	\begin{figure}[H]
	\centering
	\includegraphics[width=.3\textwidth]{imagenes/imagenes08/T08IM15.png}
\end{figure}
\end{multicols}

\begin{equation}
\subrayado{ \ \boldsymbol{h=\dfrac{2\sigma \cos \varphi}{\rho g R}}	\ } \quad \text{Ley de Jurin}
\end{equation}

Si $R\to 0 \  \rightarrow \ h\to \infty.\ $ Cuanto más pequeño es el capilar, mayor es la altura alcanzada.

Para un capilar $R$ dado, los ascensos y descensos capilares son directamente proporcionales a la tensión superficial del líquido y al coseno del ángulo de conjunción e inversamente proporcionales a la densidad del líquido.

\subsection{Capilaridad en láminas paralelas.}

\begin{multicols}{2}
La altura alcanzada por un líquido entre dos láminas paralelas es la mitad que la que alcanzaría en un tubo cuyo diámetro fuese la mitad de la distancia que separa las láminas.

La fuerza por unidad de longitud, $\sigma \cos \varphi$, actúa sobre los dos lados, $2l$. Esta fuerza se equilibra con el peso del paralelepípedo de fluído $2 r l \cdot h$.
	\begin{figure}[H]
	\centering
	\includegraphics[width=.55\textwidth]{imagenes/imagenes08/T08IM17.png}
\end{figure}
\end{multicols}

\vspace{-4mm} %************************
$$2l\ \sigma \cos \varphi=2rlh\ \rho g \quad \to \quad h=\dfrac{\sigma \cos \varphi}{r\rho g}$$ 

\subsection{Estalagmómetro.}
El estalagmómetro es un aparato de medida de la tensión superficial. Consta de un depósito conectado a un tubo capilar por el que se dejan caer gotas que se pesan. 

 Está ideado de manera que cuando el líquido salga al exterior lo haga en forma de gotas regulares. La zona ancha, tiene un volumen fijo, V, cuando el líquido está comprendido entre las marcas $A$ y $B$.

 
\begin{multicols}{2}
$M=\rho \ V$, para $n$ gotas:

Cada gota, $\ m=\dfrac {\rho\ V}{n}$

Con otro líquido:

$M'=\rho'\ V \to m'=\dfrac {\rho'\ V}{n'}$

Sus pesos:

$mg=\dfrac {\rho\ V}{n}g; \ m'g=\dfrac {\rho'\ V}{n'}g$

De la ley de Tate: 

$\ \dfrac {\rho\ V}{n} g=\pi r \sigma; \ \ \dfrac {\rho'\ V}{n'} g=\pi r \sigma'$ 

\begin{figure}[H]
	\centering
	\includegraphics[width=.4\textwidth]{imagenes/imagenes08/T08IM16.png}
\end{figure}	
\end{multicols}
Dividiendo ambas expresiones:$\quad \dfrac{\rho/n}{\rho'/n}=\dfrac \sigma {\sigma'}$

Suponiendo $\sigma'$ conocida, determinamos la tensión superficial del otro líquido $\sigma$ sin más que contar gotas:

$$\boldsymbol{ \sigma\ = \ \sigma' \ \dfrac{\rho'\ n'}{\rho\ n} }$$

\section{Problemas}
\begin{prob}.

\begin{multicols}{2}
	Una masa de líquido está girando con velocidad angular $\omega$ constante alrededor del eje vertical que pasa por el centro del recipiente cilíndrico que la contiene. Demuéstrese que la variación de la presión en la dirección radial está dada por $\displaystyle \dv{P}{r}=\rho \omega^2 r$.
\begin{figure}[H]
	\centering
	\includegraphics[width=.25\textwidth]{imagenes/imagenes08/T08IM20.png}
\end{figure}
\end{multicols}

\vspace{-4mm} %*********************************
Llamando $P_c$ a la presión en el eje de rotación ($r=0$) y llamando $P$ a la presión a $r$ metros del eje, demuéstrese que: $\ P=P_C+\frac 1 2 \rho \omega^2 r^2$.
\end{prob}


La fuerza normal a que se encuentra sometido el elemento de masa $\dd m$, de espesor $\dd r$ y situado a $r$ metros del eje es:

$\dd F= \dd m a = \dd m \dfrac {v^2}{r}=\dd m \omega^2 r$

Dividiendo por $\dd V \to  \dfrac{\dd F}{\dd V}=\rho \omega^2 r$, con $\rho=\dd m / \dd V$.

Como $V=\pi r^2 h \to \dd V= 2 \pi r h \dd r=S_L \dd r$, con esto,

$\dfrac {\dd F}{\dd V}=\dfrac{\dd F}{S_L \ \dd r}=\dfrac{\dd P}{\dd r}$, con $\dd P=\dfrac{\dd F}{S_L}$

Por lo que, efectivamente: $\ \dfrac{\dd P}{\dd r}=\rho \omega^2 r$

Por ortra parte: $\ \displaystyle \dd P = \rho \omega^2 r \dd r \to \int_{P_c}^P=\int_{0}^r \rho \omega^2 r$, 

de donde: $\ P=P_c+\frac 1 2 \rho \omega^2 r^2$ 

\begin{prob}
Se tiene una suspensión de gotas esféricas de aceite en una mezcla hidroalcoholica de igual densidad ($0.92$). Calcúlese es trabajo realizado por $200$ de estas gotitas al juntarse para formar una sola gota. La tensión superficial del aceite es de $33 \ \mathrm{dinas/cm}$, en el sistema CGS y el diámetro de cada gotita es de $0.2\ \mathrm{mm}$.
\end{prob}

$\dd W = \sigma \dd S \to \ W=\sigma \Delta S=\sigma (4\pi R^2 -n\  4\pi r^2)$, siendo $r$ el radio de las gotitas, $R$ el de la gota grande y $n$ el número de gotitas que formarán la gota.

Falta averiguar $R$, el tamaño de la gota final. Como se conserva la masa de aceite:

$\rho \frac 4 3 \pi R^3= n\ \rho \frac 4 3 \pi r^3 \to \ R=r\ n^{1/3}$

Luego: $W=\sigma \ 4\pi [\ (rn^{1/3})^2-nr^2 \ ]$

$\sigma:\ \ 1 \ \mathrm{dina}=10^{-5}\ \mathrm{N};\ \ 1\ \mathrm{cm} = 10^{-2}\ \mathrm{m} \to \ 33 \ \mathrm{dinas/cm} = \dfrac {33\cdot 10^{-5}}{10^{-2}}=0.033 \ \mathrm{N/m}; \quad r=10^{-4} \ \mathrm{m}; \quad n=200$

Con los datos del problema: $\ \ W=-0.161 \mathrm{J}$, negativo, el campo de fuerza de la tensión superficial realiza el trabajo, la formación de la gota grande es espontánea, no requiere que se realice trabajo exterior.

\begin{prob}
Una gota esférica de radio $3\ \mathrm{mm}$ se divide en dos gotas iguales. Calcula el tamaño de las gotas resultantes y el trabajo que hay que realizar contra la fuerza de la tensión superficial	para dividir la gota grande. La tensión superficie¡al del aceite es $0.47\ \mathrm{J\ m}^{-2}$.
\end{prob}

Como en el problema anterior, $\rho \frac 4 3 \pi R^3 = 2\cdot \frac 4 3 \pi r^3 \to \ r=R\ 2^{-1/3}=2.4\ \mathrm{mm}$

$W=\sigma \Delta S = 4 \pi \sigma (2\cdot r^2-R^2)=14\ \mu \mathrm{J}$


\begin{prob}
\small{Se introduce un capilar de vidrio de $1,0 \ \mathrm{mm}$ de diámetro hasta el fondo de un vaso de agua de $15 \ \mathrm{cm}$ de profundidad y se sopla hasta formar burbujas de aire. a) Determina el exceso de presión del aire respecto a la presión atmosférica en el interior de las burbujas, suponiendo que éstas tienen el mismo diámetro que el capilar. b) Calcula lo que asciende o desciende el agua por el capilar respecto a la superficie del agua cuando se deja de soplar. Supón que el ángulo de contacto del agua con la superficie de vidrio es aproximadamente $0^o$. c) Si se sustituye el agua por glicerina, ?`habrá que soplar con mayor o menor fuerza para formar las burbujas? d) ?es mayor o menor el nivel de la glicerina respecto al agua en el capilar cuando se deja de soplar? Supón que el ángulo de contacto de la glicerina con la superficie de vidrio también es aproximadamente $0^o$. Datos: $\sigma_{agua}=72.8 \ \mathrm{m\ N/m}$, $\sigma_{glicerina}=63.1 \ \mathrm{m\ N/m}$, $\rho_{glicerina}=1.26  \times 10^3 \ \mathrm{kg}/\mathrm{m}^3$	}\normalsize{.}
\end{prob}


---a) El exceso de presión en el interior de las burbujas será:

$P=P_0+\rho g h + \dfrac {2\sigma}{R} \to P-P_0=\rho g h + \dfrac {2\sigma}{R}$, con los datos del problema: $P=1760\ \mathrm{Pa}$

--- b) La altura a la que sube el agua por el capilar es:

$h=\dfrac{2\sigma \cos \varphi}{\rho g R}$, con los datos del problema: $h=0.030 \ \mathrm{m}$

--- c) Si cambiamos por glicerina, el exceso de presión será:

$ P'-P_0=\rho' g h + \dfrac {2\sigma'}{R}$, con los datos del problema: $P'=2100\ \mathrm{Pa}$

--- d) La altura a la que sube la glicerina es:

$h'=\dfrac{2\sigma' \cos \varphi'}{\rho' g R}$, con los datos del problema: $h=0.020\  \mathrm{m}$

\newpage %**********************************************
\begin{myblock}{Fuerzas intermoleculares. Cohesión.}

\vspace{2mm} Las moléculas de los líquidos no oscilan respecto a posiciones fijas, como ocurre en los sólidos, sino que gozan de cierta libertad de movimiento. 

\vspace{2mm} Las distancias entre moléculas son lo suficientemente pequeñas como para que se ejerzan fuerzas atractivas de cohesión entre las moléculas. 

\vspace{2mm} Como resultado de tales fuerzas, el líquido ocupa un volumen determinado, aunque su forma sea la del recipiente que lo contiene. 

\vspace{2mm} En los gases, las fuerzas de cohesión son muy pequeñas porque las distancias intermoleculares son muy grandes, de modo que los gases tiende a ocupar todo el espacio del que dispongan. 

\vspace{2mm} La existencia de fuerzas de cohesión en los líquidos determina la existencia de una superficie libre y de los fenómenos asociados a ella. 


\vspace{2mm} 
\begin{multicols}{2}
$\quad$

Las fuerzas intermoleculares son de naturaleza electromagnética (\emph{fuerzas de Van der Waals}). Aunque las moléculas sean eléctricamente neutras, cuando dos moléculas se encuentran separadas una distancia suficientemente pequeña, sus distribuciones de carga se alternan, lo que da lugar a la aparición de una fuerza neta, atractiva o repulsiva, entre ambas. 

\begin{figure}[H]
	\centering
	\includegraphics[width=.35\textwidth]{imagenes/imagenes08/T08IM21.png}
\end{figure}
\end{multicols}

\vspace{2mm} La intensidad de las fuerzas intermoleculares varía con la distancia. 

\vspace{2mm} La fuerza repulsiva intermolecular es la responsable de la impenetrabilidad de la materia (de que una molécula pueda atravesar a otra). 

\vspace{2mm} En cualquier caso las fuerzas intermoleculares son de corto alcance, como máximo algunos radios moleculares, 

\vspace{2mm} El efecto de las fuerzas de cohesión sólo es perceptible a una distancia $r$, que llamaremos radio de acción molecular, que es del orden de $10 \ \textup{\r{A}}$. 
\footnote{http://www.ugr.es/~esteban/earth/apuntesbasesfisicas/tr3.pdf}
	
\end{myblock}
















\include{TEMA09_chapter-A4}
\chapter{Movimiento relativo}

\vspace{15mm} %********************
\begin{miparrafo}

En mecánica newtoniana, un sistema de referencia inercial es un sistema de referencia en el que las leyes del movimiento cumplen las leyes de Newton y, por tanto, la variación del momento lineal del sistema es igual a las fuerzas reales sobre el sistema, es decir, un sistema en el que:
$\quad \vec F_{real} = \displaystyle \dv{\vec p}{t}$


\vspace{2mm} En cambio, la descripción newtoniana de un sistema no inercial requiere la introducción de fuerzas ficticias o inerciales, de tal manera que:
$\quad \displaystyle \dv{\vec p}{t}=\vec {F} _{real}-\vec {F}_{fict}$

\vspace{2mm} Esto lleva a una definición alternativa, un sistema inercial es aquel en que el movimiento de las partículas puede describirse empleando solo fuerzas reales sin necesidad de considerar fuerzas ficticias.

\vspace{2mm} El concepto de sistema de referencia inercial también es aplicable a teorías más generales que la mecánica newtoniana. Así, en la teoría de la relatividad especial también se pueden introducir los sistemas inerciales. Aunque en relatividad especial la caracterización matemática no coincide con la que se da en mecánica newtoniana, debido a que la segunda ley de Newton, tal como la formuló, no se cumple en la Teoría de la relatividad. El concepto de sistema de referencia no fue establecido hasta dos siglos después de la formulación de las leyes de Newton (1687), cuando Ludwig Lange (1885) introdujo el concepto en un intento de eliminar la necesidad de un espacio y un tiempo absolutos del tipo que Newton conjeturaba. Estas ideas fueron, poco más tarde, consideradas en la formulación de la teoría de la relatividad especial.
	
\end{miparrafo}

\vspace{35mm} %********************

\section{Sistema de Referencia inercial}

\begin{multicols}{2}
	
$\displaystyle \vec v_A=\dv{\vec r_A}{t}; \qquad \vec v_B=\dv{\vec r_B}{t}$

$\quad$

$\vec r_{AB}=\vec r_B-\vec r_A$

$\quad$

?`Con que \emph{velocidad relativa} se mueve una partícula respecto de la otra?


\begin{figure}[H]
	\centering
	\includegraphics[width=.35\textwidth]{imagenes/imagenes10/T10IM01.png}
\end{figure}	
\end{multicols}

$\vec v_{AB}=\displaystyle \dv{\vec r_{AB}}{t}=\dv{\vec r_B-\vec r_A}{t}=\vec v_B-\vec v_A$

Análogamente: $\vec v_{BA}=\vec v_B-\vec v_A \to\ $ conclusión: 

\begin{equation}
\boldsymbol{ \vec v_{AB}=-\vec v_{BA} }	
\end{equation}

Derivando respecto al tiempo: $\quad \vec a_{AB}=-\vec a_{BA}$

\section[Movimiento de traslación uniforme. Principio de la relatividad de Galileo]{Movimiento de traslación uniforme. Principio de la relatividad de Galileo\sectionmark{Principio de relatividad de Galileo}}
\sectionmark{Principio de relatividad de Galileo}
\begin{multicols}{2}
Supongamos dos observadores $\mathcal O$ y $\mathcal O'$ que se mueven, uno respecto al otro, con velocidad uniforme $\vec V=\overrightarrow{cte}$, velocidad constante en módulo y dirección ( y sentido).

En $t=0$ los origenes coinciden, $\mathcal O \equiv \mathcal O'$.
\begin{figure}[H]
	\centering
	\includegraphics[width=.5\textwidth]{imagenes/imagenes10/T10IM02.png}
\end{figure}
\end{multicols}
$\mathcal O$, observa una partícula en movimiento. ?`Cómo debe pasarle la información a $\mathcal O'$?

\begin{equation}
\subrayado{\vec r=\vec V \ t + \vec r \ '} \quad \text{Transformación de Galileo de posiciones}	
\end{equation}

derivando respecto al tiempo,

\begin{equation}
\subrayado{\vec v=\vec V \ t + \vec v \ '} \quad \text{Transformación de Galileo de velocidades}	
\end{equation}

si volvemos a derivar respecto del tiempo,

\begin{equation}
\subrayado{\vec a= \vec a \ '} \qquad \qquad \textcolor{gris}{\vec V=\overrightarrow{cte}\to \dv{\vec V}{t}=0}	
\end{equation}

Como la masa es invariante (no depende de $t$), \emph{``las leyes de la dinámica son las mismas en ambos sistemas.''}

En tres dimensiones, si es $X$ el eje de movimiento de $\mathcal O$ respecto de $\mathcal O'$:

\begin{equation}
\begin{cases}
\ \subrayado{x=Vt+x'} \\ \  \subrayado{y=y'} \\ \  \subrayado{z=z'}	
\end{cases}
\qquad \qquad 
\begin{cases}
\  \subrayado{v_x=V+v'_x} \\ \ \subrayado{v_y=v'_y} \\ \ \subrayado{v_z=v'_z}	
\end{cases}	
\end{equation}


\textbf{Principio de relatividad de Galileo:} \emph{``Cuando se tienen dos sistemas de referencia que se desplazan uno respecto del otro a velocidad uniforme, \colorbox{LightYellow}{las leyes de la dinámica} son las mismas en ambos sistemas.''}

Newton pone su sistema de referencia en las `estrellas fijas', le llama \emph{``sistema de referencia inercial.''}. En todos los infinitos sistemas de referencia que se muevan con $\vec V=\overrightarrow{cte}$ respecto del `sistema inercial de Newton' son válidas las leyes de Newton (dinámica).

Como la tierra gira, $\vec V \neq \overrightarrow{cte}$ respecto del sistema inercial de Newton, no serán válidas las leyes de Newton, las tendremos que transformar.

\section{Operador derivada respecto a $t$}

\textbf{Operador derivada respecto del tiempo en el cambio de sistemas de referencia en rotación.}

\begin{multicols}{2}
Ambos observadores $\mathcal O \text{ y } \mathcal O'$, sistemas de referencia, giran uno respecto al otro con velocidad angular $\vec \omega$, sin desplazarse.

Sea $\vec A$ un vecor cualquiera:

Medido por $\mathcal O$:

$\vec A=\vec i A_x+\vec j A_y+ \vec k A_z$

Medido por $\mathcal O'$:

$\vec A=\vec i\ ' A'_x+\vec j\ ' A'_y+ \vec k\ ' A'_z$
\begin{figure}[H]
	\centering
	\includegraphics[width=.3\textwidth]{imagenes/imagenes10/T10IM03.png}
\end{figure}
\end{multicols}

El observador $\mathcal O$ mide como varía $\vec A$ respecto del tiempo:

$\displaystyle \left( \dv{\vec A}{t} \right)_{\mathcal O}=
\left[ \dv{t} \left( \vec i\ ' A'_x+\vec j\ ' A'_y+ \vec kº ' A'_z \right) \right]_{\mathcal O} =$

$= \displaystyle \left[ \vec i\ ' \dv{A_x}{t} + \vec j\ ' \dv{A_y}{t} + \vec k\ ' \dv{A_z}{t} \right] \ + \ 
\left[ A'_x \dv{\vec i\ '}{t} + A'_y \dv{\vec j\ '}{t} + A'_z \dv{\vec k\ '}{t} \right] =$

$= \displaystyle  \left( \dv{\vec A}{t} \right)_{\mathcal O'} \ + \ 
\left[ A'_x \dv{\vec i\ '}{t} + A'_y \dv{\vec j\ '}{t} + A'_z \dv{\vec k\ '}{t} \right] $

Como $\displaystyle A'_x \dv{\vec i\ '}{t} + A'_y \dv{\vec j\ '}{t} + A'_z \dv{\vec k\ '}{t}= A'_x\ \vec \omega \times \vec i + A'_y\ \vec \omega \times \vec j + A'_z\ \vec \omega \times \vec k$,

\begin{equation}
	\boldsymbol{ \left( \dv{\vec A}{t} \right)_{\mathcal O}=\left( \dv{\vec A}{t} \right)_{\mathcal O'}+\vec \omega \times \vec A }
\end{equation}

Escrito en forma de \emph{operador matemático:}

$$\boldsymbol{ \left( \dv{t}\right)_{\mathcal O} =  \left( \dv{t}\right)_{\mathcal O'} + \vec \omega \times \underline{\ \ }  }$$

\section{Movimiento relativo rotacional}

Ambos observadores $\mathcal O \text{ y } \mathcal O'$, sistemas de referencia, giran uno respecto al otro con velocidad angular $\vec \omega$, sin desplazarse. 

Ambos miden el mismo vector de posición da una partícula en movimiento en un tiempo $t$: $ \ \  \vec r = \vec r'$. Veamos que ocurre con las velocidades.

$$\displaystyle \vec v = \left( \dv{\vec r}{t} \right)_{\mathcal O}=
\left( \dv{\vec r}{t} \right)_{\mathcal O'} + \vec \omega \times \vec r$$

$$ \boldsymbol{ \vec v = \vec v \ ' + \vec \omega \times \vec r } $$

Relación entre las velocidades $\vec v$ y $\vec v'$ de un mismo punto material que miden dos observadores que están girando uno respecto del otro con velocidad angular $\vec \omega$ pero sin separarse.
  
Derivando respecto del tiempo:

$\vec a = \displaystyle \left( \dv{\vec v}{t} \right)_{\mathcal O} =  \left( \dv{\vec v}{t} \right)_{\mathcal O'}+ \vec \omega \times \vec v \ $ \textcolor{gris}{ Aplicando el operador derivada respecto del tiempo para sistemas de referencia en rotación.}

Por otra parte, \textcolor{gris}{teniendo en cuenta la expresión anterior $\vec v = \vec v \ ' + \vec \omega \times \vec r$}:

$\displaystyle \vec a = \left[ \dv{t} (\  \vec v \ ' + \vec \omega \times \vec r \ ) \right]_{\mathcal O'} + \vec \omega \times (\  \vec v \ ' + \vec \omega \times \vec r \ ) = $

$\displaystyle = \left( \dv{\vec v\ '}{t} \right)_{\mathcal O'}+ \left(  \dv{\vec \omega}{t}\right)_{\mathcal O'} + \left( \dv{\vec r}{t} \right)_{\mathcal O'} \times \vec r +
\vec w \times \vec v\ ' + \vec \omega \times ( \ \vec \omega  \times  \vec r \ )$

Como $\ \vec r=\vec r\ ' \ \to \ \displaystyle \left( \dv{\vec r}{t} \right)_{\mathcal O'} =\left( \dv{\vec r \ '}{t} \right)_{\mathcal O'} = \vec v\ '$


$$\boldsymbol{ \displaystyle \vec a = \vec a\ ' + \dot{\vec \omega} \times \vec r + 2 \vec \omega \times \vec v\ '+ \vec \omega \times ( \ \vec \omega  \times  \vec r) }$$

Siendo $ \ \dot{\vec \omega}=\displaystyle \dv{\vec \omega}{t}$

--- El término $\dot{\vec \omega} \times \vec r$ no figura cuando $\vec \omega=\overrightarrow{cte}$.

--- El término $2\vec \omega \times \vec v\ '$ recibe el nombre de \textbf{\emph{aceleración de coriolis.}}

--- El último término, $\vec \omega (\ \vec \omega \times \vec r \ )$, es la \textbf{\emph{aceleración centrípeta}}, apunta hacia el eje de movimiento ($\vec \omega$).

Nota: El paréntesis en $\vec \omega \times ( \ \vec \omega  \times  \vec r \ )$ es muy importante pues \textbf{el producto vectorial no es}, \emph{afortunadamente}, \textbf{asociativo}. $\quad$ ;-)

\begin{figure}[H]
	\centering
	\includegraphics[width=.9\textwidth]{imagenes/imagenes02/T02IM33.png}
\end{figure}


Si $\mathcal O$ es un observador inercial $\to \mathcal O'$ no será un observador inercial, para él no serán válidas las leyes de Newton, despejando $\vec a \ '  $:

$ \displaystyle m \cdot \vec a\ ' =  m \cdot \vec a -  m \cdot \dot{\vec \omega} \times \vec r - 2  m \cdot \vec \omega \times \vec v\ '-  m \cdot \vec \omega \times ( \ \vec \omega  \times  \vec r) $


Si consideramos $\ \vec \omega=\overrightarrow{cte}$ queda

$ \displaystyle m \cdot \vec a\ ' =  m \cdot \vec a -  \cancelto{0}{m \cdot \dot{\vec \omega} \times \vec r} - 2  m \cdot \vec \omega \times \vec v\ '-  m \cdot \vec \omega \times ( \ \vec \omega  \times  \vec r) $,

$\vec F \ '= \vec F + \vec F_{\text{coriolis}}+ \vec F_{\text{centrípeta}}$

\begin{miparrafodestacado}
	 Ambas nuevas fuerzas que ahora aparecen, la \emph{fuerza de coriolis} y la \emph{fuerza centrífuga}, son \emph{fuerzas de inercia} que no tienen existencia real y solo se manifiestan cuando elegimos un sistema de referencia no inercial.
\end{miparrafodestacado}


\section{Movimiento relativo general}

Los dos observadores que desean intercambiar información, además de girar, se trasladan.

\begin{multicols}{2}
$\vec r=\vec h + \vec r\ '$

$\vec v=\displaystyle \left( \dv{\vec r}{t} \right)_{\mathcal O}=
\left( \dv{\vec h}{t} \right)_{\mathcal O} + \left( \dv{\vec r\ '}{t} \right)_{\mathcal O}$

Utilizado el operador derivada respecto del tiempo visto anteriormente,

$\vec v=\displaystyle \left( \dv{\vec h}{t} \right)_{\mathcal O} + \left( \dv{\vec r \ '}{t} \right)_{\mathcal O'}+\vec \omega \times \vec r \ '$

\begin{figure}[H]
	\centering
	\includegraphics[width=.3\textwidth]{imagenes/imagenes10/T10IM04.png}
\end{figure}
\end{multicols}

$\vec v=\displaystyle \left( \dv{\vec h}{t} \right)_{\mathcal O} + \left( \dv{\vec r \ '}{t} \right)_{\mathcal O'}+\vec \omega \times \vec r \ '$

El primer término del segundo miembro responde a una traslación y los otros dos términos a una rotación.

Cualquier movimiento arbitrario se puede descomponer como un movimiento de  traslación y un movimiento de rotación.

$$\boldsymbol{ \displaystyle \vec v=\vec v\ ' + \vec \omega \times \vec r\ '+  \left( \dv{\vec h}{t} \right)_{\mathcal O} }$$


Derivando respecto al tiempo,

$\vec a= \left( \displaystyle \dv{\vec v}{t} \right)_{\mathcal O}=
\left\{  
\displaystyle \dv{t} \left[
\vec v \ ' + \vec \omega \times \vec r\ ' + \left( \dv{\vec h}{t} \right)_{\mathcal O}
\right]
\right\}_{\mathcal{O}}$

Teniendo en cuenta el operador derivada respecto al tiempo en rotaciones,

$$\boldsymbol{ 
\vec a = \vec a\ ' + 2 \vec \omega \times \vec v\ '+ \dot{\vec \omega}\times \vec r \ ' + \vec \omega \times (\vec \omega \times \vec r \ ')+ \left( \displaystyle \dv[2]{\vec h}{t} \right)_{\mathcal O}
}$$

El último término del segundo miembro es es responsable de la traslación (aceleración con que se desplaza $\mathcal{O '}$ respecto de $\mathcal{O}$), el resto de términos responden a la rotación.

De esa expresión falta despejar $\vec a'$ y multiplicar por la masa.


\textbf{Aplicación al caso del movimiento relativo de la tierra}

Como la Tierra no es un sistema inercial, vamos a ver como se particularizan las ecuaciones anteriores para un observador solidario con la Tierra (que viaja con ella).

Aproximación: La tierra, como sistema físico, tiene dos movimientos de rotación ($\omega_{24\text{ horas}}>\omega_{24\text{ días}}$). Para periodos de tiempos pequeños ($\sim 1 \text{ día}$), la tierra se desplaza en línea recta.

Por otro lado, $\ \vec \Omega_{total}=\vec \omega_{Tierra}+ \cancelto{despreciable}{\vec \omega_{Sol}}$  

\vspace{35mm} %********************
\begin{multicols}{2}
El sistema $\mathcal O$ se desplaza, con velocidad uniforme, respectonal sistema de las estrella fijas inercial de Neqton.	

$\displaystyle \left( \dv{\vec h}{t} \right)_{\mathcal O}=\vec \omega \times \vec h;  \quad \textcolor{gris}{(\vec v=\vec \omega \times \vec r)}$

$\displaystyle \left( \dv[2]{\vec h}{t} \right)_{\mathcal O}= \dv{t} (\vec \omega \times \vec h )= $ 

$=\displaystyle \cancelto{0}{0 \times \vec h} + \vec \omega \times \vec v= \vec \omega \times (\vec \omega \times \vec h)$

\begin{figure}[H]
	\centering
	\includegraphics[width=.35\textwidth]{imagenes/imagenes10/T10IM05.png}
\end{figure}
\end{multicols}

Para el caso de que $\mathcal O'$ sea un laboratorio sobre la Tierra, $\omega$ es prácticamente constante por lo que $\dot{\vec \omega} \times \vec r\ '=0$


$\vec a = \vec a\ ' + 2 \vec \omega \times \vec v\ '+ \cancelto{0}{\dot{\vec \omega}\times \vec r \ ' }+ \vec \omega \times (\vec \omega \times \vec r \ ')+ \left( \displaystyle \dv[2]{\vec h}{t} \right)_{\mathcal O}$


$\vec a = \vec a\ ' + 2 \vec \omega \times \vec v\ '+ \vec \omega \times (\vec \omega \times \vec r \ ')+ \vec \omega \times (\vec \omega \times \vec h)$


$\vec a = \vec a\ ' + 2 \vec \omega \times \vec v\ '+ \vec \omega \times [\vec \omega \times (\vec r \ ' + \vec h)]$

$$\boldsymbol{
\vec a = \vec a\ ' + 2 \vec \omega \times \vec v\ '+ \vec \omega \times (\vec \omega \times \vec r)
}$$



Si comparamos este resultado con el del caso de rotación sin traslación en que $\mathcal O$ y $\mathcal O'$ coinciden, el resultado es el mismo. Esto significa que da lo mismo colocar al observador $\mathcal O'$ en la superficie de la tierra que en su centro, con tal de que $\mathcal O'$ gire respecto a $\mathcal O$.

\section{Problemas}
\begin{prob}
Dos trenes A y B se desplazan en rieles paralelos a $70\ \mathrm{km/h}$ y $90\ \mathrm{km/h}$, respectivamente. Calcular la velocidad relativa de B respecto de A cuando el movimiento es en a) el mismo sentido; b) sentido contrario y c) las velocidades forman un ángulo de $60^o$. 	
\end{prob}

\begin{multicols}{2}
	
	$\vec v_A+\vec v_{BA}=\vec v_B; \qquad \vec v_{BA}=\vec v_B-\vec v_A$
	
	a) $ v_{BA}=90-70=20\ \text{km/h}$
	
	b)  $ v_{BA}=90-(-70)=160\ \text{km/h}$
	\begin{figure}[H]
	\centering
	\includegraphics[width=.3\textwidth]{imagenes/imagenes10/T10IM06.png}
\end{figure}
\end{multicols}
c) Teorema del coseno: $\ v_{BA}\sqrt{v_A^2+v_B^2-2v_Av_B\cos 60^o}=170.88\ \text{km/h}$

\begin{prob}
Una persona conduce	un automóvil a través de la tormenta a $80\ \mathrm{km/h}$ y observa que las gotas de lluvia dejan trazos en las ventanas laterales que forman un ángulo de $80^o$ con la vertical. Cuando se detiene, observa que la lluvia cae verticalmente. Calcular la velocidad relativa de la lluvia respecto del coche: a) cuando está detenido y b) cuando circula a $80\ \mathrm{km/h}$.
\end{prob}

\begin{multicols}{2}
	$v_r=v_G-v_c$
	
	$\tan \theta=\dfrac {v_c}{v_G} \to v_G= \dfrac{v_c}{\tan \theta}$
	
	a) $v_G=\dfrac {80}{tan 80^o}=14.107\  \mathrm{km/h}$
	
	b) $v_r=\sqrt{v_c^2+v_G^2}=81.234\  \mathrm{km/h}$
	\begin{figure}[H]
	\centering
	\includegraphics[width=.2\textwidth]{imagenes/imagenes10/T10IM07.png}
\end{figure}
\end{multicols}

\begin{prob}
La brújula de un avión indica que va al Norte y su velocímetro indica que lo hace a $240\ \mathrm{km/h}$. 

a) Si hay un viento de $100\ \mathrm{km/h}$, ?`Cuál es la velocidad del avión relativa a la tierra?

b) ?`Qué rumbo debería tomar el piloto para volar hacia el Norte? ?`Cuál sería su velocidad relativa respecto a la Tierra?
\end{prob}
\begin{figure}[H]
	\centering
	\includegraphics[width=.8\textwidth]{imagenes/imagenes10/T10IM09.png}
\end{figure}

Llamamos $\vec v_{AvT}$ a la velocidad del avión respevto de la Tierra; $\vec v_{AvA}$; la del avión respecto del aire y $\vec v_{AT}$ a la velocidad del aire respecto a la tierra.

--- a) $\vec v_{AvA}=240 \text{km/h};\quad \vec v_{AT}=100\ \text{km/h}$

$\vec v_{AvT}=\vec v_{AvA}+\vec v_{AT}$

$v_{AvT}=\sqrt{240^2+100^2}=260\ \text{km/h}$

$\tan \alpha=\dfrac {v_{AT}}{v_{AvA}}\ \to \ \  \alpha = 23^o $ al Este desde el Norte.

--- b) $\vec v_{AvT}=\vec v_{AvA}+\vec v_{AT}$

Ahora, $\vec v_{AvT}$ es desconocida, pero se dirige al Norte; $\vec v_{AvA}=240$ km/h, pero con dirección $\beta$ desconocida  y $\vec v_{AT}=100$ km/h, en dirección W a E.

$v_{AvT}= \sqrt{240^2-100^2}=218\ $ km/h.

$\sin \beta=\dfrac{v_{AT}}{v_{AvA}}\ to \ \ \beta=25^o$ al Oeste desde el Norte.



\begin{prob}
La posición de una partícula Q en un sistema de coordenadas $\mathcal O$	viene dada por $\vec r=\{ \vec i\ (6t^2-4t)+\vec j\ (-3t^2)+\vec k\ 3\}\ \mathrm{m}$

a) Determinar del sistema $\mathcal O'$ respecto de $\mathcal O$ si la posición de Q medida desde $\mathcal O'$ es $\vec r\ '=\{ \vec i\ (6t^2+3t)+\vec j\ (-3t^2)+\vec k\ 3\}\ \mathrm{m}$.

b) Demostrar que la aceleración de la partícula es la misma en ambos sistemas de referencia.
\end{prob}

\begin{multicols}{2}
	$\vec h=\vec r-\vec r\ '$
	
	$\quad$
	
	$\displaystyle \dv{\vec h}{t} = \vec v_{\mathcal O' \mathcal O}=\dv{\vec r}{t}-\dv{\vec r\ '}{t}$
	\begin{figure}[H]
	\centering
	\includegraphics[width=.3\textwidth]{imagenes/imagenes10/T10IM08.png}
\end{figure}
\end{multicols}
a) $\vec v_{\mathcal O' \mathcal O}=[\ \vec i (12t-4)+\vec j (-6t)+\vec k 0 \ ]-[\ \vec i(12t+3)+\vec j(-6t)+\vec k 0  \ ]=-7\vec i\ \text{m}$

b) $\displaystyle \dv{\vec v_{\mathcal O' \mathcal O}}{t}=\vec a -\vec a \ '=0 \to \vec a=\vec a \ '$

%********************************************************
\newpage %***********************************************
\begin{myblock}{Relatividad de Galileo.}
\begin{small}
\vspace{2mm} Se entiende por la teoría de la relatividad clásica o relatividad de Galileo al estudio del movimiento de una partícula condicionado a un sistema de referencia arbitrariamente escogido. De este modo se establece que la percepción y la medida de las magnitudes físicas varían en función al sistema de referencia escogido. Para poner un ejemplo: no es lo mismo observar la caída de una manzana que está moviéndose en un tren si lo vemos desde fuera del tren (la manzana hace una parábola) o desde dentro (la manzana cae en vertical).

\vspace{2mm} Esta dependencia de la percepción del movimiento según el sistema de referencia escogido es lo que se conoce como relatividad clásica. Fue descrita por Galileo Galilei en el ejemplo del barco que aparece en sus Diálogos sobre los dos máximos sistemas del mundo en el siglo XVII.

\vspace{2mm} Su famosa frase: Eppur si muove (``Y sin embargo se mueve’’) es el resumen de la mentalidad de la época ante un hecho actualmente reconocido. La Tierra se mueve alrededor del Sol, si bien sus habitantes no percibimos que esta alcanza velocidades de hasta 106.000 km/h pues nosotros mismos nos movemos a esa velocidad. 

\vspace{2mm} Esta teoría de la relatividad clásica se conoce también como invariancia galileana y es el primer paso hacia la realidad física más general que se conoce como teoría de la relatividad especial desarrollado por Albert Einstein en 1905.

\vspace{2mm} La teoría clásica de la relatividad establecía que las magnitudes físicas eran dependientes del sistema de referencia escogido pero presuponía que el tiempo era un ente absoluto e independiente del sistema de referencia escogido. Sin embargo en el siglo XX, tras el experimento de Michelson y Morley quedó demostrada la invariabilidad de la velocidad de la luz lo cual condujo al descubrimiento de la relatividad de ambos espacio y tiempo.

\vspace{2mm} Los descubrimientos de Galileo sobre la relatividad de las percepciones de la realidades físicas espaciales ante distintos sistemas de referencia supusieron una auténtica revolución en su época.	
\end{small}
\end{myblock}

\include{TEMA11_chapter-A4}
\chapter{Dinámica de los sistemas de partículas}
\chaptermark{Dinámica sistemas partículas}

\begin{miparrafo}
Una partícula es una entidad física que posee masa pero carece de dimensiones geométricas.

Para un cuerpo real se nos presenta una dificultad. Cualquiera de las partículas que forman el cuerpo puede estar sometida a una fuerza exterior debido a la posición que ocupa en el universo y también sometido a una fuerza interior debida a las propias partículas que forman el cuerpo. (En $1 \text{mm}^3$ de líquido hay $\sim 10^{24}$ partículas, tendríamos una ingente cantidad de ecuaciones que plantear).

La aproximación de punto material es válida en los movimientos de traslación y en aquellos casos en los que la precisión en la localización del cuerpo es del orden de las dimensiones de éste. Por tanto, hay de proponer un nuevo modelo que permita estudiar los cuerpos, y su evolución temporal, en los casos en que la aproximación anterior no sea válida. Este modelo es el de \emph{sistemas de partículas:  un conjunto de partículas cuyas propiedades globales queremos estudiar}. 

Los sistemas de partículas se pueden clasificar en:
\begin{itemize}
	\item Sistema discreto, cuando el cuerpo se considera formado por un número finito de partículas. Dentro de este modelo podemos considerar:
		\begin{itemize}
		\item Sistemas indeformables, en los que la distancia relativa entre las partículas del sistema permanece inalterable en el tiempo.
		\item Sistemas deformables, en los que puede cambiar la distancia relativa entre las partículas.
		\end{itemize}
	\item Sistemas continuos, cuando un cuerpo puede considerarse formado por una distribución “continua” de materia (llenando todo el espacio que ocupa). Estos sistemas se dividen en deformables e indeformables (sólidos rígidos).
\end{itemize}
Vamos a estudiar si las magnitudes importantes para la física se conservan o no para el sistema de partículas.
\end{miparrafo}

\section[Teorema de conservación de la cantidad de movimiento. Sistema de referencia centro de masas y de laboratorio]{Teorema de conservación de la cantidad de movimiento. Sistema de referencia centro de masas y de laboratorio\sectionmark{Conservación cantidad de movimiento}}
\sectionmark{Conservación cantidad de movimiento}

Consideremos un sistema formado por $N$ partículas y sea $\vec F_i^{(e)}$ la fuerza externa que actúa sobre la partícula $i$. La fuerza que ejercen el resto de partículas, $N-1$, sobre la partícula $i$ es:

$\vec F_{i1}+\vec F_{2i}+ \cdots + \vec F_{(N-1)i}+\vec F_{(N+1)i}+ \cdots +\vec F_{Ni}$

Aplicando la segunda ley de Newton:

$\vec F_i^{(e)}\ + \ \vec F_{i1}+\vec F_{2i}+ \cdots + \vec F_{(N-1)i}+\vec F_{(N+1)i}+ \cdots +\vec F_{Ni}=\displaystyle \dv{\vec p_i}{t}$

Esquemáticamente: $\ \displaystyle \dv{\vec p_i}{t}= \sum_{\substack{j=1 \\ j\neq i}}^{N} \vec F_{ji} \ + \ \vec F_i^{(e)}$

Para las $N$ partículas habrá que resolver $N$ ecuaciones como esta. Sumándolas todas:$\quad \displaystyle \dv{t} \sum_{i=1}^N \vec p_i = \sum_i \sum_j \vec F_{ji}+\sum_i \vec F_i^{(e)}$

Suponiendo que se cumple la ley de acción-reacción, $\vec F_{ji}=-\vec F_{ij}$, por tanto, $\quad \displaystyle \sum_i \sum_j \vec F_{ji}=0$ y, finalmente:

\begin{equation}
\subrayado{\ \boxed{ \boldsymbol{ \dv{\vec P}{t}=\vec F^{(e)}}\ }\ } \qquad \text{ con } \vec F^{(e)}=\sum_i	\vec F_i^{(e)} \ \  \text{ y } \ \ \vec P=\sum_i \vec p_i
\end{equation}

Expresión que se conoce como \subrayado{\emph{``Teorema de la cantidad de movimiento''}} y viene a decir que:

\begin{miparrafodestacado}
	 \emph{`un sistema de partículas evoluciona de idéntica forma a lo que lo haría una partícula sobre la que actuase una fuerza externa'.}
\end{miparrafodestacado}


Cuando las coordenadas  que usamos son las del laboratorio se dice que estamos usando el \emph{sistema de referencia de laboratorio}.

Al igual a lo que pasaba para una partícula, para un sistema de partículas, si $ \subrayado{\vec F^{(e)}=\vec 0 \ \to \ \displaystyle \sum_i \vec p_i=\overrightarrow{cte}}$

Vamos a introducir un nuevo sistema de referencia, el \emph{\textbf{sistema de referencia centro de masas}}.

\begin{miparrafodestacado}
El centro de masas, $CM$, de un sistema de $N$ partículas es, por definición, aquel punto del espacio que tiene un vector de posición $\vec R$ que viene determinada por la siguiente expresión:	
\end{miparrafodestacado}

\begin{equation}
\subrayado{\boldsymbol{ \vec R=\dfrac 1 M \ \sum_{i=1}^{N} m_i \ \vec r_i}}
\end{equation}

siendo $\vec r_i$ el vector de posición de la partícula $i$ respecto del origen $\mathcal{O}$ que hayamos tomado.

Veamos como evoluciona el $CM$ respecto al tiempo:

\begin{equation}
\subrayado{\boldsymbol{\vec v_{CM}=\dv{\vec R}{t}}}= \dfrac 1 M \ \sum_{i=1}^N m_i\ \vec v_i=\frac 1 M \sum_{i=1}^N \vec p_i= \subrayado{ \boldsymbol{\dfrac {\vec P}{M}}}
\end{equation}

Aplicando el \emph{teorema de la cantidad de movimiento},

\begin{equation}
\subrayado{\ \boxed{\ \boldsymbol{M\ \dv{\vec v_{CM}}{t}}\ =\ \vec F^{(e)}  \ } \ }	
\end{equation}

\begin{miparrafodestacado}
\emph{El $CM$ de un sistema de partículas se comporta como si se tratase de una partícula de masa $M$ igual a la masa total del sistema y que se desplaza con la velocidad del centro de masas, $\vec v_{CM}$	}
\end{miparrafodestacado}

\emph{Un \textbf{sistema} $\boldsymbol{CM}$ es un sistema de referencia centrado en el $CM$}

Obviamente, si estamos en el $CM$, la velocidad respecto al centro de masas $\vec v_{CM}=0$
Si  $\vec v_{CM}=\frac {\vec P}{M}=0 \ to \ \vec P=\sum_i \vec p_i=0$,
lo que da una nueva definición al centro de masas: es aquel punto del espacio en que $\sum_i \vec p_i=0=\sum_i m_i\ \vec v_{i, \ CM}$.

\begin{figure}[H]
	\centering
	\includegraphics[width=1\textwidth]{imagenes/imagenes12/T12IM02.png}
\end{figure}

\begin{figure}[H]
	\centering
	\includegraphics[width=1\textwidth]{imagenes/imagenes12/T12IM05.png}
\end{figure}



\section[Momento Cinético. Teorema de König. Teorema del momento angular]{Momento Cinético. Teorema de König. Teorema del momento angular\sectionmark{Teorema del momento angular}}
\sectionmark{Teorema del momento angular}

\textcolor{gris}{ \textsf{ \small{Momento cinético = momento angular}\normalsize{.} } }

\begin{multicols}{2}
$\vec L_i=\vec r_i \times m\vec v_i$

$\vec L =\displaystyle \sum_{i_1}^N \vec L_i  = \sum_{i=1}^N\vec r_i \times m\vec v_i$

$\boldsymbol{\vec r_i=\overrightarrow R + \vec r_{i,CM}}$

Derivando respecto al tiempo:

$\vec v_i=\vec v_{CM}+\vec v_{i,CM}$
\begin{figure}[H]
	\centering
	\includegraphics[width=.4\textwidth]{imagenes/imagenes12/T12IM03.png}
\end{figure}
\end{multicols}

Sustituyendo en la expresión de $\vec L$:

$\vec L =\displaystyle \sum_{i=1}^N\vec r_i \times m \ \vec v_i =
\sum_{i=1}^N \  m_i\ \left[ \ ( \ \vec R +\vec r_{i,CM} \ ) \ \times \ ( \ \vec v_{CM}+\vec v_{i,CM} \ ) \ \right]$

$\vec L = \displaystyle \sum_{i=1}^N \left( 
m_i \vec R \times \vec v_{CM}+m_i \vec R \times \vec v_{i,CM}+m_i \vec r_{i,CM} \times \vec v_{CM}  +m_i \vec r_{i,CM} \times  \vec v_{i,CM}
\right)=$

\small{$\displaystyle =
\vec R \times M\  \vec v_{CM}+\vec R \times \sum_{i=1}^N m_i \vec v_{i,CM}+   \left( \sum_{i=1}^N  m_i \vec r_{i,CM} \right) \times \vec v_{CM}  +  \sum_{i=1}^N   \vec r_{i,CM} \times m_i \vec v_{i,CM}
$}\normalsize{$=$}

Como, 

\hspace{7mm} $\  \ \displaystyle \vec R \times \sum_{i=1}^N m_i \vec v_{i,CM}=\vec R \times \cancelto{0}{\vec P_{CM}}=0 \ $ (cantidad de movimiento del CM respecto al CM) y
  
\hspace{7mm} $\displaystyle  \left( \sum_{i=1}^N  m_i \vec r_{i,CM} \right) \times \vec v_{CM} =0\ $ (posición del CM respecto al CM), 

finalmente:

\begin{equation}
\subrayado{\ \boxed{\  \boldsymbol{ \vec L \ =\ \vec R \times M \ \vec v_{CM}\ + \ \sum_{i=1}^N \ \vec r_{i,CM} \times m_i \ \vec v_{i,CM} }\ }\ }
\end{equation}

\begin{miparrafodestacado}
El \textbf{teorema de König} dice que el momento angular de un sistema de $N$ partículas es la suma de dos térmicos o contribuciones:
\begin{itemize}
\item primer término: momento angular de una sola partícula de masa $M$, la masa total del sistema, desplazándose a la velocidad del centro de masas, $\vec v_{CM}$
\item  Momento angular relativo de las N partículas respecto el $CM$.
\end{itemize}
\end{miparrafodestacado}

$$\subrayado{\ \boxed{\ \boldsymbol{\overrightarrow{L}=\overrightarrow{L}_{CM}+\overrightarrow{L}_{Rel}}\ }\ }$$
\vspace{-8mm} %***************************************
\begin{figure}[H]
	\centering
	\includegraphics[width=1\textwidth]{imagenes/imagenes12/T12IM04.png}
\end{figure}


\textbf{Teorema del momento angular}, veamos como varía el momento angular con el tiempo:

$ \displaystyle  \dv{\vec L}{t} =\dv{t} \sum_{i=1}^N m_i \vec r_i \times \vec v_i= \cancelto{0,\text{ \tiny{paralelos}}}{\sum_i \vec v_i \times m_i \vec v_i} + \sum_i{ \vec r_i \times m_i \dv{\vec v_i}{t} } $

Teniendo en cuenta la segunda ley de Newton,

$ \displaystyle  \dv{\vec L}{t}=\sum_i \vec r_i \times \left[ 
\sum_j \vec F_{ji} \ + \ \vec F_i^{(e)}
\right]$

Luego:

$ \displaystyle  \dv{\vec L}{t}=\sum_i \sum_j \vec r_i \times \vec F_{ji}\ + \ \sum_i \vec r_i \times \vec F_i^{(e)}$

Por la tercera de Newton (acción y reacción) $\vec F_{ji}=-\vec F_{ij}$, aparecen $N$ parejas del tipo: $\vec r_i\times \vec F_{ji}+ \vec r_j \times \vec F_{ij}=(\vec r_i - \vec r_i)\times \vec F_{ji}=\vec r_{ji}\times \vec F_{ji}=0$, al ser vectores paralelos. Con ello:
\vspace{-3mm} %**********************************************
\begin{equation}
	\subrayado{\ 
	\boxed{\ \boldsymbol{
	\dv{\overrightarrow L}{t}\ =\ \overrightarrow{M}^{(e)}
	}
	\ }
	\ }
	\ ; \qquad
	\text{ siendo } \ \overrightarrow{M}^{(e)}=\sum_i \vec r_i \times \vec F_i^{(e)}
\end{equation}
\vspace{-7mm}  %**********************************************
\begin{miparrafodestacado}
En el caso particular de que $\overrightarrow{M}^{(e)}=0 \ \to \ \overrightarrow{L}=\overrightarrow{cte}$, el momento angular total del sistema será un vector constante.
\end{miparrafodestacado}

\section{Teorema de la energia}


Supongamos que tenemos una particula que cambia de una configuración $1$ a una configuración $2$; entonces, el trabajo efectuado será:
$ \int_1^2 \vec F_i \cdot \dd \vec r_i$, para el sistema de $N$ partículas:

$ \displaystyle W_{12}= \sum_{i=1}^N \int_1^2 \vec F_i \cdot \dd \vec r_i= \sum_{i=1}^N \int_1^2 m_i \ \dv{\vec v_i}{t} \cdot \vec v_i \ \dd t = \sum_{i=1}^N \int_1^2 m_i \ \vec v_i \cdot \dd \vec v_i = $

Como $\vec v \cdot  \dd \vec v= v \dd v$, tendremos que 

$\boldsymbol{ W_{12} }=\displaystyle \sum_{i=1}^N \int_1^2 m_i \  v_i \ \dd  v_i = \sum_{i=1}^N \int_1^2 \dd \left( \dfrac 1 2 m_i v_i^2 \right)= \sum_{i=1}^N \left[ \ \mathcal E_{c_i}(2)-\mathcal E_{c_i}(1) \ \right] = \boldsymbol{ \mathcal E_c(2)-\mathcal E_c(1) }, \ $ 

donde $\mathcal E_c(c)=\displaystyle \sum_{i=1}^N \dfrac 1 2 m_i v_i^2 \ $ es la energía cinética del sistema en una configuración dada $c$.

$\vec r_i=\vec R + \vec r_{i,CM}  \to  \vec v_i=\vec v_{CM}+\vec v_{i,CM}  \to  v_i^2=v_{CM}^2 +v_{i,CM}^2 +2v_{CM}v_{i,CM}$

$\mathcal E_c(c)=\displaystyle \sum_{i=1}^N \dfrac 1 2 m_i v_i^2 = \sum_{i=1}^N \dfrac 1 2 m_i \left( v_{CM}^2 +v_{i,CM}^2 +2v_{CM}v_{i,CM} \right)$

$\mathcal E_c(c)=\displaystyle \dfrac 1 2 M  v_{CM}^2 + \sum_{i=1}^N \dfrac 1 2 m_i v_{i,CM}^2 + 2 \frac 1 2 \ v_{CM}\ \cancelto{0, {\text{\tiny{P respecto CM}}}}{\sum_{i}m_i v_{i,CM}}=0$

\begin{equation}
\subrayado{\ 
	\boxed{\ \boldsymbol{
\mathcal E_c(c)=\displaystyle \dfrac 1 2 M  \ v_{CM}^2 \ + \ \sum_{i=1}^N \dfrac 1 2 \ m_i \ v_{i,CM}^2
	}
	\ }
	\ }
\end{equation}

\textbf{Teorema de König de la energía para un sistema de $N$ partículas.}

\begin{miparrafodestacado}
La energía cinética total del sistema de partículas es igual a la energía cinética de una partícula de masa la total del sistema, $M$, que se desplaza a la velocidad del centro de masas, $v_{CM}$ más las energías cinéticas de todas las partículas medidas respecto del centro de masas.
\end{miparrafodestacado}

Por otra parte, como

$W_{12}=\displaystyle \sum_{i=1}^N \vec F_i \cdot \dd \vec r_i= 
\sum_{i=1}^N \int_1^2 \left[ \vec F_i^{(e)}+\sum_j \vec F_{ji} \right] \cdot \dd \vec r_i=$

$= \displaystyle \sum_{i_i}^N \int_1^2 \vec F_i^{(e)} \dd \vec r_i + \sum_i \sum_j \int_1^2 \vec F_{ji} \dd \vec r_i =
W_{ext}+ W_{int} \quad \to $



$$\subrayado{\ \boldsymbol{ \mathcal E_c(2)\ -\ \mathcal E_c(1)=W_{ext}\ +\ W_{int} } \ }$$

Que es el \textbf{teorema más general de la energía}.

Vamos a imponer unas restricciones al sistema. Los resultados que obtengamos serán aplicables solo a sistemas que cumplas esas restricciones.

------ Desarrollemos primero el trabajo realizado por las fuerzas interiores:


$\displaystyle W_{int}=\sum_i \sum_j \int_1^2 \vec F_{ji} \cdot \dd \vec r_i =\sum_i { \int_1^2 \left( \vec F_{1i} \cdot \dd \vec r_i +  \vec F_{2i} \cdot \dd \vec r_i + \cdots + \vec F_{Ni} \cdot \dd \vec r_i   \right)  }$

donde aparecen $N$ parejas del tipo $\vec F_{ij} \cdot \dd \vec r_i + \vec F_{ji} \cdot \dd \vec r_j  $

\textbf{Primera hipótesis:} \emph{Las fuerzas interiores que intervienen cumplen la tercera ley de la dinámica}, ley de acción y reacción: $\vec F_{ij}=-\vec F_{ji}$.

$\vec F_{ij} \cdot \dd \vec r_i + \vec F_{ji} \cdot \dd \vec r_j= \vec F_{ij}\cdot (\dd \vec r_i - \dd \vec r_j)=\vec F_{ij}\cdot \dd \vec r_{ij}  $

\textbf{Segunda hipótesis:} \emph{Admitimos que las fuerzas interiores son conservativas.}

$\vec F_{ij}=-\overrightarrow{\grad}\mathcal E_{p_{ij}} \to \vec F_{ij}\cdot \dd \vec r_{ij}=-\overrightarrow{\grad}\mathcal E_{p_{ij}}\cdot \dd \vec r_{ij}$

\small{ $\dd \vec r_{ij} \cdot \overrightarrow{\grad}\mathcal E_{p_{ij}}= \displaystyle
 \left( \vec i \ \pdv{\mathcal E_{p_{ij}}}{x_{ij}} + 
        \vec j \ \pdv{\mathcal E_{p_{ij}}}{y_{ij}}
        \vec k \ \pdv{\mathcal E_{p_{ij}}}{z_{ij}}   \right) \cdot  
 \left( \vec i \dd x_{ij} + \vec j \dd y_{ij} + \vec k \dd z_{ij} \right) =\dd \mathcal E_{p{ij}}$}
 
 \normalsize{$\vec F_{ij}\cdot \dd \vec r_{ij}=-\overrightarrow{\grad}\mathcal E_{p_{ij}}\cdot \dd \vec r_{ij}=-\dd \mathcal E_{p_{ij}}$}


\textcolor{gris}{Hemos usado que: $\ f(x,y,z)\to \ \displaystyle \pdv{f}{x}\dd x+ \pdv{f}{y} \dd y + \pdv{f}{z} \dd z \ = \ \dd f$}

$W_{int}=\displaystyle -\sum_i \sum_j \int_1^2 \vec F_{ij} \cdot \vec \dd r_{ij}=-\sum_i \sum_{j>i} \int_1^2 \dd \mathcal E _{p_{ij}}$

\textcolor{gris}{Hemos usado $j>i$ para asegurar $j\neq i$}

Luego, para fuerzas interiores conservativas que cumplan la tercera ley de la mecánica (acción y reacción):

$\displaystyle \boldsymbol{ W_{int}=} -\eval{\sum_i \sum_{j>i} \ \mathcal E_{p_{ij}}}_1^2= -\left( \mathcal E_{p,int}(2)-\mathcal E_{p,int}(1) \right) \boldsymbol{ =\mathcal E_{p,int}(1)-\mathcal E_{p,int}(2) }$

Sustituyendo en el teorema general de la energía,

$\displaystyle 
\left[ \  \mathcal E_c(2) + \mathcal E_{p,int}(2) \  \right  ]-
\left[ \  \mathcal E_c(1) + \mathcal E_{p,int}(1) \  \right  ] \ = \
W_{ext}$


Llamando $\subrayado{ \ \boldsymbol{\mathcal U(c)=\mathcal E_c(c) + \mathcal E_{p,int}(c)} \ }$, \colorbox{LightYellow}{\textbf{\emph{energía propia}}} de la configuración c, podemos escribir:

$$\boldsymbol{\mathcal U(2)-\mathcal U(1)=W_{ext}}$$

------ Vamos, ahora, a por el trabajo de las fuerzas externas.

\textbf{Tercera hipótesis:} \emph{Supongamos que también las fuerzas exteriores son conservativas}

\textcolor{gris}{Cada vez que añadimos hipótesis restringimos más el problema.}

$\boldsymbol{\vec F_i^{e}=-\overrightarrow{\grad}\mathcal E_{p_{i,ext}}}$

$\displaystyle W_{ext}=\sum_i \int_1^2 \vec F_i^{e} \cdot \dd \vec r_i=-\sum_i \int_1^2 \overrightarrow{\grad} \mathcal E_{p_{i,ext}} \cdot \dd \vec r_i=-\sum_i \int_1^2 \dd \mathcal E_{p_{i,ext}}=\mathcal E_{p, ext}(1)-\mathcal E_{p,ext}(2)$ 

Substituyendo, también, en la ecuación del teorema general de la energía, obtenemos que en el caso de que las fueras externas cumplan la tercera ley de Newton (acción-reacción) y que las fuerzas internas y externas sean conservativas:

\begin{equation}
\subrayado{
\boldsymbol{ 
	\left[ \mathcal E_c(2) + \mathcal E_{p,int}(2)+ \mathcal E_{p,ext}(2)  \right]    -  \left[ \mathcal E_c(1) + \mathcal E_{p,int}(1)+ \mathcal E_{p,ext}(1) \right]   =  0
	}
	}
\end{equation}

Si llamamos $ \mathcal E_c(c) + \mathcal E_{p,int}(c)+ \mathcal E_{p,ext}(c) = E(c)$, energía total del sistema en el estado $c$, 


\begin{equation}
\subrayado{\boldsymbol{ \ E \ = \ cte\ }}	
\end{equation}

Muchos autores llaman a este resultado \emph{principio de conservación de la energía}, pero no se trata de un principio sino de un teorema basado en las tres leyes de Newton de la mecánica. Es un caso particular (\footnotesize{en que las fueras externas cumplen la tercera ley de Newton (acción-reacción) y que las fuerzas internas y externas son conservativas}) \normalsize{que} se suele presentar en casos macroscópicos.


\section{Problemas}

\begin{prob}
Una granada que cae verticalmente explota en dos fragmentos de igual masa cuando se encuentra a una altura de $2000 \ \mathrm{m}$ y tiene una velocidad hacia abajo de $$60\ \mathrm{m \ s}^{-1}$$. Inmediatamente después de la explosión, uno de los fragmentos se mueve hacia abajo con velocidad $80\ \mathrm{m \ s}^{-1}$	. Encontrar la posición del centro de masas pasados $10\ \mathrm{s}$ de la explosión.
\end{prob}

--- Primer método:

Como resultado de la explosión, las fuerzas exteriores no cambian. El $CM$ continuará moviéndose en línea recta como si no hubiese ocurrido la explosión. Por ello, estará a una distancia $y=y_0+v_0t+\frac 1 2 g t^2$, con $y_0=200\mathrm{m}$, $v_0=-60\ \mathrm{m \ s}^{-1}$, $g=-9.8  \mathrm{m \ s}^{-2}$, con lo que para $t=10\ \text{s}$ se obtiene $\boldsymbol{y_{CM}=910\ \text{m}}$.

--- Segundo método:

Podemos encontrar la posición del centro de masa a partir de las posiciones de los dos fragmentos iguales en que se divide la granada al explotar. Como $\vec F^{(e)}=\vec 0 \ \to \ \displaystyle \sum_i \vec p_i=\overrightarrow{cte}$ y la cantidad de movimiento se conserva.

En el momento de la explosión, $mv_0=m_1v_1+m_2v_2$, 

como $m_1=m_2=\frac 1 2 m$, luego $2v_0=v_1+v_2$, resulta que

$v_0=-60\ \mathrm{m \ s}^{-1};\ v_1=-80\ \mathrm{m \ s}^{-1} \to \ v_2=-40\ \mathrm{m \ s}^{-1}$

Después de $10 \ \text{s}$ de la explosión, cada fragmento se encuentra en la posición $y_i=y_{0i}+v_it+\frac 1 2 g t^2$, con $y_{01}=y_{02}=2000\ \text{m}$, $v_1=-80\ \mathrm{m \ s}^{-1}$, $v_2=-40\ \mathrm{m \ s}^{-1}$, $g=-9.8  \mathrm{m \ s}^{-2}$, $t=10\ \text{s}$, sustituyendo: $y_1=710\ \text{m}$ e $y_2=110\ \text{m}$

El $CM$ estará en: $\boldsymbol{y_{CM}}=\dfrac{\frac 1 2 m \ y_1 + \frac 1 2 m \ y_2}{m}= \dfrac 1 2 (y_1+y_2)=\boldsymbol{910\ \text{m}}$


\begin{prob}.
	\begin{figure}[H]
	\centering
	\includegraphics[width=1\textwidth]{imagenes/imagenes12/T12IM06.png}
\end{figure}
\end{prob}

------ Figura I:

$\vec R_{CM}=(x_{CM},y_{CM})=\dfrac {3((2,2)+1(1,1)+1(3,0)}{3+1+1}=\left( 2, \dfrac 7 5 \right)$

------ Figura II:

Para cuerpos planos  con la masa $M$ uniformemente distribuida en un área $A$, se define la densidad superficial uniforme de carga como $\sigma =\frac M A$. 

Así, podemos considerar nuestro cuerpo como formado por dos: el primero, rectángulo grande, de masa  $m_1= \sigma A_1=\sigma  \ 0.4 \times 0.8=0.32 \ \sigma\  \text{kg}$ situado en su centro de gravedad , que por simetría, $CM_{1}=(0.4,0.2)\ \text{m}$. Con un razonamiento análogo: $m_2=\sigma \ 0.2 \times 0.2=0.04 \ \sigma \ \text{kg}$ situado en $CM_2=(0.6+0.1, 0.4+0.1)=(0.7,0,5)\ \text{m}$

Usando la misma fórmula anterior: $(x_{CM},y_{CM})=(0.43,0.26)\ \text{m}$

--- Figura III

Para superficies continuas con densidad superficial uniforme $\sigma=\frac M A$:

$\displaystyle \vec R_{CM}=(x_{CM},y_{CM})=\dfrac 1 M \int \vec r \ \dd m
= \dfrac 1 {\cancel{\sigma} \ A} \int \vec r \ \cancel{\sigma} \dd A= \dfrac  1 A \int \vec r \ \dd A=
\dfrac 1 A \int (x,y) \ \dd A $

Luego, $\displaystyle \quad x_{CM}=\dfrac 1 M \int x \dd m;\qquad y_{CM}=\dfrac 1 M \int y \dd m$

Calculemos primero el área $A$ de la placa parabólica. 

\begin{multicols}{2}
Por matemáticas, el área bajo una curva es $\int f(x)\dd x$, pero nosotros estamos interesados en el área por encima de la curva. Este área será igual al área del rectángulo menos el área bajo la curva.

$y=ax^2=b \to x=\pm \sqrt{b/a}$
\begin{figure}[H]
	\centering
	\includegraphics[width=.4\textwidth]{imagenes/imagenes12/T12IM07.png}
\end{figure}
\end{multicols}

Área bajo la curva:
$\displaystyle A=\int_{-\sqrt{b/a}}^{+\sqrt{b/a}} ax^2 dx \eval{a \dfrac {x^3}{3}}_{-\sqrt{b/a}}^{+\sqrt{b/a}}=\dfrac {2b}{3}\sqrt{\dfrac{b}{a}}$

Área del rectángulo: $A_{rectan}=2b\sqrt{\dfrac b a}$

Área de la placa metálica: $\boldsymbol{A}=2b\sqrt{\dfrac b a}- \dfrac{2b}3 \sqrt{\dfrac b a}=\boldsymbol{\dfrac{4b}{3} \sqrt{\dfrac b a}}$

\begin{multicols}{2}
Para calcular la coordenada $x$ del $CM$ será conveniente dividir la placa en diferenciales de área cuyos puntos posean una coordenada $x$ la misma para todos ellos. Vamos por lo tanto a dividir la placa en bandas verticales de espesor $\dd x$. 
\begin{figure}[H]
	\centering
	\includegraphics[width=.4\textwidth]{imagenes/imagenes12/T12IM08.png}
\end{figure}
\end{multicols}

$\displaystyle x_{CM}=\dfrac 1 A \int_{-\sqrt{b/a}}^{+\sqrt{b/a}} x \dd A = \dfrac {3}{2b} \sqrt{\dfrac a b} \int_{-\sqrt{b/a}}^{+\sqrt{b/a}} x(b-ax^2) \dd x = $

$=\dfrac {3}{2b} \sqrt{\dfrac a b}\eval{\left( \dfrac {bx^2}{2} - \dfrac  {ax^4}{4} \right) }_{-\sqrt{b/a}}^{+\sqrt{b/a}} = 0$, como era de esperar por la simetría del sistema \textcolor{gris}{Nos hubiésemos podido evitar este cálculo.}

\begin{multicols}{2}
Para calcular la coordenada $y$ del $CM$ sería conveniente dividir la placa en diferenciales de área cuyos puntos poseyeran una coordenada $y$ la misma para todos ellos, es decir en bandas horizontales de espesor $\dd y$. 
\begin{figure}[H]
	\centering
	\includegraphics[width=.4\textwidth]{imagenes/imagenes12/T12IM09.png}
\end{figure}
\end{multicols}

$\displaystyle y_{CM}=\dfrac 1 A \int_0^b  y \dd A = \dfrac 1 A\int_0^b y\ 2 \sqrt{\dfrac y a} \dd y = \dfrac {3}{2b} \sqrt{\dfrac a b} \dfrac {2}{\sqrt{a}} \int_0^b y^{3/2} \dd y = $

$= \displaystyle \dfrac{3}{b\sqrt{b}} \eval{\dfrac { 2y^{5/2} } {5} }_0^b=\dfrac{3b}{5}$

Finalmente, el $CM$ de la placa metálica parabólica está en: $\boldsymbol{ \ \left( 0,\dfrac{3b}{5} \right) }$.




\newpage %*************************************
%\begin{comment}
\begin{myblock}{Centro de masas del sistema Tierra - Luna}
 \begin{small}
Vamos a calcular la posición del centro de masas del sistema Tierra-Luna, situando el origen del sistema de referencia en el centro de la Tierra.
 
\vspace{1mm} Los datos que se necesitan para calcularlo son la distancia entre los centros de la Tierra y la Luna, y las masas de ambos cuerpos: 

\vspace{1mm} $d_{TL}= 384100 \ \mathrm{km}:\  M_T= 5.973\times 10^{24} \  \mathrm{kg};\  M_L= 7.349\times 10^{22}\  \mathrm{kg}$

\vspace{1mm} El vector de posición del centro de masas es: 
$\vec r_{CM}=\dfrac{\sum m_i \vec r_i}{\sum m_i}$

\vspace{1mm}  Podemos particularizar para una dimensión, puesto que los centros de la Tierra y la Luna están sobre una línea y después sustituir los datos de la tabla anterior: $r_{CM}=4668\ \mathrm{Km}$

\begin{multicols}{2}
Como el radio de la Tierra es de $6378\ \mathrm{km}$, el centro de masas del sistema Tierra Luna se encuentra debajo de la superficie de la Tierra, como se muestra en la siguiente figura. 
\begin{figure}[H]
	\centering
	\includegraphics[width=.5\textwidth]{imagenes/imagenes12/T12IM01.png}
\end{figure}

\end{multicols}
Como estamos analizando únicamente el movimiento del sistema Tierra - Luna, podemos considerar que está aislado, por lo que la fuerza gravitatoria que la Tierra ejerce sobre la Luna y su reacción, la fuerza gravitatoria que la Luna ejerce sobre la Tierra, son fuerzas internas y, por tanto, desde el punto de vista de un observador en reposo, el centro de masas del sistema no tiene aceleración (está en reposo). La Luna está orbitando alrededor de la Tierra y, para que el centro de masas del sistema Tierra - Luna permanezca en reposo, el centro de la Tierra ha de estar también en movimiento con respecto a dicho centro de masas. 

\vspace{1mm} Este fenómeno (denominado \emph{wobbling} en inglés) se da entre pares de cuerpos celestes de distinta naturaleza: entre un planeta y su luna (o sus lunas), entre un sol y sus planetas... Dependiendo de las masas de ambos el centro de masas del sistema estará situado entre los dos cuerpos, o bien en el interior de alguno de los dos y, por tanto, el movimiento del sistema de dos cuerpos será diferente en cada caso. El \emph{wobbling} se emplea para detectar planetas que gravitan en torno a estrellas lejanas. 	
\end{small}
\end{myblock}

%\end{comment}











\chapter{El problema de los dos cuerpos}

\begin{miparrafo}
Problema de los dos cuerpos

En mecánica, el problema de los dos cuerpos consiste en determinar el movimiento de dos partículas puntuales que solo interactúan entre sí. Los ejemplos comunes incluyen la Luna orbitando la Tierra y en ausencia del Sol, es decir aislados, un planeta orbitando una estrella, dos estrellas que giran en torno al centro de masas (estrella binaria), y un electrón orbitando en torno a un núcleo atómico. 

Como veremos en el tema, las leyes de Newton nos permite reducir el problema de dos-cuerpos a un problema de un-cuerpo equivalente, es decir, a resolver el movimiento de una partícula sometida a un campo gravitatorio conservativo y que por tanto deriva de un potencial externo.  El problema puede resolverse exactamente, por el contrario, el problema de los tres cuerpos (y, más generalmente, el problema de n$n$ cuerpos con $n\geq 3$) no puede resolverse, excepto en casos especiales.	
\end{miparrafo}

\section[El problema de los dos cuerpos. Masa reducida]{El problema de los dos cuerpos. Masa reducida\sectionmark{Problema 2 cuerpos. Masa reducida}}
\sectionmark{Problema 2 cuerpos. Masa reducida}

\begin{multicols}{2}
Consideremos dos cuerpos que ejercen fuerzas entre sí y, además, están inmersos en un campo exterior a ellos.

Tercera de Newton: $\ \vec F_{12}=-\vec F_{21}$

Posiciones: $\vec r_{12}=\vec r_1- \vec r_2$

\begin{figure}[H]
	\centering
	\includegraphics[width=.3\textwidth]{imagenes/imagenes13/T13IM01.png}
\end{figure}
\end{multicols}
Nuestro objetivo va a ser encontar las ecuaciones de movimiento que rigen el sistema de los dos cuerpos como tal y que hacen que se muevan como un ente.

Cada uno de los dos cuerpos, por separado:
$ \quad \begin{cases}
 \displaystyle \ m_1 \ \dv{\vec v_1}{t} = \vec F_{12}+\vec F_1^{(e)} \\ \\	 
 \displaystyle \ m_2 \ \dv{\vec v_2}{t} = \vec F_{21}+\vec F_2^{(e)}
 \end{cases}$
 
 Vamos a intentar refundir estas dos ecuaciones de movimiento en una sola.

Para ello, consideremos:
$ \quad \begin{cases}
 \displaystyle \overrightarrow {R}_{CM}=\dfrac {m_1 \vec r_1+m_2 \vec r_2}{m_1+m_2} \\ 
 \displaystyle \vec r_{12}=\vec r_1-\vec r_2
 \end{cases}$

Despejando de aquí $\vec r_1$ y $\vec r_2$,
$\ \  \vec r_1=\dfrac {m_1+m_2}{m_1} \vec R_{CM}-\dfrac {m_2}{m_1}\vec r_2; \ \  \vec r_2=\vec r_1-\vec r_{12}$

\small{$\vec r_1=\dfrac {m_1+m_2}{m_1} \vec R_{CM}-\dfrac {m_2}{m_1}\vec r_1+\dfrac{m_2}{m_1}\vec r_{12} \to \vec r_1\left( 1+\dfrac {m_2}{m_1} \right) =\dfrac{m_1+m_2}{m_2}\vec R_{CM} +\dfrac{m_2}{m_1}\vec r_{12}$}

\normalsize{Finalmente,} obtenemos el siguiente \emph{cambio de coordenadas}:

$$ \vec r_1 = \overrightarrow {R}_{CM}\ + \ \dfrac{m_2}{m_1+m_2}\ \vec r_{12} \qquad \qquad \vec r_2 = \overrightarrow {R}_{CM}\ + \ \dfrac{m_1}{m_1+m_2}\ \vec r_{12}$$

Tenemos las ecuaciones de movimiento de cada partícula por separado, tenemos la tercera de Newton y este cambio de coordenadas. Con todo esto vamos a hacer un \emph{truco matemático} \textbf{sumando} las ecuaciones de movimiento de los cuerpos independientes.

$\displaystyle m_1 \dv{\vec v_1}{t}+m_2 \dv{\vec v_2}{t} = \vec F_1^{(e)}+\vec F_2^{(e)}+\cancel{\vec F_{12}}+\cancel{\vec F_{21}}$

Suponiendo que las masas son constantes, $\displaystyle \dv{m_i}{t}=0$ y derivando respecto del tiempo el vector de posición del $CM$,

$\displaystyle \dv{\overrightarrow R_{CM}}{t}=\dfrac 1{m_1+m_2} \left( m_1 \dv{\vec r_1}{t} + m_2 \dv{\vec r_2}{t} \right)=\dfrac 1{m_1+m_2}(m_1\vec v_1+m_2 \vec v_2)$

Volviendo a derivar respecto al tiempo,

\small{$\displaystyle \dv[2]{\overrightarrow {R}_{CM}}{t}=\dfrac 1 {m_1+m_2} \left( m_1 \dv{\vec v_1}{t} + m_2\dv{\vec v_2}{t} \right)=\dfrac 1 {m_1+m_2} \left(\vec F_1^{(e)}+\vec F_2^{(e)}\right)=\dfrac{\vec F^{(e)}}{m_1+m_2}$}

\normalsize{Finalmente,} obtenemos una ecuación de movimiento que unifica a los dos cuerpos:

\begin{equation}
\subrayado{\boldsymbol{
(m_1+m_2) \ \dv[2]{\overrightarrow {R}_{CM} }{t}	 \ = \ \overrightarrow F^{(e)}
}}
\end{equation}

\begin{miparrafodestacado}
Ecuación de movimiento del sistema donde la masa es la masa total del sistema, el vector de posición es el del centro de masas y la fuerza externa que actúa es la que actúa sobre el centro de masas.	
\end{miparrafodestacado}

\textbf{Restando}, ahora, las dos ecuaciones de movimiento de los dos cuerpos por separado:

$m_1 m_2 \ \displaystyle \left( \dv{\vec v_1}{t} - \dv{\vec v_2}{t} \right) \ = \ m_2 \vec F_{12} \ - \  m_1 \vec F_{21}\ + \ m_2 \vec F_1^{(e)} \ - \ m_1 \vec F_2^{(e)} \ = \quad (m_1+m_2) \vec F_{12} + m_2 \vec F_1^{(e)}-m_1 \vec F_2^{(e)}$

$\vec r_{12}=\vec r_1 - \vec r_2 \ \to \ \displaystyle \dv{\vec r_{12}}{t}=\dv{\vec r_1}{t}-\dv{\vec r_2}{t} \ \to \  \dv[2]{\vec r_{12}}{t}=\dv{\vec v_1}{t}-\dv{\vec v_2}{t}$

$m_1 m_2 \ \displaystyle \left( \dv{\vec v_1}{t} - \dv{\vec v_2}{t} \right) \ =m_1m_2 \dv[2]{\vec r_{12}}{t}= (m_1+m_2) \vec F_{12}+ m_2\vec F_1^{(e)}-m_1\vec F_a^{(e)}$

$\displaystyle m_1m_2 \dv[2]{\vec r_{12}}{t}= (m_1+m_2) \vec F_{12}+ m_1m_2 \left( \dfrac {\vec F_1^{(e)}}{m_1}-  \dfrac {\vec F_2^{(e)}}{m_2} \right)$

Como caso particular, exigimos que $\dfrac{\vec F_i^{(e)}}{m_i}=\overrightarrow{cte},\quad i=1,2$ con lo que el último paréntesis anterior será cero. Esto significa que la aceleración es un vector constante, relación común en muchos campos de la física.

$\displaystyle \dfrac{m_1m_2}{m_1+m_2}\  \dv[2]{\vec r_{12}}{t}= \vec F_{12} \to \ \mu \  \dv[2]{\vec r_{12}}{t}= \vec F_{12}, \ \ con \ \ \mu=\dfrac{m_1m_2}{m_1+m_2}$, 

donde $\mu$ es la llamada \emph{masa reducida} del sistema.

\begin{equation}
\subrayado{\boldsymbol{
 \mu \  \dv[2]{\vec r_{12}}{t}= \vec F_{12}
}}
\end{equation}

\begin{miparrafodestacado}
	El movimiento relativo de dos partículas sujetas a una interacción mutua y a un campo exterior	que satisface la relación $\vec F/m=\overrightarrow{cte}$ es equivalente al movimiento de una  partícula cuya masa sea la masa relativa del sistema y sometida a una fuerza igual a la fuerza medida por un observador inercial.
\end{miparrafodestacado}

El sistema Tierra-Luna, p.e., puede ser estudiado con las dos ecuaciones siguientes, donde hemos llamado $M=m_1+m_2$ a la masa total del sistema.

\begin{equation}
\subrayado{
\boldsymbol{ \boxed{\ 
\begin{cases}
\quad \displaystyle 	\mu \  \dv[2]{\vec r_{12}}{t} &= \vec F_{12} \\
\quad \displaystyle   M \ \dv[2]{\overrightarrow {R}_{CM} }{t}	 \ &= \ \overrightarrow F^{(e)}
\end{cases}\ } }
}
\end{equation}

Análisis de la masa reducida:
$\quad \mu=\dfrac{m_1m_2}{m_1+m_2}=\dfrac{m_1}{1+\dfrac{m_1}{m_2}}$

Usando el conocido desarrollo es serie de Maclaurin: $\ \dfrac 1{1+x} \ \underset {x\to 0}{=} \  1-x$, podemos escribir:

\begin{table}[H]
\begin{tabular}{lll}
$\quad \text{si }\ m_1<<m_2$  &$\to$&$\mu \ = \ m_1 \left( 1-\dfrac{m_1}{m_2} 	\right)$ \\
$\quad \text{si }\ m_1<<<m_2$ &$\to$&$\mu \ = \ m_1 	$
\end{tabular}
\end{table}

$\text{si }\ m_1=m_2 \to \mu \ = \ \dfrac{m^2}{2m} \ = \ \dfrac{m}{2},\   $ por ejemplo, para el caso de protón y neutrón dentro del $\ _1^2H\ $ (Deuterio).

\section[Energía, cantidad de movimiento y momento angular respecto al CM en el problema de los dos cuerpos]{Energía, cantidad de movimiento y momento angular respecto al CM en el problema de los dos cuerpos\sectionmark{Magnitudes en CM problema dos cuerpos}}
\sectionmark{Magnitudes en CM problema dos cuerpos}

Situamos el origen $\mathcal O$ del sitema de referencia en el $CM$, evidentemente ocurrirá que: $\quad \boldsymbol {\overrightarrow{R}_{CM}=\vec 0}$

\textbf{------ Energía}: $\ E=\mathcal E_c+\mathcal E_{p,int} + \mathcal E_p^{(e)}$

En el caso del problema de los dos cuerpos:

$\mathcal E_c=\sum_i \dfrac 1 2 m_i v_{i,CM}^2+\dfrac 1 2 M \cancelto{0}{v_{CM}^2}=\dfrac 1 2 m_1 v_{1,CM}^2 +\dfrac 1 2 m_2 v_{2,CM}^2$

Por otro lado:

$\vec r_1=\cancelto{0}{\vec R_{CM}}+\dfrac{m_2}{m_1+m_2}\vec r_{12}; \qquad \qquad \vec r_2=\cancelto{0}{\vec R_{CM}}+\dfrac{m_1}{m_1+m_2}\vec r_{12}$

Derivando:

$\vec v_{1,CM}=\dfrac {m_2}{m_1+m_2}\ vec v_{12}; \qquad \qquad \vec v_{2,CM}=\dfrac {m_1}{m_1+m_2}\ vec v_{12}$

Sustituyendo en la $\ \mathcal E_c$:

\small{$\mathcal E_c= \dfrac 1 2 \left( \dfrac {m_1m_2^2}{(m_1+m_2)^2}\ v_{12}^2+  \dfrac {m_1^2m_2}{(m_1+m_2)^2}\ v_{12}^2  \right) = \dfrac 1 2 \ \dfrac{m_1m_2}{(m_1+m^2)^{\cancel{}2}}\ \cancel{(m_1+m_2)} \ v_{12}^2$}
 
 \normalsize{Luego,}
 
 \begin{equation}
 \boldsymbol{ \mathcal E_c \ = \ \dfrac 1 2 \ \mu \ v_{12}^2	 }
 \end{equation}

Energía cinética de los dos cuerpos en el sistema $CM$, $\mu$ es la masa reducida y $v_{12}$ la velocidad relativa de una partícula respecto a la otra.

Si las fuerza son conservativas, la energía potencial no es función del origen de coordenadas.

\textbf{------ Cantidad de movimiento:}

$\overrightarrow R_{CM}=0=\dfrac{m_1 \vec r_1 + m_2 \vec r_2}{m_1+m_2} \ \to \ m_1 \vec r_1 + m_2 \vec r_2=0$. 

Derivando respecto al tiempo:

\begin{equation}
\boldsymbol{
m_1 \ \vec v_{1,CM }\ = - \ m_2 \ \vec v_{2,CM}
}	
\end{equation}

La cantidad de movimiento sigue conservándose en el problema de los dos cuerpos y en el sistema centro de masas.

\textbf{------ Momento angular:}

El teorema de König, $\overrightarrow L=\overrightarrow R_{CM}\times M \vec V_{CM}+\sum_i \vec r_{i,CM} \times m_i \vec V_{i,CM}$, en nuestro caso: \textcolor{gris}{($\vec R_{CM}=0$)}

$\overrightarrow L =  \vec r_{1,CM} \times m_1 \vec V_{1,CM} +  \vec r_{2,CM} \times m_2 \vec V_{2,CM}$. 

Sustituyendo los $\vec r_i$ en función de $\vec r_{12}$ y $\vec R_{CM}$

$\overrightarrow L = \left( \dfrac{m_2}{m_1+m_2}\vec r_{12}\right) \times m_1 \left( \dfrac {m_2}{m_1+m_2} \vec v_{12}\right) \ + \ 
\left( \dfrac{m_1}{m_1+m_2}\vec r_{12}\right) \times m_2 \left( \dfrac {m_1}{m_1+m_2} \vec v_{12}\right)$


\small{$\overrightarrow L =  \dfrac{m_1m_2^2}{(m_1+m_2)^2} \ ( \vec r_{12} \times \vec v_{12} )  +   \dfrac{m_1^2m_2}{(m_1+m_2)^2} \ ( \vec r_{12} \times \vec v_{12} ) $}

\normalsize{$\overrightarrow L = \dfrac{m_1m_2\cancel{(m_1+m_2)}}{(m_1+m_2)^{\cancel{2}}} \ ( \vec r_{12} \times \vec v_{12} ) $}

\begin{equation}
\boldsymbol{
\overrightarrow L \ = \ \vec r_{12} \ \times \  \mu \ \vec v_{12}
}	
\end{equation}

El momento angular del sistema de los dos cuerpos en el sistema de referencia $CM$ es igual al producto vectorial del vector de posición relativo por la masa reducida y la velocidad relativa.

\section{Ley de las áreas}


Observador situado en el $CM$

Th. momento angular: $\displaystyle \dv{\overrightarrow L}{t}=\overrightarrow M^{(e)}$.
En nuestro caso: $\overrightarrow M^{(e)}=\sum_i \vec r_i \times \vec F_i^{e}$

$\displaystyle \dv{\overrightarrow L}{t}=\vec r_1 \times \vec F_1^{(e)} + \vec r_2 \times \vec F_2^{(e)} $
\begin{multicols}{2}
$ \vec r_1 = \overrightarrow {R}_{CM}\ + \ \dfrac{m_2}{m_1+m_2}\ \vec r_{12}$

$\vec r_2 = \overrightarrow {R}_{CM}\ + \ \dfrac{m_1}{m_1+m_2}\ \vec r_{12}$

$\quad$

Observador $\mathcal O$ situado en $CM$:

$\overrightarrow R_{CM}=0$
\begin{figure}[H]
	\centering
	\includegraphics[width=.4\textwidth]{imagenes/imagenes13/T13IM02.png}
\end{figure}
\end{multicols}

$\displaystyle \dv{\vec L}{t}=\dfrac{m_2}{m_1+m_2}\vec r_{12}\times \vec F_1^{(e)}-\dfrac{m_1}{m_1+m_2}\vec r_{12}\times \vec F_2^{(e)}$

$\displaystyle \dv{\vec L}{t}=\mu \vec r_{12} \times \left( \dfrac {\vec F_1^{(e)}}{m_1} - \dfrac {\vec F_2^{(e)}}{m_2} \right)$

Exigiendo, de nuevo, que $\dfrac {\vec F^{(e)}}{m}=\overrightarrow{cte}$, el paréntesis anterior se anula, entonces:

\begin{equation}
\dv{\overrightarrow L}{t}=\vec 0 \quad \to \quad \subrayado{ \ \boldsymbol{ \overrightarrow   L=\overrightarrow {cte}} \ }  
\end{equation}

\begin{miparrafodestacado}
El momento angular es un vector constante, luego los vectores $\vec r_{12}$ y $\vec v_{12}$ están siempre en un plano, \emph{la órbita es única}.
\end{miparrafodestacado}

Sabemos que para movimientos en un plano, órbitas circulares o elípticas, son mejores las coordenadas polares: $\vec r_{12},\ \theta$.

Como vimos en cinemática, en la ecuación \ref{veloc-polares} (velocidad en polares),

$$\displaystyle \vec v_{12}\ =\ \dv{\vec r_{12}}{t} \ \vec u_{r_{12}} \ + \  \vec r_{12} \ \dv{\theta}{t} \ \vec u_{\theta}$$

Vamos a eliminar los subíndices pero hemos de recordar que estamos en movimiento relativo.


$\vec v=\displaystyle \dv{r}{t} \vec u_r + r \dv{\theta}{t} \vec u_{\theta}; \quad \vec L=\vec r \times \mu \vec v \quad \to$
$\quad \displaystyle \vec L = \vec r \times \mu \left(  \dv{r}{t} \vec u_r + r \dv{\theta}{t} \vec u_{\theta} \right) $ 

\textcolor{gris}{$(\vec r \times \vec u_r=0)$}; $\ \vec r=r\vec u_r \ \to \ \vec L=\displaystyle 
\mu r \dv{\theta}{t} \left( \vec r \times \vec u_\theta \right)=
\mu r^2 \dv{\theta}{t}  \left( \vec u_r \times \vec u_\theta  \right) $

$\vec u_r \times \vec u_\theta=\vec u_p$, perpendicular a ambos vectores.

$\displaystyle \vec L =\mu r^2 \dv{\theta}{t} \vec u_p \ \to \ L=\mu r^2 \dv{\theta}{t} \ \Rightarrow $

\begin{equation}
\label{ele-mu}
\subrayado{ \ \boldsymbol{r^2 \ \dv{\theta}{t}=\dfrac L \mu} \ }
\end{equation}

Como $\dfrac L \mu$ es una constante del movimiento, $\boldsymbol{r^2 \ \dv{\theta}{t}}$ ha de ser una \textbf{constante del movimiento.}

El área barrida es el área del sector de radio $r$ \textcolor{gris}{($r+\dd  r \approx \dd r$)} y ángulo $\theta$ \textcolor{gris}{en radianes} es: $\dd S = \frac 1 2 \theta r^2$.

Derivando respecto al tiempo, $\displaystyle \ \boldsymbol{\dv{S}{t}}=\dfrac 1 2 r^2 \dv{\theta}{t}= \dfrac{L}{2\mu}=\boldsymbol{cte}$, una constante del movimiento.

\begin{miparrafodestacado}
\emph{El radiovector barre áreas iguales en tiempos iguales} (segunda ley de Keppler).	
\end{miparrafodestacado}




\newpage %******************************************************
\begin{myblock}{El problema de los $n$ cuerpos}

\vspace{2mm} Se trata de un problema matemático originado en un modelo físico para explicar el movimiento de los astros en el sistema solar. Éste ha sido paradigmático para la ciencia desde la antigüedad hasta nuestros días. Las técnicas, que ingeniosamente desarrollaron célebres científicos a lo largo de la historia para abordar el problema, se revelaron extremadamente fructíferas para el estudio de la mayoría de los modelos matemáticos provenientes de otras disciplinas. Sin embargo, estos numerosos científicos, entre quienes podemos citar a Isaac Newton, Leonard Euler, Lagrange, Weierestrass, Hamilton, Jacobi, Poincaré, Kolmogorov, Lyaponov, no lograron dar una solución satisfactoria al problema original que consiste en predecir la evolución final de las trayectorias de los astros. 

\vspace{2mm} El problema físico puede formularse, informalmente, de la siguiente manera: dadas en un instante las posiciones y las velocidades de dos o más partículas que se mueven bajo la acción de sus atracciones gravitatorias mútuas, siendo conocidas las masas de las partículas, calcular sus posiciones y velocidades para otro instante. 

\vspace{2mm} En la actualidad, sólo se conocen algunos resultados particulares dependiendo del valor de $n$.	

\begin{figure}[H]
	\centering
	\includegraphics[width=1\textwidth]{imagenes/imagenes13/T13IM03.png}
\end{figure}
\end{myblock}

\chapter{Ley de Gravitación Universal}

\begin{miparrafo}
\begin{multicols}{2}
En la antigüedad, se observó que la órbita de Marte describía la trayectoria que aparece en la figura. Las órbitas no eran pues circulares tal como imaginaban. Fue Ptolomeo (s. II) quien describió este movimiento, suma de rotación y traslación, como \emph{epicicloide}.
\begin{figure}[H]
	\centering
	\includegraphics[width=.5\textwidth]{imagenes/imagenes14/T14IM01.png}
\end{figure}
\end{multicols}
\vspace{-10mm} %****************************************
A medida que avanzaron las técnicas de astronómicas observación (telescopio de Galileo) se vieron confirmadas las teorías de Ptolomeo.

Allá por el XV, Copérnico (final de la edad media, principio del renacimiento) niega que el centro del Universo sea la Tierra, por lo que le tacharían de hereje; sus teorías eran peligrosas para el orden establecido.

Cuando Newton enfocó su telescopio a Júpiter y vio cuatro satélites girando a su alrededor se afianzan las teorías de Copérnico y empiezan a perder importancia las creencias.

Copérnico planteó un sistema solar como hoy lo conocemos, con el Sol en su centro (visión heliocentrista, concebida en primera instancia por Aristarco de Samos --s. III a.C.-- frente a la geocentrista). Sus teorías fueron comprobadas por los estudios de momentos angulares de planetas que realizó Kepler (s. XVII) y matemáticamente por Tycho Brahe. Más tarde, Newton, introduciría los conceptos de aceleración, fuerza y campo central.

Las tres leyes de Kepler

Kepler accedió a los datos de las órbitas de los planetas que durante años se habían ido recolectando por Tycho Brahe que se centró en Marte, con una órbita elíptica muy acusada. De otra manera le hubiera sido imposible a Kepler darse cuenta de que las órbitas de los planetas eran elípticas. Inicialmente, Kepler intentó la circunferencia por ser la más perfecta de las trayectorias, pero los datos observados impedían un ajuste correcto, lo que entristeció a Kepler, ya que no podía saltarse un pertinaz error de ocho minutos de arco. Kepler comprendió que debía abandonar la circunferencia, lo que implicaba abandonar la idea de un "mundo perfecto". De profundas creencias religiosas, le costó llegar a la conclusión de que la tierra era un planeta imperfecto, asolado por las guerras. En esa misma misiva incluyó la cita clave: "Si los planetas son lugares imperfectos, ¿por qué no han de serlo las órbitas de los mismos?". Finalmente utilizó la fórmula de la elipse, una rara figura descrita por Apolonio de Pérgamo (s. III a.C.)  y descubrió que encajaba perfectamente en las mediciones de Tycho.

Había descubierto su primera ley, la primera ley de Kepler:

\emph{1.- Los cuerpos celestes tienen movimientos elípticos alrededor del Sol, estando éste situado en uno de los dos focos que tiene la elipse.}

Pasó a comprobar la velocidad del planeta a través de las órbitas llegando a la segunda ley:
\begin{multicols}{2}
\begin{figure}[H]
	\centering
	\includegraphics[width=.35\textwidth]{imagenes/imagenes14/T14IM02.png}
\end{figure}
\emph{2.-  El radio vector que une un planeta y el Sol recorre áreas iguales en tiempos iguales.}

Durante mucho tiempo, Kepler sólo pudo confirmar estas dos leyes en el resto de planetas. Aun así fue un logro espectacular, pero faltaba relacionar las trayectorias de los planetas entre sí. Tras varios años, descubrió la tercera ley e importantísima ley del movimiento planetario:
\end{multicols}
\emph{3.-  Para cualquier planeta, el cuadrado de su período orbital es directamente proporcional al cubo de la longitud del semieje mayor de su órbita elíptica. $T/a^3=cte$}

Esta ley, llamada también ley armónica, junto con las otras leyes, permitía ya unificar, predecir y comprender todos los movimientos de los astros.
\end{miparrafo}

\section{Ley de Gravitación Universal}

Disponemos de $N$ cuerpos que se mueven produciendo variaciones y fluctuaciones, pero nos limitaremos a estudiar el problema de dos cuerpos sometidos a la fuerza de un campo exterior.

Sabemos que, del tema anterior: $\ \displaystyle \mu \dv[2]{\vec r_{12}}{t} \ = \ \vec F_{12} \ + \ \mu \left[ \dfrac {\vec F_1^{(e)}}{m_1}-\dfrac {\vec F_2^{(e)}}{m_2} \right]$

Exigíamos que $\dfrac {\vec F_i^{(e)}}{m_i}=\overrightarrow{cte}$, con lo que el corchete anterior se anula y se obtiene, precindiendo de los subíndices:


$$  \displaystyle \mu \dv[2]{\vec r}{t} \ = \ \vec F\ ; \quad r^2 \dv{\theta}{t}=\dfrac L \mu=cte  \ \to \ \  \dv{S}{t}=\dfrac 1 2 \dfrac L \mu = cte$$


Vamos a pasar la ecuación $\mu \dv[2]{\vec r}{t}=\vec F$ a coordenadas polares:

$\displaystyle \left[ \dv[2]r{}{t}- r\left(\dv{\theta}{t} \right)^2\right]\vec u_r + \left[ 2 \dv{r}{t}\dv{\theta}{t}+r\dv[2]{\theta}{t} \right]\vec u_\theta = \dfrac F \mu \vec u_r$

\vspace{3mm} %*************************************
\begin{multicols}{2}
$\quad$

La fuerza es el la dirección $\vec u_r$, no hay fuerza en la dirección $\vec u_\theta$.
\begin{figure}[H]
	\centering
	\includegraphics[width=.3\textwidth]{imagenes/imagenes14/T14IM04.png}
\end{figure}
\end{multicols}

Podemos extraer dos ecuaciones escalares:

$$\displaystyle \dv[2]{r}{t}- r\left(\dv{\theta}{t} \right)^2 =\dfrac F \mu \ ; \qquad  \qquad  2 \dv{r}{t}\dv{\theta}{t}+r\dv[2]{\theta}{t}=0 $$

De la segunda ecuación, multiplicando ambos términos por $r$:

$\displaystyle 2r \dv{r}{t}\dv{\theta}{t}+r^2\dv[2]{\theta}{t}= \dv{t}\left[r^2 \dv{\theta}{t} \right] =0  \quad \to \quad \boldsymbol{r^2\  \dv{\theta}{t}\ =\ cte}$

Trabajemos ahora con el otro término: $\displaystyle \dv[2]{r}{t}- r\left(\dv{\theta}{t} \right)^2 =\dfrac F \mu $

\vspace{3mm} %**********************************************
\textcolor{gris}{\rule{60mm}{0.4pt}
\hspace{5mm} Inciso matemático: la elipse.}
\begin{multicols}{2}
$r+r'=2a \to r'=2a-r$

Th. coseno: $r'^2=r^2+4c^2-4rc\cos \theta$

Luego: $(2a-r)^2=r^2+4c^2-4rc\cos \theta$	

$4a^2 + \cancel{r^2} -4ar=\cancel{r^2}+4c^2-4rc\cos \theta$

$4a^2  -4ar=4c^2-4rc\cos \theta$

De donde: $\ a^2-c^2=r(a-\cos \theta)$
\begin{figure}[H]
	\centering
	\includegraphics[width=.45\textwidth]{imagenes/imagenes14/T14IM05.png}
\end{figure}
\end{multicols}

Como $a^2-c^2=b^2 \to b^2=ra\left(1-\dfrac c a \cos \theta \right)$

Se llama \emph{excentricidad} o achatamiento de la elipse a la cantidad $\boldsymbol{ \epsilon=\dfrac c a }$

$\boldsymbol{ \ l \ = \dfrac{b^2}{a}=r\left(1-\epsilon \cos \theta \right) }$

\textcolor{gris}{ Fin inciso matemático. \hspace{5mm} \rule{67mm}{0.4pt} }
\vspace{3mm} %**********************************************

Seguimos con nuestro problema.

$l =\dfrac{b^2}{a}=r\left(1-\epsilon\cos \theta \right) \to \epsilon \cos \theta=1-\dfrac l r $, 

derivando respecto de $t$, varían $r$ y $\theta$,

$\displaystyle -\epsilon \sin \theta \dv{\theta}{t}  = - \dfrac{l}{ r^2 } \dv{r}{t}\ \to \ \displaystyle -\epsilon \sin \theta = - \dfrac{l}{ r^2 \dv{\theta}{t} } \dv{r}{t}\ \ $, como $\displaystyle r^2 \dv{\theta}{t}=\dfrac {L_{CM}}{\mu}$

$\displaystyle \epsilon \sin \theta =\dfrac{l\ \mu}{L_{CM}}\ \dv{r}{t} \ \to \ \ \dv{r}{t}=\dfrac {\epsilon \ L_{CM} \sin \theta}{\mu \ l}$

Volviendo a derivar:
$\ \ \displaystyle \dv[2]{r}{t}=-\dfrac {\epsilon L_{CM}}{\mu \ l}\cos \theta \dv{\theta}{t} $

Sustituyendo $\displaystyle \ \epsilon \cos \theta = 1-\dfrac l r\ $ y $\ \displaystyle \dv{\theta}{t}=\dfrac {L_{CM}}{\mu\ r^2}$ en la exprersión anterior:

$\displaystyle \dv[2]{r}{t}=\dfrac { L_{CM}}{\mu \ l} \left( \dfrac l r - 1 \right) \dfrac {L_{CM}}{\mu\ r^2} \ = \ \dfrac{L^2_{CM}}{\mu^2\ r^3} - \dfrac{L^2_{CM}}{\mu^2\ r^2\ l}$

Sustituyendo lo encontrado en $\ \displaystyle \dv[2]{r}{t}-r \left( \dv{\theta}{t} \right)^2=\dfrac L \mu \ $ y teniendo en cuenta que $\displaystyle \dv{\theta}{t}=\dfrac {L_{CM}}{\mu\ r^2}$, obtenemos:

$\displaystyle \ \cancel{\dfrac{L^2_{CM}}{\mu^2\ r^3}} - \dfrac{L^2_{CM}}{\mu^2\ r^2\ l}-\cancel{r\dfrac {L^2_{CM}}{\mu^2\ r^4}}=\dfrac F \mu$


$$\displaystyle - \dfrac{L^2_{CM}}{\cancel{\mu} \ r^2 \ l}=\dfrac F {\cancel{\mu}} \quad \to \qquad \boldsymbol{F\ = \ - \dfrac {c}{r^2}} \qquad \text{con\ } c=\frac{L^2_{CM}}{\mu\ l}=cte$$
 
Vectorialmente: 

\begin{equation}
\boldsymbol{
\overrightarrow{F} \ = \ -\dfrac c{r^2} \ \vec u_r
}
\end{equation}
\begin{miparrafodestacado}
La fuerza que ejercen entre sí dos cuerpos en atractiva ($-$) e inversamente proporcional al cuadrado de la distancia.	
\end{miparrafodestacado}

Para determinar el modo en que $F$ depende de $r^2$, Newton estableció un 
\emph{postulado}: ``la fuerza también varia con la masa de los cuerpos y, para todas las sustancias, se cumple que:

\begin{equation}
\subrayado {\  \boxed { \boldsymbol { 
\ \overrightarrow  F \ = \ - G\ \dfrac {M\ m}{r^2} \ \vec u_r \ 
} } \ }	
\end{equation}
\centerline{\emph{\textbf{Ley de Gravitación Universal de Newton}}}


\section[Verificación de la tercera ley de Kepler]{Verificación de la tercera ley de Kepler\sectionmark{Tercera ley de Kepler}}
\sectionmark{Tercera ley de Kepler}

\emph{El cuadrado de los periodos de revolución es igual al cubo de los semiejes mayores.}


Tenemos que $\quad \dfrac {L^2_{CM}}{\mu\ l}\ = \ c\ = \ G\ M\ m$

$l=\dfrac b a\ , \quad \text{ con } b=\text{semieje mayor y }c=\text{semieje menor}$

$L^2_{CM}=c\mu l=GMm\mu l=G\mu M m \dfrac {b^2}{a}; \quad \mu=\dfrac{mM}{m+M} \quad \to $

$\to \quad  L^2_{CM}=G\mu^2 (M+m) \dfrac {b^2}{a}$, expresión que usaremos más adelante.

Ley áreas Kepler: 

$\displaystyle \quad \dv{S}{t}=\dfrac 1 2 \dfrac {L_{CM}}{\mu} \to \int_0^S \dd S = \int_0^T \dfrac 1 2 \dfrac {L_{CM}}{\mu} \ \dd t \ \to $
$\displaystyle \quad S= \dfrac 1 2 \dfrac{L_{CM}}{\mu}\cdot T$


\textcolor{gris}{\rule{50mm}{0.4pt} \hspace{5mm} Inciso matemático: área de la elipse.}
\begin{multicols}{2}
\begin{figure}[H]
	\centering
	\includegraphics[width=.5\textwidth]{imagenes/imagenes14/T14IM06.png}
\end{figure}
$\dfrac {x^2}{a^2}+\dfrac{y^2}{b^2}=1 \ \to \ y=b\ \sqrt{1-\dfrac{x^2}{a^2}};\quad S=2 \int y \dd x = \displaystyle 2\int_{-a}^{a} b\  \sqrt{1-\dfrac{x^2}{a^2}} \ \dd x$
\end{multicols}

Cambio de variable: $\dfrac x a =\sin t \to \begin{cases}
\ \dd x= a \cos t \dd t \\
x=a \ \ \to \sin t = 1\ \  \to t=\pi/2 \\
x=-a \to \sin t = -1 \to t=-\pi/2  	
 \end{cases}$

$\displaystyle S = 2 \int_{-\pi/2}^{\pi/2} b \sqrt{1-\sin^2 t}\ a \cos t\ \dd t=2ab\int_{-\pi/2}^{\pi/2} \cos^2 t\ \dd t$

Trigonometría, ángulo doble: $	\cos^2 t =\dfrac{1+\cos 2t}{2}$, luego

$\displaystyle S=\cancel{2} \int_{-\pi/2}^{\pi/2} \dfrac{1+\cos 2t}{\cancel{2}}\ \dd t = ab {\left[ t+\dfrac {\sin 2t}{2} \right]}_{-\pi/2}^{\pi/2} =\pi \ a \ b$

\textcolor{gris}{ Fin inciso matemático. \hspace{5mm} \rule{67mm}{0.4pt} }
\vspace{3mm} %**********************************************

Siguiendo con lo que teníamos, $\ \displaystyle  S= \dfrac 1 2 \dfrac{L_{CM}}{\mu}\cdot T \  \to \ \ \pi \ a \ b = \dfrac 1 2 \dfrac{L_{CM}}{\mu}\cdot T$

$T^2 = \dfrac{4 \pi^2 a^2 b^2 \mu^2}{L^2_{CM}} $ junto con la expresión anterior $\ L^2_{CM}=G\mu^2 (M+m) \dfrac {b^2}{a}$

\begin{equation}
\boldsymbol{ T^2 } = \dfrac{4 \pi^2 a^2 b^2 \mu^2 a}{G \mu^2 (M+m) b^2} =  \boldsymbol{ \dfrac {4\pi^2}{G(M+m)} \cdot a^3 }
\end{equation}

cqd: los cuadrados de los periodos de revolución son proporcionales a los cubos de los semiejes mayores (Newton realizó este razonamiento al revés que nosotros).

`!Un momento!: nosotros exigimos que la constante de proporcionalidad fuese la misma para todos los planetas y $M+m$ no lo es, depende de la masa del planeta $m$.

?`Cuál es la contradicción?: resulta que $M>>m$, por lo que $m$ es despreciable frente a $M$ y, entonces, sí que es \emph{aproximadamente} constante, se comete un error pequeño.

\section[Energía potencial gravitatoria. Campo gravitatorio. Potencial gravitatorio]{Energía potencial gravitatoria. Campo gravitatorio. Potencial gravitatorio\sectionmark{Potencial gravitatorio}}
\sectionmark{Potencial gravitatorio}

Como $\ \overrightarrow  F \ = \ - G\ \dfrac {M\ m}{r^2} \ \vec u_r \ $, la fuerza en \emph{central}, luego es conservativa y, por ello, deriva de un potencial escalar.

$\displaystyle \vec F=-\overrightarrow{\grad} \mathcal E_p;\qquad \overrightarrow{\grad}=\vec u_r \ \dv{r};\qquad \vec F=-\vec u_r \dv{\mathcal E_p}{r}=-\vec u_r \ G\ \dfrac {M\ m}{r^2} $

$\displaystyle \dv{\mathcal E_p}{r}= \ G\ \dfrac {M\ m}{r^2} \quad \to \qquad \int_{\mathcal E_p(r_1)}^{\mathcal E_p(r_2)} \dd \mathcal E_p=\int_{r_1}^{r_2} G \dfrac {Mm}r^2 \dd r  $ 

Luego $\ \ \mathcal E_p(r_2) - \mathcal E_p(r_1)=-GMm\left(\dfrac 1 {r_2^2}-\dfrac 1{r_1^2} \right)$

Establecemos la constante de la energía potencial exigiendo que $\mathcal E_p(\infty)=0$ \textcolor{gris}{$\displaystyle \ \lim_{r\to \infty}\mathcal E_p(r)=0$}. Con ello, tenemos que

\begin{equation}
	\subrayado{\ \boldsymbol{\mathcal E_p(r)=-\dfrac{G\ M\ m}{r}}\ }
\end{equation}

$r$ es la distancia al punto en que se calcula la energía potencial desde el centro de masas, $CM$

$\vec F$ lo podemos asociar a la acción de un \emph{campo} $\vec E$, de modo que $\vec F=m\vec E$. Podremos escribir que:

\begin{equation}
	\subrayado{\ \boldsymbol{\vec E=-\dfrac {G\ M}{r^2} \ \vec u_r}\ }
\end{equation}

El potencial gravitatorio $V$ que crea una masa puntual $M$ es $\ \vec E=-\overrightarrow{\grad} V \ \to \ V=-\dfrac {GM}{r}\ $, un escalar.

\begin{multicols}{2}
El concepto de potencial gravitatorio es muy útil cuando se considera más de una fuente puntual gravitatoria: 

$\displaystyle V=-G \sum_i \dfrac {M_i}{r_i}$
\begin{figure}[H]
	\centering
	\includegraphics[width=.4\textwidth]{imagenes/imagenes14/T14IM08.png}
\end{figure}
\end{multicols}

\begin{figure}[H]
	\centering
	\includegraphics[width=.4\textwidth]{imagenes/imagenes14/T14IM09.png}
	\caption*{Línea de fuerza y superficies equipotenciales del campo gravitatorio producido por una masa puntual.}
\end{figure}

\section{Campo gravitatorio debido a un cuerpo esférico}

Eliminamos la simplificación que hacíamos has ahora de considerar las masas puntuales. Supongamos que tenemos un cuerpo esférico hueco, con toda su masa distribuida uniformemente sobre su superficie y estudiaremos el campo gravitatorio que produce.

\begin{figure}[H]
	\centering
	\includegraphics[width=.6\textwidth]{imagenes/imagenes14/T14IM10.png}
\end{figure}

Elemento de área de la esfera.

$\dd S= (a\dd \theta)\cdot (2\pi a \sin \theta)=2\pi a^2 \sin \theta \ \dd \theta$

Masa del elemento de área: la masa está distribuida uniformemente en la superficie esférica:

$\dd m=\dfrac m{4\pi a^2}\cdot 2\pi a^2 \sin \theta \dd \theta = \dfrac{m \sin \theta \ \dd \theta}2$

Como $\displaystyle V=-G \sum_i \dfrac {m_i}{r_i}$, el potencial gravitatorio que esta masa elemental $\dd m$ crea en un punto $P$ será:

$\dd V =- G \dfrac{m \sin \theta \ \dd \theta}{2R} \quad \Rightarrow$

Geometría del dibujo, teorema del coseno: $R^2=r^2+a^2-2aR\cos \theta$

$a, R = ctes \ \to \ 2R \ \dd R=2ar \ \dd r \cos \theta \ \to \ \sin \theta \ \dd \theta =\dfrac {R \ \dd R}{ar}$, por lo que:

$\Rightarrow \quad \dd V=-G\dfrac{m R \dd R}{2aRr}=-\dfrac{Gm}{2ar} \ \dd R \ \to $

$\to \  V=-\dfrac{Gm}{2ar} \displaystyle \int_{r-a}^{r+a} \ \dd R 
=-\dfrac{Gm}{2ar} \ \eval{ R }_{r-a}^{r+a}= -\dfrac{Gm}{r} $

El potencial creado por una esfera hueca de masa $m$ a una distancia al centro $r$ de un punto situado es su exterior es:

$$\boldsymbol{ V \ = \  -\dfrac{Gm}{r}};  \qquad \boldsymbol{ \vec E=-}\overrightarrow{\grad} V =-\vec u_r \ \displaystyle \dv{V}{r} \boldsymbol{ = -\dfrac{Gm}{r^2} \ \vec u_r}; \qquad r\geq a$$

Si $P$ está dentro de la esfera, $r<a$, los límites de integración serán $a+r$ y $a-r$, por lo que:
$V=-\dfrac{GM}{2ar} \displaystyle \int_{a-r}^{a+r} \dd R
=-\dfrac{GM}{a}=cte;\quad r<a$

Tenemos un potencial gravitacional independiente de $P$. Al ser cte. $\ \vec E =\vec 0,\ r<a$

Resumiendo:

$$\boldsymbol{
V=\begin{cases}
-\dfrac{GM}{a}=cte & r<a \\ \\ 
-\dfrac{GM}{r} & r\geq a	
\end{cases} \ \rightarrow \ 
\ \vec E=
\begin{cases}
	\qquad \vec 0 & r<a \\ \\ 
	-\dfrac{GM}{r^2}\ \vec u_r & r\geq a	
\end{cases}
}$$


------ ?`A qué conclusión se puede llegar?

\emph{El campo gravitatorio y el potencial, en un punto exterior a una masa uniformemente distribuida en una capa esférica es \textbf{como si} se tratase del campo y potencial gravitatorios de una partícula de la misma masa situada en el centro de la esfera. En puntos del interior de la capa esférica, el campo es cero y el potencial constante.}

\begin{figure}[H]
	\centering
	\includegraphics[width=.75\textwidth]{imagenes/imagenes14/T14IM13.png}
\end{figure}

------ Pero, ?`qué ocurre con una esfera maciza (toda la masa distribuida homogéneamente en su volumen)?

En este caso, podemos considerar la esfera maciza como una serie de capas de esferas huecas de masa $m_i$ igual en todas las capas.

$\displaystyle V=-G \sum_i \dfrac {m_i}{r_i} =-\dfrac G r \sum_i m_i = -\dfrac G r M_T \ \to \vec E=-\dfrac {GM_T} r \ \vec u_r $

\begin{multicols}{2}
También \emph{el potencial y el campo gravitatorio creado por una esfera maciza en un punto exterior a ella es como el creado por una masa puntual}.
\begin{figure}[H]
	\centering
	\includegraphics[width=.25\textwidth]{imagenes/imagenes14/T14IM11.png}
\end{figure}	
\end{multicols}

Para el campo en el interior de una esfera homogénea, es decir, cuando $P$ está a distancia $r<a$, las capas con $r>a$ no contribuyen al campo. El campo resultante de todas las capas con $r\leq a$ produce un efecto \emph{como si} se tratase de una masa puntual $m'$ correspondiente a estas capas internas y situada en el centro de la circunferencia.
$E=-\dfrac{Gm'}{r};\ \ r\leq a$

Calculo de la masa $m'$ interna: $m'=\dfrac{m}{\dfrac 4 3 \pi a^3} \dfrac 4 3 \pi r^3=\dfrac{mr^3}{a^3}$

Luego el campo será $\ E=-\dfrac{Gmr}{a^3}\ \vec u_r$

\begin{multicols}{2}
El campo gravitacional de una esfera homogénea en un punto $P$ de su interior es proporcional a la distancia $r$ al centro de la esfera.

La disminución por la ley de la inversa del cuadrado se ve compensada por el aumento de masa, que es proporcional al cubo de la distancia al centro. (Ver figura adjunta.)
\begin{figure}[H]
	\centering
	\includegraphics[width=.4\textwidth]{imagenes/imagenes14/T14IM14.png}
\end{figure}
\end{multicols}

Se puede demostrar que el potencial gravitacional en un punto interior de una esfera homogénea de masa $m$ y radio $a$ viene dado por

$V=\dfrac{GM}{2a^3}\left( r^2-3a^2 \right);\quad r<a$ 

y el potencial en el exterior de la esfera, $V=\dfrac{-Gm}{r};\quad r>a$


Por todo lo visto, para el caso de estudio de los efectos de la gravedad entre la Tierra y un satélite ya sabemos la distancia a considerar, la distancia entre sus centros (podemos considerar ambos como puntuales).

\begin{figure}[H]
	\centering
	\includegraphics[width=.95\textwidth]{imagenes/imagenes15/T15IM05.png}
	\caption*{--- imagen de `fisicalab.com' ---}
\end{figure}

\section{Experimento de Cavendish. Determinación de $G$}

El experimento de Cavendish o de la \emph{balanza de torsión} permitió obtener implícitamente en 1798 la primera medida de la constante de gravitación universal  $G$  y, con este dato, a partir de la ley de gravitación universal de Isaac Newton y de las características orbitales de los cuerpos del sistema solar, la primera determinación de la masa de los planetas y del Sol. Debe señalarse que Henry Cavendish no calculó esta constante directamente (ya que no la necesitaba para sus mediciones; esto se hizo mucho después, aprovechando sus experiencias), pues su objetivo era determinar la densidad de la Tierra, o, más concretamente, ``pesar la Tierra'', lo que consiguió lograr con una precisión excepcional para su época. Sin embargo, dado que el producto de la constante universal por la masa de la Tierra era conocido desde tiempos de Newton, Henry Cavendish pudo dar la primera estimación del valor de $G$.

\begin{figure}[H]
	\centering
	\includegraphics[width=.6\textwidth]{imagenes/imagenes14/T14IM12.png}
\end{figure}

Cuando las masas $m'$ se colocan cerca de las masas $m$, su atracción gravitatoria produce un momento de la fuerza en la barra horizontal que da lugar a una torsión de la cuerda $OC$. El equilibrio se establece cuando los momentos gravitaorio y de torsión se igualan, el de torsión es proporcional a $\theta$ medido por la deflexión de un rayo reflejado en un espejo solidario con la fibra. Repitiendo el experimento a varias distancias $r$ y usando diferentes masas $m$ y $m'$ se puede determinar la constante $G=6.67 \times 10^{-1}\ \mathrm{Nw} \ \mathrm{m}^2 \ \mathrm{kg}^{-2}$.


\section{Masa de la Tierra y de los cuerpos celestes}

\textbf{Masa de la Tierra}: Un cuerpo cualquiera de masa $m$, en las cercanías de la superficie de la Tierra, es atraído por esta por la atracción gravitatoria, su peso:

$F=G\dfrac {Mm}{r^2}=mg \ \to M=\dfrac{g R^2}{G} = 5,97\times 10^{24}\ \mathrm{kg}$.

El radio de la tierra, $R=6371\ \mathrm{km}$, ya lo determinó Eratóstenes en el 276 a.C.


\textbf{Masa de los cuerpos celestes:} Se obtiene aplicando la tercera ley de Keppler, por ejemplo, para la Luna tenemos

$T_L=\dfrac {4 \pi a_L^2}{G(m_L+m_T)}\  \to \ m_L$

La masa de la Tierra se puede comparar con las de otros cuerpos celestes y equivale a $81.3$ veces la masa lunar ($M_L$), $0.00315$ veces la masa de Júpiter, ...
La masa de Júpiter es $M_{Jupiter}=317.83$ veces la masa de la Tierra, $M_\oplus$;  $M_{Saturno}=95.16 M_\oplus$; La masa solar es $M_\odot=332946\ M_\oplus$







\newpage %*****************************************************
\begin{myblock}{ley de gravitación universal}
Es una ley física clásica que describe la interacción gravitatoria entre distintos cuerpos con masa. Fue formulada por Isaac Newton en su libro Philosophiae Naturalis Principia Mathematica, publicado el 5 de julio de 1687, donde establece por primera vez una relación proporcional (deducida empíricamente de la observación) de la fuerza con que se atraen dos objetos con masa. Así, Newton dedujo que la fuerza con que se atraen dos cuerpos tenía que ser proporcional al producto de sus masas dividido por la distancia entre ellos al cuadrado. Para grandes distancias de separación entre cuerpos se observa que dicha fuerza actúa de manera muy aproximada como si toda la masa de cada uno de los cuerpos estuviese concentrada únicamente en su centro de gravedad, es decir, es como si dichos objetos fuesen únicamente un punto, lo cual permite reducir enormemente la complejidad de las interacciones entre cuerpos complejos.

\begin{figure}[H]
	\centering
	\includegraphics[width=.75\textwidth]{imagenes/imagenes14/T14IM03.png}
\end{figure}

Así, con todo esto resulta que la ley de la gravitación universal predice que la fuerza ejercida entre dos cuerpos de masas $m_1$  y  $m_2$  separados una distancia $r$  es igual al producto de sus masas e inversamente proporcional al cuadrado de la distancia, es decir:
$F=G\dfrac {m_1m_2}{r}$	
\end{myblock}

\include{TEMA15_chapter-A4}
\chapter{Dinámica del sólido rígido}

\begin{miparrafo}
	Un cuerpo rígido es un caso especial de sistemas de muchas partículas en el cual las distancias entre todas sus componentes permanecen constantes bajo la acción de una fuerza o momento.
	
	Un cuerpo rígido conservará, pues, su forma durante su movimiento. Podemos distinguir dos tipos de movimientos en un sólido rígido:
	
	\begin{itemize}
		\item El movimiento es de \emph{traslación} cuando todas las partículas describen lineas paralelas.
		\item El movimiento es de \emph{rotación} alrededor de un eje cuando todas las partículas describen trayectorias circulares alrededor de una línea llamada \emph{eje de roración}.	
	\end{itemize}
	
	El movimiento más general de un sólido rígido puede considerarse como una combinación de una traslación y una rotación.

\begin{multicols}{2}
$\quad$ 

El movimiento del $CM$ es \textbf{como si} se tratase de una partícula cuya masa sea igual la masa del cuerpo y sobre la que actúa una fuerza igual a la suma de todas las fuerzas exteriores aplicadas al cuerpo.
\begin{figure}[H]
	\centering
	\includegraphics[width=.3\textwidth]{imagenes/imagenes16/T16IM01.png}
\end{figure}
\end{multicols}

$\boldsymbol{ \displaystyle m\dv{\vec v_{CM}}{t}=\vec F^{(e)} }$

En este capítulo nos centraremos en el movimiento de rotación del un cuerpo rígido alrededor de un eje que pase por un punto fijo de un sistema inercia, 
$\displaystyle \boldsymbol{\dv{\vec L}{t}=\vec M^{(e)}}$, o a través de su $CM$, $\ \displaystyle \boldsymbol{\dv{\vec L_{CM}}{t}=\vec M_{CM}^{(e)}}$
\end{miparrafo}

\section{Momento angular de un cuerpo rígido}

\begin{multicols}{2}
Consideremos un cuerpo rígido que gira alrededor de un eje $Z$ con velocidad angular $\vec \omega$.

Cada una de sus partículas describirá una órbita circular centrada en el eje $Z$. Por ejemplo, la partícula $A_i$ describe un círculo de radio $\vec R_i=\overrightarrow{A_iB_1}$, con una velocidad $\vec v_i=\vec \omega \times \vec r_i$, siendo $\vec r_i$ el vector de posición de la partícula $i$ respecto al origen $\mathcal O$, que se escogerá como un punto fijo de un sistema inercial o el centro de masas del cuerpo.
\begin{figure}[H]
	\centering
	\includegraphics[width=.3\textwidth]{imagenes/imagenes16/T16IM02.png}
\end{figure}
\end{multicols}

En módulo, $\ v_i=\omega r_i \sin \theta_i=\omega R_i$ \textcolor{gris}{$\ \omega$ es la misma para todas las partículas del cuerpo rígido.}

El momento angular de la partícula sita en $A_i$ respecto al origen $\mathcal O$, es:
$\vec L_i=m\vec r_i \times \vec v_i$. Su dirección es $\bot$ al plano formafo por los vectores $\vec r_i \text{ y } \vec v_i$ y está contenido en el plano formado por $\vec r_i$ el eje $Z$, forma un ángulo de $90^o$ con $\vec r_i$ y de $90^o - \theta$ con el eje de rotación $Z$.

La componente de $\vec L_i$ en el eje $Z$ es: $\ L_{iz}=m_ir_iv_i \cos (90^o-\theta_i)$

$L_{iz}=m_ir_iv_i
\sin \theta_i=\textcolor{gris}{(\ v_i=\omega R_i \ )}=mr_i (\omega R_i) \sin \theta_i =m_i R_i^2 \omega$ 

Obsérvese que este resultado es equivalente al de una partícula que se desplaza en un círculo.

La componente en el eje $Z$ del momento angular de todo el cuerpo será:

$L_z=\displaystyle \sum_i L_{iz}=\sum_i m_i R_i^2\ \omega$

\vspace{7mm} %*********************************
Llamamos \emph{\textbf{Momento de Inercia, $I$}} a la expresión:

\vspace{-3mm} %*********************************
\begin{equation} 
\subrayado{\ I=\sum_i m_i R_i^2	\ } \quad \to \qquad  \subrayado{ \ L_z=I\ \omega \ }
\end{equation}


El momento angular de todo el cuerpo es $\ \vec L=\sum_i \vec L_i \ $ en general no es paralelo al eje de rotación.

\vspace{7mm} %*********************************
Para cada cuerpo hay algún eje de rotación para el cual el momento angular total es paralelo al eje. Para cada cuerpo, sin importar su forma, hay tres direcciones perpendiculares para las que el momento angular es paralelo al eje de rotación, estas tres direcciones son los llamados \emph{ejes principales de inercia}, $X_0, Y_0, Z_0$ y a sus momentos de inercia correspondientes les llamaremos \emph{momentos principales de inercia}, que denotaremos por $I_1, I_2, I_3$. Los ejes principales de inercia constituyen un sistema de referencia fijo en el cuerpo que rota respecto al observador.

\begin{figure}[H]
	\centering
	\includegraphics[width=.75\textwidth]{imagenes/imagenes16/T16IM03.png}
\end{figure}


Los ejes principales de inercia en cuerpos simétricos coinciden con ejes de simetría.

Cuando el cuerpo rota alrededor de un eje principal de inercia, $\vec L \ || \ \vec \omega\ $ \textcolor{gris}{($\vec \omega$ siempre en el eje de rotación)}. Podremos escribir, en este caso:

\begin{equation}
\subrayado{\vec L \ = \ I \ \vec \omega}	
\end{equation}

donde $I$ representa el momento principal de inercia correspondiente. 

\begin{miparrafodestacado}
Esta relación vectorial es únicamente válida para rotaciones alrededor de un eje principal de inercia.	
\end{miparrafodestacado}

En el \colorbox{LightYellow}{caso general} de rotación de un cuerpo rígido alrededor de un eje arbitrario, el momento angular $\vec L$ se puede expresar en función de los ejes principales de inercia $x_0, y_0, z_0$ como:

$$\subrayado{\vec L=\vec u_{x0}\ I_1 \omega_{x0}+\vec u_{y0}\ I_2 \omega_{y0}+\vec u_{z0}\ I_3 \omega_{z0}}$$

Con $\vec u_{x0}, \ \vec u_{y0}, \ \vec u_{z0}$ son los vectores unitarios en la dirección de los ejes principales de inercia $x_0,\ y_0,\ z_0\ $ y $\ \omega_{x0}, \ \omega_{y0}, \ \omega_{z0}$ las componentes de $\vec \omega$ respecto de estos ejes.

En este caso, $\vec L$ y $\vec \omega$ tienen distintas direcciones. La ventaja de utilizar esta expresión para $\vec L$ es que  $I_1,\ I_2;\ I_3$ son cantidades fijas que se pueden evaluar para un cuerpo determinado, sin embargo, $\vec u_{x0}, \ \vec u_{y0}, \ \vec u_{z0}$ rotan con el cuerpo y no tienen por qué ser constantes en cuanto a dirección.

\begin{figure}[H]
	\centering
	\includegraphics[width=.85\textwidth]{imagenes/imagenes16/T16IM04.png}
\end{figure}

\section{Cálculo del momento de inercia}

Momento de inercia para sistemas de partículas y para objetos extensos.

\begin{equation}
\subrayado{ \ 
I \ = \ \sum_i m_i R_i^2 \ = \ \int R^2 \dd m
\ }	
\end{equation}

Si $\rho$ es a densidad del cuerpo y tenemos una dirección de simetría, $\dd m=\rho \dd z$, tendremos que $I=\int \rho R^2 \dd z$. Si además el cuerpo es homogéneo, $\rho=cte$, podremos escribir $I=\rho \int R^2 \dd z$ y el momento de inercia se reduce a un factor geométrico igual para todos los cuerpos de la misma forma y del mismo tamaño.

Los momentos de inercia respecto a ejes paralelos están relacionados por una fórmula muy simple. Sea $z$ un eje arbitrario y $z_{CM}$ un eje paralelo que pasa por el $CM$ del cuerpo rígido y sea $a$ la separación entre ambos ejes, entonces:

\begin{equation}
\label{Th.Steiner}
\subrayado{ \ \boldsymbol{\ 
I\ =\ I_{CM} + M\ a^2 \ } \ }	 \qquad  \text{Teorema de Steiner}
\end{equation}

\begin{proof}.
\begin{figure}[H]
	\centering
	\includegraphics[width=.9\textwidth]{imagenes/imagenes16/T16IM05.png}
\end{figure}
Escogemos los ejes $X_{CM},\ Y_{CM},\ Z_{CM}$ de modo que pasen por el $CM$ del cuerpo y que el eje $Y_{CM}$ se encuentre en el plano $ZZ_{CM}$. Los ejes $X,\ Y,\ Z$ se escogen de modo que $Y$ coincida con $Y_{CM}$. 

Sea $P$ un punto arbitrario del cuerpo rígido de masa $M$. De la figura, obrsevamos que $\overrightarrow{P'A}=x\bot Y_{CM}$ y que $\overrightarrow{\mathcal OA}=y;\ \overrightarrow{\mathcal OCM}=a$

$R^2=x^2+y^2=x^2+(y+a)^2=x^2+y^2+2ay+a^2=R^2_{CM}+2ay+a^2$

ahora, el momento de inercia respecto al eje $Z$ es:

$I=\sum_i mR^2=\sum_i m(R^2_{CM} +2ay+a^2)$

$I=\sum_i mR^2_{CM} \ \textcolor{gris}{(I_{CM})}\ +2a\cancelto{0}{\sum_i my} + a^2\sum_i m \ \textcolor{gris}{ (\sum_i m=M) } \ $

pues $y_{CM}=0=\dfrac{\sum_i my}{\sum m} \to \sum_i my=0$

por lo que:

$$ \boldsymbol{ I \ = \ I_{CM} \ + \ M a^2 }\ ; \quad  \qquad [I]=[ML^2] \ \rightsquigarrow  \ \mathrm{Kg\ m}^2$$
\end{proof}



\begin{figure}[h]
	\centering
	\includegraphics[width=1\textwidth]{imagenes/imagenes16/T16IM06.png}
\end{figure}



\section[Ecuación de movimiento de rotación de un cuerpo rígido]{Ecuación de movimiento de rotación de un cuerpo rígido\sectionmark{Ecuación de movimiento de rotación}}
\sectionmark{Ecuación de movimiento de rotación}
La ecuación básica para discutir el movimiento de rotación del cuerpo rígido es:

$$\displaystyle \boxed{\ \boldsymbol{ \dv{\vec L}{t}=\vec M^{(e)} }\ } \ , \qquad \text{donde}\quad \vec L=\sum_i \vec L_i; \quad \vec M^{(e)}=\sum_i \vec M_i$$ 

------ En primer lugar vamos a estudiar el caso en que el cuerpo rígido rota alrededor de un eje principal que tiene un punto fijo en un sistema de referencia inercial.
$\to \ \displaystyle \vec L=I\vec \omega \to 	\dv{(I\vec \omega)}{t}=\vec M^{(e)}$, 
momento respecto del punto fijo del sistema inercial.

Si el eje permanece fijo respecto al cuerpo rígido, el momento de inercia permanece constante, entonces:

$\displaystyle I\dv{\omega}{t}=M^{(e)} \ \leftrightarrow \ \boldsymbol{ I\alpha=M^{(e)}}$,

donde $\alpha$ es la aceleración angular, $\alpha=\displaystyle \dv{\omega}{t}=\dot{\omega}$. 
Ecuación muy similar al movimiento de una partícula: $ \quad ma=F \quad \sim \quad  I\alpha=M^{(e)}$

Si, p.ej., $\ \vec M^{(e)}=\vec 0 \ \to \ I\vec \omega=\overrightarrow{cte}$ y, si $I=cte ,\ \to \ \vec \omega =\overrightarrow {cte}$.

Es decir, un cuerpo rígido que rota alrededor de un eje principal se mueve con $\vec \omega=\overrightarrow{cte}$
cuando no se aplica $\vec M^{(e)}$. Podemos considerar esta conclusión con \emph{ley de inercia para el movimiento de rotación.}

Si $I\neq cte$, lo es $I \omega=cte$, es decir, si $I$ aumenta (o disminuye), entonces $\omega$ disminuye (o aumenta). Este hecho tiene muchas aplicaciones.

------ Consideremos, ahora, un segundo caso. Veamos cuando el cuerpo rígido no rota respecto a un eje principal.

Sabemos que $\displaystyle \dv{L_z}{t}=M_z^{(e)}$.

Si la orientación del eje es fija con respecto al cuerpo de modo que $I=cte$, tendremos que $\displaystyle I \dv{\omega}{t}= M_z^{(e)} $

Si el eje de rotación no tiene un punto fijo en un sistea inercial, no podemos usar la ecuación $\displaystyle \dv{\vec L}{t}=\vec M^{(e)}$ y tendremos que calcular el momento exterior de fuerzas en el centro de masas del cuerpo, así:

$$ \displaystyle \dv{\vec L_{CM}}{t} \ = \ \vec M_{CM}^{(e)} $$

Si la rotación es alrededor de un eje principal, esta ecuación se convierte en:

$ I_{CM}\displaystyle \dv{\vec \omega}{t}=\vec M_{CM}^{(e)}\ $ y si $\ \vec M_{CM}^{(e)}=\vec 0 \ (*)\ \Rightarrow \ \vec \omega=\overrightarrow{cte} $

$(*)\ $ Cosa que ocurre en el caso de que la única fuerza externa aplicada al cuerpo sea el peso, entonces $\omega$ es constante.

\section{Energía cinética de rotación}

Energía cinética de un sistema de partículas $\ \displaystyle \mathcal E_c=\sum_i \dfrac 1 2 m_i v_i^2$

En el caso de un cuerpo rígido rotando con $\omega$ entorno a un eje, $v_i=\omega R_i$, con $R_i$ la distancia al eje de rotación, por lo que:

$\displaystyle \boldsymbol{ \mathcal E_c=} \ \sum_i \dfrac 1 2 m_i v_i^2=\sum_i \dfrac 1 2 m_i R_i^2 \omega^2 =\dfrac 1 2 \left( \sum_i m_i R_i^2 \right) \omega^2 \ \boldsymbol{ =\dfrac 1 2 I \omega^2 }$

Cuando la rotación se produce entrono a un eje principal de inercia, se tiene que $\ L=I\omega \to \omega=\dfrac L I \ $ y la energía cinética es

$\boldsymbol{ \mathcal E_c=}\  \dfrac 1 2 \  I \   \dfrac {L^2} {I^2} = \ \boldsymbol{ \dfrac{L^2}{2I}}$

Podemos obtener una expresión más general usando las componentes de $\vec \omega$ a lo largo de los ejes principales $x_0, y_0, z_0$. El resultado es:

$\mathcal E_c=\dfrac 1 2 \left( I_1 \omega^2_{x_0}+I_2 \omega^2_{y_0}+I_3 \omega^2_{x_0}\right)$.

Como $\vec L=\vec u_{x_0} \ I_1 \omega_{x_0}+\vec u_{y_0} \ I_2 \omega_{y_0} +\vec u_{z_0} \ I_3 \omega_{z_0}$,

$\mathcal E_c= \dfrac 1 2 \left(  \dfrac{L^2_{x_0}}{I_1}+  \dfrac{L^2_{y_0}}{I_2}+  \dfrac{L^2_{z_0}}{I_3} \right)$

Un caso interesante es cuando un cuerpo presenta simetría de revolución (como en las rotaciones moleculares), por ejemplo entorno a $x_0$, de modo que $I_1=I_2$

$\mathcal E_c=\dfrac 1 2 \left[ \dfrac 1 {I_1} ( L^2_{x_0}+L^2_{y_0} \ )+\dfrac 1 {I_3} L^2_{z_0}  \right]$

Consideremos ahora que el eje de rotación del cuerpo rígido pasa por el $CM$ y, además de rotar, tiene un movimiento relativo de traslación respecto al observador.

$\mathcal E_c=\dfrac 1 2 M V^2_{CM} \ + \ \mathcal E_{C,CM}$

En el cuerpo rígido, $\dfrac 1 2 M v_{CM}^2 = \mathcal E_{c,\text{traslación}} \to \mathcal E_{C,CM}$ es la energía cinética de rotación respecto al $CM$. Podemos escribir:

\vspace{-3mm} $$\mathcal E_c= \dfrac 1 2 M V^2_{CM}+ \dfrac 1 2 I_{CM} \omega^2$$

\vspace{-3mm} Como las distancias entre las partículas internas del cuerpo rígido no cambian, $\mathcal E_{P,int}=cte$ y nos las consideramos al explicar el intercambio energético del cuerpo rígido con sus alrededores:

$$\mathcal E_c \ - \mathcal E_{c,0} \ = \ W_{ext}$$

Para el caso de fuerzas conservativas, $\ W_{ext}=(\ \mathcal E_{p}-\mathcal E_{p,0}\ )_{ext}$

por lo que $\ \mathcal E_c+\mathcal E_p=(\mathcal E_c+\mathcal E_p)_0$

Con lo que $\boxed{\  \subrayado{ \  \boldsymbol{E=\dfrac 1 2 Mv^2_{CM} + \dfrac 1 2 I_{CM} \omega^2 + \mathcal E_p = cte} \ } \ }$

Si el cuerpo cae bajo la acción de la fuerza gravitatoria, $\mathcal E_p=Mh$, donde $h$ es la altura del $CM$ respecto a un plano horizontal de referencia.

Si hay fuerzas no conservativas (p.ej., fuerzas de fricción):  $\ W_{ext}=\mathcal E_{p,0}-\mathcal E_p+W'$, donde $W'$ es el trabajo efectuado por las fuerzas no conservativas. En este caso:

$$(\mathcal E_c+\mathcal E_p)\ - \ (\mathcal E_c+\mathcal E_p)_0\ =\ W'$$


\section{Problemas}
\begin{prob}

\begin{multicols}{2}.

Calcular el momento angular de la figura adjunta, que consta de 2 esferas de masa $m$ cada una, montadas sobre brazos de longitud $R$ conectados a un eje que les permie rotar a su alrededor. 	
\begin{figure}[H]
	\centering
	\includegraphics[width=.4\textwidth]{imagenes/imagenes16/T16IM08.png}
\end{figure}
\end{multicols}	
\end{prob}

Cada esfera describe un círculo de radio $R$, con velocidad $v=\omega R$ alrededor del eje $Z$ de giro. Respecto a $\mathcal O$, el momento angular de cada esfera es $\vec L=mR^2\omega \ \vec u_z$ y el momento total $\vec L=2mR^2\omega \ \vec u_z=2mR^2 \ \vec \omega$, el sistema rota entorno a su eje principal $Z_0=Z$.

En estas condiciones, $L=I\omega \ to \ I=2mR^2$ es el momento principal de inercia.


\begin{prob}
Calcular el momento de inercia de una varilla delgada homogénea respecto a un eje perpendicular a la varilla que pase por  a)  un extremo	, y b) el centro.
\end{prob}
\begin{figure}[H]
	\centering
	\includegraphics[width=.9\textwidth]{imagenes/imagenes16/T16IM09.png}
\end{figure}

--- a) Sea $L$ la longitud de la varilla y $S$ su superficie, que supondremos muy pequeña. Dividimos la varilla en pequeños segmentos, a distancia $x$ del eje de rotación $A$, de espesor $\dd x$ y volumen $\dd V=S\dd x$, con ello:

$\displaystyle I_A=\displaystyle \int_0^L \lambda x^2 (S\dd x)=\lambda S \int_O^L x^2 \dd x= \dfrac 1 3 \lambda S L^3$

Donde hemos llamado $\lambda$ a la densidad lineal de masa de la varilla, que por ser homogénea, $\lambda=\dfrac M{LS}$, con lo que
$\ I_A=\dfrac 1 3 \dfrac{M}{LS} SL^3=\dfrac 1 3 M L^2$

--- b) Para calcular el momento de inercia respecto a un eje paralelo que pasa por su centro, que es el $CM$ del cuerpo, como $Y\; || \; Y_C$, usaremos el Teorema de Steiner:
$I_A=I_{CM}+M\left( \dfrac 1 2 L \right)^2$, pues, en este caso, $a=\dfrac 1 2 L$ 

$\dfrac 1 3 M L^2=I_{CM}+\dfrac 1 4 M L^2 \ \to \ I_{CM}=\dfrac{1}{12}ML^2$

\textcolor{gris}{También podríamos haber llegado a este resultado integrando ahora desde $-L/2$ hasta $+L/2$: $\ \displaystyle I_{CM}=\lambda S \int_{-L/2}^{L/2}x^2 \dd x=\dfrac {M}{LS} S \left[ \dfrac {x^3}{3} \right]_{-L/2}^{L/2} = \dfrac {1}{12} ML^2$}

\textcolor{gris}{Incluso podríamos haber obtenido $I_{CM}$ suponiendo la varilla dividida en dos, con masas $M/2$ y longitud $L/2$ cada una de ellas girando sobre un extremo, $C$, así,
$\ I_{CM}=2\ \dfrac 1 3 \ (M/2)\ (L/2)^2 = \dfrac 1 {12} ML^2$}


\vspace{10mm} %*****************************************
\begin{prob}
Calcular el momento de inercia de una aro respecto de un eje que pasa por uno de sus diámetros.	
\end{prob}

\begin{multicols}{2}
Anillo homogéneo, densidad lineal de masa:

$\lambda=\dfrac {M}{2\pi R}$

$ \dd m=\lambda \dd l=\lambda R\dd l =\dfrac {M}{2\pi \cancel{R}}\cancel{R}\dd \theta$

$x=R\sin \theta$

$I=\displaystyle \int_M x^2 \dd m$
\begin{figure}[H]
	\centering
	\includegraphics[width=.25\textwidth]{imagenes/imagenes16/T16IM15.png}
\end{figure}	
\end{multicols}

$\displaystyle I=\int_0^{2\pi} (R\sin \theta)^2 \dfrac {M}{2\pi} \dd \theta
= \dfrac{MR^2}{2\pi} \int_0^{2\pi} \sin^2 \theta \dd \theta$

\small{$\begin{cases} \ \ \sin^2 \theta+\cos^2\theta &=1 \\ -\sin^2\theta+\cos^2\theta &=\cos 2\theta \end{cases} \to 2\sin^2\theta=1-\cos 2\theta;\quad \sin^2\theta=\dfrac 1 2 -\dfrac 1 2 \cos 2 \theta$}

\normalsize{$\displaystyle I=\dfrac{MR^2}{2\pi} \left[ \dfrac 1 2 \int_0^{2\pi} \dd \theta - \dfrac 1 {2\ \boldsymbol{2}}\int_0^{2\pi}\boldsymbol{2}\cos 2\theta \dd \theta \right] $}$=\displaystyle \dfrac{MR^2}{2\pi} \left[ \left(\theta \right)_0^{2\pi}-\dfrac 1 4 \left( \sin 2\theta \right)_0^{2\pi} \right] $

$\displaystyle I=\dfrac{MR^2}{2\pi} \left[ 2\pi-\dfrac 1 4 \cdot 0 \right]=
\dfrac{MR^2}{\cancel{2\pi}} \cancel{2\pi}=MR^2$

\vspace{10mm} %*****************************************
\begin{prob}
Calcular el momento de inercia de un disco homogéneo con respecto a un eje perpendicular que pasa por su centro.	
\end{prob}

\vspace{10mm} %*****************************************
\begin{multicols}{2}
Dada la simetría del problema, usaremos con elemento de volumen $\dd V$ un anillo de radio $r$ y espesor $\dd r$. Sea $h$ el espesor del disco, que suponemos muy pequeño. La densidad superficial de masa es $\sigma=\dfrac M{Sh}=\dfrac{M}{h\pi R^2}$ y la masa de anillo, situado a distancia $r$ del eje de giro, será:

$\dd m =\sigma \dd V = \sigma h \dd S= \sigma h 2\pi r \dd r$
\begin{figure}[H]
	\centering
	\includegraphics[width=.4\textwidth]{imagenes/imagenes16/T16IM10.png}
\end{figure}
\end{multicols}

$I=\displaystyle \int_0^R \sigma h \ r^2 \ 2\pi r \dd r=2\pi \sigma h \int_0^R r^3 \dd r=\dfrac 1 2 \pi \sigma h R^4 = \dfrac 1 2 \pi \dfrac {M}{\pi R^2 h} h R^4 $

Por lo que, finalmente, $\ I=\dfrac 1 2 M R^2$

\begin{prob}
Calcular el momento de inercia de una esfera maciza respecto de un eje que pasa por su centro.	
\end{prob}

Resolveremos el problema de dos formas distintas.

------ \underline{Método I}:  Primero calcularemos el momento de inercia de una esfera hueca para, después, considerar la esfera maciza como una serie de cáscaras de esferas huecas, como una cebolla.

\vspace{35mm} %****************************************************
\begin{multicols}{2}
La esfera hueca de masa $M$ y radio $R$ tiene una densidad superficial de masa: $\sigma=\dfrac{M}{4\pi R^2}$

El elemento de área mostrado tiene longitud $2\pi x$ y anchura $R\dd \theta$, por lo. que $\dd S= 2\pi x R \dd \theta$

$I=\displaystyle \int_M x^2 \dd m=\int_M x^2 \sigma \dd S=\int_M x^2 \dfrac {M}{4\pi R^2} 2 \pi x R \dd \theta =\dfrac{M}{2R}\int_M x^3 \dd \theta= \dfrac {M}{2R}\int_{-\pi/2}^{\pi/2}R^3 \cos^3 \theta \dd \theta$
\begin{figure}[H]
	\centering
	\includegraphics[width=.35\textwidth]{imagenes/imagenes16/T16IM16.png}
\end{figure}
\end{multicols}

\textcolor{gris}{\footnotesize{$\displaystyle \int \cos^3(\theta)\dd \theta = \int \cos \theta (1-\sin^2 \theta) \dd \theta = \int  \cos \theta \dd \theta - \int \sin^2 \theta \cos \theta \dd \theta = \sin \theta - \dfrac {\sin^3 \theta}{3} + \mathcal C$}}\normalsize{.}

$\displaystyle I=\dfrac{MR^2}{2} \left[\eval{\sin \theta - \dfrac {\sin^3 \theta}{3}}_{-\pi/2}^{\pi/2}\right.=\dfrac 2 3 MR^2$


\begin{multicols}{2}
Vamos ahora a por el momento de inercia de la esfera hueca que consideramos como formada por capas de esferas huecas a modo de capas de cebolla. Las capas estarán situadas a distancia $r$ del centro, de espesor $\dd r$ y con momento de inercia $\dd I=\dfrac 2 3 r^2 \dd m$. 

$r$ variará desde $0$ hasta $R$.

Para la esfera homogénea, 
$\rho=\dfrac {M}{\frac 4 3 \pi R^3}$
\begin{figure}[H]
	\centering
	\includegraphics[width=.3\textwidth]{imagenes/imagenes16/T16IM17.png}
\end{figure}
\end{multicols}

$V=\dfrac 4 3 \pi r^3 \to \dd V=4\pi r^2 \dd r;\qquad \dd m= \rho \dd V$

$\displaystyle I=\int_M \dd I =\int_M \dfrac 2 3 r^2 \dd m = \int_M \dfrac 2 3 r^2 \rho  4  \pi r^2 \dd r = \dfrac {8\pi}{3}\rho \int_O^R r^4 \dd r$

$\displaystyle I=
\dfrac {8\pi}{3} \dfrac {M}{\frac 4 3 \pi R^3} \left[ \eval{\dfrac {R^5}{5}}_0^R \right. = 
\dfrac {8\cancel{\pi}}{\cancel{3}} \dfrac{M}{\dfrac{4}{\cancel{3}}\cancel{\pi} R^3} \dfrac{R^5}{5} = \dfrac 2 5 M R^2$



------ \underline{Método II}: Consideraremos la esfera maciza formada por una serie de discos de espesor diferencial girando sobre un eje perpendicular a ellos que pasa por su centro ($I_{disco}=\frac 1 2 MR^2$).


Consideremos uno de estos discos, a altura $z$ desde el centro de la esfera, con radio r y espesor $\dd r$. Como muestra la figura.

Su momento de inercia será $\dd I=\dfrac 1 2 r^2 \dd m$

La masa de cada disco es:
$\ \dd m=\rho \dd V = \rho \pi r^2 \dd z;$
$\quad \rho=\dfrac{M}{\frac 4 3 \pi R^3}=\dfrac{3M}{4\pi R^3}$

\vspace{40mm} %***************************************************

\begin{multicols}{2}
Luego, $\dd I=\dfrac 1 2 r^2 \dfrac{3M}{4\pi R^3} \pi r^2 \dd z= \dfrac{3M}{8R^3}r^4 \dd z$

$\displaystyle I= \dfrac{3M}{8R^3} \int_{-R}^{R} r^4 \dd z;\qquad r^2+z^2=R^2 \to $

$\displaystyle I=\dfrac{3M}{8R^3} \left[ \eval{R^2x-\dfrac{x^3}{3}}_{-R}^{R}\right. = \dfrac 2 5 MR^2$
\begin{figure}[H]
	\centering
	\includegraphics[width=.3\textwidth]{imagenes/imagenes16/T16IM18.png}
\end{figure}
\end{multicols}

\begin{prob}
Un disco de $0.5\ \mathrm{m}$  de radio y $20\ \mathrm{kg}$ de masa	puede rotar libremente alrededor de un eje horizontal fijo que pasa por su centro. Se aplica una fuerza de $F=9.8\ \mathrm{N}$ tirando de una cerda atada al borde del disco. Encontrar la aceleración angular del disco y su velocidad angular al cabo de $2\ \mathrm{s}$
\end{prob}


\begin{multicols}{2}
$\quad$

Las únicas fuerzas externas sobre el disco son su peso $Mg$, la fuerza de la cuerda $F$, ambas hacia abajo y las reacciones $F'$ en los soportes del montaje. $ZZ'$ es un eje principal.
\begin{figure}[H]
	\centering
	\includegraphics[width=.5\textwidth]{imagenes/imagenes16/T16IM11.png}
\end{figure}
\end{multicols}



Calculamos el momento de las fuerzas respecto al $CM$, el correspondiente al peso es cero pues lo es su distancia a $C$. El de las fuerzas $F'$ combinadas también es cero (regla sacacorchos), con lo que $M^{(e)}=FR$.

Como hemos visto antes, el momento de inercia de un disco de masa $M$ y radio $R$ girando alrededor de un eje perpendicular que pasa por el entro es $I=\dfrac 1 2 MR^2$, tendremos que: 

$M=I\alpha \ \to \ FR=\dfrac 1 2 M R^2 \alpha \ \to \ \alpha=\dfrac{2F}{MR}=cte$

Tenemos un movimiento circular uniformemente acelerado: $\omega=	alpha t$. Sustituyendo los datos del problema:

$\alpha = 1.96 \ \mathrm{rad\ s}^{-2};\quad \omega (t=2)=3,92 \mathrm{rad\ s}^{-1}$

Como $C$ está fijo, la reacción en los soportes es: $2F'-Mg-F=0 \ \to \ F'=102,9\ \mathrm{N}$

\vspace{40mm} %*********************************************
\begin{prob}
\begin{multicols}{2}
$\quad$

Calcular la aceleración angular del sistema de la figura adjunta para un cuerpo de $1\ \mathrm{kg}$ de masa. Los datos del disco son los mismos que los del problema anterior, se trata de un disco de $0.5\ \mathrm{m}$  de radio y $20\ \mathrm{kg}$ de masa	que puede rotar libremente alrededor de un eje horizontal fijo que pasa por su centro.
\begin{figure}[H]
	\centering
	\includegraphics[width=.35\textwidth]{imagenes/imagenes16/T16IM12.png}
\end{figure}	
\end{multicols}
\end{prob}

$m=1\ \mathrm{kg} \ \to \ F=9.8\ \mathrm{N}$, como en el problema anterior. Parece que el resultado debe ser el mismo que en el problema anterior, pero no es cierto: la masa al caer ejerce una fuerza $F$ hacia abajo sobre el disco, pero el disco ejerce una fuerza $F$ hacia arriba por acción-reacción. Como $m$ cae con $MRUA$, la fuerza total sobre ella no puede ser cero, por lo que será $mg-F$, y el momento también será menor.

Ecuación de movimiento de la masa $m$: 

$\ mg-F=ma=MR\alpha$

Ecuación de movimiento del disco:

$(I=\frac 1 2 MR^2)$: $\ I\alpha=FR; \ \frac 1 2 M R^{\cancel{2}} \alpha=F\cancel{R}$

Eliminando $F$ de ambas ecuaciones, se encuentra 

$\alpha=\dfrac{mg}{\left( m+\dfrac 1 2 M \right)R}=1.8 \ \mathrm{rad\ s}^{-2}$

La aceleración hacia abajo de $m$ es 

$\ a=R\alpha=\dfrac{mg}{\left( m+\dfrac 1 2 M \right)}=0.90 \ \mathrm{m\ s}^{-2} < 9.8 \ \mathrm{m\ s}^{-2}$ que sería la que sentiría $m$ en caída libre.

La reacción $F'$ en los soportes se puede calcular como en el problema anterior. Al estar $C$ fijo:

$2F'+F-F-Mg-mg=0 \ \to \ F'=102,9\ \mathrm{N}$

\begin{prob}
\begin{multicols}{2}
Determinar la aceleración angular del disco y la aceleración hacia abajo del $CM$ de la figura adjunta.

Los datos del disco son los mismos que los del problema anterior, se trata de un disco de $0.5\ \mathrm{m}$  de radio y $20\ \mathrm{kg}$ de masa	que puede rotar libremente alrededor de un eje horizontal fijo que pasa por su centro.
\begin{figure}[H]
	\centering
	\includegraphics[width=.35\textwidth]{imagenes/imagenes16/T16IM13.png}
\end{figure}	
\end{multicols}
\end{prob}

$ZZ_0$ es el eje principal de rotación pero en este caso, al contrario que en los dos anteriores, el $CM$ no está fijo, por lo que la ecuación a usar ahora es:
$\displaystyle \dv{\vec L}{t}=\vec M^{(e)}_{CM}$. 
El movimiento es como el de un ``yo-yo''.

El momento de $Mg$ respecto a $CM$ es cero, por lo que la rotación del disco vienen dada por $\ I\alpha=FR$, con $I=\frac 1 2 MR^2$, entonces:

$\ I\alpha=FR; \ \frac 1 2 M R^{\cancel{2}} \alpha=F\cancel{R} \ \to \ F=\dfrac 1 2 M R \alpha$

El movimiento del $CM$ hacia abajo se realiza con una aceleración $a=R\alpha$

$F$ es la reacción de la cuerda que sujeta el disco deslizante y la fuerza resultante sobre él es: $mg-F$ y según la segunda de Newton $F_{res}=Ma$, tendremos que $\ Mg-F=Ma=MR\alpha$

Sustituyendo el valor de $F$ encontrado antes, $ \ \cancel{M}g-\dfrac 1 2 \cancel{M}R\alpha=\cancel{M}R\alpha \to g=\dfrac 3 2 R\alpha \to \alpha =\dfrac{2g}{3R}=13.16\ \mathrm{rad\ s}^{-1}$

La aceleración hacia abajo del $CM$ será $a=R\alpha=\dfrac 2 3 g= 6.53 \  \mathrm{m\ s}^{-1}$, menor que en caída libre e independiente del tamaño $R$ u de la masa $M$ del disco.

\begin{prob}
Una esfera, un cilindro y un aro, todos de la misma masa y el mismo radio, ruedan hacia abajo por un plano inclinado partiendo de una altura $y_0$. Encontrar la velocidad con que llegan a la base del plano.	
\end{prob}

\begin{multicols}{2}
Las fuerzas que actúan sobre el cuerpo rodante son, su peso $Mg$, la reacción normal del plano $N$ y la fuerza de rozamiento $F$. Podemos hacer un estudio dinámico o energético, que es el que haremos en este caso.
\begin{figure}[H]
	\centering
	\includegraphics[width=.35\textwidth]{imagenes/imagenes16/T16IM14.png}
\end{figure}	
\end{multicols}
Aplicando el principio de conservación de la energía:

$B=\dfrac 1 2 Mv^2_{CM} + \dfrac 1 2 I_{CM} \omega^2 + \mathcal E_p = cte.\quad$ En $B,\  E_B=Mgy$

En cualquier punto intermedio, el $CM$ se mueve con velocidad de traslación $v$ y el cuerpo rota respecto del $CM$ con velocidad angular $\omega$, ambas relacionadas por $v=\omega R$ y la energía será:

$E=\dfrac 1 2 Mv^2_{CM} + \dfrac 1 2 I_{CM} \omega^2 + Mgy$ 
Al final delmplano, en $A,\ y=0$ y $E_A=\dfrac 1 2 Mv^2_{CM} + \dfrac 1 2 I_{CM} \omega^2$ 

$E_A=E_B \ \to \ \dfrac 1 2 Mv^2_{CM} + \dfrac 1 2 I_{CM} \omega^2=Mgy_0$

Si en lugar de un cuerpo rodante tuviésemos un cuerpo que se desliza sin rodar ($\omega=0$), no tendríamos que incluir la energía cinética de rotación y el resultado sería $\dfrac 1 2 Mv^2_{CM} =Mgy_0$, como el de una partícula simple. El movimiento de rotación hace que el de traslación sea más lento. En un cuerpo rodante, la energía inicial ha de invertirse en energía de rotación y de traslación

Sustituyamos ahora $I_{CM}$ por los tres problemas propuestos, esfera, cilindro y aro ($E,\ C,\ A$). En cualquier tabla de. momento de inercia encontramos que

$I_E=\dfrac 2 5 MR^2;\quad I_C=\dfrac 1 2 MR^2;\quad I_A=MR^2$, sustituyendo y desejando,

$v_E^2=\dfrac {10}{7}gy_0;\qquad v^2_D=\dfrac 4 3 gy_0;\qquad V^2_A=gy_0$ 

En la partícula simple, $v^2=2gy_0$. Así, el cuerpo más rápido es la esfera, le sigue el cilindro y, por último, el aro.

Este resultado indica que la velocidad de un cuerpo que desciende sobre una pendiente no depende ni de su masa ni de sus dimensiones sino solamente de su forma.

\begin{prob}
Dos niños, cada uno de $25\ \mathrm{kg}$ de masa están sentados en extremos opuestos de una plancha horizontal de $2.6\ \mathrm{kg}$ de masa y $2.6\ \mathrm{m}$ de longitud. La plancha está girando a $5\ \mathrm{rpm}$ 	sobre un pivote situado en su centro. ?`Cuál es la velocidad angular con que gira el sistema cuando ambos niños se acercan $60\ \mathrm{cm}$ cada uno hacia el centro? ?`Cuál es el cambio en la energía cinética de rotación del sistema?
\end{prob}

$m=25\ \mathrm{Kg};\quad M=10\ \mathrm{kg};\quad L=2.6\ \mathrm{m};\quad d=1.3\ \mathrm{m}; \quad d'=0.9\ \mathrm{m}$

$\omega= 5 \ \mathrm{rpm}=5\dfrac {2\pi}{60} = \dfrac \pi 6 =0.52 \ \mathrm{rad\ s}^{-1}$

\begin{figure}[H]
	\centering
	\includegraphics[width=.8\textwidth]{imagenes/imagenes16/T16IM19.png}
\end{figure}	

Situación inicial, velocidad angular $\omega$

$I_{plancha}=I_{barra}=\dfrac 1{12} ML^2; \quad I_{nena}=md^2;\quad I_T=\dfrac 1{12} ML^2+2md^2$

Situación final, velocidad angular $\omega'$

$I'_{plancha}=\dfrac 1{12} ML^2; \quad I'_{nena}=md'^2;\quad I'_T=\dfrac 1{12} ML^2+2md'^2$

No se ejercen fuerzas externas, el momento angular se conserva: $L=I\omega=cte$:

$I\omega=I'\omega' \ \to \omega'=\dfrac I{I'}\omega$, sustituyendo valores, $\ \omega' =1.02  \ \mathrm{rad\ s}^{-1}$

$\mathcal{E}_{c,R}=\dfrac 1 2 I \omega^2; \quad \mathcal{E'}_{c,R}=\dfrac 1 2 I' \omega'^2 \to \Delta \mathcal{E}_{c,R}= \mathcal{E'}_{c,R}- \mathcal{E}_{c,R}$, solo queda sustituir valores.

\begin{prob}
	En el problema anterior y en la posición inicial, se aplica una fuerza horizontal de $10\ \mathrm{N}$ a $1\ \mathrm{m}$ del eje. Encontrar la aceleración del sistema.
\end{prob}

\begin{multicols}{2}
$\displaystyle \dv{\vec L}{t}=\vec M^{(e)}$

$\displaystyle \dv{I\omega}{t}=I\dv{\omega}{t}=I\alpha=$

$=M^{(e)}=Fr\sin 90^o=Fr$
\begin{figure}[H]
	\centering
	\includegraphics[width=.4\textwidth]{imagenes/imagenes16/T16IM20.png}
\end{figure}	
\end{multicols}

$Fr=I\alpha \to \alpha = \dfrac{Fr}{I}$, sustituyendo valores, $\ \alpha=1.33 \ \mathrm{rad\ s}^{-2}$

\begin{prob}
Un cilindro de $20 \ \mathrm{Kg}$ de masa y $0.25\ \mathrm{m}$ de radio está girando a $1200\ \mathrm{rpm}$ con respecto a un eje que pasa por su centro. ?`Cuál es la fuerza tangencial necesaria que habría que aplicar para que se detuviese en $1800$ revoluciones.
\end{prob}

$m=20 \ \mathrm{Kg}; \quad r=0.25\ \mathrm{m}; \quad \omega_0=1200\ \mathrm{rpm}=40\pi\  \mathrm{rad\ s}^{-1};$

$\theta=1800$ revoluciones = $3600\pi \ \mathrm{rad};\quad \omega=0 \  \mathrm{rad\ s}^{-1}$

$F=cte \to I\alpha=Fr,\quad \alpha=cte:\quad MCUA$, movimiento circular uniformemente acelerado, con aceleración angular negativa (de frenado, en realidad el movimineto es retardado)

Cuando se para: $\cancelto{0}{\omega}=\omega_0-\alpha t \to t=\dfrac{\omega_0} {\alpha}$

Ángulo barrido: $\theta=\cancelto{0}{\theta_0}+\omega_0 t - \dfrac 1 2 \alpha t^2$, sustituyendo ${t}$,

$\theta = \dfrac {\omega_0^2}{\alpha}-\dfrac 1 2 \cancel{\alpha}\dfrac{\omega_0^2}{\omega^{\cancel{2}}} \to \theta=\dfrac{\omega_0^2}{2\alpha} \ \Rightarrow \ \alpha=\dfrac{\omega_0^2}{2\theta}$

Por otro lado, $\displaystyle \dv {\vec L}{t}=\vec M^{(e)} \to I\alpha=Fr\sin 90^o=Fr \ \Rightarrow \ F=\dfrac{I\alpha}{r}$

$I=I{cilindro}=\dfrac 1 2 m r^2$, sustituyendo valores: $\ F=2.5 \times 10^4\ \mathrm{N}$

\rule{5cm}{.4pt}

\emph{Cálculo del momento de inercia de un cilindro homogéneo de masa $m$ y radio $r$ que gira entorno al eje longitudinal de simetría que pasa por el centro de masas}.

\begin{multicols}{2}
Cilindro homogéneo $M,\ H,\ R$

$\rho=\dfrac{M}{\pi R^2 H}$

$\dd m= \rho \dd V =\rho  \pi \left[ (r+\dd r)^2-r^2 \right] H=$

$=\rho \pi H \left[\cancel{r^2}+2r\dd r +\cancelto{0}{\dd^2 r}-\cancel{r^2} \right]$

$=2\pi \rho H r \dd r$

$\quad$

$I=\displaystyle \int_M r^2 \dd m$
\begin{figure}[H]
	\centering
	\includegraphics[width=.45\textwidth]{imagenes/imagenes16/T16IM22.png}
\end{figure}	
\end{multicols}

$\displaystyle I=2\pi \rho  H \int_0^R r^3 \dd r = 2 \pi \dfrac{M}{\pi R^2 H}H \dfrac{R^4}{4}=\dfrac 1 2 M R^2$

\begin{prob}
Demostrar que el momento de inercia	de un cuerpo rígido respecto a un eje que forma ángulos $\alpha,\ \beta,\ \gamma$ con los tres ejes principales es:
$\ \ I=I_1 \cos^2 \alpha + I_2 \cos^2 \beta + I_3 \cos^2 \gamma$
\end{prob}
\begin{multicols}{2}
$\quad$

$\mathcal E_c=\dfrac 1 2 I\omega = \dfrac 1 2 I_1 \omega_1^2+\dfrac 1 2 I_2 \omega_2^2+\dfrac 1 2 I_3 \omega_3^2$

$\quad$

$\begin{cases}
\quad \omega_1=\omega \cos \alpha \\
\quad\omega_2=\omega \cos \beta \\
\quad \omega_3=\omega \cos \gamma
\end{cases}$

\begin{figure}[H]
	\centering
	\includegraphics[width=.3\textwidth]{imagenes/imagenes16/T16IM23.png}
\end{figure}	
\end{multicols}
\vspace{-3mm} %***************************
$\mathcal E_c=\dfrac 1 2 \left( I_1 \cos^2 \alpha + I_2 \cos^2 \beta + I_3 \cos^2 \gamma \right)\omega^2$

$\text{Luego, } \quad   I=I_1 \cos^2 \alpha + I_2 \cos^2 \beta + I_3 \cos^2 \gamma \ ; \qquad \text{Teorema de Poinsot}$

\begin{prob}
Una esfera maciza rueda por dos planos inclinados distintos, di igual altura pero distinta inclinación.	?`Llegará la esfera al extremo inferior de los planos con la misma velocidad?, ?`empleará el mismo tiempo en llegar?
\end{prob}

\begin{multicols}{2}
$mgh=\frac 1 2 m V_{CM}^2+\frac 1 2 I \omega^2$

$v_{CM}=\omega R;\quad I=\frac 2 5 mR^2$

Luego $\ v=\sqrt{\frac {10}{7}gh}=cte$
\begin{figure}[H]
	\centering
	\includegraphics[width=.4\textwidth]{imagenes/imagenes16/T16IM24.png}
\end{figure}	
\end{multicols}
La esfera llega al final del plano con la misma velocidad, independientemente del ángulo $\theta$ de inclinación.

$v_{CM}=\displaystyle \dv{s}{t}; \quad h=s \sin \theta \to \dd x= \sqrt{\frac {10}{7}g \sin \theta \ s }\ \dd t \to \int_0^s \frac{\dd s}{\sqrt{s}}=\int_0^{t_\theta} \sqrt{\frac {10}{7}g\sin \theta}\dd t$

$\displaystyle 2\sqrt{x}= \sqrt{\frac {10}{7}g\sin \theta} \ t_\theta \to x=\frac {h}{\sin \theta}=\frac 5{14} g \sin \theta \ t^2_\theta \to h=\frac 5{14} g \sin^2 \theta \ t^2_\theta =cte$

Por lo que, $\sin^2 \theta \ t_\theta^2=cte \to \sin \theta \ t_{\theta}=cte$ y tendremos que

$t_\alpha \sin \alpha = t_\beta \sin \beta \ \Rightarrow \ \text{ si } \theta \uparrow \text{ entonces } t_\theta \downarrow\ $, la esfera cae antes (menor tiempo en caer) en el plano con mayor inclinación (ángulo mayor).


\rule{5cm}{0.3pt}
\vspace{1cm}
\begin{figure}[H]
	\centering
	\includegraphics[width=1\textwidth]{imagenes/imagenes16/T16IM25.png}
\end{figure}



\newpage %******************************************************

\begin{myblock}{Momento de inercia y conservación del momento angular}

\vspace{2mm} \small{Si ningún momento externo actúa sobre un sistema en rotación, la cantidad de movimiento angular de ese sistema permanecerá constante.} 

\vspace{2mm} \small{Esto significa que, si no hay un momento externo, el producto de la inercia por la velocidad angular en un momento será igual que en cualquier otro momento $L=I\omega=cte$.}


\begin{figure}[H]
	\centering
	\includegraphics[width=.9\textwidth]{imagenes/imagenes16/T16IM07.png}
\end{figure}


\vspace{2mm} \small{Un ejemplo interesante que ilustra la conservación del momento angular se ve en la figura adjunta. El hombre está de pie sobre una mesa giratoria sin fricción, con las pesas extendidas. Su inercia $I$, con ayuda de las pesas extendidas, es relativamente grande en esa posición. Cuando gira con lentitud, su momento angular es el producto de su inercia por la velocidad de rotación, $\omega$. Cuando junta las pesas con su cuerpo, la inercia de su cuerpo y de las pesas se reduce en forma considerable. ?`Cuál es el resultado? !`Aumenta su rapidez de rotación! Este ejemplo lo aprecia mejor la persona que gira, que siente cambios de rapidez de rotación que le parecen misteriosos. ¡Pero es física en acción! Este procedimiento lo usan los patinadores artísticos que comienzan a girar con los brazos, y quizá una pierna, extendidos, para después juntar los brazos y la pierna, y así obtener una mayor rapidez de rotación. Siempre que un cuerpo que gira se contrae, aumenta su rapidez de rotación}\normalsize{.} 
	
\end{myblock}



\include{TEMA17_chapter-A4}
\include{TEMA18_chapter-A4}
\include{TEMA19_chapter-A4}
\chapter{Otros sistemas oscilantes}

\vspace{-5mm}%***********************************************
\begin{miparrafo}
	\small{Al observar la Naturaleza nos damos cuenta de que muchos procesos físicos (por ejemplo la rotación de la tierra en torno al eje polar) son repetitivos, sucediéndose los hechos cíclicamente tras un intervalo de tiempo fijo. En estos casos hablamos de movimiento periódico y lo caracterizamos mediante su período, que es el tiempo necesario para un ciclo completo del movimiento, o su frecuencia, que representa el número de ciclos completos por unidad de tiempo.}

\small{Un caso interesante de movimiento periódico aparece cuando un sistema físico oscila alrededor de una posición de equilibrio estable. El sistema realiza la misma trayectoria, primero en un sentido y después en el sentido opuesto, invirtiendo el sentido de su movimiento en los dos extremos de la trayectoria. Un ciclo completo incluye atravesar dos veces la posición de equilibrio. La masa sujeta al extremo de un péndulo o de un resorte, la carga eléctrica almacenada en un condensador, las cuerdas de un instrumento musical, y las moléculas de una red cristalina son ejemplos de sistemas físicos que a menudo realizan movimiento oscilatorio.}

\small{El caso más sencillo de movimiento oscilatorio se denomina movimiento armónico simple y se produce cuando la fuerza resultante que actúa sobre el sistema es una fuerza restauradora lineal. El Teorema de Fourier nos da una razón de la importancia del movimiento armónico simple. Según este teorema, cualquier clase de movimiento periódico u oscilatorio puede considerarse como la suma de movimientos armónicos simples}\normalsize{.}
\end{miparrafo}

\vspace{15mm} %***************************************************
\section{Oscilaciones armónicas}

Características el MAS: $\qquad F=-kx;\qquad \mathcal E_p=\dfrac 1 2 k x^2$

Movimiento vibratorio: un cuerpo o una partícula se mueve sucesivamente de un lado a otro de una posición de equilibrio, repitiendo a intervalos de tiempo regulares sus variables cinemáticas (posición, velocidad y aceleración).Cuando las oscilaciones son muy rápidas se denominan vibraciones y el movimiento correspondiente es un movimiento vibratorio

En movimiento vibratorio, llamando $x_0$ al punto de equilibrio, se tiene:
$F=-k(x-x_0); \qquad \mathcal E_p=\dfrac 1 2 k (x-x_0)^2$

\begin{multicols}{2}
$\quad$ 

La representación gráfica $\mathcal E_p - x$ en forma de parábola es característica del movimiento vibratorio. La parábola es simétrica respecto de $x_0$ en que presenta un mínimos, por el teorema de Lejeune Dirichlet (ver seccción \ref{Lejeune-Dirichlet}) en $x_0$ hay un mínimo por lo que el equilibrio es estable.

\begin{figure}[H]
		\centering
		\includegraphics[width=.35\textwidth]{imagenes/imagenes20/T20IM01.png}
	\end{figure}
\end{multicols}

\begin{multicols}{2}
En física hay otro tipo de movimientos vibratorios en los que la representación de la energía potencial no es una parábola pero sí tiene bien definido un mínimo en el punto de equilibrio.

Por consideraciones energéticas, el sistema oscila entre $x_2$ y $x_2$.

Este tipo de oscilaciones se las conoce con el nombre de \emph{oscilaciones anarmónicas}.
\begin{figure}[H]
		\centering
		\includegraphics[width=.3\textwidth]{imagenes/imagenes20/T20IM02.png}
	\end{figure}
\end{multicols}

La energía potencial será función de la posición, pero no del tiempo (no estaríamos tratando con campos conservativos) $\ \mathcal E_p=\mathcal E_p(x)$.

Experimentalmente sabemos que $a)$ la partícula oscila y $b)$ lo hace alrededor de una posición de equilibrio.

Establcemos la hipótesis matemática de que la función de la energía potencial en función de la posición que ocupa la partícula es \emph{continua}, por Taylor\footnote{Ver apéndice \ref{McLaurin}}:

$\displaystyle \mathcal E_p(x)=\mathcal E_p(x_0)+\left(\dv{\mathcal E_p}{x}\right)_{x_0}(x-x_0)+ \dfrac 1{2!}\left(\dv[2]{\mathcal E_p}{x}\right)_{x_0}(x-x_0)^2$

$\displaystyle +\dfrac 1{3!}\left(\dv[3]{\mathcal E_p}{x}\right)_{x_0}(x-x_0)^3+\cdots$

$\mathcal E_p(x_0)$ es la constante arbitraria de la energía potencial, la neregía potencial en el punto de equilibrio.

Como $x_0$ es un mínimo, $\displaystyle \left(\dv{\mathcal E_p}{x}\right)_{x_0}=0$ y además $\displaystyle \left(\dv[2]{\mathcal E_p}{x}\right)_{x_0}=k>0$, por el mismo motivo.

Por todo ello: $\displaystyle \mathcal E_p(x)=\mathcal E_p(x_0)+ \dfrac 1{2!} k(x-x_0)^2 +\dfrac 1 {3!} k' (x-x_0)^3 + \cdots$

En el MAS: $\displaystyle \mathcal E_p=\dfrac 1 2 k (x-x_0)^2 \to \dv{\mathcal E_p}{x}=k(x-x_0) \to \dv[2]{\mathcal E_p}{x}=k=cte$ elástica o de recuperación de la fuerza. 

Es por analogía con el más que hemos llamado, más arriba, $k=\displaystyle \left(\dv[2]{\mathcal E_p}{x}\right)_{x_0}$, pero no por otra cosa.

Suponemos que estamos estudiando un caso particular para el cual la energía mecánica $E_T$ sea lo suficientemente pequeña para que $(x-x_0)^2<<(x-x_0)$; en estas condiciones:
$\ \displaystyle \mathcal E_p \approx \mathcal E_p(x_0)+\dfrac 1 2 k (x-x_0)^2$

La energía potencial estará constituida por dos partes, una constante y una función oscilador armónico simple.

Calculando ahora la fuerza:  $\ F=-\displaystyle \dv{\mathcal E_p}{x}=-K(x-x_0)$

Toda partícula que oscila sobre un valor de equilibrio presenta un mínimo en ese punto de equilibrio.

Si los resultados no coinciden con la experiencia es porque hay que considerar más términos del desarrollo de Taylor de la energía potencial.

\section{Oscilaciones amortiguadas}

En la naturaleza, los movimientos oscilatorios no existen. Van perdiendo amplitud paulatinamente hasta que se detienen.

Vamos a ver, para una dimensión, el caso de que las fuerzas de rozamiento hacen que el cuerpo oscilante se pare.

$F_T=F_e+F_m$; $\ F_e$ es la fuerza elástica ($-kx$) y $F_m$ la fuerza del medio (de rozamiento). Experimentalmente se comprueba que las fuerzas de medio son proporcionales a una determinada potencia de la velocidad, en esta caso supondremos que dependen de la primera potencia de la velocidad: $F_m=-bv$ (negativa, pues se opone al movimiento).

Segunda de Newton:  $\ F_T=ma=-Kx-bv$, luego

$$\subrayado{\ \displaystyle m\ \dv[2]{x}{t}\ +\ b\ \dv{x}{t}+kx\ = \ 0 \ }$$

ecuación diferencial lineal homogénea de segundo orden.\footnote{Para más información, ver apéndice \ref{EDO}.}

Ecuación característica: $\ mp^2+bp+k=0 \to p=-\dfrac {b}{2m} \pm \left[ \left( \dfrac {b}{2m} \right)^2 - \left( \dfrac k m \right) \right]^{1/2}$

Llamamos $\gamma=\dfrac {b}{2m}$, \emph{coeficiente de amortiguamiento} y $\omega^2=\dfrac{k}{m}$, que en el caso del oscilador armónico es la frecuencia angular.

Las soluciones de la ecuación característica son, ahora, $p=-\gamma \pm \sqrt{\gamma^2-\omega^2}$

Vamos a estudiar tres casos:

\begin{itemize}
\item $\gamma < \omega \qquad$ (raíces $p$ imaginarias - \emph{infraamortiguado})

$(\gamma^2-\omega^2)^{1/2}=i(\omega^2-\gamma^2)^{1/2}=\pm i \omega_1 \to p=-\gamma \pm i \omega_1$

\begin{multicols}{2}
La solución general es estos casos es: $\quad x=Ae^{-\gamma t}\cos(\omega_1t+\alpha)$

$A e^{-\gamma t}$ es la amplitud instantánea. Se trata del tipo de movimiento del péndulo simple.
\begin{figure}[H]
		\centering
		\includegraphics[width=.5\textwidth]{imagenes/imagenes20/T20IM03.png}
	\end{figure}
\end{multicols}

\item $\gamma > \omega \qquad$ (raíces $p$ reales y distintas - \emph{sobreamortiguado})

$p_1=-\gamma + (\gamma^2-\omega^2)^{1/2}=-\gamma_1; \quad
p_2=-\gamma - (\gamma^2-\omega^2)^{1/2}=-\gamma_2 $

La solución más general es: $x=Ae^{-\gamma_1 t}+Be^{-\gamma_2 t}$

\begin{multicols}{2}
Exponenciales decrecientes; en este caso, la amplitud decae más rápidamente que en el anterior.

Es el caso de un péndulo muy ligero oscilando sumergido en agua.
\begin{figure}[H]
		\centering
		\includegraphics[width=.5\textwidth]{imagenes/imagenes20/T20IM04.png}
	\end{figure}
\end{multicols}
\item $\gamma = \omega \qquad$	($p$ es una raíz real doble - \emph{amortiguamiento crítico})

$p=-\gamma = \to x=(A+Bt)e^{-\gamma t}$

El amortiguamiento crítico proporciona la forma más rápida de aproximar a cero la amplitud de un oscilador amortiguado. Con menor amortiguamiento (subamortiguación) alcanza el cero más rápidamente, pero oscila alrededor de él. Con mas amortiguamiento (sobreamortiguación), el acercamiento a cero es más lento. La amortiguación crítica, ocurre cuando el coeficiente de amortiguación es igual a la frecuencia de resonancia sin amortiguación del oscilador.
\end{itemize}

\begin{figure}[H]
		\centering
		\includegraphics[width=.95\textwidth]{imagenes/imagenes20/T20IM06.png}
	\end{figure}

\section[Oscilaciones forzadas. Resonancia. Transmisión de energía]{Oscilaciones forzadas. Resonancia. Transmisión de energía\sectionmark{Oscilaciones forzadas}}
\sectionmark{Oscilaciones forzadas}

Vamos a estudiar el caso en el que sobre nuestro oscilador actúa, además de la fuerza elástica ($F_e=-kx$) y la fuerza del medio ($f_m=-bv$) otra fuerza esta vez periódica $F_p=F\cos \omega t$).

Segunda de Newton: $F_t=\displaystyle ma=-kx-bv+F\cos \omega t$

$$\subrayado{\ \displaystyle m\ \dv[2]{x}{t}\ +\ b\ \dv{x}{t}+kx\ = \ F\cos \omega t \ }$$

ecuación diferencial lineal \emph{completa} de segundo orden.

La solución más general consta de la solución de la homogénea, $x_h$, más una solución particular, $x_i$: $\quad x=x_h+x_i$

$x_h=Ae^{-\gamma t}\cos (\omega_1t+\varphi)$, con $\omega_1=(\omega_0^2-\gamma^2)^{1/2} \text{ y } \gamma=\dfrac{b}{2m}$; $\ \omega_0=\sqrt{\dfrac k m}$, frecuencia propia del oscilador.

$x_i=p\cos \omega t + q \sin \omega t$. En este caso, $\omega$ es la frecuencia externa de la fuerza periódica. $p \text{ y } q$ son cosntantes a determinar por las condiciones físicas del problema ($m, b, x, F, \cdots)$).

Sustituiremos la  $x_i$ por la $x$ de la ecuación diferencial de segundo orden e identificaremos los coeficientes en senos y cosenos de ambos miembros.

$\displaystyle \dv{x_i}{t}=-p\omega \sin \omega t +q\omega
\cos \omega t; \qquad \displaystyle \dv[2]{x_i}{t}=-p\omega^2 \cos \omega t - q \omega^2 \sin \omega t$

$-mp\omega^2 \cos \omega t-mq\omega^2 \sin \omega t -bp\omega \sin \omega t +bq\omega \cos \omega t +kp\cos \omega t +kq\sin \omega t=F\cos \omega t$

Identificando: $\quad \begin{cases}
 \ \ (k-m\omega^2)p+b\omega q=F \\ -b\omega p+(k-m\omega^2)q=0	
 \end{cases}$
 
 Dos ecuaciones con dos incógnitas: $\quad \begin{cases}
\ \ p=\dfrac{F(k-m\omega^2)}{(k-m\omega^2)^2+b^2\omega^2} \\ \ \ q=\dfrac{Fb\omega)}{\sqrt{(k-m\omega^2)^2+b^2\omega^2}}	
\end{cases}$

Llevando estos resultados a $x_i=p\cos \omega t + q \sin \omega t \ \to$

\small{$x_i=\dfrac{F}{\sqrt{(k-m\omega^2)^2+b^2\omega^2}} \left[
\dfrac{(k-m\omega^2)}{\sqrt{(k-m\omega^2)^2+b^2\omega^2}}+
\dfrac{b\omega}{\sqrt{(k-m\omega^2)^2+b^2\omega^2}}\sin \omega t
\right]=$}

\normalsize{$\dfrac{F}{\sqrt{(k-m\omega^2)^2+b^2\omega^2}} [A+B\sin \omega t]$}

Llamando $\cos \alpha=A;\ \ \sin \alpha =B \ \to \ \text{Pitágoras} \ \ A^2+B^2=1$ y teniendo en cuenta el desarrollo en serie de potencias de McLaurin del $\sin (\omega t + \alpha)$, \textcolor{gris}{$\cos \alpha=\dfrac{(k-m\omega^2)}{\sqrt{(k-m\omega^2)^2+b^2\omega^2}}$} y \textcolor{gris}{$\sin \alpha=\dfrac{b\omega}{\sqrt{(k-m\omega^2)^2+b^2\omega^2}}\sin \omega t\ $} podremos escribir.

$x_i=\dfrac{F}{\sqrt{(k-m\omega^2)^2+b^2\omega^2}} \sin (\omega t + \alpha)$

Como $\sqrt{(k-m\omega^2)^2+b^2\omega^2}=cte=G$ y como $\tan \alpha= \dfrac{k-m\omega^2}{b\omega} \ \to \  x_i=\dfrac F G \sin (\omega t +\alpha ) = C  \sin (\omega t +\alpha ) \qquad C=\dfrac F G$

La solución general es:

$$ \subrayado{ \ x= A\ e^{-\gamma t} \ \cos (\omega_1t+ \varphi) \ + \ C \ \sin (\omega t + \alpha) \ } $$

Analizando la solución general, aparecen dos términos: el primero corresponde a ondas que se amortiguan (caen muy deprisa) y el segundo corresponde al MAS.

Para un determinado tiempo en que el primer término ya haya caído, $\tau$, solo queda el segundo término, $x=C\sin(\omega t + \alpha),\quad t\leq \tau$

$A\ e^{-\gamma t} \ \cos (\omega_1t+ \varphi) \ $ es la \emph{solución transitoria}; $C \ \sin (\omega t + \alpha) \ $ es la \emph{solución estacionaria}.

Trabajemos con la solución estacionaria y vemos que el sistema vibra con la misma frecuencia $\omega$ de la fuerza periódica: $x=A \sin (\omega t + \alpha) $
con $A=\dfrac F G= \dfrac A {\sqrt{m^2(\omega_0^2-\omega^2)^2+b^2\omega^2}}$ y $\tan \alpha=\dfrac{\omega_0^2- \omega^2}{\frac b m \omega^2}$

\begin{multicols}{2}
Representación de la amplitud $A$ del movimiento resultante en función de la frecuencia propia $\omega$ 
$\omega_A$ se obtiene derivando, $\displaystyle \dv{A}{\omega}=0$, se obtiene: $\ \omega_A=\sqrt{\omega_0^2-\left( \dfrac{b}{\sqrt{2m}} \right)^2}$
\begin{figure}[H]
		\centering
		\includegraphics[width=.4\textwidth]{imagenes/imagenes20/T20IM07.png}
	\end{figure}	
\end{multicols}

A esta frecuencia para la cual la amplitud del sistema es máxima dados un medio $b$ y una fuerza periódica $F$ se le conoce con el nombre de \emph{frecuencia de resonancia en amplitud}.

Representemos $A$ frente a $\omega$ para distintos valores de $b$

Para $b=0$ y $\omega=\omega_0$, en la ecuación de $A=\frac F G$, el denominador tiende a cero y el valor de la amplitud se dispara a infinito.

\vspace{10mm} %****************************************
\begin{figure}[H]
		\centering
		\includegraphics[width=.8\textwidth]{imagenes/imagenes20/T20IM08.png}
\end{figure}	
\vspace{10mm} %****************************************
Para un oscilador dado, a medida que $b$ crece, $\omega_A$ se va haciendo más pequeño lo que significa que el máximo se desplaza hacia la izquierda.

$\omega_1=\sqrt{\omega_0^2-\left( \dfrac b{2m} \right)^2}$. \hspace{5mm} Se verifica que: $\omega_0>\omega_1>\omega_A$

Vamos ahora a calcular la velocidad con que se mueve el sistema que tiene asociada una fuerza periódica.

$\displaystyle v=\dv{x}{t}=\omega t \cos(\omega t + \alpha); \qquad F(t)=F\cos \omega t$

Podemos interpretar $\alpha$ como el desfase entre la velocidad y la fuerza aplicada.

\vspace{30mm} %****************************************
\begin{multicols}{2}
En módulo, 

$\ v_0=\omega A=\dfrac{F}{\sqrt{m^2 \left( \dfrac{\omega_0^2}{\omega}-\omega \right)^2+b^2}} \ \to \ v_0=v_0(\omega)$

El módulo de la velocidad es función de la frecuencia que acompaña a la fuerza aplicada. Representando $v_0$ frente a $\omega$.

\begin{figure}[H]
		\centering
		\includegraphics[width=.25\textwidth]{imagenes/imagenes20/T20IM09.png}
	\end{figure}	
\end{multicols}
Para que $v_0$ sea máximo, $\boldsymbol{ \omega_{v_0}=\omega_0 }$. Para este valor de $\omega_0$, el oscilador gira con $A$ máxima $\to \alpha=0$, la velocidad del movimiento y la fuerza aplicada están en fase: \emph{condición de resonancia en la velocidad.}

Veamos cuales son las condiciones de resonancia en la energía:

Potencia, $\ \mathcal P=\displaystyle \dv{E}{t}=F(t)v(t)$. La condición experimental es $F(t)$ y $v(t)$ es la respuesta del sistema.

El valor máximo que tomará la potencia será aquel para el cual en unas condiciones determinadas el producto $F$ por $v$ sea máximo. La respuesta del sistema será máxima cuando se cumplan las condiciones de resonancia de la velocidad, la potencia será máxima.

Cuando hay resonancia en la energía, la transferencia energética de la fuerza aplicada al oscilador forzado es máxima.

Si el oscilador está en un medio en que $b$ es muy pequeño, $\omega_0, \ \omega_A, \ \omega_{v_0}$ son prácticamente iguales.

En estas condiciones, el agente impulsor del oscilador transmite máxima amplitud y máxima energía al sistema.


\section[Análisis de Fourrier del movimiento periódico. Teorema de Fourrier]{Análisis de Fourrier del movimiento periódico. Teorema de Fourrier\sectionmark{Análisis de Fourrier}}
\sectionmark{Análisis de Fourrier}

Sea $f(t)=f(t+T)$ una función periódica.
\begin{figure}[H]
		\centering
		\includegraphics[width=.75\textwidth]{imagenes/imagenes20/T20IM10.png}
	\end{figure}	

\begin{teor}{Teorema de Fourrier}
Todo movimiento periódico se puede descomponer como suma de movimientos vibratorios armónicos de frecuencias múltiplos de la del movimiento periódico considerado, que se llama frecuencia fundamental.	
\end{teor}

$f(t)=a_0+a_1\cos \omega t +a_2\cos 2\omega t +a_3\cos 3\omega t + \cdots + b_1\sin \omega t +b_2\sin 2\omega t  + \cdots$

donde: 
\begin{table}[H]
%\centering
\begin{tabular}{l}
$\qquad a_0=\displaystyle \dfrac 1 T \int_0^T f(t) \ \dd t$ \\
$\qquad a_n=\displaystyle \dfrac 1 T \int_0^T f(t) \cos n\omega t \  \dd t$ \\
$\qquad b_n=\displaystyle \dfrac 1 T \int_0^T f(t) \sin n\omega t \ \dd t$
\end{tabular}
\end{table}

Admitido el teorema de Fourrier, procedemos a integrar $f(t)$ desde $0$ hasta $T$:

\hspace{-5mm}\small{$\displaystyle \int_0^T f(t)\ \dd t= a_0 \cancelto{T}{\int_0^T \dd T} + a_1 \cancelto{0}{\int_0^T \cos \omega t \ \dd t}+ \cdots + b_1\cancelto{0}{\int_0^T \sin \omega t \ \dd t}+ \cdots =a_0T$}

\normalsize{Despejando, obtenemos:} $\ a_0=\displaystyle \dfrac 1 T \int_0^T f(t)\ \dd t$

\begin{figure}[H]
		\centering
		\includegraphics[width=1\textwidth]{imagenes/imagenes20/T20IM11.png}
	\end{figure}

\newpage
\begin{myblock}{Análisis de Fourrier}
Una serie de Fourier es una serie infinita que converge puntualmente a una función periódica y continua a trozos (o por partes). Las series de Fourier constituyen la herramienta matemática básica del análisis de Fourier empleado para analizar funciones periódicas a través de la descomposición de dicha función en una suma infinita de funciones sinusoidales mucho más simples (como combinación de senos y cosenos con frecuencias enteras). 
%\begin{multicols}{2}

\vspace{2mm} El nombre se debe al matemático francés Jean-Baptiste Joseph Fourier, que desarrolló la teoría cuando estudiaba la ecuación del calor. Fue el primero que estudió tales series sistemáticamente, y publicó sus resultados iniciales en 1807 y 1811. Esta área de investigación se llama algunas veces análisis armónico.

\vspace{2mm} Es una aplicación usada en muchas ramas de la ingeniería, además de ser una herramienta sumamente útil en la teoría matemática abstracta. Sus áreas de aplicación incluyen análisis vibratorio, acústica, óptica, procesamiento de imágenes y señales, y compresión de datos.


\begin{figure}[H]
		\centering
		\includegraphics[width=.5\textwidth]{imagenes/imagenes20/T20IM12.png}
	\end{figure}
%\end{multicols}
\emph{Primeras cuatro aproximaciones para una función periódica escalonada}
\end{myblock}




 
\chapter{Estudio general de la propagación de ondas}
\chaptermark{Ondas}

\begin{miparrafo}
Para la física, una onda (del latín unda) consiste en la propagación de una perturbación de alguna propiedad del espacio, por ejemplo, densidad, presión, campo eléctrico o campo magnético, implicando un transporte de energía sin transporte de materia. El espacio perturbado puede contener materia (aire, agua, etc.) o no (vacío).

La magnitud física cuya perturbación se propaga en el medio se expresa como una función tanto de la posición como del tiempo 

$\psi ({\vec  {r}},t)$. Matemáticamente se dice que dicha función es una onda si verifica la ecuación de ondas:

$$\displaystyle \grad^2 \psi (\vec r ,t)= \pdv[2]{\psi}{t}(\vec r ,t)$$

donde $v$ es la velocidad de propagación de la perturbación. Por ejemplo, ciertas perturbaciones de la presión de un medio, llamadas sonido, verifican la ecuación anterior.

\begin{figure}[H]
		\centering
		\includegraphics[width=.75\textwidth]{imagenes/imagenes21/T21IM01.png}
	\end{figure}

Las ondas se producen como consecuencia de oscilaciones y vibraciones de la materia, que se propagan en el tiempo según lo descrito por la Teoría de ondas, la rama de la física encargada de comprender dicho fenómeno, sumamente común en el universo.

De acuerdo al origen de las ondas o de la naturaleza del medio a través del cual se propagan, dependerán los efectos de su aparición y sus características. Así, podemos hablar de ondas de luz, de sonido, etc., cada una con propiedades físicas y frecuencias diferentes, dependiendo, entre otras cosas, del medio en el que se propagan y de cuánta energía transportan.

Algunas ondas, como las sonoras, no pueden transportarse en el vacío, requieren de un medio físico. Otras, como las ondas electromagnéticas, pueden hacerlo perfecta y velozmente: es así como operan los satélites artificiales que reenvían información a la Tierra mediante microondas.
\end{miparrafo}

\section[Propagación de una perturbación en una dirección. Ecuación de ondas]{Propagación de una perturbación en una dirección. Ecuación de ondas\sectionmark{Perturbación en una dirección}}
\sectionmark{Perturbación en una dirección}

$\psi$ es una magnitud que existe en un momento en un determinado punto del espacio que lo perturba propagándose en él. Llamemos $v$ a la velocidad con que $\psi$ se transmite. Primera hipótesis, la perturbación  se desplaza a $v_cte$. Incluimos la segunda hipótesis: la magnitud perturbada $\psi$ no sufre amortiguamiento, no se debilita.

 \begin{figure}[H]
		\centering
		\includegraphics[width=1\textwidth]{imagenes/imagenes21/T21IM02.png}
	\end{figure}
	
Consideramos la curva $\psi(x)=f(x)$ y observamos que $\psi=f(x-a)$ es la misma curva desplazada una distancia $a$ hacia la dereche. Del mismo modo, $\psi=f(x+a)$ es la curva original desplazada $a$ unidades hacia la izquierda.

Si $a=vt$ obtenemos una \emph{curva viajera}. $\psi=f(x-vt)$ es una curva que se mueve a velocidad $v$ hacia la derecha y $\psi=f(x+vt)$ una cueva que se mueve hacia la izquierda.

Concluimos que, en general, $\psi(x,t)\ = f(x \ \pm vt) =\ f \left(t \ \pm \ \dfrac x v \right) $

De otro modo, como $v=cte \to x_2-x_1=v(t_2-t_1) \to $

\begin{equation}
\subrayado{\  \bold{t_2-\dfrac {x_2}v=t_1-\dfrac{x_1}v } \ }	
\end{equation}

Debido a esto, en todas las parejas de puntos espacio-temporales la perturbación tendrá las misma características: $\psi(x,t)=f \left(t-\dfrac x v \right)$.
Si consideramos que la perturbación se desplaza hacia la izquierda, $\psi(x,t)=f \left(t+\dfrac x v \right)$.
La expresión general de como se desplaza en el tiempo una perturbación $\psi$ que se propaga a velocidad constante $v$ y que no se amortigua, en una dirección, lo hace en los dos sentidos, derecha e izquierda, como muestra la siguiente expresión:

\begin{equation}
\subrayado{\  \bold{ \psi(x,t)\ = \ f \left(t \ \pm \ \dfrac x v \right) } \ }	
\end{equation}

 \begin{figure}[H]
		\centering
		\includegraphics[width=1\textwidth]{imagenes/imagenes21/T21IM03.png}
	\end{figure}

  Veamos si somos capaces de encontrar alguna ecuación diferencial que satisfaga esta relación funcional. Llamamos $\ \boldsymbol{ z=t\pm \dfrac x v }$.
  
 $\displaystyle \pdv{\psi}{t}=\dv{f}{z}\pdv{z}{t}=\dv{f}{z} \ \to  \ \pdv[2]{\psi}{t}=\pdv{t} \dv{f}{t}=\dv[2]{f}{z}\pdv{z}{t}= \dv[2]{f}{z}$
 
 $\displaystyle \pdv{\psi}{x}=\dv{f}{z}\pdv{z}{x} =\pm \dfrac 1 v \dv{f}{z} \ \to \ \pdv[2]{\psi}{x}=\pdv{x} \left[ \pm \dfrac 1 v \dv{f}{z} \right]=\pm \dfrac 1 v \dv[2]{f}{z} \pdv{z}{x}=\dfrac 1 {v^2} \dv[2]{f}{z}$ 
 
 Despejando $\displaystyle \dv[2]{f}{z}$, se encuentra que $\displaystyle \pdv[2]{\psi}{t}=v^2 \pdv[2]{\psi}{x}$, o lo que es lo mismo:
 
\begin{equation}
\label{onda-1D}
\subrayado{ \ \boxed{ \ \boldsymbol{ \pdv[2]{\psi}{x} \ - \ \dfrac 1 {v^2} \ \pdv[2]{\psi}{t}=0 } \ } \ } 	\qquad \textbf{Ec. ondas 1-dim}
\end{equation}

que es una ecuación diferencial de segundo orden con variables independientes. Las soluciones será del tipo $\ \psi=f(t\pm \frac x v)$

La solución más general estará compuesta por una onda que se propaga hacia la derecha y otra hacia la izquierda, $\ \psi=f_1\left( t - \dfrac x v \right) + f_2\left( t + \dfrac x v \right)$

Vamos a ver un caso particular.

\section{Propagación de un fenómeno periódico}

Una gran parte de los fenómenos ondulatorios son de naturaleza periódica.

Tomaremos como origen de distancias el lugar en que se produce la perturbación, $x=0 \to \psi=f(t)$ que suponemos que se repite cada tiempo $T$

Estudiemos cómo se detecta esa perturbación en un punto a una distancia $x$ del origen: $\ t \rightsquigarrow  t-\frac x v: \ \Rightarrow \ \psi=f(t-\frac x v)$

Por el Th\footnote{Th: abreviatura de teorema} de Fourrier, cualquier función periódica se puede expresar como suma de funciones armónicas con un $T$ fundamental y multiplos de éste.

$\psi = a \cos 2\pi \left(\dfrac t T + \varphi \right) \qquad \textcolor{gris}{\omega t = \dfrac {2\pi}T=2\pi \dfrac t T }$

$\text{si }\ \varphi=0 \ \to \ \psi = a \cos 2\pi \dfrac t T $. A una distancia $x$,

$\psi = a \cos 2\pi \left(\dfrac {t-\frac x v}{T}\right)=a\cos 2\pi \left( \dfrac t T - \dfrac{x}{vT} \right)$

Congelando el tiempo y observando como varía la posición (fotografía de la perturbación): 

$\dfrac{2\pi t}{T}-\dfrac{2\pi x}{vT} \ + \ 2\pi = \dfrac{2\pi}{T}-\dfrac{2\pi (x+\lambda)}{vT} \ \to $

\begin{equation}
\subrayado{ \ \boxed{ \ \boldsymbol{\lambda=vT} \ } \ }	\qquad \textbf{longitud de onda}
\end{equation}

Relación del periodo espacial $\lambda$, \emph{longitud de onda}, con el periodo (periodo temporal) $T$

luego:  $\quad \boldsymbol{ \psi = a\cos 2\pi \left( \dfrac t T - \dfrac{x}{\lambda} \right) }$

$\psi$ depende de dos variables, $t \text{ y } x$ y tienen dos periodos, uno temporal $T$ y uno espacial $\lambda$ y una amplitud $a$.

Los puntos en que se verifica $\ 2\pi  \left( \dfrac t T - \dfrac{x}{\lambda} \right) = cte$ son los \emph{puntos equifásicos}. Los puntos equifásicos se desplazan a la misma velocidad que la perturbación $\ v=\dv{x}{t}$

\emph{Puntos en concordancia de fase} son aquellos que, \emph{en un momento determinado}, la diferencia entre sus fases es múltiplo de la longitud de onda $\lambda$. Matemáticamente:

$2\pi \left( \dfrac {\boldsymbol{t}}{T}-\dfrac {x_1}{\lambda}- \right) - 2\pi \left( \dfrac {\boldsymbol{t}}{T}-\dfrac {x_2}{\lambda}- \right) = 2\pi k; \quad \forall k\in \mathbb Z \ \to \ x_1-x_2=k\lambda$

En ese instante  $\boldsymbol{t}$, tenemos que los puntos en concordancia de fase cumplen que $\psi_1=\psi_2$

Llamamos \emph{puntos en oposición de fase} a aquellos puntos en los cuales las diferencias de fase son múltiplos impares de $\pi$; matemáticamente:

$2\pi \left( \dfrac {\boldsymbol{t}}{T}-\dfrac {x_1}{\lambda}- \right) - 2\pi \left( \dfrac {\boldsymbol{t}}{T}-\dfrac {x_2}{\lambda}- \right) =  (2k+1) \pi ; \quad \forall k\in \mathbb Z \ \to \ x_1-x_2=(2k+1) \dfrac \lambda 2$

Para estos puntos, $\ \psi_1=-\psi_2$

\section{Propagación de una perturbación en el espacio}

En una dimensión tenemos: $\quad \displaystyle  \pdv[2]{\psi}{x} \ - \ \dfrac 1 {v^2} \ \pdv[2]{\psi}{t}=0$

generalizando:

\begin{equation}
\displaystyle  \pdv[2]{\psi}{x} \ +  \pdv[2]{\psi}{y} \ + \pdv[2]{\psi}{z} \- \ \dfrac 1 {v^2} \ \pdv[2]{\psi}{t}=0
\end{equation}

Ecuación escalar de ondas en el espacio. (Suponiendo que la magnitud perturbada es una función escalar)

Matemáticamente: $\ \displaystyle \pdv[2]{x}+\pdv[2]{y}+\pdv[2]{z}=\laplacian= \bigtriangleup =\overrightarrow{\grad} \cdot \overrightarrow{\grad}$, es el llamado \textbf{operador ``laplaciana''}.

Si lo que se propaga es una magnitud escalar, $\psi$, la ecuación de ondas en el espacio es:

\begin{equation}
\subrayado{ \ \boxed{ \ \boldsymbol{\laplacian \psi \ - \dfrac 1 {v^2} \ \pdv[2]{\psi}{t}} \ } \ } \quad \textbf{ecuación escalar de ondas 3-D}
\end{equation}

Para una magnitud vectorial que se propague: $\ \vec \psi=\vec i \ \psi_x + \vec j \ \psi_y + \vec k \psi_z\ $ habrá una ecuación escalar por cada componente.

$\displaystyle 
\laplacian \psi_x \ - \dfrac 1 {v^2} \ \pdv[2]{\psi_x}{t}=0; \quad
\laplacian \psi_y \ - \dfrac 1 {v^2} \ \pdv[2]{\psi_y}{t}=0; \quad
\laplacian \psi_z \ - \dfrac 1 {v^2} \ \pdv[2]{\psi_z}{t}=0$

que también podemos escribir vectorialmente como:

\begin{equation}
\label{onda-3D}
\subrayado{ \ \boxed{ \ \boldsymbol{\laplacian \overrightarrow{\psi} \ - \dfrac 1 {v^2} \ \pdv[2]{\overrightarrow{\psi}}{t}} \ } \ } \quad \textbf{ecuación vectorial de ondas 3-D}
\end{equation}

Veamos un caso particular.

\section[Ondas esféricas en un medio homogéneo e isótropo]{Ondas esféricas en un medio homogéneo e isótropo\sectionmark{Ondas esféricas}}
\sectionmark{Ondas esféricas}

Como hipótesis, consideramos el espacio \emph{homogéneo} (igual en todas las partes) e \emph{isótropo} (las propiedades del espacio son independientes de la dirección en que se midan). Por ello, todos los puntos situados en la esfera de radio $r$ trazada desde el centro donde se origina la perturbación estarán en la misma fase, vibrarán del mismo modo.

Consideremos que la magnitud que se propaga es de tipo escalar (sonido de un grito). Veamos como es su ecuación de propagación.

$\psi=f(x,y,z,t)=f(r,t) \ $ (homogeneidad e isotropía). \textcolor{gris}{$\quad (r=\sqrt{x^2+y^2+z^2})$}

$\displaystyle \pdv[2]{\psi}{x}=
\pdv{x} \pdv{\psi}{x}=
\pdv{x} \left( \boldsymbol{\pdv{\psi}{r} \pdv{r}{x}} \right) =
 \pdv{x} \left( \boldsymbol{\dfrac x r \pdv{\psi}{r}}  \right)=$
 
 $\displaystyle
 = \dfrac 1 r \pdv{\psi}{r}+x\pdv{1/r}{x}\pdv{\psi}{r}+x \dfrac 1 r \pdv{x} \pdv{\psi}{r}=
 \dfrac 1 r \pdv{\psi}{r} - \dfrac{x^2}{r^3}\pdv{\psi}{r} + \dfrac{x^2}{r^3}\pdv[2]{\psi}{r}= $
 
 $\displaystyle
 =\dfrac{r^2-x^2}{r^3}\pdv{\psi}{r} + \dfrac{x^2}{r^2} \pdv[2]{\psi}{r}\qquad$ Análogamente para las otras componentes.

 En total tendremos: 
 $\ \displaystyle \laplacian \psi= \pdv[2]{\psi}{x}+ \pdv[2]{\psi}{y}+ \pdv[2]{\psi}{z}=  \pdv[2]{\psi}{r} + \dfrac 2 r \pdv{\psi}{r}$
 
 Por lo que la ecuación de la perturbación 3-D en espacios homogéneso e isótropos (ec. \ref{onda-3D}) queda como:
 
 \begin{equation}
 \laplacian \psi \ = \ \pdv[2]{\psi}{r} \ + \ \dfrac 2 r \ \pdv{\psi}{r} \ - \ \dfrac 1 {v^2} \ \pdv[2]{\psi}{t} \ = \ 0 	
 \end{equation}

Si multiplicamos esta ecuación por $r$: $\quad \displaystyle r\  \pdv[2]{\psi}{r}  +   2  \ \pdv{\psi}{r}  -  \dfrac r{v^2}\  \pdv[2]{\psi}{t} = 0$ 

O, lo que es lo mismo, $\quad \displaystyle \pdv[2]{\psi r}{r}-\dfrac 1 {v^2} \pdv[2]{\psi r}{t}=0$, ya que $r$ no depende de $t$.

Llamando $\quad \boldsymbol{\Psi=\psi r}$, podemos escribir:

\begin{equation} 
\subrayado{ \ \boldsymbol{\pdv[2]{\Psi}{r}\ -\ \dfrac 1 {v^2} \ \pdv[2] {\Psi}{t}=0}\ }
\end{equation}

Ecuación que coincide con la forma de la ecuación (\ref{onda-1D}) de propagación de una perturbación en una dimensión, por lo que el resultado debe ser como el de una dimensión:

$\Psi= r \psi = f_1(\left( t-\dfrac r v \right)+f_2(\left( t+\dfrac r v \right) \ \to \ \psi = \dfrac 1 r f_1(\left( t-\dfrac r v \right)+ \dfrac 1 r f_2(\left( t-\dfrac r v \right)$

Como solución particular escribiremos la de una perturbación periódica por el desarrollo de Fourrier, para una perturbación desplazándose en las $x$ positivas:

\begin{equation}
\boldsymbol{ \psi= \dfrac a r \cos 2\pi \left( \dfrac t T - \dfrac r v  \right)	}
\end{equation}

En espacio homogéneo e isótropo la perturbación. se caracteriza porque la amplitud se debilita de modo inversamente proporcional a la distancia $r$.

Se suelen definir las \emph{superficies equifases} o \emph{superficies de onda} como aquellas superficies que en todo instante tienen la misma fase. En el caso de propagación en espacios homogéneos e isótropos, las superficies de onda son esféricas.

También se habla de puntos en concordancia y en oposición de fase como aquellas superficies esféricas cuyas diferencias son múltiplos de $2\pi$ o de $(2k+1)\pi$, respectivamente: $r_2-r_1=k\lambda$ o $r_2-r_1=(2k+1)\lambda/2$

Se define la \emph{dirección de propagación de onda} en un punto dado como la dirección de la normal a la superficie de onda que pase por ese punto.

Veamos ahora como se propaga una onda vectorial en el espacio, supuesto éste homogéneo e isótropo. $\vec \psi=\psi \vec u$

La solución más general es $ \psi=\dfrac v{r^2} f(\left( t-\dfrac r v \right) + \dfrac 1 r f'(\left( t-\dfrac r v \right)$, 

con $f'$ derivada temporal de $f$. Como caso particular consideramos una perturbación periódica que desarrollada por Fourrier como movimientos armónicos podemos escribir:

$\psi  = \dfrac v{r^2} A sin 2\pi \left( \dfrac t T - \dfrac r \lambda \right)+\dfrac 1 r \dfrac {2\pi}{T} A \cos 2 \pi \left( \dfrac t T - \dfrac r \lambda \right)$

Como el primer término varía con el inverso de la distancia al cuadrado será despreciable frente al segundo término para distancias grandes y llamando $\ a=\dfrac{2\pi}{T} A$, tendremos:

\begin{equation}
\label{ondas-esfericas-grandes-distancias}
\boldsymbol{ r>>1 \ \to \  \psi \ \approx \ \dfrac a r \cos 2\pi \left( \dfrac t T - \dfrac r \lambda \right) }
\end{equation}

que coincide con el comportamiento de una onda escalar a una distancia $r$.
\vspace{10mm}  %**********************************************
\section{Ondas planas}

La onda plana asociada a la propagación de una magnitud escalar se caracteriza por que, en un instante determinado, la magnitud $\vec \psi$ es la misma en cada uno de los puntos de los planos perpendiculares  a una dirección dada (dirección de propagación). Las superficies de onda serán planos geométricos perpendiculares a la dirección de propagación de la onda.
	\begin{figure}[H]
		\centering
		\includegraphics[width=.8\textwidth]{imagenes/imagenes21/T21IM04.png}
	\end{figure}
\vspace{40mm}  %**********************************************
Elegimos $x$ como eje de propagación de la onda. Las superficies de onda son planos $YZ\ \bot \ \overrightarrow{\mathcal OX}$.

Por definición de onda plana: $\ \displaystyle \pdv{\vec \psi}{y}=\pdv{\vec \psi}{z}=0;\ \pdv{\vec \psi}{x}\neq 0$

La ecuación de ondas, para esta onda plana, se escribe como:

\begin{table}[H]
\begin{tabular}{ccc}
$\displaystyle \pdv[2]{\psi_x}{x}-\dfrac 1 v^2\ \pdv[2]{\psi_x}{t}=0\quad $ &$ \ \to \ $&$\displaystyle \quad \psi_x=a_x\ \cos 2 \pi \left( \dfrac t T - \dfrac x \lambda \right)$  \\
$\displaystyle \pdv[2]{\psi_y}{y}-\dfrac 1 v^2\ \pdv[2]{\psi_y}{t}=0\quad $ &$ \ \to \ $&$\displaystyle \quad \psi_y=a_y\ \cos 2 \pi  \left( \dfrac t T - \dfrac x \lambda \right)$  \\
$\displaystyle \pdv[2]{\psi_z}{z}-\dfrac 1 v^2\ \pdv[2]{\psi_z}{t}=0\quad $ &$ \ \to \ $&$\displaystyle \quad \psi_z=a_z\ \cos 2 \pi  \left( \dfrac t T - \dfrac x \lambda \right)$  
\end{tabular}
\end{table}

Estas ecuaciones son formalmente análogas a las de la propagación de una perturbación en una dirección.

$$\boldsymbol{\psi\ =\ \sqrt{\psi_x^2+\psi_y^2+\psi_z^2}\ =\ a \cos 2 \pi \left( \dfrac t T - \dfrac x \lambda \right) }$$

donde $ \ a=\sqrt{a_x^2+a_y^2+a_z^2}$

Comparando esta ecuación co la dela ondas esféricas a grandes distancias del cuerpo emisor, ec \ref{ondas-esfericas-grandes-distancias}, observamos que estas son similares a ondas planas. (Rayos que parten del sol y llegan a la superficie terrestre).

\begin{figure}[H]
		\centering
		\includegraphics[width=.75\textwidth]{imagenes/imagenes21/T21IM05.png}
	\end{figure}

Cuando se habla de ondas vectoriales, se suele distinguir entre \emph{ondas transversales} y \emph{ondas longitudinales}.

Una \emph{onda es transversal} cuando el vector asociado a la magnitud que se propaga vibra perpendicularmente a la dirección de propagación.

Una \emph{onda es longitudinal} cuando el vector asociado a la magnitud que se propaga vibra en la dirección de propagación.

\begin{figure}[H]
		\centering
		\includegraphics[width=.9\textwidth]{imagenes/imagenes21/T21IM06.png}
	\end{figure}
	

\section{Problemas}
\vspace{10mm} %***************************************
\begin{prob}
Un resorte tiene una constante $k$ y una masa $m$ cuelga de él. Se corta el resorte por la mitad y la misma masa de cuelga de una de las dos mitades. ¿La frecuencia de la vibración es la misma, mayor o menor antes y después de cortar el resorte.	
\end{prob}

\begin{multicols}{2}
$mg=-k_1x;\quad mg=-k_2x/2$

$k_1/k_2=1/2$

$\omega=\sqrt{k m}$

$\dfrac {\omega_1}{\omega_2}=\sqrt{\dfrac 1 2}$

$\omega_2=\sqrt{2}\ \omega_1$
\begin{figure}[H]
		\centering
		\includegraphics[width=.4\textwidth]{imagenes/imagenes21/T21IM07.png}
	\end{figure}	
\end{multicols}

El muelle cortado vibrará con mayor frecuencia que sin cortar.
\vspace{5mm} %***************************************
\begin{prob}
Supóngase un bloque de masa desconocida y un resorte de constante elástica desconocida. Explicar como puede predecirse el periodo	de oscilación de ese bloque midiendo simplemente la deformación que sufre el resorte al colgar de él el bloque.
\end{prob}

$x \ \to \ mg=km \quad \to \quad \omega=\sqrt{\dfrac k m}=\sqrt{\dfrac{g} {\boldsymbol{x}}}$
\vspace{5mm} %***************************************
\begin{prob}
Un bloque de madera, cuya densidad es $\rho$	, tiene dimensiones $a,\ b,\ c$. Mientras está flotando en el agua, con el lado $a$ en posición vertical, se empuja haca abajo y se suelta. Encontrar el periodo de oscilación resultante.
\end{prob}
\vspace{5mm} %***************************************
\begin{multicols}{2}
Cond. equil.: $Peso=Empuje$

$\rho abc g=\rho_0 x_0bcg \to \rho a=\rho_0 x_0$

Sumergimos $x_0+x \to P-E=F_T$

$\rho a \cancel{bc} g-\rho_0(x_0+x) \cancel{bc} g=\rho a \cancel{bc}\ddot{x}$

Con la condición de equilibrio, tenemos:

\begin{figure}[H]
		\centering
		\includegraphics[width=.3\textwidth]{imagenes/imagenes21/T21IM08.png}
	\end{figure}	
\end{multicols}

$\cancel{\rho a g}-\cancel{\rho_0x_0g}-\rho_0xg=\rho a \ddot{x} \quad \to \quad \ddot{x}+\dfrac{\rho_o g}{\rho a}x=0\,\quad$ ec. MAS

Luego $\quad \omega^2=\dfrac{\rho_o g}{\rho a} \ \to \ \ T=2\pi \sqrt{\dfrac{\rho a}{\rho_0 g}}$

\vspace{5mm}%******************************
\begin{prob}
Un disco sólido de radio $R$ puede colgarse de un eje horizontal a una distancia $h$ de su centro.	Encontrar la posición a que hay que colgar el disco para la cual el periodo es máximo. 
\end{prob}
\begin{multicols}{2}
$T_{pend.\ compto.}=2\pi \sqrt{\dfrac{I}{mgh}}$

Th. Steiner: $I=I_0+mh^2=\dfrac 1 2 m R^2+mh^2$

$T=2\pi \sqrt{\dfrac{\dfrac {R^2}{2}+h^2}{gh}}=T(h)$

Para buscar el máximo, derivamos
\begin{figure}[H]
		\centering
		\includegraphics[width=.3\textwidth]{imagenes/imagenes21/T21IM09.png}
	\end{figure}	
\end{multicols}
$\displaystyle \dv{T}{h}=
\dfrac
{2\pi}
{2 \sqrt{ \dfrac{\dfrac {R^2}{2}+h^2}{gh}} } 
\dfrac{2gh^2-g \left( \dfrac{R^2}{2} + h^2 \right) } {g^2R^2} \ = \ 0 \leftrightarrow\ $ numerador es cer0.

$2\cancel{g}h^2-\cancel{g}\dfrac{R^2}{2}-\cancel{g}h^2=0 \ \to \ h_{min}=\dfrac{R}{\sqrt{2}}\quad $ Para esa $h_{min} \ \to \ T_{min}=2\pi \sqrt{\dfrac{\sqrt{2}R}{g}}$

$T(R)=\sqrt{\dfrac{3R}{2g}}; \quad T(0)\to \infty$, no oscila.

\vspace{5mm}%******************************
\begin{prob}
Una aro circular de $0.61\ \mathrm{m}$ de radio y $3.63\ \mathrm{kg}$ de masa se cuelga de un clavo horizontal. ?`Cuál es el periodo de oscilación para un movimiento de pequeña amplitud? ?`Cuál sería la longitud de un péndulo simple equivalente?	
\end{prob}

Suponemos que el aro gira colgado de un punto de su superficie.

$I=I_0+mh^2=2mR^2\quad (h=R); \qquad  T=2\pi \sqrt{\dfrac{I}{hmg}}=2\pi \sqrt{\dfrac{2R}{g}}=2.22\ \mathrm{s}$

$T=T_{pen.\ simple}=2\pi \sqrt{\dfrac l g} \quad \to \quad l=2R=1.22\ \mathrm{m}$
\vspace{5mm}%******************************
\begin{prob}
Un péndulo de torsión está constituido por un bloque de madera de $8\ \mathrm{cm}	\times 12 \ \mathrm{cm}	\times 3 \ \mathrm{cm}$ con masa de $0.3\ \mathrm{Kg}$. Al suspenderlo con un alambre con el lado más corto en posición vertical se mide el periodo de oscilación, resultando ser de $2.4\ \mathrm{s}$. ?`Cuál es la constante de torsión del alambre?
\end{prob}

$T=2\pi \sqrt{\dfrac I k};\quad I=\dfrac 1 {12} m (a^2+b^2); \quad k=4\pi^2 \dfrac {I}{T^2}=3.56\times 10^{-3} \ \mathrm{N \ m}$

\vspace{5mm}%******************************
\begin{prob}
\begin{multicols}{2}.

Dos resortes están unidos y conectados como muestra la figura a una masa $m$. Si los resortes, separadamente, tienen constantes $k_1$ y $k_2$, averiguar el periodo de oscilación del sistema.
\begin{figure}[H]
		\centering
		\includegraphics[width=.4\textwidth]{imagenes/imagenes21/T21IM10.png}
	\end{figure}	
\end{multicols}	
\end{prob}

$x=x_1+x_2 \begin{cases}
 F=-k_1x_1 \\
 F=-k_2x_2	
 \end{cases}
 k_ax_1=k_2x_2 \begin{cases}
 x_1=-\dfrac{m}{k_1}\ddot{x} \\  x_2=-\dfrac{m}{k_2}\ddot{x}	
 \end{cases}$

Sumando: $\ x_1+x_2 =\ x=-\left( \dfrac {m}{k_1} + \dfrac{m}{k_2} \right) \ddot{x} \quad \to \quad \ddot{x}+\dfrac {1}{\dfrac {m}{k_1} + \dfrac{m}{k_2}} \ x = 0$

$\omega^2 = \dfrac {1}{\dfrac {m}{k_1} + \dfrac{m}{k_2}} \quad \to \quad T=2\pi \sqrt{\dfrac{m}{k_1}+\dfrac{m}{k_2}}$ 

\vspace{5mm}%******************************
\begin{prob}.

\begin{multicols}{2}
Averígüese el periodo de oscilación del sistema de la figura cuando la masa $m$ y los resortes $K_1$ y $k_2$ están unidos como se indica.
\begin{figure}[H]
		\centering
		\includegraphics[width=.4\textwidth]{imagenes/imagenes21/T21IM13.png}
	\end{figure}	
\end{multicols}
\end{prob}

Para un pequeño desplazamiento $x$ de la masa $m$ hacia la derecha, el muelle $k_1$ se comprime y ejercerá fuerza hacia la izquierda, el muelle $k_2$ se estira y la fuerza también será hacia la izquierda. Ambas en el mismo sentido.

$F=-k_1x-k_2x=m\ddot{x} \ \to \ \ddot{x}+\dfrac{k_1çk_2}{m}x=0 \ \to \ \omega=\dfrac{k_1+k_2}{m} \ \to $

$ T=2\pi \sqrt{\dfrac{m}{k_1+k_2}}$

\begin{prob}
La escala de una balanza de resorte lee desde $0$ hasta $14.5\ \mathrm{kg}$ y tiene $10\ \mathrm{cm}$ de largo. Se sabe que un paquete colgado de la balanza oscila verticalmente con una frecuencia de $2.0$ oscilaciones por segundo	. ?`Cuánto pesa el paquete?
\end{prob}

$m=14.5\ \mathrm{kg}$; $\ x=0.1\ \mathrm{m}$; $\ T=\dfrac 1 2= 0.5 \ \mathrm{s}$

$F=-kx=mg \to k=\dfrac{mg}{x}$

$mg-kx=m\ddot{x} \to \ddot{x}+\dfrac{k}{m}x=g\ \textcolor{gris}{(cte)} \quad \omega^2=\dfrac k m \ \to \dfrac{4\pi^2}{T^2}=\dfrac k m \ \to \ m=k\dfrac{T^2}{4\pi^2} = 7.8\ \mathrm{kg}$

\begin{prob}
Un péndulo	simple tiene un perido de $2$ segundos y una amplitu de $2^o$. Después de $10$ oscilaciones completas su amplitud se ha reducido a $1.5^o$, encontrar la constante de amortiguamiento.
\end{prob}

$\theta_0=2^o;\ t=10 T \to \theta =1.5^o$

Movimiento amortiguado: 

$ \theta=\theta_0 e^{-\gamma t}\cos (\omega t + \alpha) \ \to \ \theta(t)=\theta_0 e^{-\gamma t} \ \to \theta=\theta_o e^{-\gamma 10 T}$

$\gamma=\dfrac {1}{10T} \ln \dfrac {\theta}{\theta_0}=0.014 \ \mathrm{s}^{-1}$







\newpage
\begin{myblock}{Frentes de onda y rayos}
	\begin{figure}[H]
		\centering
		\includegraphics[width=1\textwidth]{imagenes/imagenes21/T21IM11.png}
	\end{figure}
	\begin{figure}[H]
		\centering
		\includegraphics[width=1\textwidth]{imagenes/imagenes21/T21IM12.png}
	\end{figure}
\end{myblock}




\include{TEMA22_chapter-A4}
\chapter{Ley de Gauss}

\begin{miparrafo}

En electromagnetismo el flujo eléctrico, o flujo electrostático, es una magnitud escalar que expresa una medida del campo eléctrico que atraviesa una determinada superficie, o expresado de otra forma, es la medida del número de líneas de campo eléctrico que penetran una superficie. Su cálculo para superficies cerradas se realiza aplicando la ley de Gauss. Por definición el flujo eléctrico parte de las cargas positivas y termina en las negativas, y en ausencia de las últimas termina en el infinito. 


El flujo eléctrico en unidades del Sistema Internacional (SI) se expresa en $\mathrm{V\ m}$, o, de forma equivalente, $\mathrm{N\ m}^2 \mathrm{C}^{-}1$ 


Faraday (s. XIX) supuso que existía un flujo eléctrico, y concluyó que era proporcional a la carga. Fue Carl Friedrich Gauss (s. XIX) quién expresó matemáticamente esta relación, dando lugar a la ley que lleva su nombre.


En física la ley de Gauss, relacionada con el Teorema de la divergencia o Teorema de Gauss, establece que el flujo de ciertos campos a través de una superficie cerrada es proporcional a la magnitud de las fuentes de dicho campo que hay en el interior de la misma superficie. Estos campos son aquellos cuya intensidad decrece como la distancia a la fuente al cuadrado. La constante de proporcionalidad depende del sistema de unidades empleado.
Se aplica al campo electrostático y al gravitatorio. Sus fuentes son la carga eléctrica y la masa, respectivamente. También puede aplicarse al campo magnetostático.

La ley fue formulada por Carl Friedrich Gauss en 1835, pero no fue publicado hasta 1867. Esta es una de las cuatro ecuaciones de Maxwell, que forman la base de electrodinámica clásica (las otras tres son la ley de Gauss para el magnetismo, la ley de Faraday de la inducción y la ley de Ampère con la corrección de Maxwell). La ley de Gauss puede ser utilizada para obtener la ley de Coulomb, y viceversa.

\end{miparrafo}

\section{Flujo de un campo vectorial}
Supóngase una región del espacio en que existe un campo vectorial $\vec V$ y sea $S$ una superficie arbitraria de esa región, en la cual, arbitrariamente, elegimos un sentido de recorrer su periferia.
\begin{multicols}{2}
$\quad$

Un elemento de superficie $\dd S$ lo representaremos por un vector $\overrightarrow{\dd S}=\dd S \vec u_N$, con $\vec u_N$ un vector unitario perpendicular a la superficie en el sentido de avance de un  sacacorchos al rotar en el sentido arbitrario que hemos escogido. 
\begin{figure}[H]
	\centering
	\includegraphics[width=.4\textwidth]{imagenes/imagenes23/T23IM01.png}
\end{figure}	
\end{multicols}

Por definición, recibe el nombre de 	\emph{flujo elemental vectorial}, $\dd \Phi$, al producto escalar del campo vectorial $V$ por el elemento diferencial de superficie $\dd \vec S$.

\begin{equation}
\subrayado{\ \boldsymbol{
\dd \Phi= \vec V \cdot \dd \vec S=\vec V \cdot \vec u_N \dd S= V \cos \theta \dd S} \ }
\end{equation}

Para calcular el flujo total que atraviesa la superficie $S$ integraremos a través de toda la superficie $S$.

\begin{equation}
\subrayado{\ \boldsymbol{ \Phi = \int_S \vec V \cdot \dd \vec S 	} \ }
\end{equation}

El flujo total puede ser positivo, negativo o nulo. Si es positivo $(0^o<\theta<90^o)$, se denomina ``saliente'', y si es negativo $(90^o<\theta<180^o)$, ``entrante''. Si la superficie es cerrada como una esfera o un elipsoide, se escribe un círculo sobre el signo integral.

\begin{equation}
	\Phi= \oint_S \vec V \cdot \dd \vec S 
\end{equation}


\begin{multicols}{2}
\small{El nombre de flujo se debe a su aplicación al estudio de los fluidos. Supongamos que tenemos un chorro de partículas, todas moviéndose hacia la derecha con velocidad $v$. Aquellas partículas que atraviesan la superficie $\dd S$ en el tiempo $t$ estarían contenidas en un cilindro de base $\dd S$, generatriz paralela a $v$ y longitud $vt$. Este volumen es $vt \dd S \cos \theta$. Suponiendo que haya $n$ partículas por unidad de volumen, el número total de partículas que pasa a través de la superficie $\dd S$ en el tiempo $t$ es $nv \dd S \cos \theta=n\vec v \cdot \vec u_n \dd S=n\vec v\cdot \dd \vec S$}, lo que concuerda formalmente con nuestra definición de flujo.\normalsize{.}
\begin{figure}[H]
	\centering
	\includegraphics[width=.55\textwidth]{imagenes/imagenes23/T23IM02.png}
\end{figure}	
\end{multicols}

Significado físico: en hidrodinámica, el flujo representa el movimiento de algo real, matemáticamente no es necesario esto.

Por convenio, el sentido del vector $\dd \vec S$ en una superficie cerrada será el que indique hacia afuera de la misma.

\section{Ángulo sólido}

\begin{multicols}{2}
Por definición, se llama \emph{ángulo sólido} subtendido por el elmento diferencial de área $\dd S=\dd a \ \dd b$, al cociente:

\begin{equation}
\boldsymbol \dd \Omega	= \dfrac {\dd S}{r^2}
\end{equation}

\begin{figure}[H]
	\centering
	\includegraphics[width=.3\textwidth]{imagenes/imagenes23/T23IM03.png}
\end{figure}
\end{multicols}

Por definición, el ángulo sólido es adimensional. Al igual que los ángulos planos se les asocia la unidad arbitraria \emph{radián}, el ángulo sólido se mide en \emph{estéreo radianes}, $\mathrm{srad }$.


\begin{multicols}{2}
Vamos a expresar en polares el ángulo sólido.

arco=ángulo $\times$ radio

$\dd a = r \sin \theta \dd \varphi$

$\dd b= r \dd \theta$
\begin{figure}[H]
	\centering
	\includegraphics[width=.4\textwidth]{imagenes/imagenes23/T23IM04.png}
\end{figure}	
\end{multicols}

$$\boldsymbol{\dd \Omega=} \dfrac{\dd S}{r^2}=\dfrac{\dd a \ \dd b}{r^2}= \dfrac {r^2 \sin \theta \dd \varphi \dd \theta}{r^2} \boldsymbol{=\sin \theta \  \dd \theta \ \dd \varphi}$$

Nos interesa el ángulo sólido finito subtendido por una superficie finita, integrando:

$\Omega= \displaystyle \int_S \dfrac{\dd S}{r^2}=\int_{\varphi_1}^{\varphi_2} \dd \varphi \cdot \int_{\theta_1}^{\theta_2} \sin \theta \dd \theta = -(\varphi_2-\varphi_1)\cdot (\cos \theta_2-\cos \theta_1)$

Vamos a estudiar el caso particular del cálculo del \emph{ángulo solido subtendido por una superficie tridimensional cerrada} desde un punto interior y desde uno exterior a la misma.

\begin{enumerate}
\item 	 \emph{ángulo solido subtendido por una superficie tridimensional cerrada desde un punto interior a la misma.}

$0\leq \varphi\leq 2\pi;\ \  0\leq \theta\leq \pi \ \ \to \ \  \boldsymbol{\Omega=}-(2\pi-0)(\cos \pi -\cos 0) \boldsymbol{= 4\pi}  \mathrm{srad}$

Una superficie cerrada $S$, con un punto $\mathcal O$ interior, subtiende un ángulo sólido de $4\pi\ \mathrm{srad}$, independientemente de las características geométricas de $S$.

\begin{figure}[H]
	\centering
	\includegraphics[width=.9\textwidth]{imagenes/imagenes23/T23IM05.png}
\end{figure}

\item  \emph{ángulo solido subtendido por una superficie tridimensional cerrada desde un punto exterior a la misma.}

Para la proyección como cirva cerrada de $S$ sobre el plano $XY$, el ángulo $\varphi=0$, no le da la vuelta al eje $Z$. Por lo que, $\ \boldsymbol{\Omega =0 \ \mathrm{srad}}$ 

Una superficie cerrada $S$, con un punto $\mathcal O$ exterior, subtiende un ángulo sólido de $0\ \mathrm{srad}$, independientemente de las características geométricas de $S$.
\end{enumerate}

\section{Ley de Gauss para el campo eléctrico}

Distribución continua de carga: $\displaystyle \ \vec E=\dfrac 1{4\pi \varepsilon_0} \int_\tau \dfrac{\rho \dd \tau}{r^2} \ \vec u_r$

Una carga puntual $q$ creará un campo eléctrico $\vec E$ y vamos a calcular el flujo que crea esta carga a atravesar una superficie cerrada $S$ en dos casos: a) $q$ es externa a $S$, y b) $q$ es interna a $S$.

\begin{figure}[H]
	\centering
	\includegraphics[width=.9\textwidth]{imagenes/imagenes23/T23IM06.png}
\end{figure}

El flujo eléctrico elemental del campo que crea $q$ al atravesar $S$ es:

$\boldsymbol{ \dd \Phi_E=}\vec E \cdot \dd \vec S =E\cos \theta \dd S=\dfrac {q}{4\pi \varepsilon_0 r^2} \cos \theta \dd S= \dfrac {q}{4\pi \varepsilon_0} \dfrac {\cos \theta \dd S}{r^2} \boldsymbol{= \dfrac {q}{4\pi \varepsilon_0} \dd \Omega}$

Considerando $q=cte$ e integrando en toda la superficie cerrada,

$\displaystyle \subrayado{ \ \Phi_E= \ } \oint_S \vec E \cdot \dd \vec S= \dfrac {q}{4\pi \varepsilon_0} \oint_S \dd \Omega = \dfrac {q}{4\pi \varepsilon_0} \Omega = \begin{cases}
\subrayado{ \  \dfrac q {\varepsilon_0} \ \ \text{	si } q \text{ interior } S \ }\\
 \subrayado{\  0 \ \ \ \  \text{	si } q \text{ exterior} \ }S 
 \end{cases}$

Que es el \emph{Teorema de Gauss de la electrostática en el caso de una carga puntual}.

A partir de este resultado podemos generalizar para un sistema discreto de cargas:

$$\boldsymbol{ \Phi_E=} \displaystyle \oint_S \vec E_{total} \cdot \dd \vec S \boldsymbol{= \dfrac 1 {\varepsilon_0} \sum_{i=1}^N q_i}$$

\begin{multicols}{2}
donde $i=1,\cdots , N$ son las cargas \textbf{interiores} a la superficie, las cargas externas no interviene a nivel de flujo eléctrico. Esto constituye el \emph{Teorema de Gauss de la electrostática para el caso de una distribución discreta de cargas}.

\begin{figure}[H]
	\centering
	\includegraphics[width=.3\textwidth]{imagenes/imagenes23/T23IM07.png}
\end{figure}
\end{multicols}

Si lo que tenemos es una distribución continua $\rho$ de cargas, el flujo es

$$\displaystyle \displaystyle{ \Phi_E=} \oint_S \vec E \cdot \dd \vec S=\dfrac 1 {\varepsilon_0} \int_\tau \dd q \boldsymbol{ = \dfrac 1 {\varepsilon_0} \int_\tau \rho \dd \tau}$$

donde $\tau$ es el volumen que contiene a  las cargas interiores a la superficie cerrada $S$.
Esto constituye el \emph{Teorema de Gauss de la electrostática para el caso de una distribución continua de cargas}.

\section{Ejemplos del teorema de Gauss}

\begin{miparrafodestacado}

Como se mencionó, la ley de Gauss es útil para determinar campos eléctricos cuando la distribución de carga está caracterizada por un alto grado de simetría.  

Las superficies gaussianas no son reales.  La superficie gaussiana es una superficie imaginaria que se elige para satisfacer las condiciones mencionadas en este caso. No tiene que coincidir con una superficie física en una situación determinada. 

Al seleccionar la superficie, siempre debe aprovechar la simetría de la distribución de la carga de manera que $E$ pueda salir de la integral. El objetivo en este tipo de cálculo es encontrar una superficie para la que cada parte de la superficie satisfaga una o más de las condiciones siguientes: 

1. Demostrar por simetría que el valor del campo eléctrico es constante sobre la porción de superficie.
 
2. Que el producto escalar $\vec E \cdot \dd \vec S$ se expresa como un producto algebraico simple $E \dd S$, porque sean paralelos entre sí. 

3. Que el producto escalar sea cero, ya que $E$ y $\dd S$ son  perpendiculares entre sí.
 
4. Que el campo eléctrico es iguala cero sobre la porción de superficie. 

Diferentes porciones de la superficie gaussiana puedan satisfacen varias condiciones en tanto que cada porción satisfaga al menos una condición.  Si la distribución de carga no tiene simetría suficiente para que una superficie gaussiana que satisfaga estas condiciones se pueda encontrar, la ley de Gauss sigue siendo cierta,  pero no es útil para determinar el campo eléctrico para esta distribución de carga. 	
\end{miparrafodestacado}


\subsection{Campo eléctrico credo por una carga distribuida uniformemente sobre un plano}

$$\displaystyle \displaystyle{ \Phi_E=} \oint_S \vec E \cdot \dd \vec S{ = \dfrac 1 {\varepsilon_0} \int_\tau \rho \dd \tau =\dfrac {Q}{\varepsilon_0}}$$

\begin{figure}[H]
	\centering
	\includegraphics[width=.9\textwidth]{imagenes/imagenes23/T23IM08.png}
\end{figure}

El campo elcétrico $\vec E$ es perpendicular al plano, si no fuese así habrían componentes tangenciales que provocarían un movimiento de las cargas y la distribución no sería uniforme.

El cilindro en gris define la superficie gaussiana (cerrada) que tomamos para aplicar el teorema de Gauss. A través de la superficie lateral del cilindro, el flujo es nulo ya que $\vec E \ \bot \ \dd \vec S$. Consecuentemente, el único flujo de líneas del campo que puede entras o salir de la superficie gaussiana es el que entre o salga por sus bases.

$\displaystyle \oint_S \vec E \cdot \dd \vec S= \int_S{S_1}  \vec E \cdot \dd \vec S +  \int_S{S_2}  \vec E \cdot \dd \vec + \cancelto{0}{\int_S{S_{lateral}}  \vec E \cdot \dd \vec S} = \vec E \cdot \vec S_1+\vec E \cdot \vec S_2 $

$\sigma=\dfrac Q S$, densidad superficial de carga en la placa.

Como $S_1=S_2 \ \to \ 2 E \cancel{S}=\dfrac 1{\varepsilon_0}Q_{int}= \dfrac 1 {\varepsilon_0} \sigma \cancel{S} \ \to $

$$ \subrayado{ \ \boldsymbol{E\ = \ \dfrac {\sigma}{2\varepsilon}} \ } $$

\emph{El campo eléctrico creado por una distribución continua de carga distribuida uniformemente sobre un plano es independiente de la distancia al plano, solo depende de la distribución de cargas.}

\subsection{Campo eléctrico producido por dos planos paralelos cargados con cargas iguales y opuestas}

\begin{figure}[H]
	\centering
	\includegraphics[width=.7\textwidth]{imagenes/imagenes23/T23IM09.png}
\end{figure}

$Q=\sigma S$

A la derecha de las placas: $\ E=\dfrac{\sigma+}{2\varepsilon_0}+\dfrac{\sigma-}{2\varepsilon_0}=\dfrac{+\sigma}{2\varepsilon_0}+\dfrac{-\sigma}{2\varepsilon_0}=0$.

 A la izquierda de las placas: $\ E=-\dfrac{\sigma+}{2\varepsilon_0}-\dfrac{\sigma-}{2\varepsilon_0}=-\dfrac{+\sigma}{2\varepsilon_0}-\dfrac{-\sigma}{2\varepsilon_0}=0$.

Entre las placas: $\ \dfrac{\sigma+}{2\varepsilon_0}-\dfrac{\sigma-}{2\varepsilon_0}=\dfrac{+\sigma}{2\varepsilon_0}-\dfrac{-\sigma}{2\varepsilon_0}=\dfrac{2\sigma}{2\varepsilon_0}=\dfrac{\sigma}{\varepsilon_0}$

\emph{El campo eléctrico entre dos placas  planoparalelas cargadas con cargas de distinto signo es constante, independiente de la distancia a las placas. Fuera de la zona entre las placas, el campo es nulo.}

\subsection{Campo eléctrico creado por una distribución esférica de cargas}

\begin{figure}[H]
	\centering
	\includegraphics[width=1\textwidth]{imagenes/imagenes23/T23IM10.png}
\end{figure}

El campo eléctrica de una esfera cargada tiene simetría radial. Las superficies gaussianas que vamos a escoger van a ser esferas concéntricas de radios $r$ mayor y menor que el radio $R$ de la esfera cargada, por ello, $\vec E \ || \ \dd \vec S$ y $E=cte$ en todos los puntos de la superficie esférica.

\begin{itemize}
\item $\vec E$ para distancias $r>R$	

$\displaystyle \Phi_E= \oint_S \vec E \cdot \dd \vec S=\oint_S E \dd S=E\oint_S \dd S= E \ 4\pi r^2 = \dfrac {Q}{\varepsilon_0}$ 

Por lo que,

$$\subrayado{\  E(r>R)=\dfrac Q {4\pi \varepsilon_o r^2} \ }$$

\emph{El campo eléctrico creado por una distribución continua de carga en una esfera, para puntos externos $r>a$, es el mismo que produciría una carga puntual $Q$ igual a toda la carga de la esfera y situada en su centro.}

\item $\vec E$ para distancias $r<R$	

Hemos de distinguir entre dos posibilidades: que la carga de la esfera está distribuida solamente en su superficie o que lo esté en todo su volumen.

	\begin{itemize}
	\item $Q$ sobre la superficie esférica $4\pi R^2$, no hay carga en el interior de la esfera, solo está en su superficie (esto ocurre en los metales).
	
	$$ \subrayado{E(r<R)=0} \quad \subrayado{\ \text{distribución superficial de carga} \ }$$
	
	\emph{El campo eléctrico de una distribución superficial esférica de cargas en el interior de la esfera es nulo. No hay cargas en su interior.}
	
	\item $Q$ sobre todo el volumen $\frac 4 3 \pi R^3$ de la esfera. $\rho=cte=\dfrac Q{\frac 4 3 \pi R^3}$. Para $r<R$,
	
	$\displaystyle \Phi_E= \dfrac {Q_{r}}{\varepsilon_0}=\dfrac{\frac 4 3 \pi r^3 \rho}{\varepsilon_0}=\dfrac 4 3 \dfrac {\pi r^3 \rho}{\varepsilon_0}$
	
	$\displaystyle \Phi_E=\oint_S \vec E \cdot \dd \vec S= \oint E \dd S = E \oint_S \dd S= E\ 4\pi r^2$
	
	Por lo que,
	
	$$ \subrayado{ \ E=\dfrac{\rho r}{3\varepsilon_0}  \ } \quad \subrayado{ \ \text{distribución de carga en el volumen} \ }$$
	\end{itemize}
	
	\emph{El campo eléctrico de una distribución volumétrica  esférica de cargas en el interior de la esfera es directamente proporcional a $r$. Para $r=a \ \to \ E=\dfrac{\rho a}{3\varepsilon_0}$}.
\end{itemize}

\section{Forma diferencial de la ley de Gauss}

\begin{multicols}{2}
Vamos a encontrar una relación local, de punto, entre el campo eléctrico y la distribución de carga.

$\dd \tau=\dd x\ \dd y \ \dd z$

$\vec E$ en el centro de simetría de la figura.

Vamos a calcular el flujo eléctrico a través de todas las caras del elemento de volumen del cubo.
 
\begin{figure}[H]
	\centering
	\includegraphics[width=.4\textwidth]{imagenes/imagenes23/T23IM11.png}
\end{figure}	
\end{multicols}

Entre los puntos $E$ y $A$ hay una variación del campo $\dd \vec E$

------ Cara $ABCD$: el campo vale $\ \vec E + \dfrac 1 2 \dd \vec E$

$\dd \Phi_{ABCD}=\left( \vec E + \dfrac 1 2 \dd \vec E \right) \cdot \vec i \ \dd y \dd z=E_x \ \dd y \dd z + \dfrac 1 2 \dd E_x \ \dd y \dd z$ 

En el caso de campos electrostáticos en que estamos, el campo puede variar con la distancia, pero no con el tiempo, $\ E=E(x,y,z)$.

$\displaystyle \dd E_x=\pdv{E_x}{x} \dd x+\pdv{E_x}{y} \dd y+\pdv{E_x}{z} \dd z$

Como, nos hemos desplazado en el eje $x$ desde el centro del elemento de volumen, solo hay variación en $\dd x$, y $\dd y=\dd z=0$, por lo que: 

$\displaystyle \dd E_x=\pdv{E_x}{x} \dd x \quad \to \quad$

$\displaystyle \dd \Phi_{ABCD}=
E_x \ \dd y \dd z + \dfrac 1 2 \dd E_x \ \dd y \dd z=
E_x \ \dd y \dd z + \dfrac 1 2 \pdv{E_x}{x} \ \dd x \dd y \dd z$

Análogamente: $\ \dd \displaystyle \Phi_{EFGH}= \left( \vec E - \dfrac 1 2  \dd \vec E \right) \cdot (-\vec i) \ \dd y \dd z= 
-E_x \ \dd y \dd z + \dfrac 1 2 \pdv{E_x}{x} \ \dd x \dd y \dd z$

$\displaystyle \dd \Phi_x= \dd \Phi_{ABCD}+\dd \Phi_{CDEF}= \pdv{E_x}{x}\ \dd x \dd y \dd z=\pdv{E_x}{x} \ \dd \tau$

Mediante razonamientos análogos, $\ \displaystyle \dd \Phi_y=\pdv{E_y}{y} \ \dd \tau;\quad \displaystyle \dd \Phi_z=\pdv{E_z}{z} \ \dd \tau$

Por lo que, $\ \displaystyle \dd \Phi=\dd \Phi_x+\dd \Phi_y+\dd \Phi_z=\left( \pdv{E_x}{x}+\pdv{E_y}{y}+\pdv{E_z}{z} \right) \ \dd \tau$

Como la ley de Gauss dice: $\quad \dd \Phi_E=\dfrac 1 {\varepsilon_0} \rho \dd \tau$, finalmente tenemos que

\begin{equation}
\label{Gauss-diferencial}
\subrayado{ \ \boldsymbol { \left( \pdv{E_x}{x}+\pdv{E_y}{y}+\pdv{E_z}{z} \right)\  =\ \dfrac {\rho}{\varepsilon_0}   }	 \ }
\end{equation}

\emph{Ley de Gauss en forma diferencial}, es una ecuación local o de punta, da cuenta de la variación del campo en función de la densidad de carga en el punto $(x,y,z)$.

Teniendo en cuenta el \emph{operador nabla}, podemos escribir la \textbf{ley de Gauss en forma diferencial} como:

\begin{equation}
\label{Gauss-diferencial2}
\subrayado{\  \boxed { \ \boldsymbol { \overrightarrow{\grad} \cdot \overrightarrow{E} \  =\ \dfrac {\rho}{\varepsilon_0}  } \ } \ }  \qquad \textbf{Divergencia del campo eléctrico.}
\end{equation}

\section{Ecuaciones de Poisson y de Laplace}

Son simples consecuencias del teorema de Gauss.

\subsection{Ecuación de Gauss}

$\vec E$ conservativo $\to \vec E=-\vec{\grad} V; \ V$ es el \textbf{potencial eléctrico} (escalar)

$\vec {\grad} \cdot \vec E= \vec {\grad} (-\vec{\grad} V)=\dfrac \rho {\varepsilon_0} \quad \to \quad \vec {\grad} \cdot ( \vec {\grad} V )=-\dfrac \rho {\varepsilon_0}$

$\displaystyle \vec {\grad} \cdot  \vec {\grad}=
\left( \vec i \pdv{x} +\vec j \pdv{y} +\vec k \pdv{z} \right) \cdot
\left( \vec i \pdv{x} +\vec j \pdv{y} +\vec k \pdv{z} \right)=$

$\displaystyle = \pdv[2]{x}+\pdv[2]{y}+\pdv[2]{z}=\grad^2=\nabla \quad \to \quad \displaystyle \vec {\grad} \cdot  \vec {\grad}=\nabla $

\begin{equation}
\label{Ec-Poisson}
\subrayado{\ \boxed{\ \boldsymbol{\grad^2 V=-\dfrac \rho {\varepsilon_0}} \ } \ } \qquad \textbf{Ecuación de Poisson}	
\end{equation}

\emph{La ecuación de Poisson es una relación local entre el potencial eléctrico y la densidad de carga}

\subsection{Ecuación de Laplace}

En los puntos donde no hay carga,

\begin{equation}
\label{Ec-Laplace}
\subrayado{\ \boxed{\ \boldsymbol{\rho=0 \ \ \to \ \  \grad^2 V=0 } \ } \ } \qquad \textbf{Ecuación de Laplace}	
\end{equation}

\begin{ejem}
Calculo del potencial eléctrico y el campo eléctrico en la región vacía comprendida entre dos planos paralelos, cargados a los potenciales $V_1$ y $V_2$	
\end{ejem}

\begin{multicols}{2}
No hay carga entre las placas: 

$\rho=0 \to $ Laplace $\to \grad^2V=0$

La simetría del problema nos induce a pensar el el campo solo tiene componente $x$, por lo que

$\displaystyle \dv[2]{V}{x}=0 \ \to \ \dv{V}{x}=cte$

Físicamente, $\displaystyle \dv{V}{x}=-E$, 

luego $E=cte$ . Integrando:
\begin{figure}[H]
	\centering
	\includegraphics[width=.35\textwidth]{imagenes/imagenes23/T23IM12.png}
\end{figure}	
\end{multicols}

$\displaystyle \int_{V_1}V \dd V=-E \int_{x_1}^x \dd x \ \to \ V-V_1=-E(x-x_1)$

$$V_2=V_1-E(x_2-x_1); \qquad x_2-x_1=d \ \to \ 	E=-\dfrac {V_2-V_1}{d}$$

La dependencia analítica de $V$ en función de la distancia $x$ es la ecuación de una recta, es decir, el potencial eléctrico varía linealmente con la distancia. Resultado que concuerda con el obtenido en el capítulo anterior (subsección \ref{EV-Ecte}).

\begin{ejem}
Resolver el mismo problema anterior pero suponiendo que entre las placas hay una distribución uniforme de carga.
\end{ejem}

Ahora debemos aplicar la ecuación de Poisson. Por la simetría del problema, el potencial solo depende de $x$ y podremos escribir:

$\displaystyle \dv[2]{V}{x}=-\rho / \varepsilon_0$, con $\rho=cte$. Integrando,

$\displaystyle \int_{x_1}^x \dv[2]{V}{x} \dd x = -\dfrac 1 {\varepsilon_0} \int_{x_1}^x \rho \dd x = -\dfrac \rho {\varepsilon_0} \int_{x_1}^x \dd x$

$\displaystyle \dv{V}{x}-\cancelto{-E_1}{\left( \dv{V}{x} \right)_{x=x_1}}=-\dfrac \rho {\varepsilon_0} (x-x_1)$

$\displaystyle \cancelto{-E}{\dv{V}{x}}=-E_1-\dfrac \rho {\varepsilon_0} (x-x_1)$

$\displaystyle E=E_1+\dfrac \rho {\varepsilon_0} (x-x_1)$

El campo eléctrico varía linealmente con $x$. Integrando de nuevo,

$\displaystyle \int_{V_1}^V \dd V = -\int_{x_1}^x E_1 \dd x - \dfrac \rho {\varepsilon_0} \int_{x_1}^x (x-x_1) \dd x$

$V=V_1-E_1(x-x_1)-\dfrac \rho {2\varepsilon_0} (x-x_1)^2$

El potencial eléctrico varia con $x^2$.

Haciendo $x_2=x_1 \quad \to \quad V_2=V_1-E_1(x_2-x_1)-\dfrac \rho {2\varepsilon_0} (x_2-x_1)^2$, de donde se despeja $E$.





\section{Problemas}

\begin{prob}
Hallar el campo eléctrico creado por una distribución cilíndrica de carga de longitud infinita.	
\end{prob}

$C$, cilindro de radio $a$ y longitud $L$; $\lambda$ densidad `lineal' de carga, $\lambda=q/L$.

Por simetría, el campo eléctrico en un punto depende solo de la distancia al eje del cilindro y está dirigido radialmente.

\begin{figure}[H]
	\centering
	\includegraphics[width=.6\textwidth]{imagenes/imagenes23/T23IM14.png}
\end{figure}


Tomamos como superficie gaussiana una superficie $S$ cilíndrica de radio $r$ coaxial con el cilindro $C$. El flujo a través de esta superficie se compone del flujo a través de la superficie lateral y el flujo a través de sus dos bases, siendo estos últimos nulos pues $\vec E \ \bot \vec S_{base}$. Solo queda el flujo a través de la superficie lateral de $S:\quad S\Phi_E=2\pi r L E$

Aplicando el teorema de Gauss, tendremos:

\begin{itemize}
\item $r>a$

La carga es $q=\lambda L \quad \to Gauss: \qquad 2\pi rL E=\dfrac{\lambda L}{\varepsilon_0}$

$$E=\dfrac \lambda {2\pi \varepsilon_0 r}$$
\item $r<a$

Ahora, la superficie gaussiana es $S'$ el la figura.

	\begin{itemize}
	\item Carga distribuida superficialmente.
	
	No hay carga en el interior de $' \quad Gauss:\qquad 2\pi r L E=0$
	
	$$E=0$$
	\item Carga distribuida volumétricamente.
	
	La carga interior a la superficie $\ S' \ $ de radio $\ r<a\ $ es $\quad q'=\lambda L \dfrac {r^2}{a^2}; \quad Gauss: \qquad 2\pi r L E=q'$
	
	$$E=\dfrac{\lambda r}{2 \pi \varepsilon_0 a^2}$$
	\end{itemize}	
\end{itemize}


\begin{prob}
Hallar el flujo eléctrico y la densidad de carga	 en el interior de un cubo de lado $a$ colocado en na región en que el valor del campo es: $\ a) \vec E=\vec i\ cx^2; \quad b)\ \vec E=c(\vec i\ y+\vec j \ x)$
\end{prob}

$a)\qquad \displaystyle \dfrac {\rho}{\varepsilon_0}=\vec \grad \cdot \vec E=
\pdv{E_x}{x}+\cancelto{0}{\pdv{E_y}{y}}+\cancelto{0}{\pdv{E_z}{z}}=2cx \ \to \ \ \rho=2c\varepsilon_0 x$

$\phi_E= \displaystyle \oint \vec E \cdot \dd \vec S=\dfrac 1 {\varepsilon_0} \int_\tau \rho \dd \tau= =
\dfrac 1 {\varepsilon_0} \int_\tau \rho a^2 \dd x
 2ca^2 \int_0â x \dd x=\dfrac 2 3 c a^5$
 
 $b) \qquad  
\displaystyle \dfrac {\rho}{\varepsilon_0}=0=\vec \grad \cdot \vec E=
\cancelto{0}{\pdv{E_x}{x}}+\cancelto{0}{\pdv{E_y}{y}}+\cancelto{0} = 0 \ \to \ \ \rho=0 x$

$\vec E=cy \vec i + cx\ \vec j=E_x \vec i + E_y \vec j$

$\Phi_E=\displaystyle \oint \vec E \cdot \dd \vec S =\dfrac 1 {\varepsilon_0} \int_\tau \rho \dd \tau =0$


\begin{prob}
Un recipiente hemiesférico de radio $R$ tiene una carga total $q$	distribuida uniformemente en su superficie interior. Encontrar el campo eléctrico en el centro de curvatura.
\end{prob}

\begin{figure}[H]
	\centering
	\includegraphics[width=.6\textwidth]{imagenes/imagenes23/T23IM15.png}
\end{figure}

$\dd E_x=\dd E \cos \theta = \dfrac {\sigma \dd S_{anillo}}{4\pi \varepsilon_0 R^2}\cos \theta$

$\dd S=2\pi r \dd l =2\pi R \sin \theta \ R\dd \theta=2\pi R^2 \sin \theta \dd \theta$

$\displaystyle \dd E_x=\dfrac{\sigma 2 \pi R^2 \sin \theta \cos \theta \dd \theta}{4\pi \varepsilon_0 R^2}=\dfrac{\sigma}{4\varepsilon_0} \int_0^{\pi/2}\sin 2 \theta \dd \theta$

\rightline{ \textbf{\textcolor{blue}{acábese}}}

\justify

\begin{prob}
Se dispone, em forma alternativa, un número infinito de cargas positivas y negativas de valor $q$ sobre una línea recta. La separación entre cargas adyacentes es siempre la misma, $r$. Determinar la energía potencial de una de estas cargas.	
\end{prob}

\begin{figure}[H]
	\centering
	\includegraphics[width=1\textwidth]{imagenes/imagenes23/T23IM16.png}
\end{figure}

$V_r=\dfrac {q}{4\pi \varepsilon_0 r}+\dfrac {q}{4\pi \varepsilon_0 r}=\dfrac {q}{2\pi \varepsilon_0 r}$

$V_{2r}=-\dfrac {q}{4\pi \varepsilon_0 (2r)}-\dfrac {q}{4\pi \varepsilon_0 (2r)}=-\dfrac {q}{2\pi \varepsilon_0 (2r)}$

$V_{3r}=\dfrac {q}{4\pi \varepsilon_0 (3r)}+\dfrac {q}{4\pi \varepsilon_0 (3r)}=\dfrac {q}{2\pi \varepsilon_0 (3r)}$

etcétera

$V=\displaystyle \sum{i=1}^\infty v_{ir}=\dfrac {q}{2\pi \varepsilon_0 r}\sum_{i=1}^\infty
\left( 1-\dfrac 1 2 + \dfrac 1 3 - \dfrac 1 4 + \cdots \right)$

En el paréntesis aparece el desarrollo en serie de McLauirin de $\ln(1+x)$ en $x=1$, por lo que

$V=\dfrac {q}{2\pi \varepsilon_0 r} \ln 2 \quad \to \quad \mathcal E_p=-2 \dfrac {q}{4\pi \varepsilon_0 r} \ln 2 = - \dfrac {q^2}{4\pi \varepsilon_0 r}\ln 2$

\newpage %********************************************************
\begin{myblock}{?´Qué es a ley de Gauss?}

\begin{figure}[H]
	\centering
	\includegraphics[width=.9\textwidth]{imagenes/imagenes23/T23IM13.png}
\end{figure}

\vspace{2mm} En física las propiedades de simetría de los sistemas constituyen una herramienta importante para simplificar los problemas. Muchos sistemas físicos tienen simetría: un cuerpo cilíndrico no se ve distinto después de hacerlo girar sobre su eje, y una esfera de metal con carga se ve igual una vez que se ha hecho girar alrededor de cualquier eje que pase por su centro. 

\vspace{2mm} La ley de Gauss trata de lo siguiente: dada cualquier distribución general de carga, se rodea con una superficie imaginaria que la encierre y luego se observa el campo eléctrico en distintos puntos de esa superficie imaginaria. Es una relación entre el campo en todos los puntos de la superficie y la carga total que ésta encierra .

\vspace{2mm} La ley de Gauss es parte de la clave para utilizar consideraciones de simetría que simplifiquen los cálculos del campo eléctrico. Pero la ley de Gauss es algo más que un método para hacer ciertos cálculos con facilidad. En realidad es un enunciado fundamental acerca de la relación que hay entre las cargas eléctricas y los campos eléctricos. Entre otras cosas, la ley de Gauss ayuda a entender cómo se distribuye la carga en los cuerpos conductores. 
\end{myblock}
\include{TEMA24_chapter-A4}
\chapter{Intensidad eléctrica y fuerza electromotriz}
\chaptermark{Intensidad eléctrica y fem}

\begin{miparrafo}
La corriente eléctrica es el flujo de carga eléctrica que recorre un material. Se debe al movimiento de las cargas (normalmente electrones) en el interior del mismo. Al caudal de corriente (cantidad de carga por unidad de tiempo) se le denomina intensidad de corriente eléctrica(representada comúnmente con la letra I). En el Sistema Internacional de Unidades se expresa en culombios por segundo (C/s), unidad que se denomina ampere, A. El nombre de ampere es en honor al físico francés André-Marie Ampère (1775-1836). 

El instrumento usado para medir la intensidad de la corriente eléctrica es el galvanómetro que, calibrado en amperios, se llama amperímetro, colocado en serie con el conductor por el que circula la corriente que se desea medir.


Históricamente, la corriente eléctrica se definió como un flujo de cargas positivas (+) y se fijó el sentido convencional de circulación de la corriente, como un flujo de cargas desde el polo positivo al negativo. Sin embargo posteriormente se observó, gracias al efecto Hall, que en los metales los portadores de carga son negativos, electrones, los cuales fluyen en sentido contrario al convencional. En conclusión, el sentido convencional y el real son ciertos en tanto que los electrones como protones fluyen desde el polo negativo hasta llegar al positivo (sentido real), cosa que no contradice que dicho movimiento se inicia al lado del polo positivo donde el primer electrón se ve atraído por dicho polo creando un hueco para ser cubierto por otro electrón del siguiente átomo y así sucesivamente hasta llegar al polo negativo (sentido convencional). Es decir la corriente eléctrica es el paso de electrones desde el polo negativo al positivo comenzando dicha progresión en el polo positivo.

En el siglo XVIII cuando se hicieron los primeros experimentos con electricidad, solo se disponía de carga eléctrica generada por frotamiento (electricidad estática) o por inducción. Se logró (por primera vez, en 1800) tener un movimiento constante de carga cuando el físico italiano Alessandro Volta inventó la primera pila eléctrica.
\end{miparrafo}

\section[Corriente eléctrica. Intensidad y densidad de corriente]{Corriente eléctrica. Intensidad y densidad de corriente\sectionmark{Corriente eléctrica}}
\sectionmark{Corriente eléctrica}

Hay una ley termodinámica que asegura que un sistema de partículas que se encuentra en un medio tiene una energía cinética que vale $\frac 3 2 KT$, 	donde $k$ es la constante de Boltzman y $T$ la temperatura absoluta.

Las partículas cargadas tienen, pues, un movimiento aleatorio en el interior de una determinado material debido a la temperatura. Supongamos que a ese material le conectamos a una batería, es decir, lo sometemos a la acción de una campo eléctrico. Ahora, las partículas, además de la componente aleatoria, tendrán una componente en la dirección del campo. Este flujo de cargas es lo que se llama \emph{corriente eléctrica}, hay un flujo neto de cargas, distinto de cero, en una dirección.

La magnitud que vamos a introducir para expresar cuantitativamente la corriente eléctrica es la \emph{intensidad de corriente eléctrica}, $I$, que representa \emph{la cantidad de carga por unidad de tiempo que atraviesa la sección normal de un conductor},

\begin{equation}
I\ = \ \dv{q}{t}	
\end{equation}

En el sistema internacional se expresa en $\mathrm{C\ s}^{–1}=\mathrm{A}$, amprère,  si tomamos el $\mathrm{C}$ como unidad fundamental. 

En un conductor homogéneo e isótropo, $I=cte$, que está relacionado con la ecuación de la continuidad de la hidrodinámica. La conservación de la carga exige que la carga no se puede almacenar ni aniquilar en diversos tramos del conductor.

Ahora, las cargas en un conductor no solo podrán circular por su superficie (conductora aislado en tema anterior) sino que lo harán por todo su interior.

Necesitamos de una \emph{fuente} que nos proporcione energía para que las cargas se puedan mover:

$$\displaystyle P\ =\ \dv{W}{t}\ = \ \dv{ q \ V}{t}\ =\ I\ V$$

Por convenio, se suele definir una dirección de la corriente eléctrica y éste es el que tomarían las cargas libres positivas, aunque el la mayoría de los metales lo que ¡realmente se mueven son los electrones, cargas negativas.

Introducimos una nueva magnitud que representará la corriente eléctrica a escala diferencial, es la \emph{densidad d corriente}, $J$:

$$J\ =\ \displaystyle \dv{I}{S'}$$

La densidad de corriente es la intensidad de corriente eléctrica por unidad de sección normal del conductor.

$\vec J=J\vec u_q$, con $\vec u_q$ vector unitario en la dirección y sentido del movimiento de las supuestas cargas positivas. 

$\vec J \cdot \dd \vec S = J \dd S \cos \theta = J \dd S'$

Con esto, $\quad \displaystyle I\ = \ \int_S \vec J \dd \vec S$

En un conductor homogéneo, $J=cte \ \to I=\int J \dd S = I \int \dd S = JS \ \to \to I=cte$, independiente del tiempo.

\section{Conductividad eléctrica. Ley de Ohm}

\begin{multicols}{2}
Los dieléctricos se caracterizan por la susceptibilidad eléctrica. Introduciremos una nueva magnitud que medirá la capacidad que tienen las cargas para moverse en el interior de un conductor (metal), es la \emph{conductividad eléctrica}.

Los átomos de los metales se disponen en una estructura cristalina en que los electrones están libres.

\begin{figure}[H]
	\centering
	\includegraphics[width=.4\textwidth]{imagenes/imagenes25/T25IM01.png}
\end{figure}
\end{multicols}

La \emph{conductividad eléctrica} $\sigma$ mide la capacidad de moverse que tienen los electrones de moverse en la red cristalina dependiendo de lo espaciada o comprimida que esté la red, da la idea de la facilidad con que un electrón puede moverse en el interior de un metal.

El físico alemán Georg Simon Ohm (1789-1854) experimentó con muchos metales y enunció la ley experimental que lleva su nombre y dice \emph{``En un conductor, a $T$ constante, la razón entre la diferencia de potencial entre dos puntos, $V$ y la intensidad de la corriente eléctica que circula, $I$, es constante''}. A la constante de proporcionalidad, que depende de la naturaleza del conductor, se le llama \emph{\textbf{resistencia}} del metal, $R$.

\begin{equation}
\subrayado{ \ \boxed{ \ \boldsymbol{ \dfrac V I \ = \ R	} \ } \ } \qquad \textbf{Ley de Ohm}
\end{equation}

Posteriores estudios demostraron que esta ley la cumplen una gran gama de metales en una gran gama de temperaturas, la ley es casi independiente de la temperatura.

La resistencia, en el sistema internacional de unidades s expresa en $\mathrm{V \ A}^{-1}= \mathrm{\Omega}$ y recibe el nombre de Omh, en honor a su descubridor.

Considenemos un conductor metálico y cilíndrico, de longitud $l$ y vamos a calcular el campo eléctrico.

En el caso unidimensional, $\ \displaystyle E=-\dv{V}{x} \ \to \ V=E\ l$. La conductividad $\sigma$ está relacionada con la estructura cristalina del metal y con $E$.

Al estar considerando un conductor eléctrico, $I=J\ S$, tendremos

$\dfrac V I = \dfrac {E\ l}{J\ S}=R \ \to \ J=\dfrac{l}{R\ S} \ E$

$\dfrac{l}{R\ S}$ depende de las características geométricas y de la naturaleza del metal y $E$ es la magnitud aplicada, así, llamamos

$$\sigma \ = \ \dfrac {l}{R\ S} \qquad \text{conductividad eléctrica del metal} \quad \to \quad J \ = \ \sigma \ E$$

$\text{En forma vectorial: } \qquad \vec J \ = \ \sigma \ \vec E$

Vamos a interpretar la ley de Ohm dede el punto de vista dinámico o microscópico. Partimos de la ecuación de movimiento de una carga, lo aplicamos para todas las cargas y comparamos con la ley de Ohm.

$\displaystyle I=\dv{q}{t}=\dfrac{e\ \dd N}{\dd t}$, $e$ es el valor absoluto de la carga. $\ \vec J =\sigma \vec E$

Supongamos que tenemos un conductor homogéneo, con las mismas propiedades físicas en cada uno de sus puntos,

$\displaystyle J=\dfrac I S = \dfrac {e\dd N}{S \dd t}=\textcolor{gris}{\left( v=\dv{l}{t} \right)}=\dfrac{e\ dd N}{S \dd l}v=n\ e\ v$

donde $n=\dfrac{\dd N}{S \dd l}=\dfrac{\dd N}{\dd \tau}$

Vectorialmente, $\quad \vec J \ = \ n\ e\ \vec v_E$, con $\vec v_E$ la velocidad de la carga debida al campo eléctrico.

Cuando la conducción es debida a los electrones, carga negativa, tenemos:

$\vec J = -ne\vec v_E$, que comparando con $\vec J=\sigma \vec E$, se obtiene:

$\vec v_E=-\dfrac{\sigma}{n e} \vec E$, velocidad de arrastre debida al campo eléctrico.

Debido a la homogeneidad del conductor (cilíndrico en este caso), el campo eléctrico $\vec E$ por lo general va a ser constante. $\sigma$ depende de las características físicas del conductor, $n$ es la densidad de partículas y $e$ la carga de éstas. Hemos obtenido que $\vec v_E$ es constante, es la velocidad límite de arrastre. Esto significa que el electrón rozará con el resto del metal de tal manera que alcanzará la velocidad límite de arrastre.

La ecuación dinámica del electrón cuando se somete el metal a un campo eléctrico $\vec E$ es
$\ \displaystyle m \dv{\vec v}{t}=-e\vec E-k\vec v.\ $ $k$ es una constante de rozamiento (como la fuerza de Stookes) que engloba la facilidad con que el electrón puede entrar en la red cristalina.

Para que exista una velocidad límite, la aceleración ha de ser cero, por lo que $\ \displaystyle m \dv{\vec v}{t}=0 \ \to \ -e\vec E-k\vec v_E=0 \to \vec v_E=\dfrac{-e\vec E}{k}$

Comparando con la velocidad límite encontrada anteriormente, 

$\dfrac{\sigma n}{e}=\dfrac e k \ \to \ k=\dfrac{ne^2}{\sigma}$

Vamos a ampliar el problema introduciendo el \emph{tiempo de relajamiento}: es el tiempo que pasa desde que la velocidad del electrón es la de arrastre hasta que sea la de arrastre dividida por el número $e$, una vez quitado el campo eléctrico exterior· $\ t_R:\ \vec v_E \ \to \ \dfrac{\vec v_E}{e}$

Al quitar el campo, $\displaystyle m\dv{\vec v}{t}=-k\vec v \ †o \ \dv{\vec v}{\vec v}=-\dfrac m n \dd t$

$\ln v_x=-\dfrac k m t + \ln \mathcal C;\quad t=t_0:\ v_x=v_{Ex} \quad v_x=v_{Ex}e^{-kt/m}$

$\displaystyle \dv{\vec v}{\vec v}=-\dfrac m n \dd t \quad \to \quad \vec v=\vec v_E e^{-kt/m}$

Cuando $\ t=t_R \ \leftrightarrow \ \vec v=\vec v_E e^{-1}=\vec v_E e^{-kt_R/m}$

De donde, $\ t_R=m/k$, que comparando con $\ k=\dfrac{ne^2}{\sigma}$, tenemos

$$t_R \ = \ \dfrac m k \ = \ \dfrac {m\ \sigma}{n\ e^2}$$

$\sigma= l/(RS)$; $\sigma$ se puede determinar experimentalmente mediante la ley de Ohm. Para la mayoría de metales, $\sigma \sim 10^7 \ \Omega^{-1} \mathrm{m}^{-1}$

Si de la expresión $t_R  =  (m\sigma)/(n e^2)$ deducimos que $\sigma \sim 10^7 \ \Omega^{-1} \mathrm{m}^{-1}$, todo el razonamiento anterior será correcto.

\emph{Hipótesis:} Para cada electrón, hay un átomo relacionado con él.  $\ n$ número de partículas por unidad de volumen. Supongamos que cada uno de los átomos del metal contribuye como máximo con $1 \ e^{-}$, así, $\ n$ representará el número de átomos por unidad de volumen. A partir de la densidad del metal y de su número atómico podemos calcular $\ n \sim 10^{28} \ e^-/\mathrm{m}^3$.

Nos falta una estimación para el tiempo de relajamiento. El $e^-$ se mueve en una red cristalina. El $t_R$ será del órden en que el electrón tarda en interaccionar con dos átomos distintos (esto constituye otra \emph{hipótesis}).

Como $\mathcal E_c=\dfrac 3 2 k T = \dfrac 1 2 m v^2$ y la distancia entre átomos en la red cristalina es del orden de $l \sim 5\times 10^{-9}\ \mathrm{m}$ y $t\sim l/v$ tendremos $t_R \sim 10^{-14} \ \mathrm{s}$.

De ello, $t_R  =  (m\sigma)/(n e^2)$ despejamo $\ \sigma \sim 10^7 \ \Omega^{-1} \mathrm{m}^{-1}$, lo que coincide con lo encontrado experimentalmente con la ley de Ohm.

\section{Efecto Joule}

Para que el electrón se mueva, hay que aplicarle una determinada energía.  El electrón roza constantemente con la red cristalina, incluso los átomos de la red van a vibrar más, una mayor agitación térmica significa mayor energía. A medida que se mueve la red cristalina puede aportar energía al exterior: \emph{radiación electromagnética}.

\emph{La radiación electromagnética} en los metales (infrarrojo o visible, calor) se produce cuando por ellos circula una corriente eléctrica.

$P=\vec F \cdot \vec v_E=e\vec E \cdot \vec v_E$ es la potencia necesaria para acelerar un electrón  a la velocidad de arrastre.

Vamos a calcular la potencia total que se puede emitir al exterior.

La potencia por unidad de volumen, $P_\tau=-ne\vec E \cdot \vec v_E$

$P=P_\tau (SL)=--ne\vec E \cdot \vec v_E Sl=neEv_EJl =(JS)(El)=IV$

Para los conductores que cumplen la ley de Ohm, $IR=V$, tenemos

\begin{equation}
P \ = \ I \ V \ = \ I^2\ R
\end{equation}

Esquemáticamente, la resistencia en un circuito se representa por --/|/|/\---.

\section{Combinación de resistencias}

\subsection{Combinación de resistencias en serie}

\begin{figure}[H]
	\centering
	\includegraphics[width=.9\textwidth]{imagenes/imagenes25/T25IM02.png}
\end{figure}

$V=V_1+V_2+V_3+\dots +V_N;\quad I$ es la misma en todas las $R_i$

Si todos los conductores obedecen la ley de Ohm,

$V=\sum V_i= IR_1+IR_2+\cdots +IR_N=I(R_1+R_2+\cdots +R_N)$, por lo que

\begin{equation}
\boldsymbol{ R_{serie} \ = \ R_1 \ + \ R_2 \ + \ \cdots \ + \ R_N	 \ = \ \sum_{i=1}^N R_i }
\end{equation}

\subsection{Combinación de resistencias en paralelo}

Todas las resistencias están conectadas a la misma $V$, pero son atravesadas por las distintas intensidades en que se divide la intensidad total.

\begin{figure}[H]
	\centering
	\includegraphics[width=.9\textwidth]{imagenes/imagenes25/T25IM03.png}
\end{figure}


$I=I_1+I_2+\cdots + I_N=\dfrac V{R_1}+\dfrac V{R_2}+\cdots +\dfrac V{R_N}= $

$=V \left( \dfrac 1{R_1}+ \dfrac 1{R_2}+ \cdots +\dfrac 1{R_N} \right)  =\dfrac V R$

\begin{equation}
\boldsymbol{ \dfrac 1 {R_{paralelo}} \ = \ \dfrac 1{R_1} \ + \ \dfrac 1{R_2} \ + \cdots \ + \ \dfrac 1{R_N} = \sum_{i=1}^N \dfrac 1{R_i} }
\end{equation}


\vspace{10mm} %******************************************

\begin{figure}[H]
	\centering
	\includegraphics[width=1.05\textwidth]{imagenes/imagenes25/T25IM04.png}
\end{figure}

\section{Fuerza electromotriz}


La fuerza electromotriz (representado fem o $\varepsilon$)
es toda causa capaz de mantener una diferencia de potencial entre dos puntos de un circuito abierto o de producir una corriente eléctrica en un circuito cerrado. Es una característica de cada generador eléctrico.

El trabajo por unidad de carga que se ha de realizar para que una carga de una vuelta a un circuito es
$\ \textcolor{gris}{\left( \vec E = \dfrac {\vec F}{q} \right)} \qquad \displaystyle
\oint_C \vec E \cdot \dd \vec l = \varepsilon,\ $ \emph{fem}.

Como $\vec E$ es conservativo, $\ \displaystyle \int_1^2 \vec E \cdot \dd \vec l = V_2-V_1=V=IR$

donde, por hipótesis, hemos supuesto que el conductor es óhmico (verifica la ley de Ohm)

En campos conservativos, $\ \displaystyle \oint \vec E \cdot \dd \vec l =0 \ \to \ I=0$, en un conductor cerrado en campo conservativo no circula corriente eléctrica. Nunca podremos construir corriente eléctrica si el campo es conservativo o estacionario.

Para ello se construyen los \emph{generadores}, son capaces de ceder energía a los electrones del circuito para que se muevan en un sentido o en otro. Esta energía, por efecto Joule, desprenderá calor. Los generadores obtinen su energía de reacciones químicas (pilas Volta) o por induccción magnética.

Si el conductor es óhmico, $\varepsilon=IR$, en el $SI$ se mide en $\mathrm{V}$.

\begin{multicols}{2}
Esquema de circuito eléctrico (en corriente continua).

$R_i$ es la resistencia interna del propio generador.

$\varepsilon=I(R+R_i)$

$\varepsilon=IR_e+IR_i=V_A-V_B+IR_i$
	\begin{figure}[H]
	\centering
	\includegraphics[width=.4\textwidth]{imagenes/imagenes25/T25IM05.png}
\end{figure}
\end{multicols}
$V_A-V_B$ es la diferencial de potencial a que se conecta la resistencia externa, tiene lugar en el interior del generador.

Por ello, $ \ \varepsilon > V$

Cuando se conecta un motor $G'$ a un generador $G$, aplicando el teorema de conservación de la energía, tenemos

$\varepsilon I=I^2(R+R_i+r_i)+\varepsilon' I \ \to \ I=\dfrac{\varepsilon -\varepsilon'}{R+R_i+r_i}= \dfrac{\sum_i \varepsilon_i}{\sum_i R_i}$

donde $r_i$ representa la resistencia interna del motor o paratos conectados y $\varepsilon'$ es la \emph{fuerza contra electromotriz}.

\section{Leyes de Kirchoff. Redes eléctricas}

\begin{multicols}{2}
Una \emph{red} eléctrica no es más que una combinación de fem y fcem (fuerza electromotriz y contra electromotriz).

Un \emph{nudo} eléctrico es la región donde confluyen dos o más conductores.
	\begin{figure}[H]
	\centering
	\includegraphics[width=.5\textwidth]{imagenes/imagenes25/T25IM07.png}
\end{figure}
\end{multicols}
Se asignan llas intensidades que van a ciercular por cafa rama y se aplican las \colorbox{LightYellow}{ \emph{\textbf{leyes de Kirchoff}} }. Se basan en dos teoremas de conservación: el teorema de conservación de la carga y el teorema de conservación de la energía.

\begin{miparrafodestacado}
\begin{enumerate}
	\item En cualquier nudo de la red, la suma de las intensidades ha de ser cero.
	\item La suma de las caídas de potencial a lo largo de cualquier camino cerrado en una red es nula.
\end{enumerate}
\end{miparrafodestacado}

Analicemos la sigiuiente red red donde aparecen 3 corrientes de rama.

\begin{multicols}{2}
Se elige, arbitrariamente,  un sentido de recorrido de malla.

Se toman como positivas las intensidades que salen del nudo y negativas la que entran. Aplicando la $1^a$ de Kirchoff al nudo superior de la imagen:
	\begin{figure}[H]
	\centering
	\includegraphics[width=.55\textwidth]{imagenes/imagenes25/T25IM08.png}
\end{figure}
\end{multicols}
$-I1+I_2+I_3=0$

En cuanto a la segunda ley:

\begin{itemize}
\item El sentido de las fem es el siguiente, si el sentido que hemos elegido de malla hace que la `corriente de malla' entre por el borne negativo y salga por el postivo, la femm es positiva; en caso contrario la tomaremos negariva. En la figura anterior, $\varepsilon_1>0,\ \varepsilon_2<0$

\item El sentido de las caídas de tensión en las resistencias será positivo si la intensidad que la atraviesa lleva el mismo sentido que la `corriente de malla' y negativo en caso contrario. Así, tendremos $+I_1R_1; \ +I_2 R_2; \ -I_3R_3$ en la malla II. 
\end{itemize}

\vspace{5mm} %************************************
\begin{miparrafodestacado}
	Plantearemos la primera ley de Kirchoff a $n-1$ nodos de la red y completaremos las ecuaciones necesarias aplicando la segunda ley de kirchoff a las mallas que haga falta.
\end{miparrafodestacado}

$$ \begin{cases}
 \ \text{nudo superior} & -I_1+I_2+I_3=0 \\
 \ \text{malla I} & -\varepsilon_1+I_1(R_1+r_1)+I_3R_3=0 \\
 \ \text{Malla II} & +\varepsilon_2+I_2(R_2+r_2)-I_3R_3=0
 \end{cases}$$
 
 Tenemos 3 ecuaciones con 3 incógnitas, $I_1,\ I_2,\ I_3$.
 

 
 \begin{multicols}{2}
En los circuitos sencillos solo hay una malla y ningún nodo.

 $$\varepsilon - I(R+R_1)=0$$
 
 $\quad$
	\begin{figure}[H]
	\centering
	\includegraphics[width=.3\textwidth]{imagenes/imagenes25/T25IM09.png}
\end{figure}
\end{multicols}
 
 \section{Problemas}

\begin{prob}
Demostrar que si en una bateria de fem fija $\varepsilon$ y resistencia interna $r_i$ se conecta a una resistencia exterior $R$, se suministra máxima energía a la resistencia exterior cuando $R=r_i$	
\end{prob}

$\varepsilon I=I^2R+I^2r_i;\quad P_{ext}=I^2R \ \to \ P_{ext}=\varepsilon I-I^2 r_i$

Ohm: $\ \varepsilon =I(R+r_i) \to I=\dfrac{\varepsilon}{R+r_i}$, con lo que

$P_{ext}=\dfrac{\varepsilon^2}{R+r_i}-\dfrac{\varepsilon^2}{(R+r_i)^2}Rî=\dfrac{\varepsilon^2}{(R+r_i)^2} \ [(R+r_i)-r_i]=\dfrac{\varepsilon^2 R}{(R+r_i)^2}$

$\varepsilon,\ r_i$ son datos, por lo que $\ P_{ext}=P_{ext}(R)$

máximo cuando $\ \displaystyle \dv{P_{ext}}{R}=0:$

$\displaystyle \dv{P_{ext}}{R}=\dfrac{\varepsilon^2(R+r_i)^2-\varepsilon R 2(r+r_i)}{(R+r_i)^4}=\dfrac{\varepsilon^2}{(R+r_i)^4} [ (R+r_i)^2-2R^2-2Rr_i ]=\dfrac{\varepsilon^2}{(R+r_i)^4} [-R^2-r_i^2]=0 \leftrightarrow R^2= r_i^2 \ \Rightarrow \ R=r_i$


\begin{prob}
Calcula las corrientes de rama en los circuitos de las figuras adjuntas y averigua la ddp entre los puntos A y B	
\end{prob}

\begin{figure}[H]
	\centering
	\includegraphics[width=1\textwidth]{imagenes/imagenes25/T25IM10.png}
\end{figure}

\begin{figure}[H]
	\centering
	\includegraphics[width=1\textwidth]{imagenes/imagenes25/T25IM11.png}
\end{figure}


\begin{prob}
Se desea convertir un galvanómetro con $r_i=20\ \Omega$ que puede medir hasta $I_a=1\ \mathrm{mV}$ en

a) un amperímetro que pueda medir intensidades de $I=20\ \mathrm{mA}$

b) un voltímetro que pueda medir ddp de hasta $V=10\ \mathrm{V}$
\end{prob}

Los amperímetros se conectan en serie al circuito y los voltímetros en paralelo.

--- a) Al conectarle al amperímetro una resistencia externa $R$ en paralelo y montar el sitema en serie en  la rama del circuito cuya $I=2\times 10^{-2} \ \mathrm{A}$ que deseamos medir, ambas ramas, las del amperímetro y las de la resistencia externa, son atravesadas por intensidades $I_1 \text{ y } I_R$ tales que $I=I_a+I_R$

Al estar en paralelo en el circuito, ambas están conectadas a la misma ddp:

$I_a r_i =(I-I_a) R \ \to \ R=\dfrac{I_a r_i}{I-I_a}=\dfrac{10^{-3} \cdot 20}{5\times 10^{-2}-10^{-3}}=0.408\ \Omega$

--- b) Para que el galvanómetro anterior, que  mide una ddp $V=I r = 0.02 \ \mathrm{V}$ pueda medir mayores voltajes se le añade una resistencia $R$ en serie y se conecta el sistema en paralelo a los extremos del circuito donde se desea medir la $V=10\ \mathrm{V}$

La intensidad que atraviesa la resistencia interna del aparato y la externa en serie añadida es la misma y ha de ser la que puede soportar el aparato, $I=10^{-3} \ \mathrm{A}$ 

Como El potencial que se desea medir es de $V=20 \ \mathrm{V}$, lo intercalamos en paralelo al circuito y tendremos que

$V=I(R+r_i) \ \to \ R=\dfrac V I - r_i = \dfrac{10}{10^{-3}}-20=9980\ \Omega$


\newpage %*********************************************


\begin{myblock}{Esquemas circuito eléctrico (continua)}
	\begin{figure}[H]
	\centering
	\includegraphics[width=1\textwidth]{imagenes/imagenes25/T25IM06.png}
\end{figure}	
\end{myblock}







\include{TEMA26_chapter-A4}
\include{TEMA27_chapter-A4}
\chapter{Propiedades magnéticas de la materia}
\chaptermark{Propiedades magnéticas materia}

\begin{miparrafo}
	En ausencia de un campo magnético la mayor parte de la materia no manifiesta propiedades magnéticas; eso es debido a que internamente, los campos magnéticos generados por el movimiento de los electrones están compensados unos con otros. Sin embargo al someter a un material, sea el que sea, a la acción de un campo magnético exterior, se produce una distorsión del movimiento electrónico lo que provoca la aparición de un momento magnético opuesto al campo exterior. Además, se da el caso de materiales que poseen de antemano un momento magnético y al ser sometidos a la acción del campo se produce una alineación de dichos momentos, lo que favorece la propagación del campo magnético.
\end{miparrafo}


\section[Comportamiento de la materia en un campo magnético: paramagnetismo, diamagnetismo, ferromagnetismo]{Comportamiento de la materia en un campo magnético: paramagnetismo, diamagnetismo, ferromagnetismo\sectionmark{Para, diam y ferro - magnetismo}}
\sectionmark{Para, diam y ferro - magnetismo}


\begin{multicols}{2}
Consideremos un campo magnético creado por un solenoide. Introducimos una muestra con un poco de material sujeta por un dinamómetro en el interior de solenoide y apreciamos que aparece una fuerza que, curiosamente, no es más intensa en el centro del solenoide sino en sus bordes, donde el campo magnético es menor:
$\ \displaystyle \pdv{B}{z}$ es mayor en el borde que en el centro.

$B_{centro}=\dfrac{nI\mu_0}{L}$ 

$B_{bordes}=\dfrac 1 2 \dfrac{nI\mu_0}{L}$
\begin{figure}[H]
	\centering
	\includegraphics[width=0.3\textwidth]{imagenes/imagenes28/T28IM01.png}
\end{figure}	
\end{multicols}

Haremos una gran gama de experiencias como esta variando el material. La muestra debe ser pequeña.

Se observa:

\begin{itemize}
\item Experimentalmente se pone de manifiesto que la fuerza es directamente proporcional a la masa e independiente de su forma geométrica.
\item Algunas sustancias son atraídas hacia abajo, donde el campo magnético crece; otras son repelidas hacia arriba, donde el campo decrece. Y esto ocurre con independencia de la orientación del campo magnético (sentido de la corriente).	
\end{itemize}

Elegiremos el signo $-$ hacia arriba, hacia donde el campo decrece y $+$ hacia abajo, donde crece el campo.

La fuerza sobre $1 \ \mathrm{g}$ de nuestra muestra en un campo magnético con $B_z=1.8\times 10^4\ \mathrm{Gaus}$ y $\displaystyle \pdv {B_z} {z}=1.7\times 10^3 \ \mathrm{G\ cm}^{-2}$, a $21\ ^o$ es:



\begin{table}[H]
\centering
\begin{tabular}{lllll}
\multicolumn{1}{l|}{\textit{Sustancia}} & \textit{Fuerza(dinas)} & \textit{$\quad$} & \multicolumn{1}{l|}{\textit{Sustancia}} & \textit{Fuerza(dinas)} \\
\multicolumn{2}{c}{\textbf{Diamagnéticas}}                       &                  & \multicolumn{2}{c}{\textbf{Paramagnéticas}}                      \\ \cline{1-2} \cline{4-5} 
\multicolumn{1}{l|}{$H_2O$}             & $-22$                  &                  & \multicolumn{1}{l|}{$Na$}               &                        \\
\multicolumn{1}{l|}{$Cu$}               & $-2.6$                 &                  & \multicolumn{1}{l|}{$Al$}               &                        \\
\multicolumn{1}{l|}{$Pb$}               & $-3.7$                 &                  & \multicolumn{1}{l|}{$Cl_2Cu$}           &                        \\
\multicolumn{1}{l|}{$ClNa$}             & $-15$                  &                  & \multicolumn{1}{l|}{$SO_4Ni$}           &                        \\
\multicolumn{1}{l|}{$SiO_2$}            & $-16$                  &                  & \multicolumn{1}{l|}{$O_2\ (liq)$}       &                        \\
\multicolumn{1}{l|}{$C$, diamante}      & $-16$                  &                  & \multicolumn{2}{c}{\textbf{Ferromagnéticas}}                     \\ \cline{4-5} 
\multicolumn{1}{l|}{$C$, grafito}       & $-110$                 &                  & \multicolumn{1}{l|}{$Fe$}               & +400000                \\
\multicolumn{1}{l|}{$N_2\ (liq)$}       & $-10 \ (78 K)$         &                  & \multicolumn{1}{l|}{$F_e3O_4$}          & +120000               
\end{tabular}
\end{table}

Por definición, las sustancias que reaccionan hacia arriba se llaman \emph{\textbf{diamagnéticas}}, a las que lo hacen hacia abajo se les llama \emph{\textbf{paramagnéticas}}. En este tipo de sustancias, el efecto aumenta al aumentar la temperatura (\emph{Ley de Curie}). Cuando la fuerza es elevada, las sustancias se llaman 
\emph{\textbf{ferromagnéticas}}.


El $Na$ es paramagnético, pero el $ClNa$ es diamagnético. En cambio, el $Cu$ es diamagnético y el $Cl_2Cu$ es paramagnético.

Hay una cosa que tienen en común las sustancias paramagnéticas y diamagnéticas que las diferencia de las ferromagnéticas: imaginemos que disminuimos a la mitad el valor de la intensidad de la corriente que circula por el solenoide, el valor del campo magnético también disminuirá a la mitad. Se observa que tanto las sustancias paramagnéticas como las diamagnéticas reaccionan con una fuerza que es la cuarta parte del valor absoluto encontrado antes de disminuir la corriente mientras que las sustancias ferromagnéticas reaccionan con una fuerza reducida solo a la mitad, en consecencia:

$$F_{\text{Dia y Para-magnéticas}} \propto I^2; \qquad F_{\text{Ferromagnéticas}} \propto I$$

Esto es lo que hay que explicar teóricamente.

\section[Fuerzas en un dipolo magnético en un campo externo]{Fuerzas en un dipolo magnético en un campo externo\sectionmark{Fuerza en dipolo magnético}}

\sectionmark{Fuerza en dipolo magnético}

Siguiendo la idea de Ampère, suponemos la materia sometida a prueba constituida por corrientes moleculares que dan lugar un momento dipolar.


\begin{multicols}{2}
El circuito representa la corriente molecular. Supongamos que está contenido en el plano $XY$

En cada elemento del circuito, ley de Ampère:

$\dd \vec F = I \dd \vec l \times \vec B=$

$=I\vec \dd l \times (\vec u_r B_r + \vec k B_z)=$

$=I\dd \vec l \times \vec u_r B_r + I\dd \vec l \times \vec k B_z$
\begin{figure}[H]
	\centering
	\includegraphics[width=0.35\textwidth]{imagenes/imagenes28/T28IM02.png}
\end{figure}	
\end{multicols}
El primer término lleva el sentido de $-\vec k$ y el segundo término el de $\vec u_r$, por lo que éste último término no contribuirá a la fuerza total.

Para toda la espira, integrando:

$\displaystyle F=\oint_C I\dd l B_r = I 2\pi r B_r$

Veamos si podemos expresar $I$ en función del gradiente $\displaystyle \pdv{B_z}{z}$. Para ello usaremos el th. de Gauss para el campo magnético, $\displaystyle \oint_S \vec B \cdot \dd \vec S=0$

\begin{multicols}{2}
Suponemos, $z\to B_z;$

$ z+\dd z \to B_z+\dd B_z$

$B_z+\dd B_z=\displaystyle B_z+ \pdv{B_z}{z}\dd z$

Calculamos el flujo:
\begin{figure}[H]
	\centering
	\includegraphics[width=0.3\textwidth]{imagenes/imagenes28/T28IM03.png}
\end{figure}	
\end{multicols}

Flujo = Flujo que entra por abajo + flujo que sale por arriba + flujo que sale por la superficie lateral:

$\displaystyle -B_z \pi r^2 + \left( B_z+ \pdv{B_z}{z} \right) \pi r^2 + 2\pi r \dd z B_r = 0$

De aquí, se obtiene: $\ B_r=-\dfrac r 2 \displaystyle \pdv{B_z}{z}$, por lo que

$$F \ = \ -I\pi r^2 \displaystyle \pdv{B_z}{z} \ = \ -m\pdv{B_z}{z}$$

\section{Explicación clásica del magnetismo}


Imaginemos un campo como el del solenoide tal que al aumentar $z$, $B_z$ disminuye. Esto es debido al signo menos del gradiente.

Como $\vec m$ es un vector y $\displaystyle \pdv{B_z}{z}$ hace referencia al la componente vectorial del campo, se obtiene:

$$\vec B \uparrow \ \downarrow \vec m \ \to \ F \ \uparrow ;\qquad 
  \vec B \uparrow \ \uparrow \vec m \ \to \ F \ \downarrow$$

Si el momento dipolar magnético $\vec m$ asociado a una espira está asociado a la dirección paralela de un campo magnético $\to$ la fuerza que actúa sobre el dipolo va en el sentido en el cual la intensidad del campo es creciente. Si $\vec m$ es antiparalelo al campo externo $\to$ la fuerza actúa en el sentido de la intensidad del campo decreciente.

si $\displaystyle \pdv{B_z}{z}=0 \ \to \ F=0$, si no hay gradiente, variación del campo $\to$ no hay fuerza.

\begin{miparrafodestacado}
	El quid de la cuestión estriba en el por qué un material tiene un momento magnético que se orienta paralelamente al campo magnético externo para reaccionar paramagnéticamente o se orienta antiparalelamente al campo para reaccionar diamagnéticamente.
	
	Además, cuando la $I$ se reduce a la mitad, las sustancias para y diamagnéticas ven el efecto reducido a la cuarta parte mientras que en las ferromagnéticas solo se reduce a la mitad.
\end{miparrafodestacado}

Tendremos que exigir que $\ m \propto I;\ \displaystyle \pdv{B_z}{z} \propto I \ \to \ F \propto I^2$, para sustancias para o diamagnéticas.

Para ferromagnéticas, habrá de ocurrir que $\ m $ independiente de $I$ , $\ \ \displaystyle \pdv{B_z}{z} \propto I \ \to \ F \propto I$

\textbf{Problemas que se plantean:}

\begin{itemize}
	\item ?`Por qué unas sustancias tienen $\vec m$  paralelo o antiparalelo a $\vec B$?
	\item ?`Por qué el momento dipolar magnético $m$ de las sustancias para o diamagnéticas ha de ser proporcional a $B$ (o a $I$) mientras que en las sustancias ferromagnéticas ha de ser independiente del mismo?
\end{itemize}

\textcolor{gris}{Para explicar bien el electromagnétisno es necesaria la mecánica cuántica.}

\section{Teoría del diamagnetismo}

$\vec m = -\dfrac{e}{2m_e} \ \vec L;\qquad \vec L= I_{nercia}\  \vec \omega = m_er^2 \ \vec \omega \quad \to \quad \vec m=-\dfrac 1 2 e r^2 \ \vec \omega$

Tenemos las dos posibles situaciones para cualquier electrón en su órbita:

\begin{figure}[H]
	\centering
	\includegraphics[width=0.75\textwidth]{imagenes/imagenes28/T28IM04.png}
\end{figure}

Estudiemos como influye el campo magnético externo sobre el movimiento de los electrones alrededor del núcleo.

$\vec M=\vec m \times \vec B$; $\ \displaystyle \dv{\vec L}{t}= \vec m\times \vec B=-\dfrac{e}{2m_e}\vec L \times \vec B$

Tanto para en un caso u otro el sentido de rotación del extremos de $\vec L$ es el mismo, sentido contrario a las agaujas del reloj.

\begin{figure}[H]
	\centering
	\includegraphics[width=0.75\textwidth]{imagenes/imagenes28/T28IM05.png}
\end{figure}

\begin{multicols}{2}
\emph{arco = ángulo $\times$ radio}

$\dd L= L \sin \varphi \dd \theta$

$\dd \vec L = - \vec L \times \dfrac{\vec B}{B} \ \dd \theta$

$\displaystyle \dv{\vec L}{t}=-\vec L \times  \dfrac{\vec B}{B} \ \dv{\theta}{t}= - \vec L \times  \dfrac{\vec B}{B} \ \Omega$
\begin{figure}[H]
	\centering
	\includegraphics[width=0.3\textwidth]{imagenes/imagenes28/T28IM06.png}
\end{figure}	
\end{multicols}

Comparando con el resultado anterior, $\ \Omega=-\dfrac{eB}{2m_e}$, en forma vectorial:

$\vec \Omega \ = \ \dfrac{e\ \vec B}{2m_e}\ $  representa una \emph{precesión} de la órbita del electrón.

Así, el movimiento del electrón se deberá a dos causas: un movimiento electrostático respecto al núcleo, $\vec \omega$ y el debido al campo magnético externo, $\vec \Omega$. Finalmente, $\vec \omega_T=\vec \omega+\vec \Omega$

Aparece un momento dipolar inducido por la presencia de $\Omega$, que llamaremos $\vec m_{ind}$ que será:

$$m_{ind} \ = \ -\dfrac 1 2 e r^2 \ \vec \Omega \ =\ - \dfrac 1 4 \dfrac {e^2\ r^2}{m_e}\ \vec B$$

Cualquier electrón, en presencia de un campo magnético externos, tendrá un momento dipolar que será suma del que tenía más el inducido.

Por término medio y desde el punto de vista clásico, todos los sentidos de giros posibles se van a cancelar pero los momentos dipolares inducidos no se cancelan nunca por tener siempre la misma dirección y el mismo sentido (contrario a las agujas del reloj).

$\vec m_{total}=\sum \vec m_{ind} \neq 0$, todos tienen la misma dirección y sentido.

$$ \vec m \ = \ - \dfrac 1 4 \dfrac {e^2\ r^2}{m_e}\ \vec B$$

la fuerza se dirige hacia donde la intensidad del campo magnético crece.

Según este resultado, todas las sustancias deberían ser diamagnéticas. \emph{?`Dónde están las paramagnéticas?}

Efectivamente, todas las ustancias tienen una componente diamagnética, pero en las  paramagnéticas ha de ocurrir algo que haga aparecer otro momento dipolar magnético orientado en el mismo sentido del campo magnético externo. \textbf{\emph{Hasta aquí llega el electromagnetismo clásico.}}

Vamos a usar una mecánica clasico-cuántica.

$\text{electrones en átonos } \to 
	\begin{cases}
		\text{número par } &\to  \text{\ diamagnéticas} \\ 
		\text{número impar } &\to  \text{\ paramagnéticas}
 	\end{cases}$

En las sustancias diamagnéticas, el único momento dipolar que se manifiesta es el inducido. En las sustancias paramagnéticas aparece un nuevo momento dipolar magnético. La naturaleza solo presenta una excepción a esta regla, el $Cu^{29}$ que con número impar de electrones se comporta como diamagnético.

\begin{miparrafodestacado}
	\textbf{Conclusiones}: \emph{Todos los átomos de la naturaleza, en principio, tienen caracter diamagnético. Aquellos con número par de electrones cancelan dos a dos sus momentos dipolares y se comportan diamagnéticamente. En las sustancias con número impar de electrones (excepto el $Cu^{29}$) tienen un momento dipolar que no se cancela y supera al $\vec m_{ind}$ por lo que son paramagnéticas. }
\end{miparrafodestacado}

\subsection{Cálculo de la susceptibilidad diamagnética}

Vector magnetización $\ \vec M=n\vec m=\chi_m \vec H$, $n$ número de átomos, $\chi_m$ susceptibilidad magnética.

$\mu'=\dfrac{\mu}{\mu_0}=1+\chi_m$. $\mu'$ permeabilidad relativa.

En las sustancias diamagnéticas, $\vec M$ antiparalelo a $\vec H$ (o $\vec B$), para esas sustancias $\chi_m<0$ y $\mu'<1$

Para sustancias paramagnéticas, $\vec M$ paralelo $\vec H$, por lo que $\chi:m>0$ y $\mu'>1$, es decir $\mu>\mu_0$

En las sustancias ferromagnéticas, también ocurre que $\vec M$ paralelo a $\vec H$, pero $\chi_m>>0 \ \to \mu' >> 1 \ \to \ \mu >>\mu_0$

Como $\vec M=\chi_m \vec H$ y $\vec m_{imd}=-\dfrac 1 4 \dfrac{e^2r^2}{m_e}\vec B$, entonces:

$\vec M_{\text{diamagnético}}= n\  \vec m_{ind} = - \dfrac 1 4 \dfrac{e^2r^2n}{m_e} \ \vec B=- \dfrac 1 4 \dfrac{e^2r^2n}{m_e} \ \mu_o \vec H=\chi_m \ \vec H$

$$\chi_m \ = \ -\dfrac 1 4 \ \dfrac{e^2\ r^2\ n}{m_e} \ \mu_0$$

\section{Teoría del paramagnetismo}

El paramagnetismo solo se manifiesta en átomos con número impar de electrones (excepto el $Cu^{29}$). Sometemos una muestra de un material paramagnético, de número impar de electrones, a la acción de un campo magnético externo, $\vec B$. Para cada uno de estos átomos, 

$\mathcal E_p=-\vec m \cdot \vec B=-mB\cos \varphi$, donde $\varphi$ es el ángulo entre $\vec m$ y $\vec B$

Un resultado de \emph{mecánica estadística} asegura quepara ua muestra con $\mathcal E_4$, se cumple:
$\ \displaystyle \dv{n(\varphi)}{\Omega}=n_0\ e^{-\mathcal E_p / KT}; \quad n(\varphi)=\dv{m}{\tau}.\ $ Donde, $n(\varphi)$ es el número de elementos por unidad de volumen orientados en la dirección $\varphi$; $K=R/N_A$ es la constante de Blotzman, $1.3805\times 10^{-23}\ \mathrm{J\ K}^{-1}$, y $T$ es la temperatura absoluta.

En el caso del dipolo magnético, $\ \displaystyle \dv{n(\varphi)}{\Omega}= n_0\ e^{mB\cos \varphi / KT}$

Aproximando, $\ \dfrac{mB\cos \varphi}{KT} \sim \dfrac{10^{-23}\times 2 \times 0.5}{1.4\times 10^{-23} \times 300} \sim 0.002$, o sea, el argumento de la exponencial es del orden de milésimas.

Desarrollo en serie McLaurin: $\ e^x=1+x+\dfrac{x^2}{2!}¡\dfrac{x^3}{3!}+\cdots$, quedándonos a primer orden (despreciamos billonésimas frente a milésimas,

$$\displaystyle \dv{n(\varphi)}{\Omega} \ \approx \ n_0 \left[ 1 + \dfrac {mB\cos \varphi}{KT} \right]$$

$ \text{Para } \vec B \ \uparrow \uparrow \ \vec m \ \to \  \varphi=0 \ \to \ \cos \varphi = 1 \\
\text{Para } \vec B \ \uparrow \downarrow \ \vec m \ \to \  \varphi=\pi \ \to \ \cos \varphi = -1 $

Despejando e integrando, obtenemos el número de dipolos por unidad de volumne en la dirección $\varphi$

$\displaystyle \int \dd n(\varphi) \ = \ n_0\ \int \left[ 1 + \dfrac {mB\cos \varphi}{KT} \right] \ \dd \Omega$

$\displaystyle M=\int_0^n  m \cos \varphi \dd n(\varphi) \ = \ \int_\Omega m \cos \varphi \left[ 1 + \dfrac {mB\cos \varphi}{KT} \right] \ \dd \Omega$

En el caso axial, $\ \dd \Omega = 2\pi \sin \varphi \dd \varphi = -2\pi \dd (\cos \varphi) \ \Rightarrow $

$ M=\dfrac{nm^2B}{3KT}\ , \qquad \text{con } n_0=n/4\pi$

Como $M=\chi_m H=\chi_m \dfrac B {\mu_0}$, finalmente

$$\chi_m \ = \ \dfrac {n\ m^2 \ \mu_0}{K \ \ T}$$

Esta es la demostración de \emph{Lancevin}. Hemos obtenido una \textbf{susceptibilidad paramagnética} positiva, con $\vec m \ || \ \vec B$ y que es inversamente proporcional a la temperatura absoluta, $\chi_m \propto \dfrac 1 T$ para materiales paramagnéticos, como establece la ley experimental de \emph{Pierre Curie}.

\section[Ferromagnetismo, ferrimagnetismo y antiferromagnetismo]{Ferromagnetismo, ferrimagnetismo y antiferromagnetismo\sectionmark{Ferro, ferri y antiferro - magnetismo}}
\sectionmark{Ferro, ferri y antiferro - magnetismo}

\begin{footnotesize}

El ferromagnetismo es un fenómeno físico en el que se produce ordenamiento magnético de todos los momentos magnéticos de una muestra, en la misma dirección y sentido. La interacción magnética que hace que los polos magnéticos tiendan a disponerse en la misma dirección y sentido ha de extenderse por todo el sólido para alcanzar el ferromagnetismo.

Los ferromagnetos están divididos en dominios magnéticos, separados por superficies conocidas como paredes de Bloch. En cada uno de estos dominios, todos los momentos magnéticos están alineados. 

Al someter un material ferromagnético a un campo magnético intenso, los dominios tienden a alinearse con este, de forma que aquellos dominios en los que los dipolos están orientados con el mismo sentido y dirección que el campo magnético inductor aumentan su tamaño. Este aumento de tamaño se explica por las características de las paredes de Bloch, que avanzan en dirección a los dominios cuya dirección de los dipolos no coincide; dando lugar a un monodominio. Al eliminar el campo, el dominio permanece durante cierto tiempo.

\begin{figure}[H]
	\centering
	\includegraphics[width=1\textwidth]{imagenes/imagenes28/T28IM07.png}
\end{figure}


El ferrimagnetismo es un fenómeno físico en el que se produce ordenamiento magnético de los momentos magnéticos de una muestra de modo que todos los momentos magnéticos están alineados en la misma dirección pero no en el mismo sentido. Así que algunos de ellos están opuestos y se anulan entre sí, en parte o completamente. Sin embargo estos momentos magnéticos que se pueden anular están distribuidos aleatoriamente y no consiguen anular por completo la magnetización espontánea. Un ferrimagneto es una muestra de material que presenta ferrimagnetismo. 

El ferrimagnetismo también presenta, como el ferromagnetismo, magnetizaciones permanentes y de saturación (punto en el que ya no aumenta la magnetización por más que aumentemos la fuerza del campo). Otra similitud es que por encima de la temperatura de Curie se pierde el ferrimagnetismo y el material pasa a ser paramagnético.

Los materiales ferrimagnéticos proceden normalmente de la ferrita. Las ferritas, siendo materiales cerámicos, son buenos aislantes eléctricos. 

El antiferromagnetismo es el ordenamiento magnético de todos los momentos magnéticos de una muestra, durante la aplicación de un campo magnético, en la misma dirección. Al cesar el campo magnético externo la mitad de los momentos magnéticos de la muestra cambian en sentido inverso (por pares, por ejemplo, o una subred frente a otra). La interacción antiferromagnética, interacción magnética que hace que los momentos magnéticos tiendan a disponerse en la misma dirección y en sentido inverso, cancelándolos si tienen el mismo valor absoluto, o reduciéndolos si son distintos, ha de extenderse por todo un sólido para alcanzar el antiferromagnetismo.

Como el ferromagnetismo, la interacción antiferromagnética se destruye a alta temperatura. La temperatura por encima de la cual no se aprecia el antiferromagnetismo se llama temperatura de Neel, nombrada en honor del químico francés  Louis Néel (1904 – 2000), que había identificado por primera vez este tipo de ordenamiento magnético. Por encima de esta, los compuestos son típicamente paramagnéticos.

Generalmente, los antiferromagnetos están divididos en dominios magnéticos. En cada uno de estos dominios, todos los momentos magnéticos están alineados. En las fronteras entre dominios hay cierta energía potencial.

Al someter un material antiferromagnético a un campo magnético intenso, algunos de los momentos magnéticos se alinean paralelamente con él, aun a costa de alinearse también paralelo a sus vecinos (superando la interacción antiferromagnética). Generalmente, se requiere un campo magnético muy intenso para conseguir alinear todos los momentos magnéticos de la muestra.


\begin{figure}[H]
	\centering
	\includegraphics[width=.6\textwidth]{imagenes/imagenes28/T28IM08.png}
\end{figure}

\end{footnotesize}


\newpage %***********************************************
\begin{myblock}{Electromagnetismo}
En este capítulo hemos discutido los campos eléctrico y magnético estáticos como dos entidades separadas, sin relación alguna entre ellas, excepto que las fuentes del campo eléctrico son las cargas electricas y las del campo magnético son las corrientes eléctricas. En consecuencia hemos obtenido dos conjuntos de ecuaciones separadas, las cuales aparecen en la tabla en ambas formas, integral y diferencial. 

\begin{table}[H]
\centering
\begin{tabular}{l|l|l}
\multicolumn{1}{c|}{\textbf{Ley}}        & \multicolumn{1}{c|}{\textbf{Forma integral}}                                     & \multicolumn{1}{c}{\textbf{Forma diferencial}}                                       \\ \hline
Ley de Gauss para \\ el campo eléctrico  & $\quad \displaystyle \oint_S \vec E \cdot \dd \vec S = \dfrac{q}{\varepsilon_0}$  & $\quad \overrightarrow{\nabla}\cdot \overrightarrow{E}=\dfrac {\rho}{\varepsilon_0}$ \\
Ley de Gauss para \\  el campo magnético  & $\quad \displaystyle \oint_S \vec B \cdot \dd \vec S = 0$                        & $\quad \overrightarrow{\nabla}\cdot \overrightarrow{B}=0$                            \\
Circulación del \\ campo eléctrico        & $\quad \displaystyle \int_L \vec E \cdot \dd \vec l = 0$                         & $\quad \overrightarrow{\nabla}\times \overrightarrow{E}=0$                           \\
Circulación del \\ campo magnético        & $\quad \displaystyle \int_L \vec B \cdot \dd \vec l = \mu_0\ I$                  & $\quad \overrightarrow{\nabla}\times \overrightarrow{B}=\mu_0 \vec J$               
\end{tabular}
\end{table}

Estas ecuaciones permiten calcular el campo eléctrico $\vec E$ y el campo magnético $\vec B$ si se conocen las cargas y las corrientes, y recíprocamente. De este modo parece como si los campos eléctrico y magnético se pudieran considerar como dos campos independientes. Sin embargo esto no es cierto, según las transformaciones de Lorentz (relatividad especial), dos observadores en movimiento relativo observan que $\vec E$ y $\vec B$.



Hemos estudiado los campos eléctrico y magnético estáticos, es decir, independientes de tiempo. En los casos que los campos dependan del tiempo, las ecuaciones precedentes requerirán algunas modificaciones. 

Todo ello formará parte de cursos superiores de \emph{electromagnetismo.}
	
\end{myblock}

\appendix
\include{APENDICES-A4}
		
\end{document}