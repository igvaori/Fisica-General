%\chapter{Intro}
\section{Intoducción}



\centering{
\fcolorbox{black}{fondoblau}{
\parbox{0.95\textwidth}{
	\textit{Este material es un conjunto de apuntes personales que comparto gratuitamente en la red. Se agradecería la comunicación de la detección de cualquier error.}
}}}
\justify

\begin{resumen}{Agradecimientos}

Me he basado, para su confección, en mis viejos apuntes de Física General de la facultad de física de la UV. He consultado y tomado material de los libros clásicos de física general: Alonso y Finn (parece ser que mi profesor seguía este libros, en ocasiones, al pie de la letra), Tipler y Mosca, Sears y Zemanski, Serway y Jewett; así como de mucho material que he encontrado en internet. 

Este texto no tiene más pretensiones que las citadas, una puesta en \LaTeX de mis antiguos apuntes que me sirva para desenpolvar mis conocimientos. Ni la totalidad del contenido ni su organización son del todo de mi gusto. Quizás, en un futuro .. ?`Quién sabe?, de todos modos me decido a compartirlo en la web por si a alguien le resulta de utilidad, aunque dejadme dudarlo. 
\end{resumen}





\vspace{5mm}
\emph{Este documento se comparte bajo licencia `Attribution-NonCommercial 4.0 International (CC BY-NC 4.0)'}
\vspace{5mm}

\begin{multicols}{2}
\begin{figure}[H]
	\centering
	\includegraphics[width=.4
	\textwidth]{imagenes/imagenes00/licencia.png}
\end{figure}
\begin{figure}[H]
	\centering
	\includegraphics[width=.3
	\textwidth]{imagenes/firma.png}
\end{figure}
\end{multicols}

\newpage %****************************************

\begin{myalertblock}{Para aprender física}
La resolución de problemas en física es primordial, aprender a resolver problemas es absolutamente indispensable; es imposible saber física sin poder hacer física.

\vspace{2mm}A la hora de resolver problemas de física no se debe nunca olvidar que las ecuaciones que obtengamos deben ser dimensionalmente correctas. Un análisis de dimensiones siempre son ayudará a decidir si lo que hemos hecho está mal.

\vspace{2mm}La meta de la resolución de problemas en física no es sólo obtener un número o una fórmula; es entender mejor. Ello implica examinar la respuesta para ver qué nos dice. En particular,  deberíamos preguntarnos: ¿Es lógica esta respuesta? ¿Es posible este resultado?

\vspace{2mm}Y lo más importante, al acabar el problema se deben analizar los casos extremos, ?`qué ocurriría si ...?
\end{myalertblock}








